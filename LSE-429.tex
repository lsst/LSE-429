\documentclass[DM,lsstdraft,STR,toc]{lsstdoc}
\usepackage{geometry}
\usepackage{longtable,booktabs}
\usepackage{enumitem}
\usepackage{arydshln}

\input meta.tex

\providecommand{\tightlist}{
  \setlength{\itemsep}{0pt}\setlength{\parskip}{0pt}}

\setcounter{tocdepth}{4}

\begin{document}

\def\milestoneName{Commissioning Sv: Single-Visit And Full-Survey Performance W/ Comcam}
\def\milestoneId{LVV-P2}
\def\product{Commissioning Science Verification}

\setDocCompact{true}

\title{ LVV-P2 Commissioning Sv: Single-Visit And Full-Survey Performance W/ Comcam Test Plan and Report}
\setDocRef{\lsstDocType-\lsstDocNum}
\date{\vcsdate}
\setDocUpstreamLocation{\url{https://github.com/lsst/lsst-texmf/examples}}
\author{ Keith Bechtol }

\input history_and_info.tex


\setDocAbstract{
This is the test plan and report for LVV-P2 (Commissioning Sv: Single-Visit And Full-Survey Performance W/ Comcam),
an LSST level 2 milestone pertaining to the Data Management Subsystem.
}


\maketitle

\section{Introduction}
\label{sect:intro}


\subsection{Objectives}
\label{sect:objectives}

 Initial verification of the single-visit and full survey performance at
the raft scale using ComCam. Bulk data production from sustained on-sky
observations.



\subsection{System Overview}
\label{sect:systemoverview}

 \subsection{Applicable Documents}\label{applicable-documents}

\citeds{LSE-419} Commissioning Science Verification Test Specification


\subsection{Document Overview}
\label{sect:docoverview}

This document was generated from Jira, obtaining the relevant information from the 
\href{https://jira.lsstcorp.org/secure/Tests.jspa#/testPlan/LVV-P2}{LVV-P2}
~Jira Test Plan and related Test Cycles (
  \href{https://jira.lsstcorp.org/secure/Tests.jspa#/testCycle/LVV-C4}{LVV-C4}
  \href{https://jira.lsstcorp.org/secure/Tests.jspa#/testCycle/LVV-C5}{LVV-C5}
  \href{https://jira.lsstcorp.org/secure/Tests.jspa#/testCycle/LVV-C36}{LVV-C36}
  \href{https://jira.lsstcorp.org/secure/Tests.jspa#/testCycle/LVV-C37}{LVV-C37}
).

Section \ref{sect:intro} provides an overview of the test campaign, the system under test (\product{}),
the applicable documentation, and explains how this document is organized.
Section \ref{sect:configuration}  describes the configuration used for this test.
Section \ref{sect:personnel} describes the necessary roles and lists the individuals assigned to them.
%Section \ref{sect:plannedtestactivities} provides the list of planned test cycles and test cases,
including all relevant information that fully describes the test campaign.

Section \ref{sect:overview} provides a summary of the test results, including an overview in Table \ref{table:summary},
an overall assessment statement and suggestions for possible improvements.
Section \ref{sect:detailedtestresults} provides detailed results for each step in each test case.

The current status of test plan LVV-P2 in Jira is \textbf{ Draft }.

\subsection{References}
\label{sect:references}
\renewcommand{\refname}{}
\bibliography{lsst,refs,books,refs_ads}
\section{Test Configuration}
\label{sect:configuration}

\subsection{Data Collection}

  Observing is not required for this test campaign.

\subsection{Verification Environment}
\label{sect:hwconf}

  \subsection{Entry Criteria}
  ComCam has passed electro-optical testing and is ready to begin
sustained on-sky observing.



\newpage
\section{Personnel}
\label{sect:personnel}

The personnel involved in the test campaign are shown in the following table.

\begin{longtable}{p{3cm}p{3cm}p{3cm}p{6cm}}
\hline
\multicolumn{2}{r}{Test Plan (LVV-P2) owner:} &
\multicolumn{2}{l}{\textbf{ Keith Bechtol } }\\\hline
\multicolumn{2}{r}{ LVV-C4 owner:} &
\multicolumn{2}{l}{\textbf{
    Undefined
}
} \\\hline
\textbf{Test Case} & \textbf{Assigned to} & \textbf{Executed by} & \textbf{Additional Test Personnel} \\ \hline
\href{https://jira.lsstcorp.org/secure/Tests.jspa#/testCase/LVV-T293}{LVV-T293}
& {\small Keith Bechtol } & {\small Keith Bechtol } &
\begin{minipage}[]{6cm}
\smallskip
{\small  }
\medskip
\end{minipage}
\\ \hline
\href{https://jira.lsstcorp.org/secure/Tests.jspa#/testCase/LVV-T295}{LVV-T295}
& {\small Keith Bechtol } & {\small Keith Bechtol } &
\begin{minipage}[]{6cm}
\smallskip
{\small  }
\medskip
\end{minipage}
\\ \hline
\href{https://jira.lsstcorp.org/secure/Tests.jspa#/testCase/LVV-T961}{LVV-T961}
& {\small Scott Daniel } & {\small  } &
\begin{minipage}[]{6cm}
\smallskip
{\small  }
\medskip
\end{minipage}
\\ \hline
\href{https://jira.lsstcorp.org/secure/Tests.jspa#/testCase/LVV-T959}{LVV-T959}
& {\small Scott Daniel } & {\small  } &
\begin{minipage}[]{6cm}
\smallskip
{\small  }
\medskip
\end{minipage}
\\ \hline
\href{https://jira.lsstcorp.org/secure/Tests.jspa#/testCase/LVV-T960}{LVV-T960}
& {\small Scott Daniel } & {\small  } &
\begin{minipage}[]{6cm}
\smallskip
{\small  }
\medskip
\end{minipage}
\\ \hline
\href{https://jira.lsstcorp.org/secure/Tests.jspa#/testCase/LVV-T956}{LVV-T956}
& {\small Scott Daniel } & {\small  } &
\begin{minipage}[]{6cm}
\smallskip
{\small  }
\medskip
\end{minipage}
\\ \hline
\href{https://jira.lsstcorp.org/secure/Tests.jspa#/testCase/LVV-T957}{LVV-T957}
& {\small Scott Daniel } & {\small  } &
\begin{minipage}[]{6cm}
\smallskip
{\small  }
\medskip
\end{minipage}
\\ \hline
\href{https://jira.lsstcorp.org/secure/Tests.jspa#/testCase/LVV-T595}{LVV-T595}
& {\small Scott Daniel } & {\small  } &
\begin{minipage}[]{6cm}
\smallskip
{\small  }
\medskip
\end{minipage}
\\ \hline
\href{https://jira.lsstcorp.org/secure/Tests.jspa#/testCase/LVV-T594}{LVV-T594}
& {\small Scott Daniel } & {\small  } &
\begin{minipage}[]{6cm}
\smallskip
{\small  }
\medskip
\end{minipage}
\\ \hline
\href{https://jira.lsstcorp.org/secure/Tests.jspa#/testCase/LVV-T593}{LVV-T593}
& {\small Scott Daniel } & {\small  } &
\begin{minipage}[]{6cm}
\smallskip
{\small  }
\medskip
\end{minipage}
\\ \hline
\href{https://jira.lsstcorp.org/secure/Tests.jspa#/testCase/LVV-T592}{LVV-T592}
& {\small Scott Daniel } & {\small  } &
\begin{minipage}[]{6cm}
\smallskip
{\small  }
\medskip
\end{minipage}
\\ \hline
\href{https://jira.lsstcorp.org/secure/Tests.jspa#/testCase/LVV-T591}{LVV-T591}
& {\small Scott Daniel } & {\small  } &
\begin{minipage}[]{6cm}
\smallskip
{\small  }
\medskip
\end{minipage}
\\ \hline
\href{https://jira.lsstcorp.org/secure/Tests.jspa#/testCase/LVV-T590}{LVV-T590}
& {\small Scott Daniel } & {\small  } &
\begin{minipage}[]{6cm}
\smallskip
{\small  }
\medskip
\end{minipage}
\\ \hline
\href{https://jira.lsstcorp.org/secure/Tests.jspa#/testCase/LVV-T589}{LVV-T589}
& {\small Scott Daniel } & {\small  } &
\begin{minipage}[]{6cm}
\smallskip
{\small  }
\medskip
\end{minipage}
\\ \hline
\href{https://jira.lsstcorp.org/secure/Tests.jspa#/testCase/LVV-T588}{LVV-T588}
& {\small Scott Daniel } & {\small  } &
\begin{minipage}[]{6cm}
\smallskip
{\small  }
\medskip
\end{minipage}
\\ \hline
\href{https://jira.lsstcorp.org/secure/Tests.jspa#/testCase/LVV-T587}{LVV-T587}
& {\small Scott Daniel } & {\small  } &
\begin{minipage}[]{6cm}
\smallskip
{\small  }
\medskip
\end{minipage}
\\ \hline
\href{https://jira.lsstcorp.org/secure/Tests.jspa#/testCase/LVV-T554}{LVV-T554}
& {\small Scott Daniel } & {\small  } &
\begin{minipage}[]{6cm}
\smallskip
{\small  }
\medskip
\end{minipage}
\\ \hline
\href{https://jira.lsstcorp.org/secure/Tests.jspa#/testCase/LVV-T548}{LVV-T548}
& {\small Scott Daniel } & {\small  } &
\begin{minipage}[]{6cm}
\smallskip
{\small  }
\medskip
\end{minipage}
\\ \hline
\href{https://jira.lsstcorp.org/secure/Tests.jspa#/testCase/LVV-T545}{LVV-T545}
& {\small Scott Daniel } & {\small  } &
\begin{minipage}[]{6cm}
\smallskip
{\small  }
\medskip
\end{minipage}
\\ \hline
\href{https://jira.lsstcorp.org/secure/Tests.jspa#/testCase/LVV-T389}{LVV-T389}
& {\small Imram Hasan } & {\small  } &
\begin{minipage}[]{6cm}
\smallskip
{\small  }
\medskip
\end{minipage}
\\ \hline
\href{https://jira.lsstcorp.org/secure/Tests.jspa#/testCase/LVV-T390}{LVV-T390}
& {\small Sam Schmidt } & {\small  } &
\begin{minipage}[]{6cm}
\smallskip
{\small  }
\medskip
\end{minipage}
\\ \hline
\href{https://jira.lsstcorp.org/secure/Tests.jspa#/testCase/LVV-T297}{LVV-T297}
& {\small Keith Bechtol } & {\small Keith Bechtol } &
\begin{minipage}[]{6cm}
\smallskip
{\small  }
\medskip
\end{minipage}
\\ \hline
\href{https://jira.lsstcorp.org/secure/Tests.jspa#/testCase/LVV-T298}{LVV-T298}
& {\small Keith Bechtol } & {\small Keith Bechtol } &
\begin{minipage}[]{6cm}
\smallskip
{\small  }
\medskip
\end{minipage}
\\ \hline
\href{https://jira.lsstcorp.org/secure/Tests.jspa#/testCase/LVV-T299}{LVV-T299}
& {\small Keith Bechtol } & {\small Keith Bechtol } &
\begin{minipage}[]{6cm}
\smallskip
{\small  }
\medskip
\end{minipage}
\\ \hline
\href{https://jira.lsstcorp.org/secure/Tests.jspa#/testCase/LVV-T360}{LVV-T360}
& {\small Keith Bechtol } & {\small Keith Bechtol } &
\begin{minipage}[]{6cm}
\smallskip
{\small  }
\medskip
\end{minipage}
\\ \hline
\multicolumn{2}{r}{ LVV-C5 owner:} &
\multicolumn{2}{l}{\textbf{
    Undefined
}
} \\\hline
\textbf{Test Case} & \textbf{Assigned to} & \textbf{Executed by} & \textbf{Additional Test Personnel} \\ \hline
\href{https://jira.lsstcorp.org/secure/Tests.jspa#/testCase/LVV-T986}{LVV-T986}
& {\small Scott Daniel } & {\small  } &
\begin{minipage}[]{6cm}
\smallskip
{\small  }
\medskip
\end{minipage}
\\ \hline
\href{https://jira.lsstcorp.org/secure/Tests.jspa#/testCase/LVV-T969}{LVV-T969}
& {\small Scott Daniel } & {\small  } &
\begin{minipage}[]{6cm}
\smallskip
{\small  }
\medskip
\end{minipage}
\\ \hline
\href{https://jira.lsstcorp.org/secure/Tests.jspa#/testCase/LVV-T968}{LVV-T968}
& {\small Scott Daniel } & {\small  } &
\begin{minipage}[]{6cm}
\smallskip
{\small  }
\medskip
\end{minipage}
\\ \hline
\href{https://jira.lsstcorp.org/secure/Tests.jspa#/testCase/LVV-T176}{LVV-T176}
& {\small Robert Gruendl } & {\small  } &
\begin{minipage}[]{6cm}
\smallskip
{\small  }
\medskip
\end{minipage}
\\ \hline
\href{https://jira.lsstcorp.org/secure/Tests.jspa#/testCase/LVV-T966}{LVV-T966}
& {\small Scott Daniel } & {\small  } &
\begin{minipage}[]{6cm}
\smallskip
{\small  }
\medskip
\end{minipage}
\\ \hline
\href{https://jira.lsstcorp.org/secure/Tests.jspa#/testCase/LVV-T963}{LVV-T963}
& {\small Scott Daniel } & {\small  } &
\begin{minipage}[]{6cm}
\smallskip
{\small  }
\medskip
\end{minipage}
\\ \hline
\href{https://jira.lsstcorp.org/secure/Tests.jspa#/testCase/LVV-T962}{LVV-T962}
& {\small Scott Daniel } & {\small  } &
\begin{minipage}[]{6cm}
\smallskip
{\small  }
\medskip
\end{minipage}
\\ \hline
\href{https://jira.lsstcorp.org/secure/Tests.jspa#/testCase/LVV-T950}{LVV-T950}
& {\small Scott Daniel } & {\small  } &
\begin{minipage}[]{6cm}
\smallskip
{\small  }
\medskip
\end{minipage}
\\ \hline
\href{https://jira.lsstcorp.org/secure/Tests.jspa#/testCase/LVV-T943}{LVV-T943}
& {\small Scott Daniel } & {\small  } &
\begin{minipage}[]{6cm}
\smallskip
{\small  }
\medskip
\end{minipage}
\\ \hline
\href{https://jira.lsstcorp.org/secure/Tests.jspa#/testCase/LVV-T942}{LVV-T942}
& {\small Scott Daniel } & {\small  } &
\begin{minipage}[]{6cm}
\smallskip
{\small  }
\medskip
\end{minipage}
\\ \hline
\href{https://jira.lsstcorp.org/secure/Tests.jspa#/testCase/LVV-T941}{LVV-T941}
& {\small Scott Daniel } & {\small  } &
\begin{minipage}[]{6cm}
\smallskip
{\small  }
\medskip
\end{minipage}
\\ \hline
\href{https://jira.lsstcorp.org/secure/Tests.jspa#/testCase/LVV-T939}{LVV-T939}
& {\small Scott Daniel } & {\small  } &
\begin{minipage}[]{6cm}
\smallskip
{\small  }
\medskip
\end{minipage}
\\ \hline
\href{https://jira.lsstcorp.org/secure/Tests.jspa#/testCase/LVV-T940}{LVV-T940}
& {\small Scott Daniel } & {\small  } &
\begin{minipage}[]{6cm}
\smallskip
{\small  }
\medskip
\end{minipage}
\\ \hline
\href{https://jira.lsstcorp.org/secure/Tests.jspa#/testCase/LVV-T938}{LVV-T938}
& {\small Scott Daniel } & {\small  } &
\begin{minipage}[]{6cm}
\smallskip
{\small  }
\medskip
\end{minipage}
\\ \hline
\href{https://jira.lsstcorp.org/secure/Tests.jspa#/testCase/LVV-T597}{LVV-T597}
& {\small Scott Daniel } & {\small  } &
\begin{minipage}[]{6cm}
\smallskip
{\small  }
\medskip
\end{minipage}
\\ \hline
\href{https://jira.lsstcorp.org/secure/Tests.jspa#/testCase/LVV-T596}{LVV-T596}
& {\small Scott Daniel } & {\small  } &
\begin{minipage}[]{6cm}
\smallskip
{\small  }
\medskip
\end{minipage}
\\ \hline
\href{https://jira.lsstcorp.org/secure/Tests.jspa#/testCase/LVV-T549}{LVV-T549}
& {\small Scott Daniel } & {\small  } &
\begin{minipage}[]{6cm}
\smallskip
{\small  }
\medskip
\end{minipage}
\\ \hline
\href{https://jira.lsstcorp.org/secure/Tests.jspa#/testCase/LVV-T547}{LVV-T547}
& {\small Scott Daniel } & {\small  } &
\begin{minipage}[]{6cm}
\smallskip
{\small  }
\medskip
\end{minipage}
\\ \hline
\href{https://jira.lsstcorp.org/secure/Tests.jspa#/testCase/LVV-T544}{LVV-T544}
& {\small Scott Daniel } & {\small  } &
\begin{minipage}[]{6cm}
\smallskip
{\small  }
\medskip
\end{minipage}
\\ \hline
\href{https://jira.lsstcorp.org/secure/Tests.jspa#/testCase/LVV-T532}{LVV-T532}
& {\small Scott Daniel } & {\small  } &
\begin{minipage}[]{6cm}
\smallskip
{\small  }
\medskip
\end{minipage}
\\ \hline
\href{https://jira.lsstcorp.org/secure/Tests.jspa#/testCase/LVV-T533}{LVV-T533}
& {\small Scott Daniel } & {\small  } &
\begin{minipage}[]{6cm}
\smallskip
{\small  }
\medskip
\end{minipage}
\\ \hline
\href{https://jira.lsstcorp.org/secure/Tests.jspa#/testCase/LVV-T294}{LVV-T294}
& {\small Keith Bechtol } & {\small Keith Bechtol } &
\begin{minipage}[]{6cm}
\smallskip
{\small  }
\medskip
\end{minipage}
\\ \hline
\href{https://jira.lsstcorp.org/secure/Tests.jspa#/testCase/LVV-T296}{LVV-T296}
& {\small Keith Bechtol } & {\small Keith Bechtol } &
\begin{minipage}[]{6cm}
\smallskip
{\small  }
\medskip
\end{minipage}
\\ \hline
\multicolumn{2}{r}{ LVV-C36 owner:} &
\multicolumn{2}{l}{\textbf{
    Undefined
}
} \\\hline
\textbf{Test Case} & \textbf{Assigned to} & \textbf{Executed by} & \textbf{Additional Test Personnel} \\ \hline
\href{https://jira.lsstcorp.org/secure/Tests.jspa#/testCase/LVV-T1071}{LVV-T1071}
& {\small Keith Bechtol } & {\small  } &
\begin{minipage}[]{6cm}
\smallskip
{\small  }
\medskip
\end{minipage}
\\ \hline
\multicolumn{2}{r}{ LVV-C37 owner:} &
\multicolumn{2}{l}{\textbf{
    Undefined
}
} \\\hline
\textbf{Test Case} & \textbf{Assigned to} & \textbf{Executed by} & \textbf{Additional Test Personnel} \\ \hline
\href{https://jira.lsstcorp.org/secure/Tests.jspa#/testCase/LVV-T965}{LVV-T965}
& {\small Scott Daniel } & {\small  } &
\begin{minipage}[]{6cm}
\smallskip
{\small  }
\medskip
\end{minipage}
\\ \hline
\href{https://jira.lsstcorp.org/secure/Tests.jspa#/testCase/LVV-T964}{LVV-T964}
& {\small Scott Daniel } & {\small  } &
\begin{minipage}[]{6cm}
\smallskip
{\small  }
\medskip
\end{minipage}
\\ \hline
\href{https://jira.lsstcorp.org/secure/Tests.jspa#/testCase/LVV-T967}{LVV-T967}
& {\small Scott Daniel } & {\small  } &
\begin{minipage}[]{6cm}
\smallskip
{\small  }
\medskip
\end{minipage}
\\ \hline
\end{longtable}

\newpage

\section{Test Campaign Overview}
\label{sect:overview}

\subsection{Summary}
\label{sect:summarytable}

\begin{longtable}{p{2cm}p{2.5cm}p{9cm}p{2.5cm}}
\toprule
\multicolumn{3}{p{13.5cm}}{ Test Plan {\bf LVV-P2: Commissioning SV: Single-visit and Full-survey Performance w/ ComCam }} & Draft \\\hline

  \multicolumn{3}{p{13.5cm}}{ Test Cycle {\bf LVV-C4: Commissioning SV: Single-visit Performance w/ ComCam }} & Not Executed \\\hline

  {\bf \footnotesize test case} & {\bf \footnotesize status} & {\bf \footnotesize comment} & {\bf \footnotesize issues} \\\toprule

\href{https://jira.lsstcorp.org/secure/Tests.jspa#/testCase/LVV-T293}{LVV-T293}
    & Not Executed &
    \begin{minipage}[]{9cm}
    \smallskip
    
    \medskip
    \end{minipage}
    &
    \\\hline
\href{https://jira.lsstcorp.org/secure/Tests.jspa#/testCase/LVV-T295}{LVV-T295}
    & Not Executed &
    \begin{minipage}[]{9cm}
    \smallskip
    
    \medskip
    \end{minipage}
    &
    \\\hline
\href{https://jira.lsstcorp.org/secure/Tests.jspa#/testCase/LVV-T961}{LVV-T961}
    & Not Executed &
    \begin{minipage}[]{9cm}
    \smallskip
    
    \medskip
    \end{minipage}
    &
    \\\hline
\href{https://jira.lsstcorp.org/secure/Tests.jspa#/testCase/LVV-T959}{LVV-T959}
    & Not Executed &
    \begin{minipage}[]{9cm}
    \smallskip
    
    \medskip
    \end{minipage}
    &
    \\\hline
\href{https://jira.lsstcorp.org/secure/Tests.jspa#/testCase/LVV-T960}{LVV-T960}
    & Not Executed &
    \begin{minipage}[]{9cm}
    \smallskip
    
    \medskip
    \end{minipage}
    &
    \\\hline
\href{https://jira.lsstcorp.org/secure/Tests.jspa#/testCase/LVV-T956}{LVV-T956}
    & Not Executed &
    \begin{minipage}[]{9cm}
    \smallskip
    
    \medskip
    \end{minipage}
    &
    \\\hline
\href{https://jira.lsstcorp.org/secure/Tests.jspa#/testCase/LVV-T957}{LVV-T957}
    & Not Executed &
    \begin{minipage}[]{9cm}
    \smallskip
    
    \medskip
    \end{minipage}
    &
    \\\hline
\href{https://jira.lsstcorp.org/secure/Tests.jspa#/testCase/LVV-T595}{LVV-T595}
    & Not Executed &
    \begin{minipage}[]{9cm}
    \smallskip
    
    \medskip
    \end{minipage}
    &
    \\\hline
\href{https://jira.lsstcorp.org/secure/Tests.jspa#/testCase/LVV-T594}{LVV-T594}
    & Not Executed &
    \begin{minipage}[]{9cm}
    \smallskip
    
    \medskip
    \end{minipage}
    &
    \\\hline
\href{https://jira.lsstcorp.org/secure/Tests.jspa#/testCase/LVV-T593}{LVV-T593}
    & Not Executed &
    \begin{minipage}[]{9cm}
    \smallskip
    
    \medskip
    \end{minipage}
    &
    \\\hline
\href{https://jira.lsstcorp.org/secure/Tests.jspa#/testCase/LVV-T592}{LVV-T592}
    & Not Executed &
    \begin{minipage}[]{9cm}
    \smallskip
    
    \medskip
    \end{minipage}
    &
    \\\hline
\href{https://jira.lsstcorp.org/secure/Tests.jspa#/testCase/LVV-T591}{LVV-T591}
    & Not Executed &
    \begin{minipage}[]{9cm}
    \smallskip
    
    \medskip
    \end{minipage}
    &
    \\\hline
\href{https://jira.lsstcorp.org/secure/Tests.jspa#/testCase/LVV-T590}{LVV-T590}
    & Not Executed &
    \begin{minipage}[]{9cm}
    \smallskip
    
    \medskip
    \end{minipage}
    &
    \\\hline
\href{https://jira.lsstcorp.org/secure/Tests.jspa#/testCase/LVV-T589}{LVV-T589}
    & Not Executed &
    \begin{minipage}[]{9cm}
    \smallskip
    
    \medskip
    \end{minipage}
    &
    \\\hline
\href{https://jira.lsstcorp.org/secure/Tests.jspa#/testCase/LVV-T588}{LVV-T588}
    & Not Executed &
    \begin{minipage}[]{9cm}
    \smallskip
    
    \medskip
    \end{minipage}
    &
    \\\hline
\href{https://jira.lsstcorp.org/secure/Tests.jspa#/testCase/LVV-T587}{LVV-T587}
    & Not Executed &
    \begin{minipage}[]{9cm}
    \smallskip
    
    \medskip
    \end{minipage}
    &
    \\\hline
\href{https://jira.lsstcorp.org/secure/Tests.jspa#/testCase/LVV-T554}{LVV-T554}
    & Not Executed &
    \begin{minipage}[]{9cm}
    \smallskip
    
    \medskip
    \end{minipage}
    &
    \\\hline
\href{https://jira.lsstcorp.org/secure/Tests.jspa#/testCase/LVV-T548}{LVV-T548}
    & Not Executed &
    \begin{minipage}[]{9cm}
    \smallskip
    
    \medskip
    \end{minipage}
    &
    \\\hline
\href{https://jira.lsstcorp.org/secure/Tests.jspa#/testCase/LVV-T545}{LVV-T545}
    & Not Executed &
    \begin{minipage}[]{9cm}
    \smallskip
    
    \medskip
    \end{minipage}
    &
    \\\hline
\href{https://jira.lsstcorp.org/secure/Tests.jspa#/testCase/LVV-T389}{LVV-T389}
    & Not Executed &
    \begin{minipage}[]{9cm}
    \smallskip
    
    \medskip
    \end{minipage}
    &
    \\\hline
\href{https://jira.lsstcorp.org/secure/Tests.jspa#/testCase/LVV-T390}{LVV-T390}
    & Not Executed &
    \begin{minipage}[]{9cm}
    \smallskip
    
    \medskip
    \end{minipage}
    &
    \\\hline
\href{https://jira.lsstcorp.org/secure/Tests.jspa#/testCase/LVV-T297}{LVV-T297}
    & Not Executed &
    \begin{minipage}[]{9cm}
    \smallskip
    
    \medskip
    \end{minipage}
    &
    \\\hline
\href{https://jira.lsstcorp.org/secure/Tests.jspa#/testCase/LVV-T298}{LVV-T298}
    & Not Executed &
    \begin{minipage}[]{9cm}
    \smallskip
    
    \medskip
    \end{minipage}
    &
    \\\hline
\href{https://jira.lsstcorp.org/secure/Tests.jspa#/testCase/LVV-T299}{LVV-T299}
    & Not Executed &
    \begin{minipage}[]{9cm}
    \smallskip
    
    \medskip
    \end{minipage}
    &
    \\\hline
\href{https://jira.lsstcorp.org/secure/Tests.jspa#/testCase/LVV-T360}{LVV-T360}
    & Not Executed &
    \begin{minipage}[]{9cm}
    \smallskip
    
    \medskip
    \end{minipage}
    &
    \\\hline

  \multicolumn{3}{p{13.5cm}}{ Test Cycle {\bf LVV-C5: Commissioning SV: Full-survey Performance w/ ComCam }} & Not Executed \\\hline

  {\bf \footnotesize test case} & {\bf \footnotesize status} & {\bf \footnotesize comment} & {\bf \footnotesize issues} \\\toprule

\href{https://jira.lsstcorp.org/secure/Tests.jspa#/testCase/LVV-T986}{LVV-T986}
    & Not Executed &
    \begin{minipage}[]{9cm}
    \smallskip
    
    \medskip
    \end{minipage}
    &
    \\\hline
\href{https://jira.lsstcorp.org/secure/Tests.jspa#/testCase/LVV-T969}{LVV-T969}
    & Not Executed &
    \begin{minipage}[]{9cm}
    \smallskip
    
    \medskip
    \end{minipage}
    &
    \\\hline
\href{https://jira.lsstcorp.org/secure/Tests.jspa#/testCase/LVV-T968}{LVV-T968}
    & Not Executed &
    \begin{minipage}[]{9cm}
    \smallskip
    
    \medskip
    \end{minipage}
    &
    \\\hline
\href{https://jira.lsstcorp.org/secure/Tests.jspa#/testCase/LVV-T176}{LVV-T176}
    & Not Executed &
    \begin{minipage}[]{9cm}
    \smallskip
    
    \medskip
    \end{minipage}
    &
    \\\hline
\href{https://jira.lsstcorp.org/secure/Tests.jspa#/testCase/LVV-T966}{LVV-T966}
    & Not Executed &
    \begin{minipage}[]{9cm}
    \smallskip
    
    \medskip
    \end{minipage}
    &
    \\\hline
\href{https://jira.lsstcorp.org/secure/Tests.jspa#/testCase/LVV-T963}{LVV-T963}
    & Not Executed &
    \begin{minipage}[]{9cm}
    \smallskip
    
    \medskip
    \end{minipage}
    &
    \\\hline
\href{https://jira.lsstcorp.org/secure/Tests.jspa#/testCase/LVV-T962}{LVV-T962}
    & Not Executed &
    \begin{minipage}[]{9cm}
    \smallskip
    
    \medskip
    \end{minipage}
    &
    \\\hline
\href{https://jira.lsstcorp.org/secure/Tests.jspa#/testCase/LVV-T950}{LVV-T950}
    & Not Executed &
    \begin{minipage}[]{9cm}
    \smallskip
    
    \medskip
    \end{minipage}
    &
    \\\hline
\href{https://jira.lsstcorp.org/secure/Tests.jspa#/testCase/LVV-T943}{LVV-T943}
    & Not Executed &
    \begin{minipage}[]{9cm}
    \smallskip
    
    \medskip
    \end{minipage}
    &
    \\\hline
\href{https://jira.lsstcorp.org/secure/Tests.jspa#/testCase/LVV-T942}{LVV-T942}
    & Not Executed &
    \begin{minipage}[]{9cm}
    \smallskip
    
    \medskip
    \end{minipage}
    &
    \\\hline
\href{https://jira.lsstcorp.org/secure/Tests.jspa#/testCase/LVV-T941}{LVV-T941}
    & Not Executed &
    \begin{minipage}[]{9cm}
    \smallskip
    
    \medskip
    \end{minipage}
    &
    \\\hline
\href{https://jira.lsstcorp.org/secure/Tests.jspa#/testCase/LVV-T939}{LVV-T939}
    & Not Executed &
    \begin{minipage}[]{9cm}
    \smallskip
    
    \medskip
    \end{minipage}
    &
    \\\hline
\href{https://jira.lsstcorp.org/secure/Tests.jspa#/testCase/LVV-T940}{LVV-T940}
    & Not Executed &
    \begin{minipage}[]{9cm}
    \smallskip
    
    \medskip
    \end{minipage}
    &
    \\\hline
\href{https://jira.lsstcorp.org/secure/Tests.jspa#/testCase/LVV-T938}{LVV-T938}
    & Not Executed &
    \begin{minipage}[]{9cm}
    \smallskip
    
    \medskip
    \end{minipage}
    &
    \\\hline
\href{https://jira.lsstcorp.org/secure/Tests.jspa#/testCase/LVV-T597}{LVV-T597}
    & Not Executed &
    \begin{minipage}[]{9cm}
    \smallskip
    
    \medskip
    \end{minipage}
    &
    \\\hline
\href{https://jira.lsstcorp.org/secure/Tests.jspa#/testCase/LVV-T596}{LVV-T596}
    & Not Executed &
    \begin{minipage}[]{9cm}
    \smallskip
    
    \medskip
    \end{minipage}
    &
    \\\hline
\href{https://jira.lsstcorp.org/secure/Tests.jspa#/testCase/LVV-T549}{LVV-T549}
    & Not Executed &
    \begin{minipage}[]{9cm}
    \smallskip
    
    \medskip
    \end{minipage}
    &
    \\\hline
\href{https://jira.lsstcorp.org/secure/Tests.jspa#/testCase/LVV-T547}{LVV-T547}
    & Not Executed &
    \begin{minipage}[]{9cm}
    \smallskip
    
    \medskip
    \end{minipage}
    &
    \\\hline
\href{https://jira.lsstcorp.org/secure/Tests.jspa#/testCase/LVV-T544}{LVV-T544}
    & Not Executed &
    \begin{minipage}[]{9cm}
    \smallskip
    
    \medskip
    \end{minipage}
    &
    \\\hline
\href{https://jira.lsstcorp.org/secure/Tests.jspa#/testCase/LVV-T532}{LVV-T532}
    & Not Executed &
    \begin{minipage}[]{9cm}
    \smallskip
    
    \medskip
    \end{minipage}
    &
    \\\hline
\href{https://jira.lsstcorp.org/secure/Tests.jspa#/testCase/LVV-T533}{LVV-T533}
    & Not Executed &
    \begin{minipage}[]{9cm}
    \smallskip
    
    \medskip
    \end{minipage}
    &
    \\\hline
\href{https://jira.lsstcorp.org/secure/Tests.jspa#/testCase/LVV-T294}{LVV-T294}
    & Not Executed &
    \begin{minipage}[]{9cm}
    \smallskip
    
    \medskip
    \end{minipage}
    &
    \\\hline
\href{https://jira.lsstcorp.org/secure/Tests.jspa#/testCase/LVV-T296}{LVV-T296}
    & Not Executed &
    \begin{minipage}[]{9cm}
    \smallskip
    
    \medskip
    \end{minipage}
    &
    \\\hline

  \multicolumn{3}{p{13.5cm}}{ Test Cycle {\bf LVV-C36: Commissioning SV: 20-year Depth w/ ComCam }} & Not Executed \\\hline

  {\bf \footnotesize test case} & {\bf \footnotesize status} & {\bf \footnotesize comment} & {\bf \footnotesize issues} \\\toprule

\href{https://jira.lsstcorp.org/secure/Tests.jspa#/testCase/LVV-T1071}{LVV-T1071}
    & Not Executed &
    \begin{minipage}[]{9cm}
    \smallskip
    
    \medskip
    \end{minipage}
    &
    \\\hline

  \multicolumn{3}{p{13.5cm}}{ Test Cycle {\bf LVV-C37: Commissioning SV: Scheduler Testing w/ ComCam }} & Not Executed \\\hline

  {\bf \footnotesize test case} & {\bf \footnotesize status} & {\bf \footnotesize comment} & {\bf \footnotesize issues} \\\toprule

\href{https://jira.lsstcorp.org/secure/Tests.jspa#/testCase/LVV-T965}{LVV-T965}
    & Not Executed &
    \begin{minipage}[]{9cm}
    \smallskip
    
    \medskip
    \end{minipage}
    &
    \\\hline
\href{https://jira.lsstcorp.org/secure/Tests.jspa#/testCase/LVV-T964}{LVV-T964}
    & Not Executed &
    \begin{minipage}[]{9cm}
    \smallskip
    
    \medskip
    \end{minipage}
    &
    \\\hline
\href{https://jira.lsstcorp.org/secure/Tests.jspa#/testCase/LVV-T967}{LVV-T967}
    & Not Executed &
    \begin{minipage}[]{9cm}
    \smallskip
    
    \medskip
    \end{minipage}
    &
    \\\hline
\caption{Test Campaign Summary}
\label{table:summary}
\end{longtable}

\subsection{Overall Assessment}
\label{sect:overallassessment}

Not yet available.

\subsection{Recommended Improvements}
\label{sect:recommendations}

Not yet available.

\newpage
\section{Detailed Test Results}
\label{sect:detailedtestresults}

\subsection{Test Cycle LVV-C4 }

Open test cycle {\it \href{https://jira.lsstcorp.org/secure/Tests.jspa#/testrun/LVV-C4}{Commissioning SV: Single-visit Performance w/ ComCam}} in Jira.

Commissioning SV: Single-visit Performance w/ ComCam\\
Status: Not Executed

Initial verification of the single-visit performance of ComCam with
respect to

\begin{enumerate}
\tightlist
\item
  Delivered image quality
\item
  Photometric performance
\item
  Astrometric performance
\item
  Image depth
\end{enumerate}

The nominal planned on-sky observations for this test are 20 fields x 5
epochs x 5 visits x 6 filters = 3K visits (\textasciitilde{}4 nights of
observations)

\begin{itemize}
\tightlist
\item
  Several of the fields should contain absolute spectrophotometric
  calibration standards.
\item
  The fields should cover a range of airmasses and source densities.
\end{itemize}

These same observations are planned to be
\href{https://jira.lsstcorp.org/secure/Tests.jspa\#/testCycle/LVV-C6}{repeated
with LSSTCam} to enable verification tests across the full focal plane.

\subsubsection{Software Version/Baseline}
Not provided.

\subsubsection{Configuration}
Not provided.

\subsubsection{Test Cases in LVV-C4 Test Cycle}

\paragraph{Test Case LVV-T293 - On-sky Observations: Single-visit Key Performance Metrics }\mbox{}\\

Open  \href{https://jira.lsstcorp.org/secure/Tests.jspa#/testCase/LVV-T293}{\textit{ LVV-T293 } }
test case in Jira.

Perform repeated observations of a set of 20 to 30 fields to evaluate
single-visit science performance metrics such as system throughput,
image quality, and astrometric and photometric repeatability. The fields
should be selected to span a range of object densities (e.g., by
sampling different Galactic latitudes) and should be observed under a
range of environmental conditions, including a range of airmass and sky
brightness. (WHAT RANGE?) The target fields should include
spectrophotometric standards such as DA white dwarfs to be used to
evaluate the absolute photometric calibration (HOW
MANY?)\\[2\baselineskip]The ~pointings in each field will be dithered to
allow tests of delivered image quality, throughput, calibration, and
astrometry across the full field of view. Photometric conditions are
required.\\[2\baselineskip]Sub-percent Photometry: Faint DA White Dwarf
Spectrophotometric Standards for Astrophysical Observatories\\
\url{https://arxiv.org/abs/1811.12534}\\[2\baselineskip]\textbf{Example
observations:}\\
20 fields x 5 visits x 6 filters x 5 epochs = 20 fields x 25 visits x 6
filters.

\textbf{ Preconditions}:\\


Execution status: {\bf Not Executed }

Final comment:\\


Detailed steps results:

\begin{longtable}{p{1cm}p{15cm}}
\hline
{Step} & Step Details\\ \hline
1 & Description \\
 & \begin{minipage}[t]{15cm}
{\footnotesize

\medskip }
\end{minipage}
\\ \cdashline{2-2}


 & Expected Result \\
 & \begin{minipage}[t]{15cm}{\footnotesize

\medskip }
\end{minipage} \\ \cdashline{2-2}

 & Actual Result \\
 & \begin{minipage}[t]{15cm}{\footnotesize

\medskip }
\end{minipage} \\ \cdashline{2-2}

 & Status: \textbf{ Not Executed } \\ \hline

\end{longtable}

\paragraph{Test Case LVV-T295 - Data Processing Campaign: Single-visit Key Performance Metrics }\mbox{}\\

Open  \href{https://jira.lsstcorp.org/secure/Tests.jspa#/testCase/LVV-T295}{\textit{ LVV-T295 } }
test case in Jira.



\textbf{ Preconditions}:\\


Execution status: {\bf Not Executed }

Final comment:\\


Detailed steps results:

\begin{longtable}{p{1cm}p{15cm}}
\hline
{Step} & Step Details\\ \hline
1 & Description \\
 & \begin{minipage}[t]{15cm}
{\footnotesize

\medskip }
\end{minipage}
\\ \cdashline{2-2}


 & Expected Result \\
 & \begin{minipage}[t]{15cm}{\footnotesize

\medskip }
\end{minipage} \\ \cdashline{2-2}

 & Actual Result \\
 & \begin{minipage}[t]{15cm}{\footnotesize

\medskip }
\end{minipage} \\ \cdashline{2-2}

 & Status: \textbf{ Not Executed } \\ \hline

\end{longtable}

\paragraph{Test Case LVV-T961 - Bright source measurement }\mbox{}\\

Open  \href{https://jira.lsstcorp.org/secure/Tests.jspa#/testCase/LVV-T961}{\textit{ LVV-T961 } }
test case in Jira.

Verify that we can adjust the exposure time to enable measurements of
sources brighter than the nominal LSST saturation limit

\textbf{ Preconditions}:\\


Execution status: {\bf Not Executed }

Final comment:\\


Detailed steps results:

\begin{longtable}{p{1cm}p{15cm}}
\hline
{Step} & Step Details\\ \hline
1 & Description \\
 & \begin{minipage}[t]{15cm}
{\footnotesize
For each band, calculate the expected saturation limit for 15 second
exposures.

\medskip }
\end{minipage}
\\ \cdashline{2-2}


 & Expected Result \\
 & \begin{minipage}[t]{15cm}{\footnotesize
A list of nominal saturation limits.

\medskip }
\end{minipage} \\ \cdashline{2-2}

 & Actual Result \\
 & \begin{minipage}[t]{15cm}{\footnotesize

\medskip }
\end{minipage} \\ \cdashline{2-2}

 & Status: \textbf{ Not Executed } \\ \hline

2 & Description \\
 & \begin{minipage}[t]{15cm}
{\footnotesize
Identify sources brighter than the saturation limits calculated in step
1 in external catalogs.\\[2\baselineskip]This will probably involve
fitting the stars to an SED so that we can extrapolate their magnitudes
in LSST bands as needed.

\medskip }
\end{minipage}
\\ \cdashline{2-2}

 & Test Data \\
 & \begin{minipage}[t]{15cm}{\footnotesize
External catalog of bright stars (presumably Gaia)

\medskip }
\end{minipage} \\ \cdashline{2-2}

 & Expected Result \\
 & \begin{minipage}[t]{15cm}{\footnotesize
Catalog of bright sources to be used for this test

\medskip }
\end{minipage} \\ \cdashline{2-2}

 & Actual Result \\
 & \begin{minipage}[t]{15cm}{\footnotesize

\medskip }
\end{minipage} \\ \cdashline{2-2}

 & Status: \textbf{ Not Executed } \\ \hline

3 & Description \\
 & \begin{minipage}[t]{15cm}
{\footnotesize
Calculate the exposure time needed to take unsaturated images of the
bright sources identified in step 2

\medskip }
\end{minipage}
\\ \cdashline{2-2}

 & Test Data \\
 & \begin{minipage}[t]{15cm}{\footnotesize
Catalog of bright sources from step 2

\medskip }
\end{minipage} \\ \cdashline{2-2}

 & Expected Result \\
 & \begin{minipage}[t]{15cm}{\footnotesize
List of exposure times to be used in this test

\medskip }
\end{minipage} \\ \cdashline{2-2}

 & Actual Result \\
 & \begin{minipage}[t]{15cm}{\footnotesize

\medskip }
\end{minipage} \\ \cdashline{2-2}

 & Status: \textbf{ Not Executed } \\ \hline

4 & Description \\
 & \begin{minipage}[t]{15cm}
{\footnotesize
In each band, take images of bright sources 1 magnitude brighter than 15
second saturation limit in that band, using the exposure time calculated
in step 3

\medskip }
\end{minipage}
\\ \cdashline{2-2}

 & Test Data \\
 & \begin{minipage}[t]{15cm}{\footnotesize
Bright source catalog from step 2\\
Exposure times calculated in step 3

\medskip }
\end{minipage} \\ \cdashline{2-2}

 & Expected Result \\
 & \begin{minipage}[t]{15cm}{\footnotesize
Images of bright sources in each LSST band

\medskip }
\end{minipage} \\ \cdashline{2-2}

 & Actual Result \\
 & \begin{minipage}[t]{15cm}{\footnotesize

\medskip }
\end{minipage} \\ \cdashline{2-2}

 & Status: \textbf{ Not Executed } \\ \hline

5 & Description \\
 & \begin{minipage}[t]{15cm}
{\footnotesize
Perform single image processing on the images from step 4. ~Verify that
the measured magnitudes of the bright sources agree with the magnitudes
inferred from the external catalog used in step 2. ~This will indicate
that we have successfully taken a science-quality image of the bright
source.

\medskip }
\end{minipage}
\\ \cdashline{2-2}

 & Test Data \\
 & \begin{minipage}[t]{15cm}{\footnotesize
Images from step 4\\
External catalog used in step 2

\medskip }
\end{minipage} \\ \cdashline{2-2}

 & Expected Result \\
 & \begin{minipage}[t]{15cm}{\footnotesize

\medskip }
\end{minipage} \\ \cdashline{2-2}

 & Actual Result \\
 & \begin{minipage}[t]{15cm}{\footnotesize

\medskip }
\end{minipage} \\ \cdashline{2-2}

 & Status: \textbf{ Not Executed } \\ \hline

\end{longtable}

\paragraph{Test Case LVV-T959 - Inter-band astrometric consistency }\mbox{}\\

Open  \href{https://jira.lsstcorp.org/secure/Tests.jspa#/testCase/LVV-T959}{\textit{ LVV-T959 } }
test case in Jira.

Verify that the separations between objects do not vary significantly
with band

\textbf{ Preconditions}:\\


Execution status: {\bf Not Executed }

Final comment:\\


Detailed steps results:

\begin{longtable}{p{1cm}p{15cm}}
\hline
{Step} & Step Details\\ \hline
1 & Description \\
 & \begin{minipage}[t]{15cm}
{\footnotesize
Image an average field in all six bands

\medskip }
\end{minipage}
\\ \cdashline{2-2}


 & Expected Result \\
 & \begin{minipage}[t]{15cm}{\footnotesize
Set of images

\medskip }
\end{minipage} \\ \cdashline{2-2}

 & Actual Result \\
 & \begin{minipage}[t]{15cm}{\footnotesize

\medskip }
\end{minipage} \\ \cdashline{2-2}

 & Status: \textbf{ Not Executed } \\ \hline

2 & Description \\
 & \begin{minipage}[t]{15cm}
{\footnotesize
Perform source detection and astrometric measurements on the images from
step 1

\medskip }
\end{minipage}
\\ \cdashline{2-2}

 & Test Data \\
 & \begin{minipage}[t]{15cm}{\footnotesize
Images from step 1

\medskip }
\end{minipage} \\ \cdashline{2-2}

 & Expected Result \\
 & \begin{minipage}[t]{15cm}{\footnotesize
Catalog of sources in images from step 1

\medskip }
\end{minipage} \\ \cdashline{2-2}

 & Actual Result \\
 & \begin{minipage}[t]{15cm}{\footnotesize

\medskip }
\end{minipage} \\ \cdashline{2-2}

 & Status: \textbf{ Not Executed } \\ \hline

3 & Description \\
 & \begin{minipage}[t]{15cm}
{\footnotesize
Find separations between all pairs of sources in catalogs from step 2

\medskip }
\end{minipage}
\\ \cdashline{2-2}

 & Test Data \\
 & \begin{minipage}[t]{15cm}{\footnotesize
Catalogs of sources from step 2

\medskip }
\end{minipage} \\ \cdashline{2-2}

 & Expected Result \\
 & \begin{minipage}[t]{15cm}{\footnotesize
Measurements of source separations in each band

\medskip }
\end{minipage} \\ \cdashline{2-2}

 & Actual Result \\
 & \begin{minipage}[t]{15cm}{\footnotesize

\medskip }
\end{minipage} \\ \cdashline{2-2}

 & Status: \textbf{ Not Executed } \\ \hline

4 & Description \\
 & \begin{minipage}[t]{15cm}
{\footnotesize
For each band, compute the RMS difference in source separations relative
to the r-band. ~Verify that this values is less than or equal to 10
milliarcseconds.

\medskip }
\end{minipage}
\\ \cdashline{2-2}

 & Test Data \\
 & \begin{minipage}[t]{15cm}{\footnotesize
Source separations from step 3

\medskip }
\end{minipage} \\ \cdashline{2-2}

 & Expected Result \\
 & \begin{minipage}[t]{15cm}{\footnotesize

\medskip }
\end{minipage} \\ \cdashline{2-2}

 & Actual Result \\
 & \begin{minipage}[t]{15cm}{\footnotesize

\medskip }
\end{minipage} \\ \cdashline{2-2}

 & Status: \textbf{ Not Executed } \\ \hline

5 & Description \\
 & \begin{minipage}[t]{15cm}
{\footnotesize
Verify that no more than 10 percent of source separation measurements in
any band vary by more than 20 milliarcseconds from the r band
measurements

\medskip }
\end{minipage}
\\ \cdashline{2-2}

 & Test Data \\
 & \begin{minipage}[t]{15cm}{\footnotesize
Source separations from step 3

\medskip }
\end{minipage} \\ \cdashline{2-2}

 & Expected Result \\
 & \begin{minipage}[t]{15cm}{\footnotesize

\medskip }
\end{minipage} \\ \cdashline{2-2}

 & Actual Result \\
 & \begin{minipage}[t]{15cm}{\footnotesize

\medskip }
\end{minipage} \\ \cdashline{2-2}

 & Status: \textbf{ Not Executed } \\ \hline

\end{longtable}

\paragraph{Test Case LVV-T960 - Relative astrometric performance (w/ Gaia) }\mbox{}\\

Open  \href{https://jira.lsstcorp.org/secure/Tests.jspa#/testCase/LVV-T960}{\textit{ LVV-T960 } }
test case in Jira.

Verify that relative astrometric separations are as accurate as
specified

\textbf{ Preconditions}:\\


Execution status: {\bf Not Executed }

Final comment:\\


Detailed steps results:

\begin{longtable}{p{1cm}p{15cm}}
\hline
{Step} & Step Details\\ \hline
1 & Description \\
 & \begin{minipage}[t]{15cm}
{\footnotesize
Image a region that overlaps the Gaia footprint (we will use Gaia as
astrometric truth)

\medskip }
\end{minipage}
\\ \cdashline{2-2}


 & Expected Result \\
 & \begin{minipage}[t]{15cm}{\footnotesize
Images taken from Gaia region

\medskip }
\end{minipage} \\ \cdashline{2-2}

 & Actual Result \\
 & \begin{minipage}[t]{15cm}{\footnotesize

\medskip }
\end{minipage} \\ \cdashline{2-2}

 & Status: \textbf{ Not Executed } \\ \hline

2 & Description \\
 & \begin{minipage}[t]{15cm}
{\footnotesize
Run source detection and astrometric measurement on images from step 1

\medskip }
\end{minipage}
\\ \cdashline{2-2}

 & Test Data \\
 & \begin{minipage}[t]{15cm}{\footnotesize
Images from step 1

\medskip }
\end{minipage} \\ \cdashline{2-2}

 & Expected Result \\
 & \begin{minipage}[t]{15cm}{\footnotesize
Catalog of sources detected in images from step 1

\medskip }
\end{minipage} \\ \cdashline{2-2}

 & Actual Result \\
 & \begin{minipage}[t]{15cm}{\footnotesize

\medskip }
\end{minipage} \\ \cdashline{2-2}

 & Status: \textbf{ Not Executed } \\ \hline

3 & Description \\
 & \begin{minipage}[t]{15cm}
{\footnotesize
Calculate the separation between all sources detected in step 2

\medskip }
\end{minipage}
\\ \cdashline{2-2}

 & Test Data \\
 & \begin{minipage}[t]{15cm}{\footnotesize
Source catalogs from step 2

\medskip }
\end{minipage} \\ \cdashline{2-2}

 & Expected Result \\
 & \begin{minipage}[t]{15cm}{\footnotesize
Measurements of source pair separations

\medskip }
\end{minipage} \\ \cdashline{2-2}

 & Actual Result \\
 & \begin{minipage}[t]{15cm}{\footnotesize

\medskip }
\end{minipage} \\ \cdashline{2-2}

 & Status: \textbf{ Not Executed } \\ \hline

4 & Description \\
 & \begin{minipage}[t]{15cm}
{\footnotesize
Compare source separations from step 3 to the same source separations as
measured by Gaia

\medskip }
\end{minipage}
\\ \cdashline{2-2}

 & Test Data \\
 & \begin{minipage}[t]{15cm}{\footnotesize
Source separations from step 3; cross-matched with Gaia catalog

\medskip }
\end{minipage} \\ \cdashline{2-2}

 & Expected Result \\
 & \begin{minipage}[t]{15cm}{\footnotesize
Distribution of astrometric errors relative to Gaia

\medskip }
\end{minipage} \\ \cdashline{2-2}

 & Actual Result \\
 & \begin{minipage}[t]{15cm}{\footnotesize

\medskip }
\end{minipage} \\ \cdashline{2-2}

 & Status: \textbf{ Not Executed } \\ \hline

5 & Description \\
 & \begin{minipage}[t]{15cm}
{\footnotesize
Examine distribution of source separation errors from step 4 for all
pairs of sources separated by \textasciitilde{} 5 arcminutes. ~Verify
that the median error in these measurements is \textless{}= 10
milliarcseconds

\medskip }
\end{minipage}
\\ \cdashline{2-2}

 & Test Data \\
 & \begin{minipage}[t]{15cm}{\footnotesize
Source separation errors from step 4

\medskip }
\end{minipage} \\ \cdashline{2-2}

 & Expected Result \\
 & \begin{minipage}[t]{15cm}{\footnotesize

\medskip }
\end{minipage} \\ \cdashline{2-2}

 & Actual Result \\
 & \begin{minipage}[t]{15cm}{\footnotesize

\medskip }
\end{minipage} \\ \cdashline{2-2}

 & Status: \textbf{ Not Executed } \\ \hline

6 & Description \\
 & \begin{minipage}[t]{15cm}
{\footnotesize
Verify that no more than 10\% of the source pairs separated by
\textasciitilde{} 5 arcminutes have separation errors greater than 20
milliarcseconds

\medskip }
\end{minipage}
\\ \cdashline{2-2}

 & Test Data \\
 & \begin{minipage}[t]{15cm}{\footnotesize
Source separation errors from step 4

\medskip }
\end{minipage} \\ \cdashline{2-2}

 & Expected Result \\
 & \begin{minipage}[t]{15cm}{\footnotesize

\medskip }
\end{minipage} \\ \cdashline{2-2}

 & Actual Result \\
 & \begin{minipage}[t]{15cm}{\footnotesize

\medskip }
\end{minipage} \\ \cdashline{2-2}

 & Status: \textbf{ Not Executed } \\ \hline

7 & Description \\
 & \begin{minipage}[t]{15cm}
{\footnotesize
Examine distribution of source separation errors from step 4 for all
pairs of sources separated by \textasciitilde{} 20 arcminutes. ~Verify
that the median error in these measurements is \textless{}= 10
milliarcseconds

\medskip }
\end{minipage}
\\ \cdashline{2-2}

 & Test Data \\
 & \begin{minipage}[t]{15cm}{\footnotesize
Source separation errors from step 4

\medskip }
\end{minipage} \\ \cdashline{2-2}

 & Expected Result \\
 & \begin{minipage}[t]{15cm}{\footnotesize

\medskip }
\end{minipage} \\ \cdashline{2-2}

 & Actual Result \\
 & \begin{minipage}[t]{15cm}{\footnotesize

\medskip }
\end{minipage} \\ \cdashline{2-2}

 & Status: \textbf{ Not Executed } \\ \hline

8 & Description \\
 & \begin{minipage}[t]{15cm}
{\footnotesize
Verify that no more than 10 percent of source pairs separated by
\textasciitilde{} 20 arcminutes have source separation errors greater
than 20 milliarcseconds

\medskip }
\end{minipage}
\\ \cdashline{2-2}

 & Test Data \\
 & \begin{minipage}[t]{15cm}{\footnotesize
Source separation errors from step 4

\medskip }
\end{minipage} \\ \cdashline{2-2}

 & Expected Result \\
 & \begin{minipage}[t]{15cm}{\footnotesize

\medskip }
\end{minipage} \\ \cdashline{2-2}

 & Actual Result \\
 & \begin{minipage}[t]{15cm}{\footnotesize

\medskip }
\end{minipage} \\ \cdashline{2-2}

 & Status: \textbf{ Not Executed } \\ \hline

9 & Description \\
 & \begin{minipage}[t]{15cm}
{\footnotesize
Examine distribution of source separation errors from step 4 for all
pairs separated by \textasciitilde{} 200 arcminutes. ~Verify that the
median error in these measurements is \textless{}= 15 milliarcseconds.

\medskip }
\end{minipage}
\\ \cdashline{2-2}

 & Test Data \\
 & \begin{minipage}[t]{15cm}{\footnotesize
Source separation errors from step 4

\medskip }
\end{minipage} \\ \cdashline{2-2}

 & Expected Result \\
 & \begin{minipage}[t]{15cm}{\footnotesize

\medskip }
\end{minipage} \\ \cdashline{2-2}

 & Actual Result \\
 & \begin{minipage}[t]{15cm}{\footnotesize

\medskip }
\end{minipage} \\ \cdashline{2-2}

 & Status: \textbf{ Not Executed } \\ \hline

10 & Description \\
 & \begin{minipage}[t]{15cm}
{\footnotesize
Verify that no more than 10 percent of sources separated by
\textasciitilde{} 200 arcminutes have source separation errors greater
than 30 milliarcseconds.

\medskip }
\end{minipage}
\\ \cdashline{2-2}

 & Test Data \\
 & \begin{minipage}[t]{15cm}{\footnotesize
Source separation errors from step 4

\medskip }
\end{minipage} \\ \cdashline{2-2}

 & Expected Result \\
 & \begin{minipage}[t]{15cm}{\footnotesize

\medskip }
\end{minipage} \\ \cdashline{2-2}

 & Actual Result \\
 & \begin{minipage}[t]{15cm}{\footnotesize

\medskip }
\end{minipage} \\ \cdashline{2-2}

 & Status: \textbf{ Not Executed } \\ \hline

\end{longtable}

\paragraph{Test Case LVV-T956 - Ghost area characterization }\mbox{}\\

Open  \href{https://jira.lsstcorp.org/secure/Tests.jspa#/testCase/LVV-T956}{\textit{ LVV-T956 } }
test case in Jira.

Verify that the area affected by significant ghosts is within specified
limits.

\textbf{ Preconditions}:\\


Execution status: {\bf Not Executed }

Final comment:\\


Detailed steps results:

\begin{longtable}{p{1cm}p{15cm}}
\hline
{Step} & Step Details\\ \hline
1 & Description \\
 & \begin{minipage}[t]{15cm}
{\footnotesize
Image a field of view with a bright star (magnitude=4 ?) in each of the
six bands.

\medskip }
\end{minipage}
\\ \cdashline{2-2}


 & Expected Result \\
 & \begin{minipage}[t]{15cm}{\footnotesize
A set of images in each band containing a bright star.

\medskip }
\end{minipage} \\ \cdashline{2-2}

 & Actual Result \\
 & \begin{minipage}[t]{15cm}{\footnotesize

\medskip }
\end{minipage} \\ \cdashline{2-2}

 & Status: \textbf{ Not Executed } \\ \hline

2 & Description \\
 & \begin{minipage}[t]{15cm}
{\footnotesize
Dither the telescope pointing so that the bright star is far off of the
field of view (so that we no longer expect it to produce ghosts).
~Re-image the dithered field in all six bands.

\medskip }
\end{minipage}
\\ \cdashline{2-2}


 & Expected Result \\
 & \begin{minipage}[t]{15cm}{\footnotesize
A set of images in each band overlapping the images from step 1, but
with the bright star far outside the field of view.

\medskip }
\end{minipage} \\ \cdashline{2-2}

 & Actual Result \\
 & \begin{minipage}[t]{15cm}{\footnotesize

\medskip }
\end{minipage} \\ \cdashline{2-2}

 & Status: \textbf{ Not Executed } \\ \hline

3 & Description \\
 & \begin{minipage}[t]{15cm}
{\footnotesize
Perform difference imaging on the overlap region between the images in
step 1 and step 2.

\medskip }
\end{minipage}
\\ \cdashline{2-2}

 & Test Data \\
 & \begin{minipage}[t]{15cm}{\footnotesize
Images from steps 1 and 2

\medskip }
\end{minipage} \\ \cdashline{2-2}

 & Expected Result \\
 & \begin{minipage}[t]{15cm}{\footnotesize
A set of difference images

\medskip }
\end{minipage} \\ \cdashline{2-2}

 & Actual Result \\
 & \begin{minipage}[t]{15cm}{\footnotesize

\medskip }
\end{minipage} \\ \cdashline{2-2}

 & Status: \textbf{ Not Executed } \\ \hline

4 & Description \\
 & \begin{minipage}[t]{15cm}
{\footnotesize
Search differenced images for ghosts that exceed 1/3 of sky noise on 1
arcsecond scales. ~Calculate percentage of image area affected by these
ghosts.

\medskip }
\end{minipage}
\\ \cdashline{2-2}

 & Test Data \\
 & \begin{minipage}[t]{15cm}{\footnotesize
Difference images from step 3.

\medskip }
\end{minipage} \\ \cdashline{2-2}

 & Expected Result \\
 & \begin{minipage}[t]{15cm}{\footnotesize
No more than 10\% of the image area is affected by ghosts that exceed
1/3 of sky noise on 1 arsecond scales.

\medskip }
\end{minipage} \\ \cdashline{2-2}

 & Actual Result \\
 & \begin{minipage}[t]{15cm}{\footnotesize

\medskip }
\end{minipage} \\ \cdashline{2-2}

 & Status: \textbf{ Not Executed } \\ \hline

\end{longtable}

\paragraph{Test Case LVV-T957 - Ghost effect on photometric repeatability }\mbox{}\\

Open  \href{https://jira.lsstcorp.org/secure/Tests.jspa#/testCase/LVV-T957}{\textit{ LVV-T957 } }
test case in Jira.

Verify that ghosting does not unduly effect our photometric
repeatability

\textbf{ Preconditions}:\\


Execution status: {\bf Not Executed }

Final comment:\\


Detailed steps results:

\begin{longtable}{p{1cm}p{15cm}}
\hline
{Step} & Step Details\\ \hline
1 & Description \\
 & \begin{minipage}[t]{15cm}
{\footnotesize
Image a field of view with a bright star (magnitude=4 ?) in each of the
six bands.

\medskip }
\end{minipage}
\\ \cdashline{2-2}


 & Expected Result \\
 & \begin{minipage}[t]{15cm}{\footnotesize
A set of images in each band containing a bright star.

\medskip }
\end{minipage} \\ \cdashline{2-2}

 & Actual Result \\
 & \begin{minipage}[t]{15cm}{\footnotesize

\medskip }
\end{minipage} \\ \cdashline{2-2}

 & Status: \textbf{ Not Executed } \\ \hline

2 & Description \\
 & \begin{minipage}[t]{15cm}
{\footnotesize
Dither the telescope pointing so that the bright star is far off of the
field of view (so that we no longer expect it to produce ghosts).
~Re-image the dithered field in all six bands.

\medskip }
\end{minipage}
\\ \cdashline{2-2}


 & Expected Result \\
 & \begin{minipage}[t]{15cm}{\footnotesize
A set of images in each band overlapping the images from step 1, but
with the bright star far outside the field of view.

\medskip }
\end{minipage} \\ \cdashline{2-2}

 & Actual Result \\
 & \begin{minipage}[t]{15cm}{\footnotesize

\medskip }
\end{minipage} \\ \cdashline{2-2}

 & Status: \textbf{ Not Executed } \\ \hline

3 & Description \\
 & \begin{minipage}[t]{15cm}
{\footnotesize
Perform difference imaging on the overlap region between the images in
step 1 and step 2.

\medskip }
\end{minipage}
\\ \cdashline{2-2}

 & Test Data \\
 & \begin{minipage}[t]{15cm}{\footnotesize
Images from steps 1 and 2

\medskip }
\end{minipage} \\ \cdashline{2-2}

 & Expected Result \\
 & \begin{minipage}[t]{15cm}{\footnotesize
A set of difference images

\medskip }
\end{minipage} \\ \cdashline{2-2}

 & Actual Result \\
 & \begin{minipage}[t]{15cm}{\footnotesize

\medskip }
\end{minipage} \\ \cdashline{2-2}

 & Status: \textbf{ Not Executed } \\ \hline

4 & Description \\
 & \begin{minipage}[t]{15cm}
{\footnotesize
Search differenced images for ghosts that exceed 1/3 of sky noise on 1
arcsecond scales. ~Calculate percentage of image area affected by these
ghosts.

\medskip }
\end{minipage}
\\ \cdashline{2-2}

 & Test Data \\
 & \begin{minipage}[t]{15cm}{\footnotesize
Difference images from step 3.

\medskip }
\end{minipage} \\ \cdashline{2-2}

 & Expected Result \\
 & \begin{minipage}[t]{15cm}{\footnotesize
No more than 10\% of the image area is affected by ghosts that exceed
1/3 of sky noise on 1 arsecond scales.

\medskip }
\end{minipage} \\ \cdashline{2-2}

 & Actual Result \\
 & \begin{minipage}[t]{15cm}{\footnotesize

\medskip }
\end{minipage} \\ \cdashline{2-2}

 & Status: \textbf{ Not Executed } \\ \hline

5 & Description \\
 & \begin{minipage}[t]{15cm}
{\footnotesize
Identify regions containing ghosts from the bright star in the first set
of images, but which hopefully do not contain ghosts in the second set
of images.

\medskip }
\end{minipage}
\\ \cdashline{2-2}


 & Expected Result \\
 & \begin{minipage}[t]{15cm}{\footnotesize

\medskip }
\end{minipage} \\ \cdashline{2-2}

 & Actual Result \\
 & \begin{minipage}[t]{15cm}{\footnotesize

\medskip }
\end{minipage} \\ \cdashline{2-2}

 & Status: \textbf{ Not Executed } \\ \hline

6 & Description \\
 & \begin{minipage}[t]{15cm}
{\footnotesize
Perform photometric measurement in both the images with and without the
bright star in the regions affected by ghosts identified in step 2.

\medskip }
\end{minipage}
\\ \cdashline{2-2}

 & Test Data \\
 & \begin{minipage}[t]{15cm}{\footnotesize
Ghost-affected regions identified in step 2

\medskip }
\end{minipage} \\ \cdashline{2-2}

 & Expected Result \\
 & \begin{minipage}[t]{15cm}{\footnotesize
Photometric measurements of sources in the presence and absence of
ghosts.

\medskip }
\end{minipage} \\ \cdashline{2-2}

 & Actual Result \\
 & \begin{minipage}[t]{15cm}{\footnotesize

\medskip }
\end{minipage} \\ \cdashline{2-2}

 & Status: \textbf{ Not Executed } \\ \hline

7 & Description \\
 & \begin{minipage}[t]{15cm}
{\footnotesize
Verify that the repeatability of photometric measurements on theses
sources has not degraded by more than 10\% due to the presence of
ghosts.

\medskip }
\end{minipage}
\\ \cdashline{2-2}

 & Test Data \\
 & \begin{minipage}[t]{15cm}{\footnotesize
Photometric measurements from step 3

\medskip }
\end{minipage} \\ \cdashline{2-2}

 & Expected Result \\
 & \begin{minipage}[t]{15cm}{\footnotesize

\medskip }
\end{minipage} \\ \cdashline{2-2}

 & Actual Result \\
 & \begin{minipage}[t]{15cm}{\footnotesize

\medskip }
\end{minipage} \\ \cdashline{2-2}

 & Status: \textbf{ Not Executed } \\ \hline

\end{longtable}

\paragraph{Test Case LVV-T595 - PSF ellipticity }\mbox{}\\

Open  \href{https://jira.lsstcorp.org/secure/Tests.jspa#/testCase/LVV-T595}{\textit{ LVV-T595 } }
test case in Jira.



\textbf{ Preconditions}:\\


Execution status: {\bf Not Executed }

Final comment:\\


Detailed steps results:

\begin{longtable}{p{1cm}p{15cm}}
\hline
{Step} & Step Details\\ \hline
1 & Description \\
 & \begin{minipage}[t]{15cm}
{\footnotesize
Take a collection of images in all six filters at a diverse range of
airmasses and atmospheric seeing condtions in uncrowded fields.

\medskip }
\end{minipage}
\\ \cdashline{2-2}


 & Expected Result \\
 & \begin{minipage}[t]{15cm}{\footnotesize
Collection of images

\medskip }
\end{minipage} \\ \cdashline{2-2}

 & Actual Result \\
 & \begin{minipage}[t]{15cm}{\footnotesize

\medskip }
\end{minipage} \\ \cdashline{2-2}

 & Status: \textbf{ Not Executed } \\ \hline

2 & Description \\
 & \begin{minipage}[t]{15cm}
{\footnotesize
Perform single image processing on the images from step 1.

\medskip }
\end{minipage}
\\ \cdashline{2-2}

 & Test Data \\
 & \begin{minipage}[t]{15cm}{\footnotesize
Images from step 1.

\medskip }
\end{minipage} \\ \cdashline{2-2}

 & Expected Result \\
 & \begin{minipage}[t]{15cm}{\footnotesize
Catalog of detected sources.

\medskip }
\end{minipage} \\ \cdashline{2-2}

 & Actual Result \\
 & \begin{minipage}[t]{15cm}{\footnotesize

\medskip }
\end{minipage} \\ \cdashline{2-2}

 & Status: \textbf{ Not Executed } \\ \hline

3 & Description \\
 & \begin{minipage}[t]{15cm}
{\footnotesize
For each full-focal plane exposure, select all of the measured,
unresolved point sources brighter than some threshold (17th magnitude?).
~Calculate the ellipticity of the PSF measured at each of these sources.

\medskip }
\end{minipage}
\\ \cdashline{2-2}

 & Test Data \\
 & \begin{minipage}[t]{15cm}{\footnotesize
Catalog of measured sources from step 2.

\medskip }
\end{minipage} \\ \cdashline{2-2}

 & Expected Result \\
 & \begin{minipage}[t]{15cm}{\footnotesize
Distribution of PSF ellipticities in the images.

\medskip }
\end{minipage} \\ \cdashline{2-2}

 & Actual Result \\
 & \begin{minipage}[t]{15cm}{\footnotesize

\medskip }
\end{minipage} \\ \cdashline{2-2}

 & Status: \textbf{ Not Executed } \\ \hline

4 & Description \\
 & \begin{minipage}[t]{15cm}
{\footnotesize
For each full-focal plane exposure, verify that the median PSF
ellipticity of the sources from step 3 is less than or equal to
0.04.\\[2\baselineskip]Verify that no more than 5\% of the sources from
step 3 have PSF ellipticity greater than 0.07.

\medskip }
\end{minipage}
\\ \cdashline{2-2}

 & Test Data \\
 & \begin{minipage}[t]{15cm}{\footnotesize
PSF ellipticity distributions from step 3.

\medskip }
\end{minipage} \\ \cdashline{2-2}

 & Expected Result \\
 & \begin{minipage}[t]{15cm}{\footnotesize

\medskip }
\end{minipage} \\ \cdashline{2-2}

 & Actual Result \\
 & \begin{minipage}[t]{15cm}{\footnotesize

\medskip }
\end{minipage} \\ \cdashline{2-2}

 & Status: \textbf{ Not Executed } \\ \hline

\end{longtable}

\paragraph{Test Case LVV-T594 - Image quality - degradation from zenith }\mbox{}\\

Open  \href{https://jira.lsstcorp.org/secure/Tests.jspa#/testCase/LVV-T594}{\textit{ LVV-T594 } }
test case in Jira.



\textbf{ Preconditions}:\\


Execution status: {\bf Not Executed }

Final comment:\\


Detailed steps results:

\begin{longtable}{p{1cm}p{15cm}}
\hline
{Step} & Step Details\\ \hline
1 & Description \\
 & \begin{minipage}[t]{15cm}
{\footnotesize
Take images at zenith, airmass=1.4, and airmass=2.0. ~Be sure to get
complete sets of images in all six filters at the three specified
airmasses, changing the airmass rapidly enough that observing conditions
do not change.

\medskip }
\end{minipage}
\\ \cdashline{2-2}


 & Expected Result \\
 & \begin{minipage}[t]{15cm}{\footnotesize
In all six filters, collections of images at airmass=1, 1.4, 2 under
identical (or very similar) observing conditions.

\medskip }
\end{minipage} \\ \cdashline{2-2}

 & Actual Result \\
 & \begin{minipage}[t]{15cm}{\footnotesize

\medskip }
\end{minipage} \\ \cdashline{2-2}

 & Status: \textbf{ Not Executed } \\ \hline

2 & Description \\
 & \begin{minipage}[t]{15cm}
{\footnotesize
Use the DIMM to measure observing conditions under which the images in
step 1 were taken.

\medskip }
\end{minipage}
\\ \cdashline{2-2}


 & Expected Result \\
 & \begin{minipage}[t]{15cm}{\footnotesize
Observing conditions for images in step 1

\medskip }
\end{minipage} \\ \cdashline{2-2}

 & Actual Result \\
 & \begin{minipage}[t]{15cm}{\footnotesize

\medskip }
\end{minipage} \\ \cdashline{2-2}

 & Status: \textbf{ Not Executed } \\ \hline

3 & Description \\
 & \begin{minipage}[t]{15cm}
{\footnotesize
Calculate the theoretical PSF size for each of the exposures in step 1
given the observing conditions measured in step 2.

\medskip }
\end{minipage}
\\ \cdashline{2-2}

 & Test Data \\
 & \begin{minipage}[t]{15cm}{\footnotesize
Metadata of images from step 1.\\
DIMM measurements from step 2.

\medskip }
\end{minipage} \\ \cdashline{2-2}

 & Expected Result \\
 & \begin{minipage}[t]{15cm}{\footnotesize
Theoretical model of PSF sizes for the images in step 1.

\medskip }
\end{minipage} \\ \cdashline{2-2}

 & Actual Result \\
 & \begin{minipage}[t]{15cm}{\footnotesize

\medskip }
\end{minipage} \\ \cdashline{2-2}

 & Status: \textbf{ Not Executed } \\ \hline

4 & Description \\
 & \begin{minipage}[t]{15cm}
{\footnotesize
Perform single image processing on the images in step 1.

\medskip }
\end{minipage}
\\ \cdashline{2-2}

 & Test Data \\
 & \begin{minipage}[t]{15cm}{\footnotesize
Images from step 1

\medskip }
\end{minipage} \\ \cdashline{2-2}

 & Expected Result \\
 & \begin{minipage}[t]{15cm}{\footnotesize
Catalog of detected sources in step 1.

\medskip }
\end{minipage} \\ \cdashline{2-2}

 & Actual Result \\
 & \begin{minipage}[t]{15cm}{\footnotesize

\medskip }
\end{minipage} \\ \cdashline{2-2}

 & Status: \textbf{ Not Executed } \\ \hline

5 & Description \\
 & \begin{minipage}[t]{15cm}
{\footnotesize
Subtract (in quadrature) the theoretical PSF sizes from the PSF sizes
measured in step 4.

\medskip }
\end{minipage}
\\ \cdashline{2-2}

 & Test Data \\
 & \begin{minipage}[t]{15cm}{\footnotesize
Theoretical PSF model from step 4\\
Catalog of measured sources in step 3

\medskip }
\end{minipage} \\ \cdashline{2-2}

 & Expected Result \\
 & \begin{minipage}[t]{15cm}{\footnotesize
Residual PSF size

\medskip }
\end{minipage} \\ \cdashline{2-2}

 & Actual Result \\
 & \begin{minipage}[t]{15cm}{\footnotesize

\medskip }
\end{minipage} \\ \cdashline{2-2}

 & Status: \textbf{ Not Executed } \\ \hline

6 & Description \\
 & \begin{minipage}[t]{15cm}
{\footnotesize
Verify that the residual PSF size at each airmass does not exceed the
following limits:\\[2\baselineskip]0.4 arcseconds at airmass=1\\
0.49 arcseconds at airmass=1.4\\
0.6 arcseconds at airmass=2\\[2\baselineskip]so that the system
contribution to PSF width degrades no more rapidly than airmass\^{}0.6

\medskip }
\end{minipage}
\\ \cdashline{2-2}


 & Expected Result \\
 & \begin{minipage}[t]{15cm}{\footnotesize

\medskip }
\end{minipage} \\ \cdashline{2-2}

 & Actual Result \\
 & \begin{minipage}[t]{15cm}{\footnotesize

\medskip }
\end{minipage} \\ \cdashline{2-2}

 & Status: \textbf{ Not Executed } \\ \hline

\end{longtable}

\paragraph{Test Case LVV-T593 - Image quality at zenith }\mbox{}\\

Open  \href{https://jira.lsstcorp.org/secure/Tests.jspa#/testCase/LVV-T593}{\textit{ LVV-T593 } }
test case in Jira.



\textbf{ Preconditions}:\\


Execution status: {\bf Not Executed }

Final comment:\\


Detailed steps results:

\begin{longtable}{p{1cm}p{15cm}}
\hline
{Step} & Step Details\\ \hline
1 & Description \\
 & \begin{minipage}[t]{15cm}
{\footnotesize
Take a series of images at zenith in all six filters.

\medskip }
\end{minipage}
\\ \cdashline{2-2}


 & Expected Result \\
 & \begin{minipage}[t]{15cm}{\footnotesize
Set of images

\medskip }
\end{minipage} \\ \cdashline{2-2}

 & Actual Result \\
 & \begin{minipage}[t]{15cm}{\footnotesize

\medskip }
\end{minipage} \\ \cdashline{2-2}

 & Status: \textbf{ Not Executed } \\ \hline

2 & Description \\
 & \begin{minipage}[t]{15cm}
{\footnotesize
Calculate theoretical PSF size for the images in step 1 based on
observing conditions.

\medskip }
\end{minipage}
\\ \cdashline{2-2}

 & Test Data \\
 & \begin{minipage}[t]{15cm}{\footnotesize
Metadata from images in step 1

\medskip }
\end{minipage} \\ \cdashline{2-2}

 & Expected Result \\
 & \begin{minipage}[t]{15cm}{\footnotesize
Theoretical model of PSF size

\medskip }
\end{minipage} \\ \cdashline{2-2}

 & Actual Result \\
 & \begin{minipage}[t]{15cm}{\footnotesize

\medskip }
\end{minipage} \\ \cdashline{2-2}

 & Status: \textbf{ Not Executed } \\ \hline

3 & Description \\
 & \begin{minipage}[t]{15cm}
{\footnotesize
Perform single image processing on images from step 1

\medskip }
\end{minipage}
\\ \cdashline{2-2}

 & Test Data \\
 & \begin{minipage}[t]{15cm}{\footnotesize
Images from step 1

\medskip }
\end{minipage} \\ \cdashline{2-2}

 & Expected Result \\
 & \begin{minipage}[t]{15cm}{\footnotesize
Catalog of detected sources

\medskip }
\end{minipage} \\ \cdashline{2-2}

 & Actual Result \\
 & \begin{minipage}[t]{15cm}{\footnotesize

\medskip }
\end{minipage} \\ \cdashline{2-2}

 & Status: \textbf{ Not Executed } \\ \hline

4 & Description \\
 & \begin{minipage}[t]{15cm}
{\footnotesize
Subtract (in quadrature) theoretical PSF model from step 2 from measured
PSF sizes in catalog from step 3

\medskip }
\end{minipage}
\\ \cdashline{2-2}

 & Test Data \\
 & \begin{minipage}[t]{15cm}{\footnotesize
Detected source catalog from step 3\\
PSF size model from step 2

\medskip }
\end{minipage} \\ \cdashline{2-2}

 & Expected Result \\
 & \begin{minipage}[t]{15cm}{\footnotesize
Residual PSF size

\medskip }
\end{minipage} \\ \cdashline{2-2}

 & Actual Result \\
 & \begin{minipage}[t]{15cm}{\footnotesize

\medskip }
\end{minipage} \\ \cdashline{2-2}

 & Status: \textbf{ Not Executed } \\ \hline

5 & Description \\
 & \begin{minipage}[t]{15cm}
{\footnotesize
Verify that residual PSF size does not exceed 0.4 arcseconds

\medskip }
\end{minipage}
\\ \cdashline{2-2}


 & Expected Result \\
 & \begin{minipage}[t]{15cm}{\footnotesize

\medskip }
\end{minipage} \\ \cdashline{2-2}

 & Actual Result \\
 & \begin{minipage}[t]{15cm}{\footnotesize

\medskip }
\end{minipage} \\ \cdashline{2-2}

 & Status: \textbf{ Not Executed } \\ \hline

\end{longtable}

\paragraph{Test Case LVV-T592 - Image quality - maximum system contribution }\mbox{}\\

Open  \href{https://jira.lsstcorp.org/secure/Tests.jspa#/testCase/LVV-T592}{\textit{ LVV-T592 } }
test case in Jira.



\textbf{ Preconditions}:\\


Execution status: {\bf Not Executed }

Final comment:\\


Detailed steps results:

\begin{longtable}{p{1cm}p{15cm}}
\hline
{Step} & Step Details\\ \hline
1 & Description \\
 & \begin{minipage}[t]{15cm}
{\footnotesize
Use DIMM to select observations taken at 0.6 arcsecond seeing.

\medskip }
\end{minipage}
\\ \cdashline{2-2}


 & Expected Result \\
 & \begin{minipage}[t]{15cm}{\footnotesize
Set of images

\medskip }
\end{minipage} \\ \cdashline{2-2}

 & Actual Result \\
 & \begin{minipage}[t]{15cm}{\footnotesize

\medskip }
\end{minipage} \\ \cdashline{2-2}

 & Status: \textbf{ Not Executed } \\ \hline

2 & Description \\
 & \begin{minipage}[t]{15cm}
{\footnotesize
Calculate theoretical PSF size for images in step 1 given observing
conditions.

\medskip }
\end{minipage}
\\ \cdashline{2-2}

 & Test Data \\
 & \begin{minipage}[t]{15cm}{\footnotesize
Metadata from images in step 1

\medskip }
\end{minipage} \\ \cdashline{2-2}

 & Expected Result \\
 & \begin{minipage}[t]{15cm}{\footnotesize
Theoretical model of PSF sizes

\medskip }
\end{minipage} \\ \cdashline{2-2}

 & Actual Result \\
 & \begin{minipage}[t]{15cm}{\footnotesize

\medskip }
\end{minipage} \\ \cdashline{2-2}

 & Status: \textbf{ Not Executed } \\ \hline

3 & Description \\
 & \begin{minipage}[t]{15cm}
{\footnotesize
Perform single image processing on images from step 1.

\medskip }
\end{minipage}
\\ \cdashline{2-2}

 & Test Data \\
 & \begin{minipage}[t]{15cm}{\footnotesize
Images from step 1

\medskip }
\end{minipage} \\ \cdashline{2-2}

 & Expected Result \\
 & \begin{minipage}[t]{15cm}{\footnotesize
Catalog of detected sources

\medskip }
\end{minipage} \\ \cdashline{2-2}

 & Actual Result \\
 & \begin{minipage}[t]{15cm}{\footnotesize

\medskip }
\end{minipage} \\ \cdashline{2-2}

 & Status: \textbf{ Not Executed } \\ \hline

4 & Description \\
 & \begin{minipage}[t]{15cm}
{\footnotesize
Subtract (in quadrature) theoretical model from step 2 from PSF sizes of
sources in catalogs from step 3

\medskip }
\end{minipage}
\\ \cdashline{2-2}

 & Test Data \\
 & \begin{minipage}[t]{15cm}{\footnotesize
Catalog of sources from step 3\\
Theoretical PSF model from step 2

\medskip }
\end{minipage} \\ \cdashline{2-2}

 & Expected Result \\
 & \begin{minipage}[t]{15cm}{\footnotesize
Residual PSF sizes

\medskip }
\end{minipage} \\ \cdashline{2-2}

 & Actual Result \\
 & \begin{minipage}[t]{15cm}{\footnotesize

\medskip }
\end{minipage} \\ \cdashline{2-2}

 & Status: \textbf{ Not Executed } \\ \hline

5 & Description \\
 & \begin{minipage}[t]{15cm}
{\footnotesize
Verify that residual PSF size, which will have been contributed by the
LSST system, is no more than 15\% of total PSF size.

\medskip }
\end{minipage}
\\ \cdashline{2-2}


 & Expected Result \\
 & \begin{minipage}[t]{15cm}{\footnotesize

\medskip }
\end{minipage} \\ \cdashline{2-2}

 & Actual Result \\
 & \begin{minipage}[t]{15cm}{\footnotesize

\medskip }
\end{minipage} \\ \cdashline{2-2}

 & Status: \textbf{ Not Executed } \\ \hline

\end{longtable}

\paragraph{Test Case LVV-T591 - Flux-enclosing radius }\mbox{}\\

Open  \href{https://jira.lsstcorp.org/secure/Tests.jspa#/testCase/LVV-T591}{\textit{ LVV-T591 } }
test case in Jira.



\textbf{ Preconditions}:\\


Execution status: {\bf Not Executed }

Final comment:\\


Detailed steps results:

\begin{longtable}{p{1cm}p{15cm}}
\hline
{Step} & Step Details\\ \hline
1 & Description \\
 & \begin{minipage}[t]{15cm}
{\footnotesize
Select pointings in sparse fields (so that we don't have to worry about
more than one object getting encircled by the test apertures below)

\medskip }
\end{minipage}
\\ \cdashline{2-2}


 & Expected Result \\
 & \begin{minipage}[t]{15cm}{\footnotesize

\medskip }
\end{minipage} \\ \cdashline{2-2}

 & Actual Result \\
 & \begin{minipage}[t]{15cm}{\footnotesize

\medskip }
\end{minipage} \\ \cdashline{2-2}

 & Status: \textbf{ Not Executed } \\ \hline

2 & Description \\
 & \begin{minipage}[t]{15cm}
{\footnotesize
Run single image processing on the exposures from step 1 to identify all
of the sources.

\medskip }
\end{minipage}
\\ \cdashline{2-2}

 & Test Data \\
 & \begin{minipage}[t]{15cm}{\footnotesize
Exposures from step 1

\medskip }
\end{minipage} \\ \cdashline{2-2}

 & Expected Result \\
 & \begin{minipage}[t]{15cm}{\footnotesize
Source catalog

\medskip }
\end{minipage} \\ \cdashline{2-2}

 & Actual Result \\
 & \begin{minipage}[t]{15cm}{\footnotesize

\medskip }
\end{minipage} \\ \cdashline{2-2}

 & Status: \textbf{ Not Executed } \\ \hline

3 & Description \\
 & \begin{minipage}[t]{15cm}
{\footnotesize
Run forced photometry on all detected unresolved point sources,
measuring fluxes inside of a 2 arcescond (or some other unseemly large)
aperture.

\medskip }
\end{minipage}
\\ \cdashline{2-2}

 & Test Data \\
 & \begin{minipage}[t]{15cm}{\footnotesize
Source catalog from step 2

\medskip }
\end{minipage} \\ \cdashline{2-2}

 & Expected Result \\
 & \begin{minipage}[t]{15cm}{\footnotesize
Catalog of source fluxes in a large aperture

\medskip }
\end{minipage} \\ \cdashline{2-2}

 & Actual Result \\
 & \begin{minipage}[t]{15cm}{\footnotesize

\medskip }
\end{minipage} \\ \cdashline{2-2}

 & Status: \textbf{ Not Executed } \\ \hline

4 & Description \\
 & \begin{minipage}[t]{15cm}
{\footnotesize
Re-run forced photometry on sources from step 3, reducing the aperture
to 1.81 arcsecond, 1.31 arcsecond, and 0.8 arcsecond. ~Verify that three
measurements contain at least 95\%, 90\%, and 80\% of the flux for all
sources.

\medskip }
\end{minipage}
\\ \cdashline{2-2}

 & Test Data \\
 & \begin{minipage}[t]{15cm}{\footnotesize
Flux catalogs from step 3\\
Sources from step 2

\medskip }
\end{minipage} \\ \cdashline{2-2}

 & Expected Result \\
 & \begin{minipage}[t]{15cm}{\footnotesize

\medskip }
\end{minipage} \\ \cdashline{2-2}

 & Actual Result \\
 & \begin{minipage}[t]{15cm}{\footnotesize

\medskip }
\end{minipage} \\ \cdashline{2-2}

 & Status: \textbf{ Not Executed } \\ \hline

\end{longtable}

\paragraph{Test Case LVV-T590 - Median image quality at 0.8 arcsecond seeing }\mbox{}\\

Open  \href{https://jira.lsstcorp.org/secure/Tests.jspa#/testCase/LVV-T590}{\textit{ LVV-T590 } }
test case in Jira.



\textbf{ Preconditions}:\\


Execution status: {\bf Not Executed }

Final comment:\\


Detailed steps results:

\begin{longtable}{p{1cm}p{15cm}}
\hline
{Step} & Step Details\\ \hline
1 & Description \\
 & \begin{minipage}[t]{15cm}
{\footnotesize
Use DIMM to select exposures taken at 0.8 arcsecond seeing in the r and
i bands

\medskip }
\end{minipage}
\\ \cdashline{2-2}


 & Expected Result \\
 & \begin{minipage}[t]{15cm}{\footnotesize

\medskip }
\end{minipage} \\ \cdashline{2-2}

 & Actual Result \\
 & \begin{minipage}[t]{15cm}{\footnotesize

\medskip }
\end{minipage} \\ \cdashline{2-2}

 & Status: \textbf{ Not Executed } \\ \hline

2 & Description \\
 & \begin{minipage}[t]{15cm}
{\footnotesize
Run single image processing on the exposures from step 1

\medskip }
\end{minipage}
\\ \cdashline{2-2}

 & Test Data \\
 & \begin{minipage}[t]{15cm}{\footnotesize
Exposures from step 1

\medskip }
\end{minipage} \\ \cdashline{2-2}

 & Expected Result \\
 & \begin{minipage}[t]{15cm}{\footnotesize
Source catalog

\medskip }
\end{minipage} \\ \cdashline{2-2}

 & Actual Result \\
 & \begin{minipage}[t]{15cm}{\footnotesize

\medskip }
\end{minipage} \\ \cdashline{2-2}

 & Status: \textbf{ Not Executed } \\ \hline

3 & Description \\
 & \begin{minipage}[t]{15cm}
{\footnotesize
Verify that median of PSF FWHM is approximately 0.89 arcsecond

\medskip }
\end{minipage}
\\ \cdashline{2-2}


 & Expected Result \\
 & \begin{minipage}[t]{15cm}{\footnotesize

\medskip }
\end{minipage} \\ \cdashline{2-2}

 & Actual Result \\
 & \begin{minipage}[t]{15cm}{\footnotesize

\medskip }
\end{minipage} \\ \cdashline{2-2}

 & Status: \textbf{ Not Executed } \\ \hline

\end{longtable}

\paragraph{Test Case LVV-T589 - Median image quality at 0.6 arcsecond seeing }\mbox{}\\

Open  \href{https://jira.lsstcorp.org/secure/Tests.jspa#/testCase/LVV-T589}{\textit{ LVV-T589 } }
test case in Jira.



\textbf{ Preconditions}:\\


Execution status: {\bf Not Executed }

Final comment:\\


Detailed steps results:

\begin{longtable}{p{1cm}p{15cm}}
\hline
{Step} & Step Details\\ \hline
1 & Description \\
 & \begin{minipage}[t]{15cm}
{\footnotesize
Use DIMM to select pointings taken at 0.6 arcsecond seeing in the r and
i bands

\medskip }
\end{minipage}
\\ \cdashline{2-2}


 & Expected Result \\
 & \begin{minipage}[t]{15cm}{\footnotesize

\medskip }
\end{minipage} \\ \cdashline{2-2}

 & Actual Result \\
 & \begin{minipage}[t]{15cm}{\footnotesize

\medskip }
\end{minipage} \\ \cdashline{2-2}

 & Status: \textbf{ Not Executed } \\ \hline

2 & Description \\
 & \begin{minipage}[t]{15cm}
{\footnotesize
Run single image processing on the exposures from step 1.

\medskip }
\end{minipage}
\\ \cdashline{2-2}

 & Test Data \\
 & \begin{minipage}[t]{15cm}{\footnotesize
Exposures from step 1

\medskip }
\end{minipage} \\ \cdashline{2-2}

 & Expected Result \\
 & \begin{minipage}[t]{15cm}{\footnotesize
Source catalog

\medskip }
\end{minipage} \\ \cdashline{2-2}

 & Actual Result \\
 & \begin{minipage}[t]{15cm}{\footnotesize

\medskip }
\end{minipage} \\ \cdashline{2-2}

 & Status: \textbf{ Not Executed } \\ \hline

3 & Description \\
 & \begin{minipage}[t]{15cm}
{\footnotesize
Verify that the median PSF FWHM of unresolved point sources is 0.72
arcseconds.

\medskip }
\end{minipage}
\\ \cdashline{2-2}

 & Test Data \\
 & \begin{minipage}[t]{15cm}{\footnotesize
Source catalog from step 2

\medskip }
\end{minipage} \\ \cdashline{2-2}

 & Expected Result \\
 & \begin{minipage}[t]{15cm}{\footnotesize

\medskip }
\end{minipage} \\ \cdashline{2-2}

 & Actual Result \\
 & \begin{minipage}[t]{15cm}{\footnotesize

\medskip }
\end{minipage} \\ \cdashline{2-2}

 & Status: \textbf{ Not Executed } \\ \hline

\end{longtable}

\paragraph{Test Case LVV-T588 - Median image quality at 0.44 arcsecond seeing }\mbox{}\\

Open  \href{https://jira.lsstcorp.org/secure/Tests.jspa#/testCase/LVV-T588}{\textit{ LVV-T588 } }
test case in Jira.



\textbf{ Preconditions}:\\


Execution status: {\bf Not Executed }

Final comment:\\


Detailed steps results:

\begin{longtable}{p{1cm}p{15cm}}
\hline
{Step} & Step Details\\ \hline
1 & Description \\
 & \begin{minipage}[t]{15cm}
{\footnotesize
Use DIMM to select pointings taken at 0.44 arcsecond seeing in the r and
i bands

\medskip }
\end{minipage}
\\ \cdashline{2-2}


 & Expected Result \\
 & \begin{minipage}[t]{15cm}{\footnotesize

\medskip }
\end{minipage} \\ \cdashline{2-2}

 & Actual Result \\
 & \begin{minipage}[t]{15cm}{\footnotesize

\medskip }
\end{minipage} \\ \cdashline{2-2}

 & Status: \textbf{ Not Executed } \\ \hline

2 & Description \\
 & \begin{minipage}[t]{15cm}
{\footnotesize
Run single image processing on exposures selected in step 1

\medskip }
\end{minipage}
\\ \cdashline{2-2}

 & Test Data \\
 & \begin{minipage}[t]{15cm}{\footnotesize
Exposures from step 1

\medskip }
\end{minipage} \\ \cdashline{2-2}

 & Expected Result \\
 & \begin{minipage}[t]{15cm}{\footnotesize
Source catalog

\medskip }
\end{minipage} \\ \cdashline{2-2}

 & Actual Result \\
 & \begin{minipage}[t]{15cm}{\footnotesize

\medskip }
\end{minipage} \\ \cdashline{2-2}

 & Status: \textbf{ Not Executed } \\ \hline

3 & Description \\
 & \begin{minipage}[t]{15cm}
{\footnotesize
Subtract (in quadrature) seeing from PSF FWHM of unresolved point
sources. ~Verify median of distribution of residual PSF FWHM is
approximately 0.59 arcseconds

\medskip }
\end{minipage}
\\ \cdashline{2-2}

 & Test Data \\
 & \begin{minipage}[t]{15cm}{\footnotesize
Source catalog from step 2

\medskip }
\end{minipage} \\ \cdashline{2-2}

 & Expected Result \\
 & \begin{minipage}[t]{15cm}{\footnotesize

\medskip }
\end{minipage} \\ \cdashline{2-2}

 & Actual Result \\
 & \begin{minipage}[t]{15cm}{\footnotesize

\medskip }
\end{minipage} \\ \cdashline{2-2}

 & Status: \textbf{ Not Executed } \\ \hline

\end{longtable}

\paragraph{Test Case LVV-T587 - PSF size in pixels }\mbox{}\\

Open  \href{https://jira.lsstcorp.org/secure/Tests.jspa#/testCase/LVV-T587}{\textit{ LVV-T587 } }
test case in Jira.



\textbf{ Preconditions}:\\


Execution status: {\bf Not Executed }

Final comment:\\


Detailed steps results:

\begin{longtable}{p{1cm}p{15cm}}
\hline
{Step} & Step Details\\ \hline
1 & Description \\
 & \begin{minipage}[t]{15cm}
{\footnotesize
Use DIMM data to select a set of pointings observed at 0.6 arcsecond
seeing.

\medskip }
\end{minipage}
\\ \cdashline{2-2}


 & Expected Result \\
 & \begin{minipage}[t]{15cm}{\footnotesize

\medskip }
\end{minipage} \\ \cdashline{2-2}

 & Actual Result \\
 & \begin{minipage}[t]{15cm}{\footnotesize

\medskip }
\end{minipage} \\ \cdashline{2-2}

 & Status: \textbf{ Not Executed } \\ \hline

2 & Description \\
 & \begin{minipage}[t]{15cm}
{\footnotesize
Run single image processing on pointings from step 1

\medskip }
\end{minipage}
\\ \cdashline{2-2}

 & Test Data \\
 & \begin{minipage}[t]{15cm}{\footnotesize
pointings from step 1

\medskip }
\end{minipage} \\ \cdashline{2-2}

 & Expected Result \\
 & \begin{minipage}[t]{15cm}{\footnotesize
Catalog of detected sources

\medskip }
\end{minipage} \\ \cdashline{2-2}

 & Actual Result \\
 & \begin{minipage}[t]{15cm}{\footnotesize

\medskip }
\end{minipage} \\ \cdashline{2-2}

 & Status: \textbf{ Not Executed } \\ \hline

3 & Description \\
 & \begin{minipage}[t]{15cm}
{\footnotesize
Verify that the minimum FWHM of the PSFs of unresolved point sources is
3 pixels or greater

\medskip }
\end{minipage}
\\ \cdashline{2-2}

 & Test Data \\
 & \begin{minipage}[t]{15cm}{\footnotesize
source catalog from step 2

\medskip }
\end{minipage} \\ \cdashline{2-2}

 & Expected Result \\
 & \begin{minipage}[t]{15cm}{\footnotesize

\medskip }
\end{minipage} \\ \cdashline{2-2}

 & Actual Result \\
 & \begin{minipage}[t]{15cm}{\footnotesize

\medskip }
\end{minipage} \\ \cdashline{2-2}

 & Status: \textbf{ Not Executed } \\ \hline

\end{longtable}

\paragraph{Test Case LVV-T554 - Single exposure dynamic range }\mbox{}\\

Open  \href{https://jira.lsstcorp.org/secure/Tests.jspa#/testCase/LVV-T554}{\textit{ LVV-T554 } }
test case in Jira.

Verify that objects within the specified dynamic range of a single image
are not saturated

\textbf{ Preconditions}:\\


Execution status: {\bf Not Executed }

Final comment:\\


Detailed steps results:

\begin{longtable}{p{1cm}p{15cm}}
\hline
{Step} & Step Details\\ \hline
1 & Description \\
 & \begin{minipage}[t]{15cm}
{\footnotesize
Find images observed at:\\
airmass = 1.0\\
r-band skybrightness = 21 magnitude/arcsec\^{}2\\
r-band seeing = 0.7 arcsec

\medskip }
\end{minipage}
\\ \cdashline{2-2}

 & Test Data \\
 & \begin{minipage}[t]{15cm}{\footnotesize
Exposures from a mini survey

\medskip }
\end{minipage} \\ \cdashline{2-2}

 & Expected Result \\
 & \begin{minipage}[t]{15cm}{\footnotesize
Set of images to test

\medskip }
\end{minipage} \\ \cdashline{2-2}

 & Actual Result \\
 & \begin{minipage}[t]{15cm}{\footnotesize

\medskip }
\end{minipage} \\ \cdashline{2-2}

 & Status: \textbf{ Not Executed } \\ \hline

2 & Description \\
 & \begin{minipage}[t]{15cm}
{\footnotesize
Run single-exposure processing on the images from step 1.

\medskip }
\end{minipage}
\\ \cdashline{2-2}

 & Test Data \\
 & \begin{minipage}[t]{15cm}{\footnotesize
Set of images identified at fiducial conditions.

\medskip }
\end{minipage} \\ \cdashline{2-2}

 & Expected Result \\
 & \begin{minipage}[t]{15cm}{\footnotesize
5-sigma limiting magnitude and list of detected sources for each image.

\medskip }
\end{minipage} \\ \cdashline{2-2}

 & Actual Result \\
 & \begin{minipage}[t]{15cm}{\footnotesize

\medskip }
\end{minipage} \\ \cdashline{2-2}

 & Status: \textbf{ Not Executed } \\ \hline

3 & Description \\
 & \begin{minipage}[t]{15cm}
{\footnotesize
Check that sources within specified dynamic range are not saturated.

\medskip }
\end{minipage}
\\ \cdashline{2-2}

 & Test Data \\
 & \begin{minipage}[t]{15cm}{\footnotesize
List of detected sources from step 2

\medskip }
\end{minipage} \\ \cdashline{2-2}

 & Expected Result \\
 & \begin{minipage}[t]{15cm}{\footnotesize

\medskip }
\end{minipage} \\ \cdashline{2-2}

 & Actual Result \\
 & \begin{minipage}[t]{15cm}{\footnotesize

\medskip }
\end{minipage} \\ \cdashline{2-2}

 & Status: \textbf{ Not Executed } \\ \hline

\end{longtable}

\paragraph{Test Case LVV-T548 - Photometric errors -- level 1 processing -- reference catalog }\mbox{}\\

Open  \href{https://jira.lsstcorp.org/secure/Tests.jspa#/testCase/LVV-T548}{\textit{ LVV-T548 } }
test case in Jira.

Test DM contribution to photometric errors by comparing LSST
measurements to external catalog

\textbf{ Preconditions}:\\


Execution status: {\bf Not Executed }

Final comment:\\


Detailed steps results:

\begin{longtable}{p{1cm}p{15cm}}
\hline
{Step} & Step Details\\ \hline
1 & Description \\
 & \begin{minipage}[t]{15cm}
{\footnotesize
Identify catalog of static sources with well measured fluxes. ~We will
also need a sense for the historical RMS variation of the sources' flux.

\medskip }
\end{minipage}
\\ \cdashline{2-2}


 & Expected Result \\
 & \begin{minipage}[t]{15cm}{\footnotesize
Catalog of static sources with flux values and uncertainties

\medskip }
\end{minipage} \\ \cdashline{2-2}

 & Actual Result \\
 & \begin{minipage}[t]{15cm}{\footnotesize

\medskip }
\end{minipage} \\ \cdashline{2-2}

 & Status: \textbf{ Not Executed } \\ \hline

2 & Description \\
 & \begin{minipage}[t]{15cm}
{\footnotesize
Image region of sky overlapping the catalog in step 1 at varying
observing conditions (airmass, seeing, etc.)

\medskip }
\end{minipage}
\\ \cdashline{2-2}


 & Expected Result \\
 & \begin{minipage}[t]{15cm}{\footnotesize
Images overlapping catalog from step 1

\medskip }
\end{minipage} \\ \cdashline{2-2}

 & Actual Result \\
 & \begin{minipage}[t]{15cm}{\footnotesize

\medskip }
\end{minipage} \\ \cdashline{2-2}

 & Status: \textbf{ Not Executed } \\ \hline

3 & Description \\
 & \begin{minipage}[t]{15cm}
{\footnotesize
Perform level 1 processing on images from step 2. ~Keep difference
images.

\medskip }
\end{minipage}
\\ \cdashline{2-2}

 & Test Data \\
 & \begin{minipage}[t]{15cm}{\footnotesize
Images from step 2

\medskip }
\end{minipage} \\ \cdashline{2-2}

 & Expected Result \\
 & \begin{minipage}[t]{15cm}{\footnotesize
Catalog of DIASources\\
Difference images

\medskip }
\end{minipage} \\ \cdashline{2-2}

 & Actual Result \\
 & \begin{minipage}[t]{15cm}{\footnotesize

\medskip }
\end{minipage} \\ \cdashline{2-2}

 & Status: \textbf{ Not Executed } \\ \hline

4 & Description \\
 & \begin{minipage}[t]{15cm}
{\footnotesize
Perform forced photometry on difference images from step 3 at location
of sources identified in the catalog from step 1.

\medskip }
\end{minipage}
\\ \cdashline{2-2}

 & Test Data \\
 & \begin{minipage}[t]{15cm}{\footnotesize
Difference images from step 3.

\medskip }
\end{minipage} \\ \cdashline{2-2}

 & Expected Result \\
 & \begin{minipage}[t]{15cm}{\footnotesize
Catalog of forced difference image photometry measurements.

\medskip }
\end{minipage} \\ \cdashline{2-2}

 & Actual Result \\
 & \begin{minipage}[t]{15cm}{\footnotesize

\medskip }
\end{minipage} \\ \cdashline{2-2}

 & Status: \textbf{ Not Executed } \\ \hline

5 & Description \\
 & \begin{minipage}[t]{15cm}
{\footnotesize
Construct model of photometric uncertainty based only on observing
conditions of images in step 1.

\medskip }
\end{minipage}
\\ \cdashline{2-2}

 & Test Data \\
 & \begin{minipage}[t]{15cm}{\footnotesize
Metadata from images in step 1

\medskip }
\end{minipage} \\ \cdashline{2-2}

 & Expected Result \\
 & \begin{minipage}[t]{15cm}{\footnotesize
Model of photometric uncertainty expected from observing conditions

\medskip }
\end{minipage} \\ \cdashline{2-2}

 & Actual Result \\
 & \begin{minipage}[t]{15cm}{\footnotesize

\medskip }
\end{minipage} \\ \cdashline{2-2}

 & Status: \textbf{ Not Executed } \\ \hline

6 & Description \\
 & \begin{minipage}[t]{15cm}
{\footnotesize
Compare force difference image photometry to intrinsic width of source
photometry measurements in step 1 and model of uncertainty from step 5.
~Compare RMS residual to specified tolerance.

\medskip }
\end{minipage}
\\ \cdashline{2-2}

 & Test Data \\
 & \begin{minipage}[t]{15cm}{\footnotesize
Photometric uncertainty model from step 5\\
Intrinsic widths of photometric ~measurements from step 1\\
Forced difference image photometry from step 4

\medskip }
\end{minipage} \\ \cdashline{2-2}

 & Expected Result \\
 & \begin{minipage}[t]{15cm}{\footnotesize

\medskip }
\end{minipage} \\ \cdashline{2-2}

 & Actual Result \\
 & \begin{minipage}[t]{15cm}{\footnotesize

\medskip }
\end{minipage} \\ \cdashline{2-2}

 & Status: \textbf{ Not Executed } \\ \hline

\end{longtable}

\paragraph{Test Case LVV-T545 - Astrometric error -- level 1 processing -- reference catalog }\mbox{}\\

Open  \href{https://jira.lsstcorp.org/secure/Tests.jspa#/testCase/LVV-T545}{\textit{ LVV-T545 } }
test case in Jira.

Measure the astrometric performance requirements by comparing actual
data to a reference catalog (e.g. Gaia)

\textbf{ Preconditions}:\\


Execution status: {\bf Not Executed }

Final comment:\\


Detailed steps results:

\begin{longtable}{p{1cm}p{15cm}}
\hline
{Step} & Step Details\\ \hline
1 & Description \\
 & \begin{minipage}[t]{15cm}
{\footnotesize
Find catalog of sources with well-measured astrometry.

\medskip }
\end{minipage}
\\ \cdashline{2-2}


 & Expected Result \\
 & \begin{minipage}[t]{15cm}{\footnotesize
Catalog of sources to be used as ground truth

\medskip }
\end{minipage} \\ \cdashline{2-2}

 & Actual Result \\
 & \begin{minipage}[t]{15cm}{\footnotesize

\medskip }
\end{minipage} \\ \cdashline{2-2}

 & Status: \textbf{ Not Executed } \\ \hline

2 & Description \\
 & \begin{minipage}[t]{15cm}
{\footnotesize
Image the area of sky overlapping ground truth catalog from step 1 under
diverse observing conditions (airmass, seeing, etc.).

\medskip }
\end{minipage}
\\ \cdashline{2-2}


 & Expected Result \\
 & \begin{minipage}[t]{15cm}{\footnotesize
Images overlapping catalog from step 1

\medskip }
\end{minipage} \\ \cdashline{2-2}

 & Actual Result \\
 & \begin{minipage}[t]{15cm}{\footnotesize

\medskip }
\end{minipage} \\ \cdashline{2-2}

 & Status: \textbf{ Not Executed } \\ \hline

3 & Description \\
 & \begin{minipage}[t]{15cm}
{\footnotesize
Perform Level 1 processing on images taken in step 2.

\medskip }
\end{minipage}
\\ \cdashline{2-2}

 & Test Data \\
 & \begin{minipage}[t]{15cm}{\footnotesize
Images from step 2\\
Coadd template constructed therefrom

\medskip }
\end{minipage} \\ \cdashline{2-2}

 & Expected Result \\
 & \begin{minipage}[t]{15cm}{\footnotesize
Catalog of DIASources

\medskip }
\end{minipage} \\ \cdashline{2-2}

 & Actual Result \\
 & \begin{minipage}[t]{15cm}{\footnotesize

\medskip }
\end{minipage} \\ \cdashline{2-2}

 & Status: \textbf{ Not Executed } \\ \hline

4 & Description \\
 & \begin{minipage}[t]{15cm}
{\footnotesize
Construct a model of astrometric errors due only to observing conditions
of images in step 2.

\medskip }
\end{minipage}
\\ \cdashline{2-2}

 & Test Data \\
 & \begin{minipage}[t]{15cm}{\footnotesize
Metadata from images taken in step 2

\medskip }
\end{minipage} \\ \cdashline{2-2}

 & Expected Result \\
 & \begin{minipage}[t]{15cm}{\footnotesize
Model of astrometric errors expected from observing conditions

\medskip }
\end{minipage} \\ \cdashline{2-2}

 & Actual Result \\
 & \begin{minipage}[t]{15cm}{\footnotesize

\medskip }
\end{minipage} \\ \cdashline{2-2}

 & Status: \textbf{ Not Executed } \\ \hline

5 & Description \\
 & \begin{minipage}[t]{15cm}
{\footnotesize
Compare DIASources in step 3 to catalog from step 1 to find measured
astrometric errors.

\medskip }
\end{minipage}
\\ \cdashline{2-2}

 & Test Data \\
 & \begin{minipage}[t]{15cm}{\footnotesize
Truth catalog from step 1\\
DIASource catalog from step 3

\medskip }
\end{minipage} \\ \cdashline{2-2}

 & Expected Result \\
 & \begin{minipage}[t]{15cm}{\footnotesize
Catalog of measured astrometric errors

\medskip }
\end{minipage} \\ \cdashline{2-2}

 & Actual Result \\
 & \begin{minipage}[t]{15cm}{\footnotesize

\medskip }
\end{minipage} \\ \cdashline{2-2}

 & Status: \textbf{ Not Executed } \\ \hline

6 & Description \\
 & \begin{minipage}[t]{15cm}
{\footnotesize
Calculate RMS residual between measured astrometric errors (step 5) and
astrometric errors expected due only to observing conditions (step 4).
~Verify that residual is less than specified limit.

\medskip }
\end{minipage}
\\ \cdashline{2-2}

 & Test Data \\
 & \begin{minipage}[t]{15cm}{\footnotesize
Measured astrometric errors from step 5\\
Model of astrometric errors from step 4

\medskip }
\end{minipage} \\ \cdashline{2-2}

 & Expected Result \\
 & \begin{minipage}[t]{15cm}{\footnotesize

\medskip }
\end{minipage} \\ \cdashline{2-2}

 & Actual Result \\
 & \begin{minipage}[t]{15cm}{\footnotesize

\medskip }
\end{minipage} \\ \cdashline{2-2}

 & Status: \textbf{ Not Executed } \\ \hline

\end{longtable}

\paragraph{Test Case LVV-T389 - Single Visit Photometric Repeatability }\mbox{}\\

Open  \href{https://jira.lsstcorp.org/secure/Tests.jspa#/testCase/LVV-T389}{\textit{ LVV-T389 } }
test case in Jira.

Verify that the RMS of magnitudes in all filters and outlier rate of
magnitudes is within specification

\textbf{ Preconditions}:\\
Multi epoch observations of bright, isolated, unresolved, un-saturated
stars, observed at varying photometric conditions, air mass, and water
vapor.

Execution status: {\bf Not Executed }

Final comment:\\


Detailed steps results:

\begin{longtable}{p{1cm}p{15cm}}
\hline
{Step} & Step Details\\ \hline
1 & Description \\
 & \begin{minipage}[t]{15cm}
{\footnotesize
Define sample data of stars to be used in subsequent tests. ~Columns
needed are camera rotation angle, magnitude in all bands, RA, Dec,
detector position.

\medskip }
\end{minipage}
\\ \cdashline{2-2}

 & Test Data \\
 & \begin{minipage}[t]{15cm}{\footnotesize
LSST Photometry from main sequence, variable, bright, non-saturated
non-resolved point sources. Observations need to span sky position,
camera position, chip position, object color, brightness, airmass, water
vapor content, and time of observation.\\[2\baselineskip]Per The LSST
SRD, bright means 1-4 magnitudes fainter than saturation
limits.\\[2\baselineskip]We estimate this will require at
\textasciitilde{} 50,000 objects in order to overcome statistical noise.
This will require about 50**2 degrees of imaging, assuming there are
1000 stars per square degree. This roughly corresponds to
\textasciitilde{}100 ComCam pointing.~

\medskip }
\end{minipage} \\ \cdashline{2-2}

 & Expected Result \\
 & \begin{minipage}[t]{15cm}{\footnotesize

\medskip }
\end{minipage} \\ \cdashline{2-2}

 & Actual Result \\
 & \begin{minipage}[t]{15cm}{\footnotesize

\medskip }
\end{minipage} \\ \cdashline{2-2}

 & Status: \textbf{ Not Executed } \\ \hline

2 & Description \\
 & \begin{minipage}[t]{15cm}
{\footnotesize
For each star, measure the RMS magnitude in each filter. This yields a
distribution of RMS values in each filter. Calculate the median of the
distributions for each filter.

\medskip }
\end{minipage}
\\ \cdashline{2-2}


 & Expected Result \\
 & \begin{minipage}[t]{15cm}{\footnotesize

\medskip }
\end{minipage} \\ \cdashline{2-2}

 & Actual Result \\
 & \begin{minipage}[t]{15cm}{\footnotesize

\medskip }
\end{minipage} \\ \cdashline{2-2}

 & Status: \textbf{ Not Executed } \\ \hline

3 & Description \\
 & \begin{minipage}[t]{15cm}
{\footnotesize
For each star, calculate the mean magnitude. Calculate the number of
observations which deviate by more than PA2gri (15) millimags from their
means for magnitudes in the g, r, and i bands and PA2uzy (22.5)
millimags ~for magnitudes in the u, z, and y bands.

\medskip }
\end{minipage}
\\ \cdashline{2-2}


 & Expected Result \\
 & \begin{minipage}[t]{15cm}{\footnotesize

\medskip }
\end{minipage} \\ \cdashline{2-2}

 & Actual Result \\
 & \begin{minipage}[t]{15cm}{\footnotesize

\medskip }
\end{minipage} \\ \cdashline{2-2}

 & Status: \textbf{ Not Executed } \\ \hline

4 & Description \\
 & \begin{minipage}[t]{15cm}
{\footnotesize
Check the median RMS values for u, z, ~are below PA1uzy (7.5)
millimagniutes

\medskip }
\end{minipage}
\\ \cdashline{2-2}


 & Expected Result \\
 & \begin{minipage}[t]{15cm}{\footnotesize

\medskip }
\end{minipage} \\ \cdashline{2-2}

 & Actual Result \\
 & \begin{minipage}[t]{15cm}{\footnotesize

\medskip }
\end{minipage} \\ \cdashline{2-2}

 & Status: \textbf{ Not Executed } \\ \hline

5 & Description \\
 & \begin{minipage}[t]{15cm}
{\footnotesize
Check the median RMS values for g, r, i are below PA1gri (5)
millimagnitudes~

\medskip }
\end{minipage}
\\ \cdashline{2-2}


 & Expected Result \\
 & \begin{minipage}[t]{15cm}{\footnotesize

\medskip }
\end{minipage} \\ \cdashline{2-2}

 & Actual Result \\
 & \begin{minipage}[t]{15cm}{\footnotesize

\medskip }
\end{minipage} \\ \cdashline{2-2}

 & Status: \textbf{ Not Executed } \\ \hline

6 & Description \\
 & \begin{minipage}[t]{15cm}
{\footnotesize
Check that less than PF1 (10\%) of measurements deviate from their means
by more than PA2 from step 3

\medskip }
\end{minipage}
\\ \cdashline{2-2}


 & Expected Result \\
 & \begin{minipage}[t]{15cm}{\footnotesize

\medskip }
\end{minipage} \\ \cdashline{2-2}

 & Actual Result \\
 & \begin{minipage}[t]{15cm}{\footnotesize

\medskip }
\end{minipage} \\ \cdashline{2-2}

 & Status: \textbf{ Not Executed } \\ \hline

\end{longtable}

\paragraph{Test Case LVV-T390 - The spatial uniformity of photometric zeropoints }\mbox{}\\

Open  \href{https://jira.lsstcorp.org/secure/Tests.jspa#/testCase/LVV-T390}{\textit{ LVV-T390 } }
test case in Jira.

The distribution width (rms) of the internal photometric zero-point
error (the system stability across the sky) will not exceed PA3/PA3u
millimag, and no more than PF2 \% of the distribution will exceed PF4
millimag. ~Applies to both bright and faint ends to constrain
non-linearity of the flux scale

\textbf{ Preconditions}:\\


Execution status: {\bf Not Executed }

Final comment:\\


Detailed steps results:

\begin{longtable}{p{1cm}p{15cm}}
\hline
{Step} & Step Details\\ \hline
1 & Description \\
 & \begin{minipage}[t]{15cm}
{\footnotesize
Identify fields spread across the sky with available standard stars so
that zero points can be determined. ~Alternatively, identify specific
marker points in color-color diagrams that can be used to define
relative zero point offsets. ~Requires both bright and faint samples.

\medskip }
\end{minipage}
\\ \cdashline{2-2}

 & Test Data \\
 & \begin{minipage}[t]{15cm}{\footnotesize
Columns needed: RA, DEC; x, y position on focal plane; magnitudes in all
six ugrizy bands

\medskip }
\end{minipage} \\ \cdashline{2-2}

 & Expected Result \\
 & \begin{minipage}[t]{15cm}{\footnotesize

\medskip }
\end{minipage} \\ \cdashline{2-2}

 & Actual Result \\
 & \begin{minipage}[t]{15cm}{\footnotesize

\medskip }
\end{minipage} \\ \cdashline{2-2}

 & Status: \textbf{ Not Executed } \\ \hline

2 & Description \\
 & \begin{minipage}[t]{15cm}
{\footnotesize
Measure photometric zero points for stars for each patch of the sky

\medskip }
\end{minipage}
\\ \cdashline{2-2}


 & Expected Result \\
 & \begin{minipage}[t]{15cm}{\footnotesize

\medskip }
\end{minipage} \\ \cdashline{2-2}

 & Actual Result \\
 & \begin{minipage}[t]{15cm}{\footnotesize

\medskip }
\end{minipage} \\ \cdashline{2-2}

 & Status: \textbf{ Not Executed } \\ \hline

3 & Description \\
 & \begin{minipage}[t]{15cm}
{\footnotesize
Calculate RMS of zero points, check that RMS is less than PA3u (20 mmag)
in the u-band, and less than PA3 (10 mmag) for grizy bands.

\medskip }
\end{minipage}
\\ \cdashline{2-2}


 & Expected Result \\
 & \begin{minipage}[t]{15cm}{\footnotesize

\medskip }
\end{minipage} \\ \cdashline{2-2}

 & Actual Result \\
 & \begin{minipage}[t]{15cm}{\footnotesize

\medskip }
\end{minipage} \\ \cdashline{2-2}

 & Status: \textbf{ Not Executed } \\ \hline

4 & Description \\
 & \begin{minipage}[t]{15cm}
{\footnotesize
Check that no more than PF2 percent (10\%) of the zero points exceed PF4
(15 mmag)

\medskip }
\end{minipage}
\\ \cdashline{2-2}


 & Expected Result \\
 & \begin{minipage}[t]{15cm}{\footnotesize

\medskip }
\end{minipage} \\ \cdashline{2-2}

 & Actual Result \\
 & \begin{minipage}[t]{15cm}{\footnotesize

\medskip }
\end{minipage} \\ \cdashline{2-2}

 & Status: \textbf{ Not Executed } \\ \hline

\end{longtable}

\paragraph{Test Case LVV-T297 - Absolute Astrometric Performance }\mbox{}\\

Open  \href{https://jira.lsstcorp.org/secure/Tests.jspa#/testCase/LVV-T297}{\textit{ LVV-T297 } }
test case in Jira.

Measure astrometric performance using Gaia as an external reference.

\textbf{ Preconditions}:\\


Execution status: {\bf Not Executed }

Final comment:\\


Detailed steps results:

\begin{longtable}{p{1cm}p{15cm}}
\hline
{Step} & Step Details\\ \hline
1 & Description \\
 & \begin{minipage}[t]{15cm}
{\footnotesize

\medskip }
\end{minipage}
\\ \cdashline{2-2}


 & Expected Result \\
 & \begin{minipage}[t]{15cm}{\footnotesize

\medskip }
\end{minipage} \\ \cdashline{2-2}

 & Actual Result \\
 & \begin{minipage}[t]{15cm}{\footnotesize

\medskip }
\end{minipage} \\ \cdashline{2-2}

 & Status: \textbf{ Not Executed } \\ \hline

\end{longtable}

\paragraph{Test Case LVV-T298 - Cross-band Astrometric Performance }\mbox{}\\

Open  \href{https://jira.lsstcorp.org/secure/Tests.jspa#/testCase/LVV-T298}{\textit{ LVV-T298 } }
test case in Jira.



\textbf{ Preconditions}:\\


Execution status: {\bf Not Executed }

Final comment:\\


Detailed steps results:

\begin{longtable}{p{1cm}p{15cm}}
\hline
{Step} & Step Details\\ \hline
1 & Description \\
 & \begin{minipage}[t]{15cm}
{\footnotesize

\medskip }
\end{minipage}
\\ \cdashline{2-2}


 & Expected Result \\
 & \begin{minipage}[t]{15cm}{\footnotesize

\medskip }
\end{minipage} \\ \cdashline{2-2}

 & Actual Result \\
 & \begin{minipage}[t]{15cm}{\footnotesize

\medskip }
\end{minipage} \\ \cdashline{2-2}

 & Status: \textbf{ Not Executed } \\ \hline

\end{longtable}

\paragraph{Test Case LVV-T299 - Relative Astrometric Performance }\mbox{}\\

Open  \href{https://jira.lsstcorp.org/secure/Tests.jspa#/testCase/LVV-T299}{\textit{ LVV-T299 } }
test case in Jira.



\textbf{ Preconditions}:\\


Execution status: {\bf Not Executed }

Final comment:\\


Detailed steps results:

\begin{longtable}{p{1cm}p{15cm}}
\hline
{Step} & Step Details\\ \hline
1 & Description \\
 & \begin{minipage}[t]{15cm}
{\footnotesize

\medskip }
\end{minipage}
\\ \cdashline{2-2}


 & Expected Result \\
 & \begin{minipage}[t]{15cm}{\footnotesize

\medskip }
\end{minipage} \\ \cdashline{2-2}

 & Actual Result \\
 & \begin{minipage}[t]{15cm}{\footnotesize

\medskip }
\end{minipage} \\ \cdashline{2-2}

 & Status: \textbf{ Not Executed } \\ \hline

\end{longtable}

\paragraph{Test Case LVV-T360 - Off Zenith Image Quality Degradation }\mbox{}\\

Open  \href{https://jira.lsstcorp.org/secure/Tests.jspa#/testCase/LVV-T360}{\textit{ LVV-T360 } }
test case in Jira.



\textbf{ Preconditions}:\\


Execution status: {\bf Not Executed }

Final comment:\\


Detailed steps results:

\begin{longtable}{p{1cm}p{15cm}}
\hline
{Step} & Step Details\\ \hline
1 & Description \\
 & \begin{minipage}[t]{15cm}
{\footnotesize

\medskip }
\end{minipage}
\\ \cdashline{2-2}


 & Expected Result \\
 & \begin{minipage}[t]{15cm}{\footnotesize

\medskip }
\end{minipage} \\ \cdashline{2-2}

 & Actual Result \\
 & \begin{minipage}[t]{15cm}{\footnotesize

\medskip }
\end{minipage} \\ \cdashline{2-2}

 & Status: \textbf{ Not Executed } \\ \hline

\end{longtable}

\subsection{Test Cycle LVV-C5 }

Open test cycle {\it \href{https://jira.lsstcorp.org/secure/Tests.jspa#/testrun/LVV-C5}{Commissioning SV: Full-survey Performance w/ ComCam}} in Jira.

Commissioning SV: Full-survey Performance w/ ComCam\\
Status: Not Executed



\subsubsection{Software Version/Baseline}
Not provided.

\subsubsection{Configuration}
Not provided.

\subsubsection{Test Cases in LVV-C5 Test Cycle}

\paragraph{Test Case LVV-T986 - MOPS -- orbit association at catalog level }\mbox{}\\

Open  \href{https://jira.lsstcorp.org/secure/Tests.jspa#/testCase/LVV-T986}{\textit{ LVV-T986 } }
test case in Jira.

Use catalog-level tests to verify that MOPS is correctly associating
orbits at the correct rate

\textbf{ Preconditions}:\\


Execution status: {\bf Not Executed }

Final comment:\\


Detailed steps results:

\begin{longtable}{p{1cm}p{15cm}}
\hline
{Step} & Step Details\\ \hline
1 & Description \\
 & \begin{minipage}[t]{15cm}
{\footnotesize
Extract catalogs with sequence of observations meeting the requirements
orbitNightlyObservationInterval = 90{[}minute{]} Interval over which a
reference test case Solar System object must be observed within a night,
orbitObservations = 2{[}integer{]} Number of detections within a single
night required to define the reference test case for Solar System
objects, orbitObservationInterval = 3{[}day{]}

\medskip }
\end{minipage}
\\ \cdashline{2-2}


 & Expected Result \\
 & \begin{minipage}[t]{15cm}{\footnotesize

\medskip }
\end{minipage} \\ \cdashline{2-2}

 & Actual Result \\
 & \begin{minipage}[t]{15cm}{\footnotesize

\medskip }
\end{minipage} \\ \cdashline{2-2}

 & Status: \textbf{ Not Executed } \\ \hline

2 & Description \\
 & \begin{minipage}[t]{15cm}
{\footnotesize
Simulate sequence of asteroid positions given these observations
(pointing and time) using a current Solar System model

\medskip }
\end{minipage}
\\ \cdashline{2-2}


 & Expected Result \\
 & \begin{minipage}[t]{15cm}{\footnotesize

\medskip }
\end{minipage} \\ \cdashline{2-2}

 & Actual Result \\
 & \begin{minipage}[t]{15cm}{\footnotesize

\medskip }
\end{minipage} \\ \cdashline{2-2}

 & Status: \textbf{ Not Executed } \\ \hline

3 & Description \\
 & \begin{minipage}[t]{15cm}
{\footnotesize
Inject simulated asteroids into LSST catalog

\medskip }
\end{minipage}
\\ \cdashline{2-2}


 & Expected Result \\
 & \begin{minipage}[t]{15cm}{\footnotesize

\medskip }
\end{minipage} \\ \cdashline{2-2}

 & Actual Result \\
 & \begin{minipage}[t]{15cm}{\footnotesize

\medskip }
\end{minipage} \\ \cdashline{2-2}

 & Status: \textbf{ Not Executed } \\ \hline

4 & Description \\
 & \begin{minipage}[t]{15cm}
{\footnotesize
Run MOPS

\medskip }
\end{minipage}
\\ \cdashline{2-2}


 & Expected Result \\
 & \begin{minipage}[t]{15cm}{\footnotesize

\medskip }
\end{minipage} \\ \cdashline{2-2}

 & Actual Result \\
 & \begin{minipage}[t]{15cm}{\footnotesize

\medskip }
\end{minipage} \\ \cdashline{2-2}

 & Status: \textbf{ Not Executed } \\ \hline

5 & Description \\
 & \begin{minipage}[t]{15cm}
{\footnotesize
Match simulated ~asteroids

\medskip }
\end{minipage}
\\ \cdashline{2-2}


 & Expected Result \\
 & \begin{minipage}[t]{15cm}{\footnotesize

\medskip }
\end{minipage} \\ \cdashline{2-2}

 & Actual Result \\
 & \begin{minipage}[t]{15cm}{\footnotesize

\medskip }
\end{minipage} \\ \cdashline{2-2}

 & Status: \textbf{ Not Executed } \\ \hline

6 & Description \\
 & \begin{minipage}[t]{15cm}
{\footnotesize
Plot fraction of recovered asteroids as a function of limiting magnitude
(possibly for different populations of asteroids)

\medskip }
\end{minipage}
\\ \cdashline{2-2}


 & Expected Result \\
 & \begin{minipage}[t]{15cm}{\footnotesize

\medskip }
\end{minipage} \\ \cdashline{2-2}

 & Actual Result \\
 & \begin{minipage}[t]{15cm}{\footnotesize

\medskip }
\end{minipage} \\ \cdashline{2-2}

 & Status: \textbf{ Not Executed } \\ \hline

\end{longtable}

\paragraph{Test Case LVV-T969 - Alert completeness }\mbox{}\\

Open  \href{https://jira.lsstcorp.org/secure/Tests.jspa#/testCase/LVV-T969}{\textit{ LVV-T969 } }
test case in Jira.

Verify that all difference image sources above SNR=5 get broadcast as
alerts

\textbf{ Preconditions}:\\


Execution status: {\bf Not Executed }

Final comment:\\


Detailed steps results:

\begin{longtable}{p{1cm}p{15cm}}
\hline
{Step} & Step Details\\ \hline
1 & Description \\
 & \begin{minipage}[t]{15cm}
{\footnotesize
Run mini-survey over a densely populated field

\medskip }
\end{minipage}
\\ \cdashline{2-2}


 & Expected Result \\
 & \begin{minipage}[t]{15cm}{\footnotesize

\medskip }
\end{minipage} \\ \cdashline{2-2}

 & Actual Result \\
 & \begin{minipage}[t]{15cm}{\footnotesize

\medskip }
\end{minipage} \\ \cdashline{2-2}

 & Status: \textbf{ Not Executed } \\ \hline

2 & Description \\
 & \begin{minipage}[t]{15cm}
{\footnotesize
Perform difference image analysis and alert production on that
mini-survey

\medskip }
\end{minipage}
\\ \cdashline{2-2}


 & Expected Result \\
 & \begin{minipage}[t]{15cm}{\footnotesize

\medskip }
\end{minipage} \\ \cdashline{2-2}

 & Actual Result \\
 & \begin{minipage}[t]{15cm}{\footnotesize

\medskip }
\end{minipage} \\ \cdashline{2-2}

 & Status: \textbf{ Not Executed } \\ \hline

3 & Description \\
 & \begin{minipage}[t]{15cm}
{\footnotesize
Revisit difference images used to produce alerts in step 2. ~Verify by
hand that all 5-sigma detections in those difference images resulted in
alerts.

\medskip }
\end{minipage}
\\ \cdashline{2-2}


 & Expected Result \\
 & \begin{minipage}[t]{15cm}{\footnotesize

\medskip }
\end{minipage} \\ \cdashline{2-2}

 & Actual Result \\
 & \begin{minipage}[t]{15cm}{\footnotesize

\medskip }
\end{minipage} \\ \cdashline{2-2}

 & Status: \textbf{ Not Executed } \\ \hline

4 & Description \\
 & \begin{minipage}[t]{15cm}
{\footnotesize
Verify that at least one image resulted in at least 10,000 alerts,
verifying that the Data Management system was able to handle a volume of
up to 10,000 alerts per image.

\medskip }
\end{minipage}
\\ \cdashline{2-2}


 & Expected Result \\
 & \begin{minipage}[t]{15cm}{\footnotesize

\medskip }
\end{minipage} \\ \cdashline{2-2}

 & Actual Result \\
 & \begin{minipage}[t]{15cm}{\footnotesize

\medskip }
\end{minipage} \\ \cdashline{2-2}

 & Status: \textbf{ Not Executed } \\ \hline

\end{longtable}

\paragraph{Test Case LVV-T968 - WCS measurement and reporting }\mbox{}\\

Open  \href{https://jira.lsstcorp.org/secure/Tests.jspa#/testCase/LVV-T968}{\textit{ LVV-T968 } }
test case in Jira.

Verify that the Data Management system can produce and report a WCS
solution for an image with the required accuracy in the required amount
of time

\textbf{ Preconditions}:\\


Execution status: {\bf Not Executed }

Final comment:\\


Detailed steps results:

\begin{longtable}{p{1cm}p{15cm}}
\hline
{Step} & Step Details\\ \hline
1 & Description \\
 & \begin{minipage}[t]{15cm}
{\footnotesize
While telescope is running in a survey-like setting, observe that WCS
data is being reported within 60 seconds of images being taken.

\medskip }
\end{minipage}
\\ \cdashline{2-2}


 & Expected Result \\
 & \begin{minipage}[t]{15cm}{\footnotesize

\medskip }
\end{minipage} \\ \cdashline{2-2}

 & Actual Result \\
 & \begin{minipage}[t]{15cm}{\footnotesize

\medskip }
\end{minipage} \\ \cdashline{2-2}

 & Status: \textbf{ Not Executed } \\ \hline

2 & Description \\
 & \begin{minipage}[t]{15cm}
{\footnotesize
Identify the reference stars used for astrometric solutions in the
images from step 1. ~Compare their positions as reported in the
reference catalog to their positions according to the reported WCS.
~Verify that the residuals between catalog and WCS position have an RMS
less than 0.2 arcseconds.

\medskip }
\end{minipage}
\\ \cdashline{2-2}


 & Expected Result \\
 & \begin{minipage}[t]{15cm}{\footnotesize

\medskip }
\end{minipage} \\ \cdashline{2-2}

 & Actual Result \\
 & \begin{minipage}[t]{15cm}{\footnotesize

\medskip }
\end{minipage} \\ \cdashline{2-2}

 & Status: \textbf{ Not Executed } \\ \hline

\end{longtable}

\paragraph{Test Case LVV-T176 - Verify implementation of Infrastructure Sizing for ``catching up'' }\mbox{}\\

Open  \href{https://jira.lsstcorp.org/secure/Tests.jspa#/testCase/LVV-T176}{\textit{ LVV-T176 } }
test case in Jira.

Demonstrate Data Management System has sufficient excess capacity
(compute infrastructure) to process one night's data (2800 exposures)
within 24 hours while also maintaining nightly Alert Production (note
this is very similar to
\href{https://jira.lsstcorp.org/secure/Tests.jspa\#/testCase/LVV-T173}{LVV-T173}).~

\textbf{ Preconditions}:\\


Execution status: {\bf Not Executed }

Final comment:\\


Detailed steps results:

\begin{longtable}{p{1cm}p{15cm}}
\hline
{Step} & Step Details\\ \hline
1 & Description \\
 & \begin{minipage}[t]{15cm}
{\footnotesize
Execute single-day operations rehearsal including catch-up after
failure, observe data products generated in time

\medskip }
\end{minipage}
\\ \cdashline{2-2}


 & Expected Result \\
 & \begin{minipage}[t]{15cm}{\footnotesize

\medskip }
\end{minipage} \\ \cdashline{2-2}

 & Actual Result \\
 & \begin{minipage}[t]{15cm}{\footnotesize

\medskip }
\end{minipage} \\ \cdashline{2-2}

 & Status: \textbf{ Not Executed } \\ \hline

\end{longtable}

\paragraph{Test Case LVV-T966 - Image storing during outage }\mbox{}\\

Open  \href{https://jira.lsstcorp.org/secure/Tests.jspa#/testCase/LVV-T966}{\textit{ LVV-T966 } }
test case in Jira.

Verify that the Data Management system can hold up to 48 hours of images
in the event that base-to-summit communications are lost

\textbf{ Preconditions}:\\


Execution status: {\bf Not Executed }

Final comment:\\


Detailed steps results:

\begin{longtable}{p{1cm}p{15cm}}
\hline
{Step} & Step Details\\ \hline
1 & Description \\
 & \begin{minipage}[t]{15cm}
{\footnotesize
While telescope is running, instruct the Data Management pipeline to
cache data as if it cannot communicate with the base facility. ~Verify
that the system can run for 48 hours in this mode without filling up the
cache of images.

\medskip }
\end{minipage}
\\ \cdashline{2-2}


 & Expected Result \\
 & \begin{minipage}[t]{15cm}{\footnotesize

\medskip }
\end{minipage} \\ \cdashline{2-2}

 & Actual Result \\
 & \begin{minipage}[t]{15cm}{\footnotesize

\medskip }
\end{minipage} \\ \cdashline{2-2}

 & Status: \textbf{ Not Executed } \\ \hline

\end{longtable}

\paragraph{Test Case LVV-T963 - Alert generation reliability }\mbox{}\\

Open  \href{https://jira.lsstcorp.org/secure/Tests.jspa#/testCase/LVV-T963}{\textit{ LVV-T963 } }
test case in Jira.

Verify that the alert production pipeline does not skip or delay alert
production on too many images

\textbf{ Preconditions}:\\


Execution status: {\bf Not Executed }

Final comment:\\


Detailed steps results:

\begin{longtable}{p{1cm}p{15cm}}
\hline
{Step} & Step Details\\ \hline
1 & Description \\
 & \begin{minipage}[t]{15cm}
{\footnotesize
Perform a mini-survey using a realistic scheduling algorithm. ~Run the
images through the full alert-production pipeline.\\[2\baselineskip]Log
all images taken by the camera and all alerts generated.

\medskip }
\end{minipage}
\\ \cdashline{2-2}


 & Expected Result \\
 & \begin{minipage}[t]{15cm}{\footnotesize
A list of images taken and the resulting alerts.

\medskip }
\end{minipage} \\ \cdashline{2-2}

 & Actual Result \\
 & \begin{minipage}[t]{15cm}{\footnotesize

\medskip }
\end{minipage} \\ \cdashline{2-2}

 & Status: \textbf{ Not Executed } \\ \hline

2 & Description \\
 & \begin{minipage}[t]{15cm}
{\footnotesize
Verify that 99\% of images resulted in alerts being broadcast within 1
minute of the image being taken.

\medskip }
\end{minipage}
\\ \cdashline{2-2}

 & Test Data \\
 & \begin{minipage}[t]{15cm}{\footnotesize
List of images and alerts from step 1

\medskip }
\end{minipage} \\ \cdashline{2-2}

 & Expected Result \\
 & \begin{minipage}[t]{15cm}{\footnotesize

\medskip }
\end{minipage} \\ \cdashline{2-2}

 & Actual Result \\
 & \begin{minipage}[t]{15cm}{\footnotesize

\medskip }
\end{minipage} \\ \cdashline{2-2}

 & Status: \textbf{ Not Executed } \\ \hline

3 & Description \\
 & \begin{minipage}[t]{15cm}
{\footnotesize
Verify that 99.9\% of images resulted in alerts being broadcast at all.

\medskip }
\end{minipage}
\\ \cdashline{2-2}

 & Test Data \\
 & \begin{minipage}[t]{15cm}{\footnotesize
List of images and alerts from step 1

\medskip }
\end{minipage} \\ \cdashline{2-2}

 & Expected Result \\
 & \begin{minipage}[t]{15cm}{\footnotesize

\medskip }
\end{minipage} \\ \cdashline{2-2}

 & Actual Result \\
 & \begin{minipage}[t]{15cm}{\footnotesize

\medskip }
\end{minipage} \\ \cdashline{2-2}

 & Status: \textbf{ Not Executed } \\ \hline

\end{longtable}

\paragraph{Test Case LVV-T962 - PSF ellipticity correlations }\mbox{}\\

Open  \href{https://jira.lsstcorp.org/secure/Tests.jspa#/testCase/LVV-T962}{\textit{ LVV-T962 } }
test case in Jira.

Verify that correlations in the ellipticity of the PSF measured on a
single image are not excessive

\textbf{ Preconditions}:\\


Execution status: {\bf Not Executed }

Final comment:\\


Detailed steps results:

\begin{longtable}{p{1cm}p{15cm}}
\hline
{Step} & Step Details\\ \hline
1 & Description \\
 & \begin{minipage}[t]{15cm}
{\footnotesize
Image patches of sky at a wide variation of galactic latitude (to vary
the number of sources used for PSF measurement) and airmasses (to vary
the effects of the atmosphere on the PSF).

\medskip }
\end{minipage}
\\ \cdashline{2-2}


 & Expected Result \\
 & \begin{minipage}[t]{15cm}{\footnotesize
A set of images.

\medskip }
\end{minipage} \\ \cdashline{2-2}

 & Actual Result \\
 & \begin{minipage}[t]{15cm}{\footnotesize

\medskip }
\end{minipage} \\ \cdashline{2-2}

 & Status: \textbf{ Not Executed } \\ \hline

2 & Description \\
 & \begin{minipage}[t]{15cm}
{\footnotesize
For each image in step 1, measure the PSF and decompose it into the
functional form we are going to use to interpolate the PSF onto the
positions of galaxies.

\medskip }
\end{minipage}
\\ \cdashline{2-2}

 & Test Data \\
 & \begin{minipage}[t]{15cm}{\footnotesize
Images from step 1

\medskip }
\end{minipage} \\ \cdashline{2-2}

 & Expected Result \\
 & \begin{minipage}[t]{15cm}{\footnotesize
PSF measurements in the images from step 1

\medskip }
\end{minipage} \\ \cdashline{2-2}

 & Actual Result \\
 & \begin{minipage}[t]{15cm}{\footnotesize

\medskip }
\end{minipage} \\ \cdashline{2-2}

 & Status: \textbf{ Not Executed } \\ \hline

3 & Description \\
 & \begin{minipage}[t]{15cm}
{\footnotesize
Use the PSF interpolation function to calculate the theoretically
expected PSF at the position of stars used to measure the PSF so that we
can accurately measure the residual between the measured PSF ellipticity
and the interpolated PSF ellipticity.

\medskip }
\end{minipage}
\\ \cdashline{2-2}

 & Test Data \\
 & \begin{minipage}[t]{15cm}{\footnotesize
images from step 1\\
PSF interpolation function from step 2

\medskip }
\end{minipage} \\ \cdashline{2-2}

 & Expected Result \\
 & \begin{minipage}[t]{15cm}{\footnotesize
Calculated PSF at positions of stars used to measure the PSF

\medskip }
\end{minipage} \\ \cdashline{2-2}

 & Actual Result \\
 & \begin{minipage}[t]{15cm}{\footnotesize

\medskip }
\end{minipage} \\ \cdashline{2-2}

 & Status: \textbf{ Not Executed } \\ \hline

4 & Description \\
 & \begin{minipage}[t]{15cm}
{\footnotesize
For each of the stars used to calculate the PSF, calculate the e1 and e2
parameters specified in equations (10) and (11) of the Science
Requirements Document.\\[2\baselineskip]Calculate e1 and e2 for the PSF
as calculated from the interpolation function.\\[2\baselineskip]Subtract
the measured e1 and e2 from the interpolated e1 and e2 to get the
residuals delta\_e1, delta\_e2.

\medskip }
\end{minipage}
\\ \cdashline{2-2}

 & Test Data \\
 & \begin{minipage}[t]{15cm}{\footnotesize
PSF measurements on images from step 1\\
Interpolated PSFs from step 3

\medskip }
\end{minipage} \\ \cdashline{2-2}

 & Expected Result \\
 & \begin{minipage}[t]{15cm}{\footnotesize
distribution of PSF ellipticity residuals across the focal plane

\medskip }
\end{minipage} \\ \cdashline{2-2}

 & Actual Result \\
 & \begin{minipage}[t]{15cm}{\footnotesize

\medskip }
\end{minipage} \\ \cdashline{2-2}

 & Status: \textbf{ Not Executed } \\ \hline

5 & Description \\
 & \begin{minipage}[t]{15cm}
{\footnotesize
Calculate the correlation between pairs of sources of the PSF
ellipticity residuals from step 4 as a function of separation angle
between sources.\\[2\baselineskip]Verify that:\\[2\baselineskip]1) The
median PSF ellipticity correlations on 1 arcminute scales is less than
2.0e-5 (on both E1 and E2)\\[2\baselineskip]2) The median PSF
ellipticity correlations on 5 arcminutes scales is less than
1.0e-7\\[2\baselineskip]3) No more than 15\% of source pairs at 1
arcminute scales have PSF ellipticity correlations that exceed 4.0e-5
(on both E1 and E2)\\[2\baselineskip]4) No more than 15\% of source
pairs at 5 arcminute scales have PSF ellipticity correlations greater
than 2.0e-7

\medskip }
\end{minipage}
\\ \cdashline{2-2}

 & Test Data \\
 & \begin{minipage}[t]{15cm}{\footnotesize
PSF ellipticity residual distributions from step 4

\medskip }
\end{minipage} \\ \cdashline{2-2}

 & Expected Result \\
 & \begin{minipage}[t]{15cm}{\footnotesize

\medskip }
\end{minipage} \\ \cdashline{2-2}

 & Actual Result \\
 & \begin{minipage}[t]{15cm}{\footnotesize

\medskip }
\end{minipage} \\ \cdashline{2-2}

 & Status: \textbf{ Not Executed } \\ \hline

\end{longtable}

\paragraph{Test Case LVV-T950 - DIASource misassociation rate }\mbox{}\\

Open  \href{https://jira.lsstcorp.org/secure/Tests.jspa#/testCase/LVV-T950}{\textit{ LVV-T950 } }
test case in Jira.

Verify that DIASources are not misassociated with DIAObjects at too
large a rate

\textbf{ Preconditions}:\\


Execution status: {\bf Not Executed }

Final comment:\\


Detailed steps results:

\begin{longtable}{p{1cm}p{15cm}}
\hline
{Step} & Step Details\\ \hline
1 & Description \\
 & \begin{minipage}[t]{15cm}
{\footnotesize
Collect images from a minisurvey

\medskip }
\end{minipage}
\\ \cdashline{2-2}


 & Expected Result \\
 & \begin{minipage}[t]{15cm}{\footnotesize
Images and calibration products from a minisurvey

\medskip }
\end{minipage} \\ \cdashline{2-2}

 & Actual Result \\
 & \begin{minipage}[t]{15cm}{\footnotesize

\medskip }
\end{minipage} \\ \cdashline{2-2}

 & Status: \textbf{ Not Executed } \\ \hline

2 & Description \\
 & \begin{minipage}[t]{15cm}
{\footnotesize
Use synpipe (or some other tool) to inject variable sources with
different degrees of variability and different signal-to-noise ratios
into the images from step 1

\medskip }
\end{minipage}
\\ \cdashline{2-2}

 & Test Data \\
 & \begin{minipage}[t]{15cm}{\footnotesize
Images from step 1

\medskip }
\end{minipage} \\ \cdashline{2-2}

 & Expected Result \\
 & \begin{minipage}[t]{15cm}{\footnotesize
Images with an artificial population of variable sources

\medskip }
\end{minipage} \\ \cdashline{2-2}

 & Actual Result \\
 & \begin{minipage}[t]{15cm}{\footnotesize

\medskip }
\end{minipage} \\ \cdashline{2-2}

 & Status: \textbf{ Not Executed } \\ \hline

3 & Description \\
 & \begin{minipage}[t]{15cm}
{\footnotesize
Perform level 1 processing on the images with artificial variable
sources in them

\medskip }
\end{minipage}
\\ \cdashline{2-2}

 & Test Data \\
 & \begin{minipage}[t]{15cm}{\footnotesize
Images with artificial variable sources from step 2

\medskip }
\end{minipage} \\ \cdashline{2-2}

 & Expected Result \\
 & \begin{minipage}[t]{15cm}{\footnotesize
Catalog of DIASources and DIAObjects

\medskip }
\end{minipage} \\ \cdashline{2-2}

 & Actual Result \\
 & \begin{minipage}[t]{15cm}{\footnotesize

\medskip }
\end{minipage} \\ \cdashline{2-2}

 & Status: \textbf{ Not Executed } \\ \hline

4 & Description \\
 & \begin{minipage}[t]{15cm}
{\footnotesize
Consider only the artificial variables injected in step 2 (for which we
know the ground truth). ~Verify that they are not misassociated with
DIAObjects at a larger than acceptable rate.

\medskip }
\end{minipage}
\\ \cdashline{2-2}

 & Test Data \\
 & \begin{minipage}[t]{15cm}{\footnotesize
DIASources and DIAObjects from step 3

\medskip }
\end{minipage} \\ \cdashline{2-2}

 & Expected Result \\
 & \begin{minipage}[t]{15cm}{\footnotesize

\medskip }
\end{minipage} \\ \cdashline{2-2}

 & Actual Result \\
 & \begin{minipage}[t]{15cm}{\footnotesize

\medskip }
\end{minipage} \\ \cdashline{2-2}

 & Status: \textbf{ Not Executed } \\ \hline

\end{longtable}

\paragraph{Test Case LVV-T943 - Provenance in Level 1 catalogs }\mbox{}\\

Open  \href{https://jira.lsstcorp.org/secure/Tests.jspa#/testCase/LVV-T943}{\textit{ LVV-T943 } }
test case in Jira.

Verify that provenance information is correctly stored in Level 1
catalog products

\textbf{ Preconditions}:\\


Execution status: {\bf Not Executed }

Final comment:\\


Detailed steps results:

\begin{longtable}{p{1cm}p{15cm}}
\hline
{Step} & Step Details\\ \hline
1 & Description \\
 & \begin{minipage}[t]{15cm}
{\footnotesize
{Ingest raw data from L1 Test Stand DAQ, simulating each observing
mode\\
}

\medskip }
\end{minipage}
\\ \cdashline{2-2}


 & Expected Result \\
 & \begin{minipage}[t]{15cm}{\footnotesize

\medskip }
\end{minipage} \\ \cdashline{2-2}

 & Actual Result \\
 & \begin{minipage}[t]{15cm}{\footnotesize

\medskip }
\end{minipage} \\ \cdashline{2-2}

 & Status: \textbf{ Not Executed } \\ \hline

2 & Description \\
 & \begin{minipage}[t]{15cm}
{\footnotesize
O{bserve image metadata is present and queryable}

\medskip }
\end{minipage}
\\ \cdashline{2-2}


 & Expected Result \\
 & \begin{minipage}[t]{15cm}{\footnotesize

\medskip }
\end{minipage} \\ \cdashline{2-2}

 & Actual Result \\
 & \begin{minipage}[t]{15cm}{\footnotesize

\medskip }
\end{minipage} \\ \cdashline{2-2}

 & Status: \textbf{ Not Executed } \\ \hline

3 & Description \\
 & \begin{minipage}[t]{15cm}
{\footnotesize
Ingest data from L1 Camera Test Stand DAQ

\medskip }
\end{minipage}
\\ \cdashline{2-2}


 & Expected Result \\
 & \begin{minipage}[t]{15cm}{\footnotesize

\medskip }
\end{minipage} \\ \cdashline{2-2}

 & Actual Result \\
 & \begin{minipage}[t]{15cm}{\footnotesize

\medskip }
\end{minipage} \\ \cdashline{2-2}

 & Status: \textbf{ Not Executed } \\ \hline

4 & Description \\
 & \begin{minipage}[t]{15cm}
{\footnotesize
Simulate all different modes

\medskip }
\end{minipage}
\\ \cdashline{2-2}


 & Expected Result \\
 & \begin{minipage}[t]{15cm}{\footnotesize

\medskip }
\end{minipage} \\ \cdashline{2-2}

 & Actual Result \\
 & \begin{minipage}[t]{15cm}{\footnotesize

\medskip }
\end{minipage} \\ \cdashline{2-2}

 & Status: \textbf{ Not Executed } \\ \hline

5 & Description \\
 & \begin{minipage}[t]{15cm}
{\footnotesize
Verify that a raw image is constructed in correct format

\medskip }
\end{minipage}
\\ \cdashline{2-2}


 & Expected Result \\
 & \begin{minipage}[t]{15cm}{\footnotesize

\medskip }
\end{minipage} \\ \cdashline{2-2}

 & Actual Result \\
 & \begin{minipage}[t]{15cm}{\footnotesize

\medskip }
\end{minipage} \\ \cdashline{2-2}

 & Status: \textbf{ Not Executed } \\ \hline

6 & Description \\
 & \begin{minipage}[t]{15cm}
{\footnotesize
Verify that a raw image is constructed with correct metadata

\medskip }
\end{minipage}
\\ \cdashline{2-2}


 & Expected Result \\
 & \begin{minipage}[t]{15cm}{\footnotesize

\medskip }
\end{minipage} \\ \cdashline{2-2}

 & Actual Result \\
 & \begin{minipage}[t]{15cm}{\footnotesize

\medskip }
\end{minipage} \\ \cdashline{2-2}

 & Status: \textbf{ Not Executed } \\ \hline

7 & Description \\
 & \begin{minipage}[t]{15cm}
{\footnotesize
Verify that time of exposure start/end, site metadata, telescope
metadata, and camera metadata are stored in DMS
system.\\[2\baselineskip]

\medskip }
\end{minipage}
\\ \cdashline{2-2}


 & Expected Result \\
 & \begin{minipage}[t]{15cm}{\footnotesize

\medskip }
\end{minipage} \\ \cdashline{2-2}

 & Actual Result \\
 & \begin{minipage}[t]{15cm}{\footnotesize

\medskip }
\end{minipage} \\ \cdashline{2-2}

 & Status: \textbf{ Not Executed } \\ \hline

8 & Description \\
 & \begin{minipage}[t]{15cm}
{\footnotesize
Detect sources on difference images from step 1

\medskip }
\end{minipage}
\\ \cdashline{2-2}

 & Test Data \\
 & \begin{minipage}[t]{15cm}{\footnotesize
Difference images from step 1

\medskip }
\end{minipage} \\ \cdashline{2-2}

 & Expected Result \\
 & \begin{minipage}[t]{15cm}{\footnotesize
Catalog of DIASources

\medskip }
\end{minipage} \\ \cdashline{2-2}

 & Actual Result \\
 & \begin{minipage}[t]{15cm}{\footnotesize

\medskip }
\end{minipage} \\ \cdashline{2-2}

 & Status: \textbf{ Not Executed } \\ \hline

9 & Description \\
 & \begin{minipage}[t]{15cm}
{\footnotesize
Verify that provenance information is correctly stored in DIASource
catalogs from step 2

\medskip }
\end{minipage}
\\ \cdashline{2-2}


 & Expected Result \\
 & \begin{minipage}[t]{15cm}{\footnotesize

\medskip }
\end{minipage} \\ \cdashline{2-2}

 & Actual Result \\
 & \begin{minipage}[t]{15cm}{\footnotesize

\medskip }
\end{minipage} \\ \cdashline{2-2}

 & Status: \textbf{ Not Executed } \\ \hline

\end{longtable}

\paragraph{Test Case LVV-T942 - Provenance on Level 2 catalogs }\mbox{}\\

Open  \href{https://jira.lsstcorp.org/secure/Tests.jspa#/testCase/LVV-T942}{\textit{ LVV-T942 } }
test case in Jira.

Verify that provenance information is stored in level 2 catalog products

\textbf{ Preconditions}:\\


Execution status: {\bf Not Executed }

Final comment:\\


Detailed steps results:

\begin{longtable}{p{1cm}p{15cm}}
\hline
{Step} & Step Details\\ \hline
1 & Description \\
 & \begin{minipage}[t]{15cm}
{\footnotesize
The DM Stack shall be initialized using the loadLSST script (as
described in DRP-00-00).

\medskip }
\end{minipage}
\\ \cdashline{2-2}


 & Expected Result \\
 & \begin{minipage}[t]{15cm}{\footnotesize

\medskip }
\end{minipage} \\ \cdashline{2-2}

 & Actual Result \\
 & \begin{minipage}[t]{15cm}{\footnotesize

\medskip }
\end{minipage} \\ \cdashline{2-2}

 & Status: \textbf{ Not Executed } \\ \hline

2 & Description \\
 & \begin{minipage}[t]{15cm}
{\footnotesize
A ``Data Butler'' will be initialized to access the repository.

\medskip }
\end{minipage}
\\ \cdashline{2-2}


 & Expected Result \\
 & \begin{minipage}[t]{15cm}{\footnotesize

\medskip }
\end{minipage} \\ \cdashline{2-2}

 & Actual Result \\
 & \begin{minipage}[t]{15cm}{\footnotesize

\medskip }
\end{minipage} \\ \cdashline{2-2}

 & Status: \textbf{ Not Executed } \\ \hline

3 & Description \\
 & \begin{minipage}[t]{15cm}
{\footnotesize
For each of the expected data products types (listed in Test Items
section ยง4.3.2) and each of the expected units (PVIs, coadds, etc), the
data product will be retrieved from the Butler and verified to be
non-empty.

\medskip }
\end{minipage}
\\ \cdashline{2-2}


 & Expected Result \\
 & \begin{minipage}[t]{15cm}{\footnotesize

\medskip }
\end{minipage} \\ \cdashline{2-2}

 & Actual Result \\
 & \begin{minipage}[t]{15cm}{\footnotesize

\medskip }
\end{minipage} \\ \cdashline{2-2}

 & Status: \textbf{ Not Executed } \\ \hline

4 & Description \\
 & \begin{minipage}[t]{15cm}
{\footnotesize
Query and verify provenance of input images, and software versions that
went into producing stack.

\medskip }
\end{minipage}
\\ \cdashline{2-2}


 & Expected Result \\
 & \begin{minipage}[t]{15cm}{\footnotesize

\medskip }
\end{minipage} \\ \cdashline{2-2}

 & Actual Result \\
 & \begin{minipage}[t]{15cm}{\footnotesize

\medskip }
\end{minipage} \\ \cdashline{2-2}

 & Status: \textbf{ Not Executed } \\ \hline

5 & Description \\
 & \begin{minipage}[t]{15cm}
{\footnotesize
Test re-generating 10 different coadds tract+patches based on the
provenance image given

\medskip }
\end{minipage}
\\ \cdashline{2-2}


 & Expected Result \\
 & \begin{minipage}[t]{15cm}{\footnotesize

\medskip }
\end{minipage} \\ \cdashline{2-2}

 & Actual Result \\
 & \begin{minipage}[t]{15cm}{\footnotesize

\medskip }
\end{minipage} \\ \cdashline{2-2}

 & Status: \textbf{ Not Executed } \\ \hline

6 & Description \\
 & \begin{minipage}[t]{15cm}
{\footnotesize
Run source detection on coadded images from step 1

\medskip }
\end{minipage}
\\ \cdashline{2-2}

 & Test Data \\
 & \begin{minipage}[t]{15cm}{\footnotesize
Coadded images from step 1

\medskip }
\end{minipage} \\ \cdashline{2-2}

 & Expected Result \\
 & \begin{minipage}[t]{15cm}{\footnotesize
Catalog of sources detected on coadds

\medskip }
\end{minipage} \\ \cdashline{2-2}

 & Actual Result \\
 & \begin{minipage}[t]{15cm}{\footnotesize

\medskip }
\end{minipage} \\ \cdashline{2-2}

 & Status: \textbf{ Not Executed } \\ \hline

7 & Description \\
 & \begin{minipage}[t]{15cm}
{\footnotesize
Verify that correct provenance information is stored in catalogs from
step 2

\medskip }
\end{minipage}
\\ \cdashline{2-2}


 & Expected Result \\
 & \begin{minipage}[t]{15cm}{\footnotesize

\medskip }
\end{minipage} \\ \cdashline{2-2}

 & Actual Result \\
 & \begin{minipage}[t]{15cm}{\footnotesize

\medskip }
\end{minipage} \\ \cdashline{2-2}

 & Status: \textbf{ Not Executed } \\ \hline

\end{longtable}

\paragraph{Test Case LVV-T941 - Level 1 reproducibility (different computer hardware) }\mbox{}\\

Open  \href{https://jira.lsstcorp.org/secure/Tests.jspa#/testCase/LVV-T941}{\textit{ LVV-T941 } }
test case in Jira.



\textbf{ Preconditions}:\\
LVV-T940 has been run

Execution status: {\bf Not Executed }

Final comment:\\


Detailed steps results:

\begin{longtable}{p{1cm}p{15cm}}
\hline
{Step} & Step Details\\ \hline
1 & Description \\
 & \begin{minipage}[t]{15cm}
{\footnotesize
Run Level 1 analysis on data from step 1 of LVV-T940, using coadded
images from first Level 2 run as templates

\medskip }
\end{minipage}
\\ \cdashline{2-2}

 & Test Data \\
 & \begin{minipage}[t]{15cm}{\footnotesize
Precursor data and coadded images from step 2 LVV-T940

\medskip }
\end{minipage} \\ \cdashline{2-2}

 & Expected Result \\
 & \begin{minipage}[t]{15cm}{\footnotesize
Catalog of DIASources

\medskip }
\end{minipage} \\ \cdashline{2-2}

 & Actual Result \\
 & \begin{minipage}[t]{15cm}{\footnotesize

\medskip }
\end{minipage} \\ \cdashline{2-2}

 & Status: \textbf{ Not Executed } \\ \hline

2 & Description \\
 & \begin{minipage}[t]{15cm}
{\footnotesize
Re-run Level 1 analysis on data from step 1 of LVV-T940 using coadded
images from second Level 2 run and using a different computer system
than step 1 of this requirement

\medskip }
\end{minipage}
\\ \cdashline{2-2}

 & Test Data \\
 & \begin{minipage}[t]{15cm}{\footnotesize
Precursor data and coadded images from step 3 of LVV-T940

\medskip }
\end{minipage} \\ \cdashline{2-2}

 & Expected Result \\
 & \begin{minipage}[t]{15cm}{\footnotesize
Catalog of DIASources

\medskip }
\end{minipage} \\ \cdashline{2-2}

 & Actual Result \\
 & \begin{minipage}[t]{15cm}{\footnotesize

\medskip }
\end{minipage} \\ \cdashline{2-2}

 & Status: \textbf{ Not Executed } \\ \hline

3 & Description \\
 & \begin{minipage}[t]{15cm}
{\footnotesize
Verify that DIASource catalogs from steps 1 and 2 agree to within some
small tolerance

\medskip }
\end{minipage}
\\ \cdashline{2-2}

 & Test Data \\
 & \begin{minipage}[t]{15cm}{\footnotesize
DIASource catalogs from steps 1 and 2

\medskip }
\end{minipage} \\ \cdashline{2-2}

 & Expected Result \\
 & \begin{minipage}[t]{15cm}{\footnotesize
DIASource catalogs should agree to within some small tolerance

\medskip }
\end{minipage} \\ \cdashline{2-2}

 & Actual Result \\
 & \begin{minipage}[t]{15cm}{\footnotesize

\medskip }
\end{minipage} \\ \cdashline{2-2}

 & Status: \textbf{ Not Executed } \\ \hline

\end{longtable}

\paragraph{Test Case LVV-T939 - Level 1 reproducibility (same computer hardware) }\mbox{}\\

Open  \href{https://jira.lsstcorp.org/secure/Tests.jspa#/testCase/LVV-T939}{\textit{ LVV-T939 } }
test case in Jira.



\textbf{ Preconditions}:\\
LVV-T938 has been run

Execution status: {\bf Not Executed }

Final comment:\\


Detailed steps results:

\begin{longtable}{p{1cm}p{15cm}}
\hline
{Step} & Step Details\\ \hline
1 & Description \\
 & \begin{minipage}[t]{15cm}
{\footnotesize
Run Level 1 analysis on images taken in LVV-T938 using coadds from the
first run as templates.

\medskip }
\end{minipage}
\\ \cdashline{2-2}

 & Test Data \\
 & \begin{minipage}[t]{15cm}{\footnotesize
Precursor survey and coadded images from LVV-T938

\medskip }
\end{minipage} \\ \cdashline{2-2}

 & Expected Result \\
 & \begin{minipage}[t]{15cm}{\footnotesize
Set of DIASources

\medskip }
\end{minipage} \\ \cdashline{2-2}

 & Actual Result \\
 & \begin{minipage}[t]{15cm}{\footnotesize

\medskip }
\end{minipage} \\ \cdashline{2-2}

 & Status: \textbf{ Not Executed } \\ \hline

2 & Description \\
 & \begin{minipage}[t]{15cm}
{\footnotesize
Run Level 1 analysis on images taken in LVV-T938 using coadds from the
first run as templates. ~Run on the same system as step 1.

\medskip }
\end{minipage}
\\ \cdashline{2-2}

 & Test Data \\
 & \begin{minipage}[t]{15cm}{\footnotesize
Precursor survey and coadded images from LVV-T938

\medskip }
\end{minipage} \\ \cdashline{2-2}

 & Expected Result \\
 & \begin{minipage}[t]{15cm}{\footnotesize
Set of DIASources

\medskip }
\end{minipage} \\ \cdashline{2-2}

 & Actual Result \\
 & \begin{minipage}[t]{15cm}{\footnotesize

\medskip }
\end{minipage} \\ \cdashline{2-2}

 & Status: \textbf{ Not Executed } \\ \hline

3 & Description \\
 & \begin{minipage}[t]{15cm}
{\footnotesize
Verify that the sets of DIASources produced in steps 1 and 2 are
identical, since they were run on the same system.

\medskip }
\end{minipage}
\\ \cdashline{2-2}

 & Test Data \\
 & \begin{minipage}[t]{15cm}{\footnotesize
DIASources detected in steps 1 and 2

\medskip }
\end{minipage} \\ \cdashline{2-2}

 & Expected Result \\
 & \begin{minipage}[t]{15cm}{\footnotesize
Sets of DIASources should be identical

\medskip }
\end{minipage} \\ \cdashline{2-2}

 & Actual Result \\
 & \begin{minipage}[t]{15cm}{\footnotesize

\medskip }
\end{minipage} \\ \cdashline{2-2}

 & Status: \textbf{ Not Executed } \\ \hline

\end{longtable}

\paragraph{Test Case LVV-T940 - Level 2 reproducibility (different computer hardware) }\mbox{}\\

Open  \href{https://jira.lsstcorp.org/secure/Tests.jspa#/testCase/LVV-T940}{\textit{ LVV-T940 } }
test case in Jira.



\textbf{ Preconditions}:\\


Execution status: {\bf Not Executed }

Final comment:\\


Detailed steps results:

\begin{longtable}{p{1cm}p{15cm}}
\hline
{Step} & Step Details\\ \hline
1 & Description \\
 & \begin{minipage}[t]{15cm}
{\footnotesize
Take precursor data and calibration products from mini survey

\medskip }
\end{minipage}
\\ \cdashline{2-2}


 & Expected Result \\
 & \begin{minipage}[t]{15cm}{\footnotesize
Calibration products\\
Images

\medskip }
\end{minipage} \\ \cdashline{2-2}

 & Actual Result \\
 & \begin{minipage}[t]{15cm}{\footnotesize

\medskip }
\end{minipage} \\ \cdashline{2-2}

 & Status: \textbf{ Not Executed } \\ \hline

2 & Description \\
 & \begin{minipage}[t]{15cm}
{\footnotesize
Run Level 2 analysis on data from step 1

\medskip }
\end{minipage}
\\ \cdashline{2-2}

 & Test Data \\
 & \begin{minipage}[t]{15cm}{\footnotesize
Calibration products and images from step 1

\medskip }
\end{minipage} \\ \cdashline{2-2}

 & Expected Result \\
 & \begin{minipage}[t]{15cm}{\footnotesize
Coadded images\\
Catalogs of detected sources

\medskip }
\end{minipage} \\ \cdashline{2-2}

 & Actual Result \\
 & \begin{minipage}[t]{15cm}{\footnotesize

\medskip }
\end{minipage} \\ \cdashline{2-2}

 & Status: \textbf{ Not Executed } \\ \hline

3 & Description \\
 & \begin{minipage}[t]{15cm}
{\footnotesize
Re-run Level 2 analysis on data from step 1 using a different hardware
system

\medskip }
\end{minipage}
\\ \cdashline{2-2}

 & Test Data \\
 & \begin{minipage}[t]{15cm}{\footnotesize
Calibration products and images from step 1

\medskip }
\end{minipage} \\ \cdashline{2-2}

 & Expected Result \\
 & \begin{minipage}[t]{15cm}{\footnotesize
Coadded images\\
Catalogs of detected sources

\medskip }
\end{minipage} \\ \cdashline{2-2}

 & Actual Result \\
 & \begin{minipage}[t]{15cm}{\footnotesize

\medskip }
\end{minipage} \\ \cdashline{2-2}

 & Status: \textbf{ Not Executed } \\ \hline

4 & Description \\
 & \begin{minipage}[t]{15cm}
{\footnotesize
Verify that the catalogs from steps 2 and 3 agree to within some small
tolerance

\medskip }
\end{minipage}
\\ \cdashline{2-2}

 & Test Data \\
 & \begin{minipage}[t]{15cm}{\footnotesize
Catalogs of detected sources from steps 2 and 3

\medskip }
\end{minipage} \\ \cdashline{2-2}

 & Expected Result \\
 & \begin{minipage}[t]{15cm}{\footnotesize
Catalogs will agree to within some tolerance

\medskip }
\end{minipage} \\ \cdashline{2-2}

 & Actual Result \\
 & \begin{minipage}[t]{15cm}{\footnotesize

\medskip }
\end{minipage} \\ \cdashline{2-2}

 & Status: \textbf{ Not Executed } \\ \hline

\end{longtable}

\paragraph{Test Case LVV-T938 - Level 2 reproducibility (same computer hardware) }\mbox{}\\

Open  \href{https://jira.lsstcorp.org/secure/Tests.jspa#/testCase/LVV-T938}{\textit{ LVV-T938 } }
test case in Jira.



\textbf{ Preconditions}:\\


Execution status: {\bf Not Executed }

Final comment:\\


Detailed steps results:

\begin{longtable}{p{1cm}p{15cm}}
\hline
{Step} & Step Details\\ \hline
1 & Description \\
 & \begin{minipage}[t]{15cm}
{\footnotesize
Take precursor mini-survey data

\medskip }
\end{minipage}
\\ \cdashline{2-2}


 & Expected Result \\
 & \begin{minipage}[t]{15cm}{\footnotesize
Images and calibration products

\medskip }
\end{minipage} \\ \cdashline{2-2}

 & Actual Result \\
 & \begin{minipage}[t]{15cm}{\footnotesize

\medskip }
\end{minipage} \\ \cdashline{2-2}

 & Status: \textbf{ Not Executed } \\ \hline

2 & Description \\
 & \begin{minipage}[t]{15cm}
{\footnotesize
Run Level 2 processing on precursor data from step 1

\medskip }
\end{minipage}
\\ \cdashline{2-2}

 & Test Data \\
 & \begin{minipage}[t]{15cm}{\footnotesize
Images and calibration products from step 1

\medskip }
\end{minipage} \\ \cdashline{2-2}

 & Expected Result \\
 & \begin{minipage}[t]{15cm}{\footnotesize
Coadded images\\
Catalogs detected on coadded images

\medskip }
\end{minipage} \\ \cdashline{2-2}

 & Actual Result \\
 & \begin{minipage}[t]{15cm}{\footnotesize

\medskip }
\end{minipage} \\ \cdashline{2-2}

 & Status: \textbf{ Not Executed } \\ \hline

3 & Description \\
 & \begin{minipage}[t]{15cm}
{\footnotesize
Re-run Level 2 processing on data from step 1, using the same system as
in step 2

\medskip }
\end{minipage}
\\ \cdashline{2-2}

 & Test Data \\
 & \begin{minipage}[t]{15cm}{\footnotesize
Images and calibration products from step 1

\medskip }
\end{minipage} \\ \cdashline{2-2}

 & Expected Result \\
 & \begin{minipage}[t]{15cm}{\footnotesize
Coadded images\\
Catalogs detected on coadded images

\medskip }
\end{minipage} \\ \cdashline{2-2}

 & Actual Result \\
 & \begin{minipage}[t]{15cm}{\footnotesize

\medskip }
\end{minipage} \\ \cdashline{2-2}

 & Status: \textbf{ Not Executed } \\ \hline

4 & Description \\
 & \begin{minipage}[t]{15cm}
{\footnotesize
Verify that catalogs from step 2 and step 3 identify all of the same
sources and give identical measurements (since the two analyses were run
on the same system).

\medskip }
\end{minipage}
\\ \cdashline{2-2}

 & Test Data \\
 & \begin{minipage}[t]{15cm}{\footnotesize
Catalogs from steps 2 and 3

\medskip }
\end{minipage} \\ \cdashline{2-2}

 & Expected Result \\
 & \begin{minipage}[t]{15cm}{\footnotesize
Catalogs are identical

\medskip }
\end{minipage} \\ \cdashline{2-2}

 & Actual Result \\
 & \begin{minipage}[t]{15cm}{\footnotesize

\medskip }
\end{minipage} \\ \cdashline{2-2}

 & Status: \textbf{ Not Executed } \\ \hline

\end{longtable}

\paragraph{Test Case LVV-T597 - Depth variation over field of view }\mbox{}\\

Open  \href{https://jira.lsstcorp.org/secure/Tests.jspa#/testCase/LVV-T597}{\textit{ LVV-T597 } }
test case in Jira.



\textbf{ Preconditions}:\\


Execution status: {\bf Not Executed }

Final comment:\\


Detailed steps results:

\begin{longtable}{p{1cm}p{15cm}}
\hline
{Step} & Step Details\\ \hline
1 & Description \\
 & \begin{minipage}[t]{15cm}
{\footnotesize
After conclusion of a mini-survey, select all pointings that meet the
depth requirement specified in LSR-REQ-0090 (LVV-263)

\medskip }
\end{minipage}
\\ \cdashline{2-2}


 & Expected Result \\
 & \begin{minipage}[t]{15cm}{\footnotesize
Set of images that meet the fiducial depth requirement.

\medskip }
\end{minipage} \\ \cdashline{2-2}

 & Actual Result \\
 & \begin{minipage}[t]{15cm}{\footnotesize

\medskip }
\end{minipage} \\ \cdashline{2-2}

 & Status: \textbf{ Not Executed } \\ \hline

2 & Description \\
 & \begin{minipage}[t]{15cm}
{\footnotesize
Perform single image processing on the images from step 1 (if not done
already)

\medskip }
\end{minipage}
\\ \cdashline{2-2}

 & Test Data \\
 & \begin{minipage}[t]{15cm}{\footnotesize
Images from step 1 that meet the fiducial depth requirement

\medskip }
\end{minipage} \\ \cdashline{2-2}

 & Expected Result \\
 & \begin{minipage}[t]{15cm}{\footnotesize
Catalogs of measured sources from the images in step 1

\medskip }
\end{minipage} \\ \cdashline{2-2}

 & Actual Result \\
 & \begin{minipage}[t]{15cm}{\footnotesize

\medskip }
\end{minipage} \\ \cdashline{2-2}

 & Status: \textbf{ Not Executed } \\ \hline

3 & Description \\
 & \begin{minipage}[t]{15cm}
{\footnotesize
Subdivide each image into small regions (\textasciitilde{}1 CCD should
be enough, since 1/189 = 5*10\^{}-3). ~In each region, examine the SNR
distribution of measured sources to determine the 5-sigma limiting
magnitude for that region of the focal plane.

\medskip }
\end{minipage}
\\ \cdashline{2-2}

 & Test Data \\
 & \begin{minipage}[t]{15cm}{\footnotesize
Measured sources from step 3

\medskip }
\end{minipage} \\ \cdashline{2-2}

 & Expected Result \\
 & \begin{minipage}[t]{15cm}{\footnotesize
Distribution of depths for subregions of the focal plane

\medskip }
\end{minipage} \\ \cdashline{2-2}

 & Actual Result \\
 & \begin{minipage}[t]{15cm}{\footnotesize

\medskip }
\end{minipage} \\ \cdashline{2-2}

 & Status: \textbf{ Not Executed } \\ \hline

4 & Description \\
 & \begin{minipage}[t]{15cm}
{\footnotesize
Verify that, for each exposure, no more than 15\% of the focal plane
area has a 5-sigma limiting magnitude 0.2 AB magnitude brighter than the
median 5-sigma limiting magnitude for the entire exposure.

\medskip }
\end{minipage}
\\ \cdashline{2-2}


 & Expected Result \\
 & \begin{minipage}[t]{15cm}{\footnotesize

\medskip }
\end{minipage} \\ \cdashline{2-2}

 & Actual Result \\
 & \begin{minipage}[t]{15cm}{\footnotesize

\medskip }
\end{minipage} \\ \cdashline{2-2}

 & Status: \textbf{ Not Executed } \\ \hline

\end{longtable}

\paragraph{Test Case LVV-T596 - Depth: r-band }\mbox{}\\

Open  \href{https://jira.lsstcorp.org/secure/Tests.jspa#/testCase/LVV-T596}{\textit{ LVV-T596 } }
test case in Jira.



\textbf{ Preconditions}:\\


Execution status: {\bf Not Executed }

Final comment:\\


Detailed steps results:

\begin{longtable}{p{1cm}p{15cm}}
\hline
{Step} & Step Details\\ \hline
1 & Description \\
 & \begin{minipage}[t]{15cm}
{\footnotesize
Upon completion of mini-survey, select all exposures taken at or
near\\[2\baselineskip]band = r\\
airmass = 1\\
seeing = 0.7 arcsecond\\
sky brightness = 21 magnitudes per square arcsecond

\medskip }
\end{minipage}
\\ \cdashline{2-2}


 & Expected Result \\
 & \begin{minipage}[t]{15cm}{\footnotesize
Set of exposures taken at reference conditions

\medskip }
\end{minipage} \\ \cdashline{2-2}

 & Actual Result \\
 & \begin{minipage}[t]{15cm}{\footnotesize

\medskip }
\end{minipage} \\ \cdashline{2-2}

 & Status: \textbf{ Not Executed } \\ \hline

2 & Description \\
 & \begin{minipage}[t]{15cm}
{\footnotesize
Run single visit processing on images from step 1

\medskip }
\end{minipage}
\\ \cdashline{2-2}

 & Test Data \\
 & \begin{minipage}[t]{15cm}{\footnotesize
images from step 1

\medskip }
\end{minipage} \\ \cdashline{2-2}

 & Expected Result \\
 & \begin{minipage}[t]{15cm}{\footnotesize
catalog of measured sources

\medskip }
\end{minipage} \\ \cdashline{2-2}

 & Actual Result \\
 & \begin{minipage}[t]{15cm}{\footnotesize

\medskip }
\end{minipage} \\ \cdashline{2-2}

 & Status: \textbf{ Not Executed } \\ \hline

3 & Description \\
 & \begin{minipage}[t]{15cm}
{\footnotesize
For each visit, find the 5-sigma limiting magnitude by examining the
distribution of sources detected at SNR=5

\medskip }
\end{minipage}
\\ \cdashline{2-2}

 & Test Data \\
 & \begin{minipage}[t]{15cm}{\footnotesize
catalog of measured sources from step 2

\medskip }
\end{minipage} \\ \cdashline{2-2}

 & Expected Result \\
 & \begin{minipage}[t]{15cm}{\footnotesize
distribution of 5-sigma limiting magnitudes

\medskip }
\end{minipage} \\ \cdashline{2-2}

 & Actual Result \\
 & \begin{minipage}[t]{15cm}{\footnotesize

\medskip }
\end{minipage} \\ \cdashline{2-2}

 & Status: \textbf{ Not Executed } \\ \hline

4 & Description \\
 & \begin{minipage}[t]{15cm}
{\footnotesize
Verify that the median of the distribution of 5-sigma limiting
magnitudes is no brighter than 24.7 AB magnitudes

\medskip }
\end{minipage}
\\ \cdashline{2-2}

 & Test Data \\
 & \begin{minipage}[t]{15cm}{\footnotesize
Distribution of 5-sigma limiting magnitudes from step 3

\medskip }
\end{minipage} \\ \cdashline{2-2}

 & Expected Result \\
 & \begin{minipage}[t]{15cm}{\footnotesize

\medskip }
\end{minipage} \\ \cdashline{2-2}

 & Actual Result \\
 & \begin{minipage}[t]{15cm}{\footnotesize

\medskip }
\end{minipage} \\ \cdashline{2-2}

 & Status: \textbf{ Not Executed } \\ \hline

5 & Description \\
 & \begin{minipage}[t]{15cm}
{\footnotesize
Verify that no more than 10\% of the images have a 5-sigma limiting
magnitudes brighter than 24.4 AB magnitudes

\medskip }
\end{minipage}
\\ \cdashline{2-2}

 & Test Data \\
 & \begin{minipage}[t]{15cm}{\footnotesize
Distribution of 5-sigma limiting magnitudes from step 3

\medskip }
\end{minipage} \\ \cdashline{2-2}

 & Expected Result \\
 & \begin{minipage}[t]{15cm}{\footnotesize

\medskip }
\end{minipage} \\ \cdashline{2-2}

 & Actual Result \\
 & \begin{minipage}[t]{15cm}{\footnotesize

\medskip }
\end{minipage} \\ \cdashline{2-2}

 & Status: \textbf{ Not Executed } \\ \hline

\end{longtable}

\paragraph{Test Case LVV-T549 - Zeropoint consistency }\mbox{}\\

Open  \href{https://jira.lsstcorp.org/secure/Tests.jspa#/testCase/LVV-T549}{\textit{ LVV-T549 } }
test case in Jira.

Verify that the CCD AP pipeline zero point is consistent within
\textbf{photoZeroPointOffset (50)} millimags of the level 2 pipeline

\textbf{ Preconditions}:\\


Execution status: {\bf Not Executed }

Final comment:\\


Detailed steps results:

\begin{longtable}{p{1cm}p{15cm}}
\hline
{Step} & Step Details\\ \hline
1 & Description \\
 & \begin{minipage}[t]{15cm}
{\footnotesize
Image a patch of sky to a specified depth (1yr? 3yr? full LSST?).

\medskip }
\end{minipage}
\\ \cdashline{2-2}


 & Expected Result \\
 & \begin{minipage}[t]{15cm}{\footnotesize
Images of the same patch of sky

\medskip }
\end{minipage} \\ \cdashline{2-2}

 & Actual Result \\
 & \begin{minipage}[t]{15cm}{\footnotesize

\medskip }
\end{minipage} \\ \cdashline{2-2}

 & Status: \textbf{ Not Executed } \\ \hline

2 & Description \\
 & \begin{minipage}[t]{15cm}
{\footnotesize
Perform Level 2 processing on images from step 1. ~Store final zeropoint
determination.

\medskip }
\end{minipage}
\\ \cdashline{2-2}

 & Test Data \\
 & \begin{minipage}[t]{15cm}{\footnotesize
Images from step 1

\medskip }
\end{minipage} \\ \cdashline{2-2}

 & Expected Result \\
 & \begin{minipage}[t]{15cm}{\footnotesize
Determination of photometric zeropoint from final photometric
calibration algorithm

\medskip }
\end{minipage} \\ \cdashline{2-2}

 & Actual Result \\
 & \begin{minipage}[t]{15cm}{\footnotesize

\medskip }
\end{minipage} \\ \cdashline{2-2}

 & Status: \textbf{ Not Executed } \\ \hline

3 & Description \\
 & \begin{minipage}[t]{15cm}
{\footnotesize
Calibrate images from step 1 using Level 1 pipeline

\medskip }
\end{minipage}
\\ \cdashline{2-2}

 & Test Data \\
 & \begin{minipage}[t]{15cm}{\footnotesize
Images from step 1

\medskip }
\end{minipage} \\ \cdashline{2-2}

 & Expected Result \\
 & \begin{minipage}[t]{15cm}{\footnotesize
Calibrated images based on raw images from step 1

\medskip }
\end{minipage} \\ \cdashline{2-2}

 & Actual Result \\
 & \begin{minipage}[t]{15cm}{\footnotesize

\medskip }
\end{minipage} \\ \cdashline{2-2}

 & Status: \textbf{ Not Executed } \\ \hline

4 & Description \\
 & \begin{minipage}[t]{15cm}
{\footnotesize
For each calibrated image produced in step 3, verify that the
photometric zeropoint agrees with the final zeropoint determine din step
2 to the specified tolerance.

\medskip }
\end{minipage}
\\ \cdashline{2-2}

 & Test Data \\
 & \begin{minipage}[t]{15cm}{\footnotesize
Calibrated images from step 3

\medskip }
\end{minipage} \\ \cdashline{2-2}

 & Expected Result \\
 & \begin{minipage}[t]{15cm}{\footnotesize

\medskip }
\end{minipage} \\ \cdashline{2-2}

 & Actual Result \\
 & \begin{minipage}[t]{15cm}{\footnotesize

\medskip }
\end{minipage} \\ \cdashline{2-2}

 & Status: \textbf{ Not Executed } \\ \hline

\end{longtable}

\paragraph{Test Case LVV-T547 - Photometric errors -- level 1 processing -- on-sky data }\mbox{}\\

Open  \href{https://jira.lsstcorp.org/secure/Tests.jspa#/testCase/LVV-T547}{\textit{ LVV-T547 } }
test case in Jira.

Test DM contribution to photometric erros with LSST images

\textbf{ Preconditions}:\\


Execution status: {\bf Not Executed }

Final comment:\\


Detailed steps results:

\begin{longtable}{p{1cm}p{15cm}}
\hline
{Step} & Step Details\\ \hline
1 & Description \\
 & \begin{minipage}[t]{15cm}
{\footnotesize
Image a patch of sky at various observing conditions (airmass, seeing
,etc.).

\medskip }
\end{minipage}
\\ \cdashline{2-2}


 & Expected Result \\
 & \begin{minipage}[t]{15cm}{\footnotesize
Images of sky at various observing conditions

\medskip }
\end{minipage} \\ \cdashline{2-2}

 & Actual Result \\
 & \begin{minipage}[t]{15cm}{\footnotesize

\medskip }
\end{minipage} \\ \cdashline{2-2}

 & Status: \textbf{ Not Executed } \\ \hline

2 & Description \\
 & \begin{minipage}[t]{15cm}
{\footnotesize
Perform Level 2 processing on images from step 1. ~Save catalog of
fluxes as well as standard deviation of flux measurements from
individual images.

\medskip }
\end{minipage}
\\ \cdashline{2-2}

 & Test Data \\
 & \begin{minipage}[t]{15cm}{\footnotesize
Images from step 1

\medskip }
\end{minipage} \\ \cdashline{2-2}

 & Expected Result \\
 & \begin{minipage}[t]{15cm}{\footnotesize
Catalog of all sources in images from step 1\\
Characterization of width of flux measurements for each source

\medskip }
\end{minipage} \\ \cdashline{2-2}

 & Actual Result \\
 & \begin{minipage}[t]{15cm}{\footnotesize

\medskip }
\end{minipage} \\ \cdashline{2-2}

 & Status: \textbf{ Not Executed } \\ \hline

3 & Description \\
 & \begin{minipage}[t]{15cm}
{\footnotesize
Perform Level 1 processing on images from step 1. ~Keep difference
images for ``forced DIA photometry'' in later steps.

\medskip }
\end{minipage}
\\ \cdashline{2-2}

 & Test Data \\
 & \begin{minipage}[t]{15cm}{\footnotesize
Images from step 1

\medskip }
\end{minipage} \\ \cdashline{2-2}

 & Expected Result \\
 & \begin{minipage}[t]{15cm}{\footnotesize
Catalog of variable sources in images from step 1\\
Difference images corresponding to images in step 1

\medskip }
\end{minipage} \\ \cdashline{2-2}

 & Actual Result \\
 & \begin{minipage}[t]{15cm}{\footnotesize

\medskip }
\end{minipage} \\ \cdashline{2-2}

 & Status: \textbf{ Not Executed } \\ \hline

4 & Description \\
 & \begin{minipage}[t]{15cm}
{\footnotesize
Identify sources in step 2 that did not appear as DIASources. ~These
will be taken as totally static objects.

\medskip }
\end{minipage}
\\ \cdashline{2-2}

 & Test Data \\
 & \begin{minipage}[t]{15cm}{\footnotesize
Catalog of sources from step 2\\
Catalog of DIASources from step 3

\medskip }
\end{minipage} \\ \cdashline{2-2}

 & Expected Result \\
 & \begin{minipage}[t]{15cm}{\footnotesize
Catalog of static sources

\medskip }
\end{minipage} \\ \cdashline{2-2}

 & Actual Result \\
 & \begin{minipage}[t]{15cm}{\footnotesize

\medskip }
\end{minipage} \\ \cdashline{2-2}

 & Status: \textbf{ Not Executed } \\ \hline

5 & Description \\
 & \begin{minipage}[t]{15cm}
{\footnotesize
Perform forced photometry on difference images from step 3 at locations
of static objects identified in step 4.~

\medskip }
\end{minipage}
\\ \cdashline{2-2}

 & Test Data \\
 & \begin{minipage}[t]{15cm}{\footnotesize
Catalog of static sources from step 4.\\
Difference images from step 3.

\medskip }
\end{minipage} \\ \cdashline{2-2}

 & Expected Result \\
 & \begin{minipage}[t]{15cm}{\footnotesize
Catalog of difference image photometry for static sources

\medskip }
\end{minipage} \\ \cdashline{2-2}

 & Actual Result \\
 & \begin{minipage}[t]{15cm}{\footnotesize

\medskip }
\end{minipage} \\ \cdashline{2-2}

 & Status: \textbf{ Not Executed } \\ \hline

6 & Description \\
 & \begin{minipage}[t]{15cm}
{\footnotesize
Construct model of photometric uncertainty based only observing
conditions of images in step 1.

\medskip }
\end{minipage}
\\ \cdashline{2-2}

 & Test Data \\
 & \begin{minipage}[t]{15cm}{\footnotesize
Metadata from images in step 1

\medskip }
\end{minipage} \\ \cdashline{2-2}

 & Expected Result \\
 & \begin{minipage}[t]{15cm}{\footnotesize
Model of photometric uncertainty expected solely due to observing
conditions

\medskip }
\end{minipage} \\ \cdashline{2-2}

 & Actual Result \\
 & \begin{minipage}[t]{15cm}{\footnotesize

\medskip }
\end{minipage} \\ \cdashline{2-2}

 & Status: \textbf{ Not Executed } \\ \hline

7 & Description \\
 & \begin{minipage}[t]{15cm}
{\footnotesize
Compare distribution of force difference image photometry measurements
from step 5 with intrinsic width of flux measurements in step 2 and
model of uncertainty due to observing conditions in step 6. ~Verify that
RMS residual is within specified tolerance.

\medskip }
\end{minipage}
\\ \cdashline{2-2}

 & Test Data \\
 & \begin{minipage}[t]{15cm}{\footnotesize
Forced difference image photometry from step 5\\
Uncertainty model from step 6\\
Intrinsic width of flux measurements from step 2

\medskip }
\end{minipage} \\ \cdashline{2-2}

 & Expected Result \\
 & \begin{minipage}[t]{15cm}{\footnotesize

\medskip }
\end{minipage} \\ \cdashline{2-2}

 & Actual Result \\
 & \begin{minipage}[t]{15cm}{\footnotesize

\medskip }
\end{minipage} \\ \cdashline{2-2}

 & Status: \textbf{ Not Executed } \\ \hline

\end{longtable}

\paragraph{Test Case LVV-T544 - Astrometric error -- level 1 processing -- on-sky data }\mbox{}\\

Open  \href{https://jira.lsstcorp.org/secure/Tests.jspa#/testCase/LVV-T544}{\textit{ LVV-T544 } }
test case in Jira.

Measure the astrometric performance requirements using actual data

\textbf{ Preconditions}:\\


Execution status: {\bf Not Executed }

Final comment:\\


Detailed steps results:

\begin{longtable}{p{1cm}p{15cm}}
\hline
{Step} & Step Details\\ \hline
1 & Description \\
 & \begin{minipage}[t]{15cm}
{\footnotesize
Perform full-depth mini-survey on a patch of sky.

\medskip }
\end{minipage}
\\ \cdashline{2-2}


 & Expected Result \\
 & \begin{minipage}[t]{15cm}{\footnotesize
Images going down to full LSST depth

\medskip }
\end{minipage} \\ \cdashline{2-2}

 & Actual Result \\
 & \begin{minipage}[t]{15cm}{\footnotesize

\medskip }
\end{minipage} \\ \cdashline{2-2}

 & Status: \textbf{ Not Executed } \\ \hline

2 & Description \\
 & \begin{minipage}[t]{15cm}
{\footnotesize
Perform Level 2 processing to get ground truth position of sources.

\medskip }
\end{minipage}
\\ \cdashline{2-2}

 & Test Data \\
 & \begin{minipage}[t]{15cm}{\footnotesize
Images from step 1

\medskip }
\end{minipage} \\ \cdashline{2-2}

 & Expected Result \\
 & \begin{minipage}[t]{15cm}{\footnotesize
Catalog of sources to be used as truth for analysis

\medskip }
\end{minipage} \\ \cdashline{2-2}

 & Actual Result \\
 & \begin{minipage}[t]{15cm}{\footnotesize

\medskip }
\end{minipage} \\ \cdashline{2-2}

 & Status: \textbf{ Not Executed } \\ \hline

3 & Description \\
 & \begin{minipage}[t]{15cm}
{\footnotesize
Perform Level 1 analysis on images from step 1.

\medskip }
\end{minipage}
\\ \cdashline{2-2}

 & Test Data \\
 & \begin{minipage}[t]{15cm}{\footnotesize
Images from step 1\\
A template coadd constructed from those images

\medskip }
\end{minipage} \\ \cdashline{2-2}

 & Expected Result \\
 & \begin{minipage}[t]{15cm}{\footnotesize
Catalog of DIASources

\medskip }
\end{minipage} \\ \cdashline{2-2}

 & Actual Result \\
 & \begin{minipage}[t]{15cm}{\footnotesize

\medskip }
\end{minipage} \\ \cdashline{2-2}

 & Status: \textbf{ Not Executed } \\ \hline

4 & Description \\
 & \begin{minipage}[t]{15cm}
{\footnotesize
Model astrometric errors as a function of observing conditions (airmass,
seeing, etc.) in images from step 1.

\medskip }
\end{minipage}
\\ \cdashline{2-2}

 & Test Data \\
 & \begin{minipage}[t]{15cm}{\footnotesize
Metadata from images in step 1

\medskip }
\end{minipage} \\ \cdashline{2-2}

 & Expected Result \\
 & \begin{minipage}[t]{15cm}{\footnotesize
Model of astrometric errors due only to observing conditions

\medskip }
\end{minipage} \\ \cdashline{2-2}

 & Actual Result \\
 & \begin{minipage}[t]{15cm}{\footnotesize

\medskip }
\end{minipage} \\ \cdashline{2-2}

 & Status: \textbf{ Not Executed } \\ \hline

5 & Description \\
 & \begin{minipage}[t]{15cm}
{\footnotesize
Compare measured positions of DIASources to ground truth catalog from
step 2 to get distribution of astrometric errors.

\medskip }
\end{minipage}
\\ \cdashline{2-2}

 & Test Data \\
 & \begin{minipage}[t]{15cm}{\footnotesize
DIASources from step 3\\
Ground truth catalog from step 2

\medskip }
\end{minipage} \\ \cdashline{2-2}

 & Expected Result \\
 & \begin{minipage}[t]{15cm}{\footnotesize
Distribution of measured astrometric errors

\medskip }
\end{minipage} \\ \cdashline{2-2}

 & Actual Result \\
 & \begin{minipage}[t]{15cm}{\footnotesize

\medskip }
\end{minipage} \\ \cdashline{2-2}

 & Status: \textbf{ Not Executed } \\ \hline

6 & Description \\
 & \begin{minipage}[t]{15cm}
{\footnotesize
Compute the RMS residual between the measured astrometric errors in step
5 and the model of errors due just to observing conditions in step 4.
~Verify that residual is within specified tolerance.

\medskip }
\end{minipage}
\\ \cdashline{2-2}

 & Test Data \\
 & \begin{minipage}[t]{15cm}{\footnotesize
Measured astrometric errors from step 5\\
Model of errors due just to observing conditions from step 4

\medskip }
\end{minipage} \\ \cdashline{2-2}

 & Expected Result \\
 & \begin{minipage}[t]{15cm}{\footnotesize

\medskip }
\end{minipage} \\ \cdashline{2-2}

 & Actual Result \\
 & \begin{minipage}[t]{15cm}{\footnotesize

\medskip }
\end{minipage} \\ \cdashline{2-2}

 & Status: \textbf{ Not Executed } \\ \hline

\end{longtable}

\paragraph{Test Case LVV-T532 - MOPS completeness threshold }\mbox{}\\

Open  \href{https://jira.lsstcorp.org/secure/Tests.jspa#/testCase/LVV-T532}{\textit{ LVV-T532 } }
test case in Jira.

Verify that spuriousness metric has a threshold value at which
completeness and purity requirements for MOPS are met

\textbf{ Preconditions}:\\


Execution status: {\bf Not Executed }

Final comment:\\


Detailed steps results:

\begin{longtable}{p{1cm}p{15cm}}
\hline
{Step} & Step Details\\ \hline
1 & Description \\
 & \begin{minipage}[t]{15cm}
{\footnotesize
Generate catalog of simulated variable/transient sources

\medskip }
\end{minipage}
\\ \cdashline{2-2}


 & Expected Result \\
 & \begin{minipage}[t]{15cm}{\footnotesize

\medskip }
\end{minipage} \\ \cdashline{2-2}

 & Actual Result \\
 & \begin{minipage}[t]{15cm}{\footnotesize

\medskip }
\end{minipage} \\ \cdashline{2-2}

 & Status: \textbf{ Not Executed } \\ \hline

2 & Description \\
 & \begin{minipage}[t]{15cm}
{\footnotesize
Inject simulated variables/transients into actual images

\medskip }
\end{minipage}
\\ \cdashline{2-2}

 & Test Data \\
 & \begin{minipage}[t]{15cm}{\footnotesize
Catalog of simulated objects from step 1

\medskip }
\end{minipage} \\ \cdashline{2-2}

 & Expected Result \\
 & \begin{minipage}[t]{15cm}{\footnotesize
Set of images with simulated variables/transients

\medskip }
\end{minipage} \\ \cdashline{2-2}

 & Actual Result \\
 & \begin{minipage}[t]{15cm}{\footnotesize

\medskip }
\end{minipage} \\ \cdashline{2-2}

 & Status: \textbf{ Not Executed } \\ \hline

3 & Description \\
 & \begin{minipage}[t]{15cm}
{\footnotesize
Run difference imaging on images with injected variables/transients

\medskip }
\end{minipage}
\\ \cdashline{2-2}

 & Test Data \\
 & \begin{minipage}[t]{15cm}{\footnotesize
Images from step 2

\medskip }
\end{minipage} \\ \cdashline{2-2}

 & Expected Result \\
 & \begin{minipage}[t]{15cm}{\footnotesize
Catalog of detected DIASources

\medskip }
\end{minipage} \\ \cdashline{2-2}

 & Actual Result \\
 & \begin{minipage}[t]{15cm}{\footnotesize

\medskip }
\end{minipage} \\ \cdashline{2-2}

 & Status: \textbf{ Not Executed } \\ \hline

4 & Description \\
 & \begin{minipage}[t]{15cm}
{\footnotesize
Rate DIASources according to spuriousness metric

\medskip }
\end{minipage}
\\ \cdashline{2-2}

 & Test Data \\
 & \begin{minipage}[t]{15cm}{\footnotesize
DIASources from step 3

\medskip }
\end{minipage} \\ \cdashline{2-2}

 & Expected Result \\
 & \begin{minipage}[t]{15cm}{\footnotesize
Catalog of DIASources with assigned spuriousness values

\medskip }
\end{minipage} \\ \cdashline{2-2}

 & Actual Result \\
 & \begin{minipage}[t]{15cm}{\footnotesize

\medskip }
\end{minipage} \\ \cdashline{2-2}

 & Status: \textbf{ Not Executed } \\ \hline

5 & Description \\
 & \begin{minipage}[t]{15cm}
{\footnotesize
Find value of spuriousness threshold that preserves injected sources at
completeness mopsCompletenessMin

\medskip }
\end{minipage}
\\ \cdashline{2-2}

 & Test Data \\
 & \begin{minipage}[t]{15cm}{\footnotesize
DIASources and spuriousness metric

\medskip }
\end{minipage} \\ \cdashline{2-2}

 & Expected Result \\
 & \begin{minipage}[t]{15cm}{\footnotesize
Threshold in spuriousness metric

\medskip }
\end{minipage} \\ \cdashline{2-2}

 & Actual Result \\
 & \begin{minipage}[t]{15cm}{\footnotesize

\medskip }
\end{minipage} \\ \cdashline{2-2}

 & Status: \textbf{ Not Executed } \\ \hline

6 & Description \\
 & \begin{minipage}[t]{15cm}
{\footnotesize
Compare to analysis of purity as a function of spuriousness metric

\medskip }
\end{minipage}
\\ \cdashline{2-2}


 & Expected Result \\
 & \begin{minipage}[t]{15cm}{\footnotesize

\medskip }
\end{minipage} \\ \cdashline{2-2}

 & Actual Result \\
 & \begin{minipage}[t]{15cm}{\footnotesize

\medskip }
\end{minipage} \\ \cdashline{2-2}

 & Status: \textbf{ Not Executed } \\ \hline

\end{longtable}

\paragraph{Test Case LVV-T533 - MOPS purity threshold }\mbox{}\\

Open  \href{https://jira.lsstcorp.org/secure/Tests.jspa#/testCase/LVV-T533}{\textit{ LVV-T533 } }
test case in Jira.



\textbf{ Preconditions}:\\


Execution status: {\bf Not Executed }

Final comment:\\


Detailed steps results:

\begin{longtable}{p{1cm}p{15cm}}
\hline
{Step} & Step Details\\ \hline
1 & Description \\
 & \begin{minipage}[t]{15cm}
{\footnotesize
Identify all truly variable sources in a mini-survey area. ~This will
either be done by waiting for the mini-survey to complete and running a
full historical light curve analysis on the region, or through human
inspection of difference images (or some combination of both).

\medskip }
\end{minipage}
\\ \cdashline{2-2}

 & Test Data \\
 & \begin{minipage}[t]{15cm}{\footnotesize
Completed min-survey images and coadds.

\medskip }
\end{minipage} \\ \cdashline{2-2}

 & Expected Result \\
 & \begin{minipage}[t]{15cm}{\footnotesize
Catalog of true variables in the region.

\medskip }
\end{minipage} \\ \cdashline{2-2}

 & Actual Result \\
 & \begin{minipage}[t]{15cm}{\footnotesize

\medskip }
\end{minipage} \\ \cdashline{2-2}

 & Status: \textbf{ Not Executed } \\ \hline

2 & Description \\
 & \begin{minipage}[t]{15cm}
{\footnotesize
Go back to the individual images in the mini-survey and perform
difference image analysis.

\medskip }
\end{minipage}
\\ \cdashline{2-2}

 & Test Data \\
 & \begin{minipage}[t]{15cm}{\footnotesize
Images and multiple epochs in the mini-survey.

\medskip }
\end{minipage} \\ \cdashline{2-2}

 & Expected Result \\
 & \begin{minipage}[t]{15cm}{\footnotesize
Catalogs of DIASources, some of them bogus.

\medskip }
\end{minipage} \\ \cdashline{2-2}

 & Actual Result \\
 & \begin{minipage}[t]{15cm}{\footnotesize

\medskip }
\end{minipage} \\ \cdashline{2-2}

 & Status: \textbf{ Not Executed } \\ \hline

3 & Description \\
 & \begin{minipage}[t]{15cm}
{\footnotesize
Use catalog from step 1 to identify which of the DIASources in step 2
are real and which are artifacts.

\medskip }
\end{minipage}
\\ \cdashline{2-2}

 & Test Data \\
 & \begin{minipage}[t]{15cm}{\footnotesize
DIASources from step 2 and catalog of true sources from step 1.

\medskip }
\end{minipage} \\ \cdashline{2-2}

 & Expected Result \\
 & \begin{minipage}[t]{15cm}{\footnotesize
Catalog of DIASources labeled as either `real' or `bogus'.

\medskip }
\end{minipage} \\ \cdashline{2-2}

 & Actual Result \\
 & \begin{minipage}[t]{15cm}{\footnotesize

\medskip }
\end{minipage} \\ \cdashline{2-2}

 & Status: \textbf{ Not Executed } \\ \hline

4 & Description \\
 & \begin{minipage}[t]{15cm}
{\footnotesize
Rate DIASource detections with spuriousness metric.

\medskip }
\end{minipage}
\\ \cdashline{2-2}

 & Test Data \\
 & \begin{minipage}[t]{15cm}{\footnotesize
DIASources from step 2.

\medskip }
\end{minipage} \\ \cdashline{2-2}

 & Expected Result \\
 & \begin{minipage}[t]{15cm}{\footnotesize
Catalog of DIASources with spuriousness metric assigned.

\medskip }
\end{minipage} \\ \cdashline{2-2}

 & Actual Result \\
 & \begin{minipage}[t]{15cm}{\footnotesize

\medskip }
\end{minipage} \\ \cdashline{2-2}

 & Status: \textbf{ Not Executed } \\ \hline

5 & Description \\
 & \begin{minipage}[t]{15cm}
{\footnotesize
Find value of spuriousness metric which gives desired purity
mopsPurityMin

\medskip }
\end{minipage}
\\ \cdashline{2-2}

 & Test Data \\
 & \begin{minipage}[t]{15cm}{\footnotesize
Catalogs from steps 2 and 3.

\medskip }
\end{minipage} \\ \cdashline{2-2}

 & Expected Result \\
 & \begin{minipage}[t]{15cm}{\footnotesize
Threshold in spuriousness metric.

\medskip }
\end{minipage} \\ \cdashline{2-2}

 & Actual Result \\
 & \begin{minipage}[t]{15cm}{\footnotesize

\medskip }
\end{minipage} \\ \cdashline{2-2}

 & Status: \textbf{ Not Executed } \\ \hline

6 & Description \\
 & \begin{minipage}[t]{15cm}
{\footnotesize
Compare to completeness threshold in spuriousness metric.

\medskip }
\end{minipage}
\\ \cdashline{2-2}


 & Expected Result \\
 & \begin{minipage}[t]{15cm}{\footnotesize

\medskip }
\end{minipage} \\ \cdashline{2-2}

 & Actual Result \\
 & \begin{minipage}[t]{15cm}{\footnotesize

\medskip }
\end{minipage} \\ \cdashline{2-2}

 & Status: \textbf{ Not Executed } \\ \hline

\end{longtable}

\paragraph{Test Case LVV-T294 - On-sky Observations: Full-survey Key Performance Metrics }\mbox{}\\

Open  \href{https://jira.lsstcorp.org/secure/Tests.jspa#/testCase/LVV-T294}{\textit{ LVV-T294 } }
test case in Jira.

Repeated observations of a smaller number of fields reaching cumulative
exposures equivalent to the 10-year stack in the wide-fast-deep survey,
specifically, 200 visits in both the r and i band. These ~observations
are designed to measure residual PSF ellipticities, ~and to test
transient, variable, and moving object detection over a range of
timescales. Three fields should be chosen along the ecliptic that
together span a range of source densities. Each field should be observed
in multiple epochsdistributed over at least 3 consecutive nights and
cover a range of airmasses. Dithered pointings will be used to
approximate the coverage pattern expected in the wide-fast-deep
survey.\\[2\baselineskip]Estimated ~observing time = 34 seconds * 200
visits * 2 filters * 3 (dither pattern) * 3 fields =
\textasciitilde{}36hrs.

\textbf{ Preconditions}:\\


Execution status: {\bf Not Executed }

Final comment:\\


Detailed steps results:

\begin{longtable}{p{1cm}p{15cm}}
\hline
{Step} & Step Details\\ \hline
1 & Description \\
 & \begin{minipage}[t]{15cm}
{\footnotesize

\medskip }
\end{minipage}
\\ \cdashline{2-2}


 & Expected Result \\
 & \begin{minipage}[t]{15cm}{\footnotesize

\medskip }
\end{minipage} \\ \cdashline{2-2}

 & Actual Result \\
 & \begin{minipage}[t]{15cm}{\footnotesize

\medskip }
\end{minipage} \\ \cdashline{2-2}

 & Status: \textbf{ Not Executed } \\ \hline

\end{longtable}

\paragraph{Test Case LVV-T296 - Data Processing Campaign: Full-survey Key Performance Metrics }\mbox{}\\

Open  \href{https://jira.lsstcorp.org/secure/Tests.jspa#/testCase/LVV-T296}{\textit{ LVV-T296 } }
test case in Jira.



\textbf{ Preconditions}:\\


Execution status: {\bf Not Executed }

Final comment:\\


Detailed steps results:

\begin{longtable}{p{1cm}p{15cm}}
\hline
{Step} & Step Details\\ \hline
1 & Description \\
 & \begin{minipage}[t]{15cm}
{\footnotesize

\medskip }
\end{minipage}
\\ \cdashline{2-2}


 & Expected Result \\
 & \begin{minipage}[t]{15cm}{\footnotesize

\medskip }
\end{minipage} \\ \cdashline{2-2}

 & Actual Result \\
 & \begin{minipage}[t]{15cm}{\footnotesize

\medskip }
\end{minipage} \\ \cdashline{2-2}

 & Status: \textbf{ Not Executed } \\ \hline

\end{longtable}

\subsection{Test Cycle LVV-C36 }

Open test cycle {\it \href{https://jira.lsstcorp.org/secure/Tests.jspa#/testrun/LVV-C36}{Commissioning SV: 20-year Depth w/ ComCam}} in Jira.

Commissioning SV: 20-year Depth w/ ComCam\\
Status: Not Executed



\subsubsection{Software Version/Baseline}
Not provided.

\subsubsection{Configuration}
Not provided.

\subsubsection{Test Cases in LVV-C36 Test Cycle}

\paragraph{Test Case LVV-T1071 - On-sky Observations: 20-year Depth Test }\mbox{}\\

Open  \href{https://jira.lsstcorp.org/secure/Tests.jspa#/testCase/LVV-T1071}{\textit{ LVV-T1071 } }
test case in Jira.



\textbf{ Preconditions}:\\


Execution status: {\bf Not Executed }

Final comment:\\


Detailed steps results:

\begin{longtable}{p{1cm}p{15cm}}
\hline
{Step} & Step Details\\ \hline
1 & Description \\
 & \begin{minipage}[t]{15cm}
{\footnotesize

\medskip }
\end{minipage}
\\ \cdashline{2-2}


 & Expected Result \\
 & \begin{minipage}[t]{15cm}{\footnotesize

\medskip }
\end{minipage} \\ \cdashline{2-2}

 & Actual Result \\
 & \begin{minipage}[t]{15cm}{\footnotesize

\medskip }
\end{minipage} \\ \cdashline{2-2}

 & Status: \textbf{ Not Executed } \\ \hline

\end{longtable}

\subsection{Test Cycle LVV-C37 }

Open test cycle {\it \href{https://jira.lsstcorp.org/secure/Tests.jspa#/testrun/LVV-C37}{Commissioning SV: Scheduler Testing w/ ComCam}} in Jira.

Commissioning SV: Scheduler Testing w/ ComCam\\
Status: Not Executed



\subsubsection{Software Version/Baseline}
Not provided.

\subsubsection{Configuration}
Not provided.

\subsubsection{Test Cases in LVV-C37 Test Cycle}

\paragraph{Test Case LVV-T965 - Sub-system health during communication outage }\mbox{}\\

Open  \href{https://jira.lsstcorp.org/secure/Tests.jspa#/testCase/LVV-T965}{\textit{ LVV-T965 } }
test case in Jira.

Verify that sub-system health reporting infrastructure continues to
function, even in the absence of base-to-summit communications

\textbf{ Preconditions}:\\


Execution status: {\bf Not Executed }

Final comment:\\


Detailed steps results:

\begin{longtable}{p{1cm}p{15cm}}
\hline
{Step} & Step Details\\ \hline
1 & Description \\
 & \begin{minipage}[t]{15cm}
{\footnotesize
While the telescope is observing, deactivate base-to-summit
communication channels for some reasonable amount of time
(\textasciitilde{} a few hours?)

\medskip }
\end{minipage}
\\ \cdashline{2-2}


 & Expected Result \\
 & \begin{minipage}[t]{15cm}{\footnotesize

\medskip }
\end{minipage} \\ \cdashline{2-2}

 & Actual Result \\
 & \begin{minipage}[t]{15cm}{\footnotesize

\medskip }
\end{minipage} \\ \cdashline{2-2}

 & Status: \textbf{ Not Executed } \\ \hline

2 & Description \\
 & \begin{minipage}[t]{15cm}
{\footnotesize
Verify that the sub-system health reporting infrastructure specified in
OSS-REQ-0065 continues to operate while communication is down.

\medskip }
\end{minipage}
\\ \cdashline{2-2}


 & Expected Result \\
 & \begin{minipage}[t]{15cm}{\footnotesize

\medskip }
\end{minipage} \\ \cdashline{2-2}

 & Actual Result \\
 & \begin{minipage}[t]{15cm}{\footnotesize

\medskip }
\end{minipage} \\ \cdashline{2-2}

 & Status: \textbf{ Not Executed } \\ \hline

\end{longtable}

\paragraph{Test Case LVV-T964 - Quick look during outage }\mbox{}\\

Open  \href{https://jira.lsstcorp.org/secure/Tests.jspa#/testCase/LVV-T964}{\textit{ LVV-T964 } }
test case in Jira.

Verify that the quick look system still functions, even when
communication between the base and the summit has been lost

\textbf{ Preconditions}:\\


Execution status: {\bf Not Executed }

Final comment:\\


Detailed steps results:

\begin{longtable}{p{1cm}p{15cm}}
\hline
{Step} & Step Details\\ \hline
1 & Description \\
 & \begin{minipage}[t]{15cm}
{\footnotesize
While the telescope is observing, deactivate all communication channels
between the base and summit for some reasonable amount of time
(\textasciitilde{} a few hours?)

\medskip }
\end{minipage}
\\ \cdashline{2-2}


 & Expected Result \\
 & \begin{minipage}[t]{15cm}{\footnotesize

\medskip }
\end{minipage} \\ \cdashline{2-2}

 & Actual Result \\
 & \begin{minipage}[t]{15cm}{\footnotesize

\medskip }
\end{minipage} \\ \cdashline{2-2}

 & Status: \textbf{ Not Executed } \\ \hline

2 & Description \\
 & \begin{minipage}[t]{15cm}
{\footnotesize
Verify that the quick look system specified in OSS-REQ-0057 (LVV-1024)
continues to operate, even in the absence of communications.

\medskip }
\end{minipage}
\\ \cdashline{2-2}


 & Expected Result \\
 & \begin{minipage}[t]{15cm}{\footnotesize

\medskip }
\end{minipage} \\ \cdashline{2-2}

 & Actual Result \\
 & \begin{minipage}[t]{15cm}{\footnotesize

\medskip }
\end{minipage} \\ \cdashline{2-2}

 & Status: \textbf{ Not Executed } \\ \hline

\end{longtable}

\paragraph{Test Case LVV-T967 - Scheduler functioning during communication outage }\mbox{}\\

Open  \href{https://jira.lsstcorp.org/secure/Tests.jspa#/testCase/LVV-T967}{\textit{ LVV-T967 } }
test case in Jira.

Verify that the scheduler can continue to operate the telescope, even
when communication between the base and the summit has been lost

\textbf{ Preconditions}:\\


Execution status: {\bf Not Executed }

Final comment:\\


Detailed steps results:

\begin{longtable}{p{1cm}p{15cm}}
\hline
{Step} & Step Details\\ \hline
1 & Description \\
 & \begin{minipage}[t]{15cm}
{\footnotesize
Run OpSim with the scheduler in a mode that simulates loss of
communication between the base and the summit. ~Verify that the
scheduler can continue to operate for 48 simulated hours of observing in
this state.

\medskip }
\end{minipage}
\\ \cdashline{2-2}


 & Expected Result \\
 & \begin{minipage}[t]{15cm}{\footnotesize

\medskip }
\end{minipage} \\ \cdashline{2-2}

 & Actual Result \\
 & \begin{minipage}[t]{15cm}{\footnotesize

\medskip }
\end{minipage} \\ \cdashline{2-2}

 & Status: \textbf{ Not Executed } \\ \hline

\end{longtable}


\input{appendix.tex}
\end{document}
