\documentclass[DM,lsstdraft,STR,toc]{lsstdoc}
\usepackage{geometry}
\usepackage{longtable,booktabs}
\usepackage{enumitem}
\usepackage{arydshln}

\input meta.tex

\providecommand{\tightlist}{
  \setlength{\itemsep}{0pt}\setlength{\parskip}{0pt}}

\begin{document}

\def\milestoneName{Commissioning sv: single-visit and full-survey performance w/ comcam}
\def\milestoneId{LVV-P2}
\def\product{Commissioning Science Verification}

\setDocCompact{true}

\title[\milestoneId{}~Test Report]{\milestoneId{} (\milestoneName{})~Test Plan and Report}
\setDocRef{\lsstDocType-\lsstDocNum}
\setDocDate{\vcsdate}
\setDocUpstreamLocation{\url{https://github.com/lsst/lsst-texmf/examples}}
\author{ Keith Bechtol }

\input history_and_info.tex


\setDocAbstract{
This is the test plan and report for \milestoneId{} (\milestoneName{}), an LSST level 2 milestone pertaining to the Data Management Subsystem.
}


\maketitle

\section{Introduction}
\label{sect:intro}


\subsection{Objectives}
\label{sect:objectives}

Initial verification of the single-visit and full survey performance at
the raft scale using ComCam. Bulk data production from sustained on-sky
observations.



\subsection{System Overview}
\label{sect:systemoverview}



\subsection{Document Overview}
\label{sect:docoverview}

This document was generated from Jira, obtaining the relevant information from the 
\href{https://jira.lsstcorp.org/secure/Tests.jspa#/testPlan/LVV-P2}{LVV-P2}
~Jira Test Plan and related Test Cycles (
  \href{https://jira.lsstcorp.org/secure/Tests.jspa#/testCycle/LVV-C4}{LVV-C4}
  \href{https://jira.lsstcorp.org/secure/Tests.jspa#/testCycle/LVV-C5}{LVV-C5}
  \href{https://jira.lsstcorp.org/secure/Tests.jspa#/testCycle/LVV-C36}{LVV-C36}
  \href{https://jira.lsstcorp.org/secure/Tests.jspa#/testCycle/LVV-C37}{LVV-C37}
).

Section \ref{sect:intro} provides an overview of the test campaign, the system under test (\product{}), the applicable documentation, and explains how this document is organized.
Section \ref{sect:configuration}  describes the configuration used for this test.
Section \ref{sect:personnel} describes the necessary roles and lists the individuals assigned to them.
%Section \ref{sect:plannedtestactivities} provides the list of planned test cycles and test cases, including all relevant information that fully describes the test campaign.

Section \ref{sect:overview} provides a summary of the test results, including an overview in Table \ref{table:summary}, an overall assessment statement and suggestions for possible improvements.
Section \ref{sect:detailedtestresults} provides detailed results for each step in each test case.

The current status of test plan LVV-P2 in Jira is Draft.

\subsection{References}
\label{sect:references}
\renewcommand{\refname}{}
\bibliography{lsst,refs,books,refs_ads}
\section{Test Configuration}
\label{sect:configuration}

\subsection{Data Collection}

  Observing is not required for this test campaign.

\subsection{Verification Environment}
\label{sect:hwconf}

  \subsection{Entry Criteria}
  ComCam has passed electro-optical testing and is ready to begin
sustained on-sky observing.




\section{Personnel}
\label{sect:personnel}

The following personnel are involved in this test activity:

\begin{itemize}
\item Test Plan (LVV-P2) owner: Keith Bechtol
\item Test Cycles:
\begin{itemize}
  \item LVV-C4 owner: 
    Undefined
  \begin{itemize}
    \item Test case \href{https://jira.lsstcorp.org/secure/Tests.jspa#/testCase/LVV-T293}{LVV-T293} tester: Keith Bechtol
    \item Test case \href{https://jira.lsstcorp.org/secure/Tests.jspa#/testCase/LVV-T295}{LVV-T295} tester: Keith Bechtol
    \item Test case \href{https://jira.lsstcorp.org/secure/Tests.jspa#/testCase/LVV-T961}{LVV-T961} tester: 
    \item Test case \href{https://jira.lsstcorp.org/secure/Tests.jspa#/testCase/LVV-T959}{LVV-T959} tester: 
    \item Test case \href{https://jira.lsstcorp.org/secure/Tests.jspa#/testCase/LVV-T960}{LVV-T960} tester: 
    \item Test case \href{https://jira.lsstcorp.org/secure/Tests.jspa#/testCase/LVV-T956}{LVV-T956} tester: 
    \item Test case \href{https://jira.lsstcorp.org/secure/Tests.jspa#/testCase/LVV-T957}{LVV-T957} tester: 
    \item Test case \href{https://jira.lsstcorp.org/secure/Tests.jspa#/testCase/LVV-T595}{LVV-T595} tester: 
    \item Test case \href{https://jira.lsstcorp.org/secure/Tests.jspa#/testCase/LVV-T594}{LVV-T594} tester: 
    \item Test case \href{https://jira.lsstcorp.org/secure/Tests.jspa#/testCase/LVV-T593}{LVV-T593} tester: 
    \item Test case \href{https://jira.lsstcorp.org/secure/Tests.jspa#/testCase/LVV-T592}{LVV-T592} tester: 
    \item Test case \href{https://jira.lsstcorp.org/secure/Tests.jspa#/testCase/LVV-T591}{LVV-T591} tester: 
    \item Test case \href{https://jira.lsstcorp.org/secure/Tests.jspa#/testCase/LVV-T590}{LVV-T590} tester: 
    \item Test case \href{https://jira.lsstcorp.org/secure/Tests.jspa#/testCase/LVV-T589}{LVV-T589} tester: 
    \item Test case \href{https://jira.lsstcorp.org/secure/Tests.jspa#/testCase/LVV-T588}{LVV-T588} tester: 
    \item Test case \href{https://jira.lsstcorp.org/secure/Tests.jspa#/testCase/LVV-T587}{LVV-T587} tester: 
    \item Test case \href{https://jira.lsstcorp.org/secure/Tests.jspa#/testCase/LVV-T554}{LVV-T554} tester: 
    \item Test case \href{https://jira.lsstcorp.org/secure/Tests.jspa#/testCase/LVV-T548}{LVV-T548} tester: 
    \item Test case \href{https://jira.lsstcorp.org/secure/Tests.jspa#/testCase/LVV-T545}{LVV-T545} tester: 
    \item Test case \href{https://jira.lsstcorp.org/secure/Tests.jspa#/testCase/LVV-T389}{LVV-T389} tester: 
    \item Test case \href{https://jira.lsstcorp.org/secure/Tests.jspa#/testCase/LVV-T390}{LVV-T390} tester: 
    \item Test case \href{https://jira.lsstcorp.org/secure/Tests.jspa#/testCase/LVV-T297}{LVV-T297} tester: Keith Bechtol
    \item Test case \href{https://jira.lsstcorp.org/secure/Tests.jspa#/testCase/LVV-T298}{LVV-T298} tester: Keith Bechtol
    \item Test case \href{https://jira.lsstcorp.org/secure/Tests.jspa#/testCase/LVV-T299}{LVV-T299} tester: Keith Bechtol
    \item Test case \href{https://jira.lsstcorp.org/secure/Tests.jspa#/testCase/LVV-T360}{LVV-T360} tester: Keith Bechtol
  \end{itemize}
  \item LVV-C5 owner: 
    Undefined
  \begin{itemize}
    \item Test case \href{https://jira.lsstcorp.org/secure/Tests.jspa#/testCase/LVV-T986}{LVV-T986} tester: 
    \item Test case \href{https://jira.lsstcorp.org/secure/Tests.jspa#/testCase/LVV-T969}{LVV-T969} tester: 
    \item Test case \href{https://jira.lsstcorp.org/secure/Tests.jspa#/testCase/LVV-T968}{LVV-T968} tester: 
    \item Test case \href{https://jira.lsstcorp.org/secure/Tests.jspa#/testCase/LVV-T176}{LVV-T176} tester: 
    \item Test case \href{https://jira.lsstcorp.org/secure/Tests.jspa#/testCase/LVV-T966}{LVV-T966} tester: 
    \item Test case \href{https://jira.lsstcorp.org/secure/Tests.jspa#/testCase/LVV-T963}{LVV-T963} tester: 
    \item Test case \href{https://jira.lsstcorp.org/secure/Tests.jspa#/testCase/LVV-T962}{LVV-T962} tester: 
    \item Test case \href{https://jira.lsstcorp.org/secure/Tests.jspa#/testCase/LVV-T950}{LVV-T950} tester: 
    \item Test case \href{https://jira.lsstcorp.org/secure/Tests.jspa#/testCase/LVV-T943}{LVV-T943} tester: 
    \item Test case \href{https://jira.lsstcorp.org/secure/Tests.jspa#/testCase/LVV-T942}{LVV-T942} tester: 
    \item Test case \href{https://jira.lsstcorp.org/secure/Tests.jspa#/testCase/LVV-T941}{LVV-T941} tester: 
    \item Test case \href{https://jira.lsstcorp.org/secure/Tests.jspa#/testCase/LVV-T939}{LVV-T939} tester: 
    \item Test case \href{https://jira.lsstcorp.org/secure/Tests.jspa#/testCase/LVV-T940}{LVV-T940} tester: 
    \item Test case \href{https://jira.lsstcorp.org/secure/Tests.jspa#/testCase/LVV-T938}{LVV-T938} tester: 
    \item Test case \href{https://jira.lsstcorp.org/secure/Tests.jspa#/testCase/LVV-T597}{LVV-T597} tester: 
    \item Test case \href{https://jira.lsstcorp.org/secure/Tests.jspa#/testCase/LVV-T596}{LVV-T596} tester: 
    \item Test case \href{https://jira.lsstcorp.org/secure/Tests.jspa#/testCase/LVV-T549}{LVV-T549} tester: 
    \item Test case \href{https://jira.lsstcorp.org/secure/Tests.jspa#/testCase/LVV-T547}{LVV-T547} tester: 
    \item Test case \href{https://jira.lsstcorp.org/secure/Tests.jspa#/testCase/LVV-T544}{LVV-T544} tester: 
    \item Test case \href{https://jira.lsstcorp.org/secure/Tests.jspa#/testCase/LVV-T532}{LVV-T532} tester: 
    \item Test case \href{https://jira.lsstcorp.org/secure/Tests.jspa#/testCase/LVV-T533}{LVV-T533} tester: 
    \item Test case \href{https://jira.lsstcorp.org/secure/Tests.jspa#/testCase/LVV-T294}{LVV-T294} tester: Keith Bechtol
    \item Test case \href{https://jira.lsstcorp.org/secure/Tests.jspa#/testCase/LVV-T296}{LVV-T296} tester: Keith Bechtol
  \end{itemize}
  \item LVV-C36 owner: 
    Undefined
  \begin{itemize}
    \item Test case \href{https://jira.lsstcorp.org/secure/Tests.jspa#/testCase/LVV-T1071}{LVV-T1071} tester: 
  \end{itemize}
  \item LVV-C37 owner: 
    Undefined
  \begin{itemize}
    \item Test case \href{https://jira.lsstcorp.org/secure/Tests.jspa#/testCase/LVV-T965}{LVV-T965} tester: 
    \item Test case \href{https://jira.lsstcorp.org/secure/Tests.jspa#/testCase/LVV-T964}{LVV-T964} tester: 
    \item Test case \href{https://jira.lsstcorp.org/secure/Tests.jspa#/testCase/LVV-T967}{LVV-T967} tester: 
  \end{itemize}
\end{itemize}
\item Additional Test Personnel involved:
  \begin{itemize}
    \item Test case \href{https://jira.lsstcorp.org/secure/Tests.jspa#/testCase/LVV-T176}{LVV-T176}: 
    \item Test case \href{https://jira.lsstcorp.org/secure/Tests.jspa#/testCase/LVV-T293}{LVV-T293}: 
    \item Test case \href{https://jira.lsstcorp.org/secure/Tests.jspa#/testCase/LVV-T294}{LVV-T294}: 
    \item Test case \href{https://jira.lsstcorp.org/secure/Tests.jspa#/testCase/LVV-T295}{LVV-T295}: 
    \item Test case \href{https://jira.lsstcorp.org/secure/Tests.jspa#/testCase/LVV-T296}{LVV-T296}: 
    \item Test case \href{https://jira.lsstcorp.org/secure/Tests.jspa#/testCase/LVV-T297}{LVV-T297}: 
    \item Test case \href{https://jira.lsstcorp.org/secure/Tests.jspa#/testCase/LVV-T298}{LVV-T298}: 
    \item Test case \href{https://jira.lsstcorp.org/secure/Tests.jspa#/testCase/LVV-T299}{LVV-T299}: 
    \item Test case \href{https://jira.lsstcorp.org/secure/Tests.jspa#/testCase/LVV-T360}{LVV-T360}: 
    \item Test case \href{https://jira.lsstcorp.org/secure/Tests.jspa#/testCase/LVV-T389}{LVV-T389}: 
    \item Test case \href{https://jira.lsstcorp.org/secure/Tests.jspa#/testCase/LVV-T390}{LVV-T390}: 
    \item Test case \href{https://jira.lsstcorp.org/secure/Tests.jspa#/testCase/LVV-T532}{LVV-T532}: 
    \item Test case \href{https://jira.lsstcorp.org/secure/Tests.jspa#/testCase/LVV-T533}{LVV-T533}: 
    \item Test case \href{https://jira.lsstcorp.org/secure/Tests.jspa#/testCase/LVV-T544}{LVV-T544}: 
    \item Test case \href{https://jira.lsstcorp.org/secure/Tests.jspa#/testCase/LVV-T545}{LVV-T545}: 
    \item Test case \href{https://jira.lsstcorp.org/secure/Tests.jspa#/testCase/LVV-T547}{LVV-T547}: 
    \item Test case \href{https://jira.lsstcorp.org/secure/Tests.jspa#/testCase/LVV-T548}{LVV-T548}: 
    \item Test case \href{https://jira.lsstcorp.org/secure/Tests.jspa#/testCase/LVV-T549}{LVV-T549}: 
    \item Test case \href{https://jira.lsstcorp.org/secure/Tests.jspa#/testCase/LVV-T554}{LVV-T554}: 
    \item Test case \href{https://jira.lsstcorp.org/secure/Tests.jspa#/testCase/LVV-T587}{LVV-T587}: 
    \item Test case \href{https://jira.lsstcorp.org/secure/Tests.jspa#/testCase/LVV-T588}{LVV-T588}: 
    \item Test case \href{https://jira.lsstcorp.org/secure/Tests.jspa#/testCase/LVV-T589}{LVV-T589}: 
    \item Test case \href{https://jira.lsstcorp.org/secure/Tests.jspa#/testCase/LVV-T590}{LVV-T590}: 
    \item Test case \href{https://jira.lsstcorp.org/secure/Tests.jspa#/testCase/LVV-T591}{LVV-T591}: 
    \item Test case \href{https://jira.lsstcorp.org/secure/Tests.jspa#/testCase/LVV-T592}{LVV-T592}: 
    \item Test case \href{https://jira.lsstcorp.org/secure/Tests.jspa#/testCase/LVV-T593}{LVV-T593}: 
    \item Test case \href{https://jira.lsstcorp.org/secure/Tests.jspa#/testCase/LVV-T594}{LVV-T594}: 
    \item Test case \href{https://jira.lsstcorp.org/secure/Tests.jspa#/testCase/LVV-T595}{LVV-T595}: 
    \item Test case \href{https://jira.lsstcorp.org/secure/Tests.jspa#/testCase/LVV-T596}{LVV-T596}: 
    \item Test case \href{https://jira.lsstcorp.org/secure/Tests.jspa#/testCase/LVV-T597}{LVV-T597}: 
    \item Test case \href{https://jira.lsstcorp.org/secure/Tests.jspa#/testCase/LVV-T938}{LVV-T938}: 
    \item Test case \href{https://jira.lsstcorp.org/secure/Tests.jspa#/testCase/LVV-T939}{LVV-T939}: 
    \item Test case \href{https://jira.lsstcorp.org/secure/Tests.jspa#/testCase/LVV-T940}{LVV-T940}: 
    \item Test case \href{https://jira.lsstcorp.org/secure/Tests.jspa#/testCase/LVV-T941}{LVV-T941}: 
    \item Test case \href{https://jira.lsstcorp.org/secure/Tests.jspa#/testCase/LVV-T942}{LVV-T942}: 
    \item Test case \href{https://jira.lsstcorp.org/secure/Tests.jspa#/testCase/LVV-T943}{LVV-T943}: 
    \item Test case \href{https://jira.lsstcorp.org/secure/Tests.jspa#/testCase/LVV-T950}{LVV-T950}: 
    \item Test case \href{https://jira.lsstcorp.org/secure/Tests.jspa#/testCase/LVV-T956}{LVV-T956}: 
    \item Test case \href{https://jira.lsstcorp.org/secure/Tests.jspa#/testCase/LVV-T957}{LVV-T957}: 
    \item Test case \href{https://jira.lsstcorp.org/secure/Tests.jspa#/testCase/LVV-T959}{LVV-T959}: 
    \item Test case \href{https://jira.lsstcorp.org/secure/Tests.jspa#/testCase/LVV-T960}{LVV-T960}: 
    \item Test case \href{https://jira.lsstcorp.org/secure/Tests.jspa#/testCase/LVV-T961}{LVV-T961}: 
    \item Test case \href{https://jira.lsstcorp.org/secure/Tests.jspa#/testCase/LVV-T962}{LVV-T962}: 
    \item Test case \href{https://jira.lsstcorp.org/secure/Tests.jspa#/testCase/LVV-T963}{LVV-T963}: 
    \item Test case \href{https://jira.lsstcorp.org/secure/Tests.jspa#/testCase/LVV-T964}{LVV-T964}: 
    \item Test case \href{https://jira.lsstcorp.org/secure/Tests.jspa#/testCase/LVV-T965}{LVV-T965}: 
    \item Test case \href{https://jira.lsstcorp.org/secure/Tests.jspa#/testCase/LVV-T966}{LVV-T966}: 
    \item Test case \href{https://jira.lsstcorp.org/secure/Tests.jspa#/testCase/LVV-T967}{LVV-T967}: 
    \item Test case \href{https://jira.lsstcorp.org/secure/Tests.jspa#/testCase/LVV-T968}{LVV-T968}: 
    \item Test case \href{https://jira.lsstcorp.org/secure/Tests.jspa#/testCase/LVV-T969}{LVV-T969}: 
    \item Test case \href{https://jira.lsstcorp.org/secure/Tests.jspa#/testCase/LVV-T986}{LVV-T986}: 
    \item Test case \href{https://jira.lsstcorp.org/secure/Tests.jspa#/testCase/LVV-T1071}{LVV-T1071}: 
  \end{itemize}
\end{itemize}

\newpage

\section{Overview of the Test Results}
\label{sect:overview}

\subsection{Summary}
\label{sect:summarytable}

\begin{longtable}{p{0.12\textwidth}p{0.2\textwidth}p{0.56\textwidth}p{0.12\textwidth}}
\toprule

  \multicolumn{3}{c}{ Test Cycle {\bf LVV-C4: Commissioning SV: Single-visit Performance w/ ComCam
 }} \\\hline

  {\bf \footnotesize test case} & {\bf \footnotesize status} & {\bf \footnotesize comment} & {\bf \footnotesize issues} \\\toprule

    \href{https://jira.lsstcorp.org/secure/Tests.jspa#/testCase/LVV-T293}{LVV-T293}
    & Not Executed &  &
    \\\hline
    \href{https://jira.lsstcorp.org/secure/Tests.jspa#/testCase/LVV-T295}{LVV-T295}
    & Not Executed &  &
    \\\hline
    \href{https://jira.lsstcorp.org/secure/Tests.jspa#/testCase/LVV-T961}{LVV-T961}
    & Not Executed &  &
    \\\hline
    \href{https://jira.lsstcorp.org/secure/Tests.jspa#/testCase/LVV-T959}{LVV-T959}
    & Not Executed &  &
    \\\hline
    \href{https://jira.lsstcorp.org/secure/Tests.jspa#/testCase/LVV-T960}{LVV-T960}
    & Not Executed &  &
    \\\hline
    \href{https://jira.lsstcorp.org/secure/Tests.jspa#/testCase/LVV-T956}{LVV-T956}
    & Not Executed &  &
    \\\hline
    \href{https://jira.lsstcorp.org/secure/Tests.jspa#/testCase/LVV-T957}{LVV-T957}
    & Not Executed &  &
    \\\hline
    \href{https://jira.lsstcorp.org/secure/Tests.jspa#/testCase/LVV-T595}{LVV-T595}
    & Not Executed &  &
    \\\hline
    \href{https://jira.lsstcorp.org/secure/Tests.jspa#/testCase/LVV-T594}{LVV-T594}
    & Not Executed &  &
    \\\hline
    \href{https://jira.lsstcorp.org/secure/Tests.jspa#/testCase/LVV-T593}{LVV-T593}
    & Not Executed &  &
    \\\hline
    \href{https://jira.lsstcorp.org/secure/Tests.jspa#/testCase/LVV-T592}{LVV-T592}
    & Not Executed &  &
    \\\hline
    \href{https://jira.lsstcorp.org/secure/Tests.jspa#/testCase/LVV-T591}{LVV-T591}
    & Not Executed &  &
    \\\hline
    \href{https://jira.lsstcorp.org/secure/Tests.jspa#/testCase/LVV-T590}{LVV-T590}
    & Not Executed &  &
    \\\hline
    \href{https://jira.lsstcorp.org/secure/Tests.jspa#/testCase/LVV-T589}{LVV-T589}
    & Not Executed &  &
    \\\hline
    \href{https://jira.lsstcorp.org/secure/Tests.jspa#/testCase/LVV-T588}{LVV-T588}
    & Not Executed &  &
    \\\hline
    \href{https://jira.lsstcorp.org/secure/Tests.jspa#/testCase/LVV-T587}{LVV-T587}
    & Not Executed &  &
    \\\hline
    \href{https://jira.lsstcorp.org/secure/Tests.jspa#/testCase/LVV-T554}{LVV-T554}
    & Not Executed &  &
    \\\hline
    \href{https://jira.lsstcorp.org/secure/Tests.jspa#/testCase/LVV-T548}{LVV-T548}
    & Not Executed &  &
    \\\hline
    \href{https://jira.lsstcorp.org/secure/Tests.jspa#/testCase/LVV-T545}{LVV-T545}
    & Not Executed &  &
    \\\hline
    \href{https://jira.lsstcorp.org/secure/Tests.jspa#/testCase/LVV-T389}{LVV-T389}
    & Not Executed &  &
    \\\hline
    \href{https://jira.lsstcorp.org/secure/Tests.jspa#/testCase/LVV-T390}{LVV-T390}
    & Not Executed &  &
    \\\hline
    \href{https://jira.lsstcorp.org/secure/Tests.jspa#/testCase/LVV-T297}{LVV-T297}
    & Not Executed &  &
    \\\hline
    \href{https://jira.lsstcorp.org/secure/Tests.jspa#/testCase/LVV-T298}{LVV-T298}
    & Not Executed &  &
    \\\hline
    \href{https://jira.lsstcorp.org/secure/Tests.jspa#/testCase/LVV-T299}{LVV-T299}
    & Not Executed &  &
    \\\hline
    \href{https://jira.lsstcorp.org/secure/Tests.jspa#/testCase/LVV-T360}{LVV-T360}
    & Not Executed &  &
    \\\hline

  \multicolumn{3}{c}{ Test Cycle {\bf LVV-C5: Commissioning SV: Full-survey Performance w/ ComCam
 }} \\\hline

  {\bf \footnotesize test case} & {\bf \footnotesize status} & {\bf \footnotesize comment} & {\bf \footnotesize issues} \\\toprule

    \href{https://jira.lsstcorp.org/secure/Tests.jspa#/testCase/LVV-T986}{LVV-T986}
    & Not Executed &  &
    \\\hline
    \href{https://jira.lsstcorp.org/secure/Tests.jspa#/testCase/LVV-T969}{LVV-T969}
    & Not Executed &  &
    \\\hline
    \href{https://jira.lsstcorp.org/secure/Tests.jspa#/testCase/LVV-T968}{LVV-T968}
    & Not Executed &  &
    \\\hline
    \href{https://jira.lsstcorp.org/secure/Tests.jspa#/testCase/LVV-T176}{LVV-T176}
    & Not Executed &  &
    \\\hline
    \href{https://jira.lsstcorp.org/secure/Tests.jspa#/testCase/LVV-T966}{LVV-T966}
    & Not Executed &  &
    \\\hline
    \href{https://jira.lsstcorp.org/secure/Tests.jspa#/testCase/LVV-T963}{LVV-T963}
    & Not Executed &  &
    \\\hline
    \href{https://jira.lsstcorp.org/secure/Tests.jspa#/testCase/LVV-T962}{LVV-T962}
    & Not Executed &  &
    \\\hline
    \href{https://jira.lsstcorp.org/secure/Tests.jspa#/testCase/LVV-T950}{LVV-T950}
    & Not Executed &  &
    \\\hline
    \href{https://jira.lsstcorp.org/secure/Tests.jspa#/testCase/LVV-T943}{LVV-T943}
    & Not Executed &  &
    \\\hline
    \href{https://jira.lsstcorp.org/secure/Tests.jspa#/testCase/LVV-T942}{LVV-T942}
    & Not Executed &  &
    \\\hline
    \href{https://jira.lsstcorp.org/secure/Tests.jspa#/testCase/LVV-T941}{LVV-T941}
    & Not Executed &  &
    \\\hline
    \href{https://jira.lsstcorp.org/secure/Tests.jspa#/testCase/LVV-T939}{LVV-T939}
    & Not Executed &  &
    \\\hline
    \href{https://jira.lsstcorp.org/secure/Tests.jspa#/testCase/LVV-T940}{LVV-T940}
    & Not Executed &  &
    \\\hline
    \href{https://jira.lsstcorp.org/secure/Tests.jspa#/testCase/LVV-T938}{LVV-T938}
    & Not Executed &  &
    \\\hline
    \href{https://jira.lsstcorp.org/secure/Tests.jspa#/testCase/LVV-T597}{LVV-T597}
    & Not Executed &  &
    \\\hline
    \href{https://jira.lsstcorp.org/secure/Tests.jspa#/testCase/LVV-T596}{LVV-T596}
    & Not Executed &  &
    \\\hline
    \href{https://jira.lsstcorp.org/secure/Tests.jspa#/testCase/LVV-T549}{LVV-T549}
    & Not Executed &  &
    \\\hline
    \href{https://jira.lsstcorp.org/secure/Tests.jspa#/testCase/LVV-T547}{LVV-T547}
    & Not Executed &  &
    \\\hline
    \href{https://jira.lsstcorp.org/secure/Tests.jspa#/testCase/LVV-T544}{LVV-T544}
    & Not Executed &  &
    \\\hline
    \href{https://jira.lsstcorp.org/secure/Tests.jspa#/testCase/LVV-T532}{LVV-T532}
    & Not Executed &  &
    \\\hline
    \href{https://jira.lsstcorp.org/secure/Tests.jspa#/testCase/LVV-T533}{LVV-T533}
    & Not Executed &  &
    \\\hline
    \href{https://jira.lsstcorp.org/secure/Tests.jspa#/testCase/LVV-T294}{LVV-T294}
    & Not Executed &  &
    \\\hline
    \href{https://jira.lsstcorp.org/secure/Tests.jspa#/testCase/LVV-T296}{LVV-T296}
    & Not Executed &  &
    \\\hline

  \multicolumn{3}{c}{ Test Cycle {\bf LVV-C36: Commissioning SV: 20-year Depth w/ ComCam
 }} \\\hline

  {\bf \footnotesize test case} & {\bf \footnotesize status} & {\bf \footnotesize comment} & {\bf \footnotesize issues} \\\toprule

    \href{https://jira.lsstcorp.org/secure/Tests.jspa#/testCase/LVV-T1071}{LVV-T1071}
    & Not Executed &  &
    \\\hline

  \multicolumn{3}{c}{ Test Cycle {\bf LVV-C37: Commissioning SV: Scheduler Testing w/ ComCam
 }} \\\hline

  {\bf \footnotesize test case} & {\bf \footnotesize status} & {\bf \footnotesize comment} & {\bf \footnotesize issues} \\\toprule

    \href{https://jira.lsstcorp.org/secure/Tests.jspa#/testCase/LVV-T965}{LVV-T965}
    & Not Executed &  &
    \\\hline
    \href{https://jira.lsstcorp.org/secure/Tests.jspa#/testCase/LVV-T964}{LVV-T964}
    & Not Executed &  &
    \\\hline
    \href{https://jira.lsstcorp.org/secure/Tests.jspa#/testCase/LVV-T967}{LVV-T967}
    & Not Executed &  &
    \\\hline

\caption{Test Results Summary}
\label{table:summary}
\end{longtable}

\subsection{Overall Assessment}
\label{sect:overallassessment}

Not yet available.

\subsection{Recommended Improvements}
\label{sect:recommendations}

Not yet available.

\newpage
\section{Detailed Test Results}
\label{sect:detailedtestresults}


  \subsection{Test Cycle LVV-C4 }

Open test cycle {\it \href{https://jira.lsstcorp.org/secure/Tests.jspa#/testrun/LVV-C4}{Commissioning SV: Single-visit Performance w/ ComCam
}} in Jira.

  Commissioning SV: Single-visit Performance w/ ComCam
\\
  Status: Not Executed

  Initial verification of the single-visit performance of ComCam with
respect to

\begin{enumerate}
\tightlist
\item
  Delivered image quality
\item
  Photometric performance
\item
  Astrometric performance
\item
  Image depth
\end{enumerate}

The nominal planned on-sky observations for this test are 20 fields x 5
epochs x 5 visits x 6 filters = 3K visits (\textasciitilde{}4 nights of
observations)

\begin{itemize}
\tightlist
\item
  Several of the fields should contain absolute spectrophotometric
  calibration standards.
\item
  The fields should cover a range of airmasses and source densities.
\end{itemize}

These same observations are planned to be
\href{https://jira.lsstcorp.org/secure/Tests.jspa\#/testCycle/LVV-C6}{repeated
with LSSTCam} to enable verification tests across the full focal plane.


  \subsubsection{Software Version/Baseline}
    Not provided.

  \subsubsection{Configuration}
    Not provided.

  \subsubsection{Test Cases in LVV-C4 Test Cycle}


    \paragraph{Test Case LVV-T293 - On-sky Observations: Single-visit Key Performance Metrics
 }\mbox{}\\

Open  \href{https://jira.lsstcorp.org/secure/Tests.jspa#/testCase/LVV-T293}{\textit{ LVV-T293 } }
test case in Jira.

    Perform repeated observations of a set of 20 to 30 fields to evaluate
single-visit science performance metrics such as system throughput,
image quality, and astrometric and photometric repeatability. The fields
should be selected to span a range of object densities (e.g., by
sampling different Galactic latitudes) and should be observed under a
range of environmental conditions, including a range of airmass and sky
brightness. (WHAT RANGE?) The target fields should include
spectrophotometric standards such as DA white dwarfs to be used to
evaluate the absolute photometric calibration (HOW
MANY?)\\[2\baselineskip]The ~pointings in each field will be dithered to
allow tests of delivered image quality, throughput, calibration, and
astrometry across the full field of view. Photometric conditions are
required.\\[2\baselineskip]Sub-percent Photometry: Faint DA White Dwarf
Spectrophotometric Standards for Astrophysical Observatories\\
\url{https://arxiv.org/abs/1811.12534}\\[2\baselineskip]\textbf{Example
observations:}\\
20 fields x 5 visits x 6 filters x 5 epochs = 20 fields x 25 visits x 6
filters.


    \textbf{ Preconditions}:\\
    

    Execution status: {\bf Not Executed }

    Final comment:\\


    Detailed step results:

    \begin{longtable}{p{1cm}p{2cm}p{13cm}}
    \hline
    {Step} & \multicolumn{2}{c}{Description, Results and Status}\\ \hline
      1 & Description &

      \begin{minipage}[t]{13cm}{\footnotesize
      
      \vspace{\dp0}
      } \end{minipage} \\
      \\ \cdashline{2-3}

      & Expected Result & 

      \begin{minipage}[t]{13cm}{\footnotesize
      
      \vspace{\dp0}
      } \end{minipage} \\
      \\ \cdashline{2-3}

      & \begin{minipage}[t]{2cm}{Actual\\ Result}\end{minipage}   & 
      \begin{minipage}[t]{13cm}{\footnotesize
      
      \vspace{\dp0}
      } \end{minipage} \\
      \\ \cdashline{2-3}


      & Status          & Not Executed \\ \hline

    \end{longtable}


    \paragraph{Test Case LVV-T295 - Data Processing Campaign: Single-visit Key Performance Metrics
 }\mbox{}\\

Open  \href{https://jira.lsstcorp.org/secure/Tests.jspa#/testCase/LVV-T295}{\textit{ LVV-T295 } }
test case in Jira.

    

    \textbf{ Preconditions}:\\
    

    Execution status: {\bf Not Executed }

    Final comment:\\


    Detailed step results:

    \begin{longtable}{p{1cm}p{2cm}p{13cm}}
    \hline
    {Step} & \multicolumn{2}{c}{Description, Results and Status}\\ \hline
      1 & Description &

      \begin{minipage}[t]{13cm}{\footnotesize
      
      \vspace{\dp0}
      } \end{minipage} \\
      \\ \cdashline{2-3}

      & Expected Result & 

      \begin{minipage}[t]{13cm}{\footnotesize
      
      \vspace{\dp0}
      } \end{minipage} \\
      \\ \cdashline{2-3}

      & \begin{minipage}[t]{2cm}{Actual\\ Result}\end{minipage}   & 
      \begin{minipage}[t]{13cm}{\footnotesize
      
      \vspace{\dp0}
      } \end{minipage} \\
      \\ \cdashline{2-3}


      & Status          & Not Executed \\ \hline

    \end{longtable}


    \paragraph{Test Case LVV-T961 - Bright source measurement
 }\mbox{}\\

Open  \href{https://jira.lsstcorp.org/secure/Tests.jspa#/testCase/LVV-T961}{\textit{ LVV-T961 } }
test case in Jira.

    Verify that we can adjust the exposure time to enable measurements of
sources brighter than the nominal LSST saturation limit


    \textbf{ Preconditions}:\\
    

    Execution status: {\bf Not Executed }

    Final comment:\\


    Detailed step results:

    \begin{longtable}{p{1cm}p{2cm}p{13cm}}
    \hline
    {Step} & \multicolumn{2}{c}{Description, Results and Status}\\ \hline
      1 & Description &

      \begin{minipage}[t]{13cm}{\footnotesize
      For each band, calculate the expected saturation limit for 15 second
exposures.

      \vspace{\dp0}
      } \end{minipage} \\
      \\ \cdashline{2-3}

      & Expected Result & 

      \begin{minipage}[t]{13cm}{\footnotesize
      A list of nominal saturation limits.

      \vspace{\dp0}
      } \end{minipage} \\
      \\ \cdashline{2-3}

      & \begin{minipage}[t]{2cm}{Actual\\ Result}\end{minipage}   & 
      \begin{minipage}[t]{13cm}{\footnotesize
      
      \vspace{\dp0}
      } \end{minipage} \\
      \\ \cdashline{2-3}


      & Status          & Not Executed \\ \hline

      2 & Description &

      \begin{minipage}[t]{13cm}{\footnotesize
      Identify sources brighter than the saturation limits calculated in step
1 in external catalogs.\\[2\baselineskip]This will probably involve
fitting the stars to an SED so that we can extrapolate their magnitudes
in LSST bands as needed.

      \vspace{\dp0}
      } \end{minipage} \\
      \\ \cdashline{2-3}

      & Expected Result & 

      \begin{minipage}[t]{13cm}{\footnotesize
      Catalog of bright sources to be used for this test

      \vspace{\dp0}
      } \end{minipage} \\
      \\ \cdashline{2-3}

      & \begin{minipage}[t]{2cm}{Actual\\ Result}\end{minipage}   & 
      \begin{minipage}[t]{13cm}{\footnotesize
      
      \vspace{\dp0}
      } \end{minipage} \\
      \\ \cdashline{2-3}


      & Status          & Not Executed \\ \hline

      3 & Description &

      \begin{minipage}[t]{13cm}{\footnotesize
      Calculate the exposure time needed to take unsaturated images of the
bright sources identified in step 2

      \vspace{\dp0}
      } \end{minipage} \\
      \\ \cdashline{2-3}

      & Expected Result & 

      \begin{minipage}[t]{13cm}{\footnotesize
      List of exposure times to be used in this test

      \vspace{\dp0}
      } \end{minipage} \\
      \\ \cdashline{2-3}

      & \begin{minipage}[t]{2cm}{Actual\\ Result}\end{minipage}   & 
      \begin{minipage}[t]{13cm}{\footnotesize
      
      \vspace{\dp0}
      } \end{minipage} \\
      \\ \cdashline{2-3}


      & Status          & Not Executed \\ \hline

      4 & Description &

      \begin{minipage}[t]{13cm}{\footnotesize
      In each band, take images of bright sources 1 magnitude brighter than 15
second saturation limit in that band, using the exposure time calculated
in step 3

      \vspace{\dp0}
      } \end{minipage} \\
      \\ \cdashline{2-3}

      & Expected Result & 

      \begin{minipage}[t]{13cm}{\footnotesize
      Images of bright sources in each LSST band

      \vspace{\dp0}
      } \end{minipage} \\
      \\ \cdashline{2-3}

      & \begin{minipage}[t]{2cm}{Actual\\ Result}\end{minipage}   & 
      \begin{minipage}[t]{13cm}{\footnotesize
      
      \vspace{\dp0}
      } \end{minipage} \\
      \\ \cdashline{2-3}


      & Status          & Not Executed \\ \hline

      5 & Description &

      \begin{minipage}[t]{13cm}{\footnotesize
      Perform single image processing on the images from step 4. ~Verify that
the measured magnitudes of the bright sources agree with the magnitudes
inferred from the external catalog used in step 2. ~This will indicate
that we have successfully taken a science-quality image of the bright
source.

      \vspace{\dp0}
      } \end{minipage} \\
      \\ \cdashline{2-3}

      & Expected Result & 

      \begin{minipage}[t]{13cm}{\footnotesize
      
      \vspace{\dp0}
      } \end{minipage} \\
      \\ \cdashline{2-3}

      & \begin{minipage}[t]{2cm}{Actual\\ Result}\end{minipage}   & 
      \begin{minipage}[t]{13cm}{\footnotesize
      
      \vspace{\dp0}
      } \end{minipage} \\
      \\ \cdashline{2-3}


      & Status          & Not Executed \\ \hline

    \end{longtable}


    \paragraph{Test Case LVV-T959 - Inter-band astrometric consistency
 }\mbox{}\\

Open  \href{https://jira.lsstcorp.org/secure/Tests.jspa#/testCase/LVV-T959}{\textit{ LVV-T959 } }
test case in Jira.

    Verify that the separations between objects do not vary significantly
with band


    \textbf{ Preconditions}:\\
    

    Execution status: {\bf Not Executed }

    Final comment:\\


    Detailed step results:

    \begin{longtable}{p{1cm}p{2cm}p{13cm}}
    \hline
    {Step} & \multicolumn{2}{c}{Description, Results and Status}\\ \hline
      1 & Description &

      \begin{minipage}[t]{13cm}{\footnotesize
      Image an average field in all six bands

      \vspace{\dp0}
      } \end{minipage} \\
      \\ \cdashline{2-3}

      & Expected Result & 

      \begin{minipage}[t]{13cm}{\footnotesize
      Set of images

      \vspace{\dp0}
      } \end{minipage} \\
      \\ \cdashline{2-3}

      & \begin{minipage}[t]{2cm}{Actual\\ Result}\end{minipage}   & 
      \begin{minipage}[t]{13cm}{\footnotesize
      
      \vspace{\dp0}
      } \end{minipage} \\
      \\ \cdashline{2-3}


      & Status          & Not Executed \\ \hline

      2 & Description &

      \begin{minipage}[t]{13cm}{\footnotesize
      Perform source detection and astrometric measurements on the images from
step 1

      \vspace{\dp0}
      } \end{minipage} \\
      \\ \cdashline{2-3}

      & Expected Result & 

      \begin{minipage}[t]{13cm}{\footnotesize
      Catalog of sources in images from step 1

      \vspace{\dp0}
      } \end{minipage} \\
      \\ \cdashline{2-3}

      & \begin{minipage}[t]{2cm}{Actual\\ Result}\end{minipage}   & 
      \begin{minipage}[t]{13cm}{\footnotesize
      
      \vspace{\dp0}
      } \end{minipage} \\
      \\ \cdashline{2-3}


      & Status          & Not Executed \\ \hline

      3 & Description &

      \begin{minipage}[t]{13cm}{\footnotesize
      Find separations between all pairs of sources in catalogs from step 2

      \vspace{\dp0}
      } \end{minipage} \\
      \\ \cdashline{2-3}

      & Expected Result & 

      \begin{minipage}[t]{13cm}{\footnotesize
      Measurements of source separations in each band

      \vspace{\dp0}
      } \end{minipage} \\
      \\ \cdashline{2-3}

      & \begin{minipage}[t]{2cm}{Actual\\ Result}\end{minipage}   & 
      \begin{minipage}[t]{13cm}{\footnotesize
      
      \vspace{\dp0}
      } \end{minipage} \\
      \\ \cdashline{2-3}


      & Status          & Not Executed \\ \hline

      4 & Description &

      \begin{minipage}[t]{13cm}{\footnotesize
      For each band, compute the RMS difference in source separations relative
to the r-band. ~Verify that this values is less than or equal to 10
milliarcseconds.

      \vspace{\dp0}
      } \end{minipage} \\
      \\ \cdashline{2-3}

      & Expected Result & 

      \begin{minipage}[t]{13cm}{\footnotesize
      
      \vspace{\dp0}
      } \end{minipage} \\
      \\ \cdashline{2-3}

      & \begin{minipage}[t]{2cm}{Actual\\ Result}\end{minipage}   & 
      \begin{minipage}[t]{13cm}{\footnotesize
      
      \vspace{\dp0}
      } \end{minipage} \\
      \\ \cdashline{2-3}


      & Status          & Not Executed \\ \hline

      5 & Description &

      \begin{minipage}[t]{13cm}{\footnotesize
      Verify that no more than 10 percent of source separation measurements in
any band vary by more than 20 milliarcseconds from the r band
measurements

      \vspace{\dp0}
      } \end{minipage} \\
      \\ \cdashline{2-3}

      & Expected Result & 

      \begin{minipage}[t]{13cm}{\footnotesize
      
      \vspace{\dp0}
      } \end{minipage} \\
      \\ \cdashline{2-3}

      & \begin{minipage}[t]{2cm}{Actual\\ Result}\end{minipage}   & 
      \begin{minipage}[t]{13cm}{\footnotesize
      
      \vspace{\dp0}
      } \end{minipage} \\
      \\ \cdashline{2-3}


      & Status          & Not Executed \\ \hline

    \end{longtable}


    \paragraph{Test Case LVV-T960 - Relative astrometric performance
 }\mbox{}\\

Open  \href{https://jira.lsstcorp.org/secure/Tests.jspa#/testCase/LVV-T960}{\textit{ LVV-T960 } }
test case in Jira.

    Verify that relative astrometric separations are as accurate as
specified


    \textbf{ Preconditions}:\\
    

    Execution status: {\bf Not Executed }

    Final comment:\\


    Detailed step results:

    \begin{longtable}{p{1cm}p{2cm}p{13cm}}
    \hline
    {Step} & \multicolumn{2}{c}{Description, Results and Status}\\ \hline
      1 & Description &

      \begin{minipage}[t]{13cm}{\footnotesize
      Image a region that overlaps the Gaia footprint (we will use Gaia as
astrometric truth)

      \vspace{\dp0}
      } \end{minipage} \\
      \\ \cdashline{2-3}

      & Expected Result & 

      \begin{minipage}[t]{13cm}{\footnotesize
      Images taken from Gaia region

      \vspace{\dp0}
      } \end{minipage} \\
      \\ \cdashline{2-3}

      & \begin{minipage}[t]{2cm}{Actual\\ Result}\end{minipage}   & 
      \begin{minipage}[t]{13cm}{\footnotesize
      
      \vspace{\dp0}
      } \end{minipage} \\
      \\ \cdashline{2-3}


      & Status          & Not Executed \\ \hline

      2 & Description &

      \begin{minipage}[t]{13cm}{\footnotesize
      Run source detection and astrometric measurement on images from step 1

      \vspace{\dp0}
      } \end{minipage} \\
      \\ \cdashline{2-3}

      & Expected Result & 

      \begin{minipage}[t]{13cm}{\footnotesize
      Catalog of sources detected in images from step 1

      \vspace{\dp0}
      } \end{minipage} \\
      \\ \cdashline{2-3}

      & \begin{minipage}[t]{2cm}{Actual\\ Result}\end{minipage}   & 
      \begin{minipage}[t]{13cm}{\footnotesize
      
      \vspace{\dp0}
      } \end{minipage} \\
      \\ \cdashline{2-3}


      & Status          & Not Executed \\ \hline

      3 & Description &

      \begin{minipage}[t]{13cm}{\footnotesize
      Calculate the separation between all sources detected in step 2

      \vspace{\dp0}
      } \end{minipage} \\
      \\ \cdashline{2-3}

      & Expected Result & 

      \begin{minipage}[t]{13cm}{\footnotesize
      Measurements of source pair separations

      \vspace{\dp0}
      } \end{minipage} \\
      \\ \cdashline{2-3}

      & \begin{minipage}[t]{2cm}{Actual\\ Result}\end{minipage}   & 
      \begin{minipage}[t]{13cm}{\footnotesize
      
      \vspace{\dp0}
      } \end{minipage} \\
      \\ \cdashline{2-3}


      & Status          & Not Executed \\ \hline

      4 & Description &

      \begin{minipage}[t]{13cm}{\footnotesize
      Compare source separations from step 3 to the same source separations as
measured by Gaia

      \vspace{\dp0}
      } \end{minipage} \\
      \\ \cdashline{2-3}

      & Expected Result & 

      \begin{minipage}[t]{13cm}{\footnotesize
      Distribution of astrometric errors relative to Gaia

      \vspace{\dp0}
      } \end{minipage} \\
      \\ \cdashline{2-3}

      & \begin{minipage}[t]{2cm}{Actual\\ Result}\end{minipage}   & 
      \begin{minipage}[t]{13cm}{\footnotesize
      
      \vspace{\dp0}
      } \end{minipage} \\
      \\ \cdashline{2-3}


      & Status          & Not Executed \\ \hline

      5 & Description &

      \begin{minipage}[t]{13cm}{\footnotesize
      Examine distribution of source separation errors from step 4 for all
pairs of sources separated by \textasciitilde{} 5 arcminutes. ~Verify
that the median error in these measurements is \textless{}= 10
milliarcseconds

      \vspace{\dp0}
      } \end{minipage} \\
      \\ \cdashline{2-3}

      & Expected Result & 

      \begin{minipage}[t]{13cm}{\footnotesize
      
      \vspace{\dp0}
      } \end{minipage} \\
      \\ \cdashline{2-3}

      & \begin{minipage}[t]{2cm}{Actual\\ Result}\end{minipage}   & 
      \begin{minipage}[t]{13cm}{\footnotesize
      
      \vspace{\dp0}
      } \end{minipage} \\
      \\ \cdashline{2-3}


      & Status          & Not Executed \\ \hline

      6 & Description &

      \begin{minipage}[t]{13cm}{\footnotesize
      Verify that no more than 10\% of the source pairs separated by
\textasciitilde{} 5 arcminutes have separation errors greater than 20
milliarcseconds

      \vspace{\dp0}
      } \end{minipage} \\
      \\ \cdashline{2-3}

      & Expected Result & 

      \begin{minipage}[t]{13cm}{\footnotesize
      
      \vspace{\dp0}
      } \end{minipage} \\
      \\ \cdashline{2-3}

      & \begin{minipage}[t]{2cm}{Actual\\ Result}\end{minipage}   & 
      \begin{minipage}[t]{13cm}{\footnotesize
      
      \vspace{\dp0}
      } \end{minipage} \\
      \\ \cdashline{2-3}


      & Status          & Not Executed \\ \hline

      7 & Description &

      \begin{minipage}[t]{13cm}{\footnotesize
      Examine distribution of source separation errors from step 4 for all
pairs of sources separated by \textasciitilde{} 20 arcminutes. ~Verify
that the median error in these measurements is \textless{}= 10
milliarcseconds

      \vspace{\dp0}
      } \end{minipage} \\
      \\ \cdashline{2-3}

      & Expected Result & 

      \begin{minipage}[t]{13cm}{\footnotesize
      
      \vspace{\dp0}
      } \end{minipage} \\
      \\ \cdashline{2-3}

      & \begin{minipage}[t]{2cm}{Actual\\ Result}\end{minipage}   & 
      \begin{minipage}[t]{13cm}{\footnotesize
      
      \vspace{\dp0}
      } \end{minipage} \\
      \\ \cdashline{2-3}


      & Status          & Not Executed \\ \hline

      8 & Description &

      \begin{minipage}[t]{13cm}{\footnotesize
      Verify that no more than 10 percent of source pairs separated by
\textasciitilde{} 20 arcminutes have source separation errors greater
than 20 milliarcseconds

      \vspace{\dp0}
      } \end{minipage} \\
      \\ \cdashline{2-3}

      & Expected Result & 

      \begin{minipage}[t]{13cm}{\footnotesize
      
      \vspace{\dp0}
      } \end{minipage} \\
      \\ \cdashline{2-3}

      & \begin{minipage}[t]{2cm}{Actual\\ Result}\end{minipage}   & 
      \begin{minipage}[t]{13cm}{\footnotesize
      
      \vspace{\dp0}
      } \end{minipage} \\
      \\ \cdashline{2-3}


      & Status          & Not Executed \\ \hline

      9 & Description &

      \begin{minipage}[t]{13cm}{\footnotesize
      Examine distribution of source separation errors from step 4 for all
pairs separated by \textasciitilde{} 200 arcminutes. ~Verify that the
median error in these measurements is \textless{}= 15 milliarcseconds.

      \vspace{\dp0}
      } \end{minipage} \\
      \\ \cdashline{2-3}

      & Expected Result & 

      \begin{minipage}[t]{13cm}{\footnotesize
      
      \vspace{\dp0}
      } \end{minipage} \\
      \\ \cdashline{2-3}

      & \begin{minipage}[t]{2cm}{Actual\\ Result}\end{minipage}   & 
      \begin{minipage}[t]{13cm}{\footnotesize
      
      \vspace{\dp0}
      } \end{minipage} \\
      \\ \cdashline{2-3}


      & Status          & Not Executed \\ \hline

      10 & Description &

      \begin{minipage}[t]{13cm}{\footnotesize
      Verify that no more than 10 percent of sources separated by
\textasciitilde{} 200 arcminutes have source separation errors greater
than 30 milliarcseconds.

      \vspace{\dp0}
      } \end{minipage} \\
      \\ \cdashline{2-3}

      & Expected Result & 

      \begin{minipage}[t]{13cm}{\footnotesize
      
      \vspace{\dp0}
      } \end{minipage} \\
      \\ \cdashline{2-3}

      & \begin{minipage}[t]{2cm}{Actual\\ Result}\end{minipage}   & 
      \begin{minipage}[t]{13cm}{\footnotesize
      
      \vspace{\dp0}
      } \end{minipage} \\
      \\ \cdashline{2-3}


      & Status          & Not Executed \\ \hline

    \end{longtable}


    \paragraph{Test Case LVV-T956 - Ghost area characterization
 }\mbox{}\\

Open  \href{https://jira.lsstcorp.org/secure/Tests.jspa#/testCase/LVV-T956}{\textit{ LVV-T956 } }
test case in Jira.

    Verify that the area affected by significant ghosts is within specified
limits.


    \textbf{ Preconditions}:\\
    

    Execution status: {\bf Not Executed }

    Final comment:\\


    Detailed step results:

    \begin{longtable}{p{1cm}p{2cm}p{13cm}}
    \hline
    {Step} & \multicolumn{2}{c}{Description, Results and Status}\\ \hline
      1 & Description &

      \begin{minipage}[t]{13cm}{\footnotesize
      Image a field of view with a bright star (magnitude=4 ?) in each of the
six bands.

      \vspace{\dp0}
      } \end{minipage} \\
      \\ \cdashline{2-3}

      & Expected Result & 

      \begin{minipage}[t]{13cm}{\footnotesize
      A set of images in each band containing a bright star.

      \vspace{\dp0}
      } \end{minipage} \\
      \\ \cdashline{2-3}

      & \begin{minipage}[t]{2cm}{Actual\\ Result}\end{minipage}   & 
      \begin{minipage}[t]{13cm}{\footnotesize
      
      \vspace{\dp0}
      } \end{minipage} \\
      \\ \cdashline{2-3}


      & Status          & Not Executed \\ \hline

      2 & Description &

      \begin{minipage}[t]{13cm}{\footnotesize
      Dither the telescope pointing so that the bright star is far off of the
field of view (so that we no longer expect it to produce ghosts).
~Re-image the dithered field in all six bands.

      \vspace{\dp0}
      } \end{minipage} \\
      \\ \cdashline{2-3}

      & Expected Result & 

      \begin{minipage}[t]{13cm}{\footnotesize
      A set of images in each band overlapping the images from step 1, but
with the bright star far outside the field of view.

      \vspace{\dp0}
      } \end{minipage} \\
      \\ \cdashline{2-3}

      & \begin{minipage}[t]{2cm}{Actual\\ Result}\end{minipage}   & 
      \begin{minipage}[t]{13cm}{\footnotesize
      
      \vspace{\dp0}
      } \end{minipage} \\
      \\ \cdashline{2-3}


      & Status          & Not Executed \\ \hline

      3 & Description &

      \begin{minipage}[t]{13cm}{\footnotesize
      Perform difference imaging on the overlap region between the images in
step 1 and step 2.

      \vspace{\dp0}
      } \end{minipage} \\
      \\ \cdashline{2-3}

      & Expected Result & 

      \begin{minipage}[t]{13cm}{\footnotesize
      A set of difference images

      \vspace{\dp0}
      } \end{minipage} \\
      \\ \cdashline{2-3}

      & \begin{minipage}[t]{2cm}{Actual\\ Result}\end{minipage}   & 
      \begin{minipage}[t]{13cm}{\footnotesize
      
      \vspace{\dp0}
      } \end{minipage} \\
      \\ \cdashline{2-3}


      & Status          & Not Executed \\ \hline

      4 & Description &

      \begin{minipage}[t]{13cm}{\footnotesize
      Search differenced images for ghosts that exceed 1/3 of sky noise on 1
arcsecond scales. ~Calculate percentage of image area affected by these
ghosts.

      \vspace{\dp0}
      } \end{minipage} \\
      \\ \cdashline{2-3}

      & Expected Result & 

      \begin{minipage}[t]{13cm}{\footnotesize
      No more than 10\% of the image area is affected by ghosts that exceed
1/3 of sky noise on 1 arsecond scales.

      \vspace{\dp0}
      } \end{minipage} \\
      \\ \cdashline{2-3}

      & \begin{minipage}[t]{2cm}{Actual\\ Result}\end{minipage}   & 
      \begin{minipage}[t]{13cm}{\footnotesize
      
      \vspace{\dp0}
      } \end{minipage} \\
      \\ \cdashline{2-3}


      & Status          & Not Executed \\ \hline

    \end{longtable}


    \paragraph{Test Case LVV-T957 - Ghost effect on photometric repeatability
 }\mbox{}\\

Open  \href{https://jira.lsstcorp.org/secure/Tests.jspa#/testCase/LVV-T957}{\textit{ LVV-T957 } }
test case in Jira.

    Verify that ghosting does not unduly effect our photometric
repeatability


    \textbf{ Preconditions}:\\
    

    Execution status: {\bf Not Executed }

    Final comment:\\


    Detailed step results:

    \begin{longtable}{p{1cm}p{2cm}p{13cm}}
    \hline
    {Step} & \multicolumn{2}{c}{Description, Results and Status}\\ \hline
      1 & Description &

      \begin{minipage}[t]{13cm}{\footnotesize
      Image a field of view with a bright star (magnitude=4 ?) in each of the
six bands.

      \vspace{\dp0}
      } \end{minipage} \\
      \\ \cdashline{2-3}

      & Expected Result & 

      \begin{minipage}[t]{13cm}{\footnotesize
      A set of images in each band containing a bright star.

      \vspace{\dp0}
      } \end{minipage} \\
      \\ \cdashline{2-3}

      & \begin{minipage}[t]{2cm}{Actual\\ Result}\end{minipage}   & 
      \begin{minipage}[t]{13cm}{\footnotesize
      
      \vspace{\dp0}
      } \end{minipage} \\
      \\ \cdashline{2-3}


      & Status          & Not Executed \\ \hline

      2 & Description &

      \begin{minipage}[t]{13cm}{\footnotesize
      Dither the telescope pointing so that the bright star is far off of the
field of view (so that we no longer expect it to produce ghosts).
~Re-image the dithered field in all six bands.

      \vspace{\dp0}
      } \end{minipage} \\
      \\ \cdashline{2-3}

      & Expected Result & 

      \begin{minipage}[t]{13cm}{\footnotesize
      A set of images in each band overlapping the images from step 1, but
with the bright star far outside the field of view.

      \vspace{\dp0}
      } \end{minipage} \\
      \\ \cdashline{2-3}

      & \begin{minipage}[t]{2cm}{Actual\\ Result}\end{minipage}   & 
      \begin{minipage}[t]{13cm}{\footnotesize
      
      \vspace{\dp0}
      } \end{minipage} \\
      \\ \cdashline{2-3}


      & Status          & Not Executed \\ \hline

      3 & Description &

      \begin{minipage}[t]{13cm}{\footnotesize
      Perform difference imaging on the overlap region between the images in
step 1 and step 2.

      \vspace{\dp0}
      } \end{minipage} \\
      \\ \cdashline{2-3}

      & Expected Result & 

      \begin{minipage}[t]{13cm}{\footnotesize
      A set of difference images

      \vspace{\dp0}
      } \end{minipage} \\
      \\ \cdashline{2-3}

      & \begin{minipage}[t]{2cm}{Actual\\ Result}\end{minipage}   & 
      \begin{minipage}[t]{13cm}{\footnotesize
      
      \vspace{\dp0}
      } \end{minipage} \\
      \\ \cdashline{2-3}


      & Status          & Not Executed \\ \hline

      4 & Description &

      \begin{minipage}[t]{13cm}{\footnotesize
      Search differenced images for ghosts that exceed 1/3 of sky noise on 1
arcsecond scales. ~Calculate percentage of image area affected by these
ghosts.

      \vspace{\dp0}
      } \end{minipage} \\
      \\ \cdashline{2-3}

      & Expected Result & 

      \begin{minipage}[t]{13cm}{\footnotesize
      No more than 10\% of the image area is affected by ghosts that exceed
1/3 of sky noise on 1 arsecond scales.

      \vspace{\dp0}
      } \end{minipage} \\
      \\ \cdashline{2-3}

      & \begin{minipage}[t]{2cm}{Actual\\ Result}\end{minipage}   & 
      \begin{minipage}[t]{13cm}{\footnotesize
      
      \vspace{\dp0}
      } \end{minipage} \\
      \\ \cdashline{2-3}


      & Status          & Not Executed \\ \hline

      5 & Description &

      \begin{minipage}[t]{13cm}{\footnotesize
      Identify regions containing ghosts from the bright star in the first set
of images, but which hopefully do not contain ghosts in the second set
of images.

      \vspace{\dp0}
      } \end{minipage} \\
      \\ \cdashline{2-3}

      & Expected Result & 

      \begin{minipage}[t]{13cm}{\footnotesize
      
      \vspace{\dp0}
      } \end{minipage} \\
      \\ \cdashline{2-3}

      & \begin{minipage}[t]{2cm}{Actual\\ Result}\end{minipage}   & 
      \begin{minipage}[t]{13cm}{\footnotesize
      
      \vspace{\dp0}
      } \end{minipage} \\
      \\ \cdashline{2-3}


      & Status          & Not Executed \\ \hline

      6 & Description &

      \begin{minipage}[t]{13cm}{\footnotesize
      Perform photometric measurement in both the images with and without the
bright star in the regions affected by ghosts identified in step 2.

      \vspace{\dp0}
      } \end{minipage} \\
      \\ \cdashline{2-3}

      & Expected Result & 

      \begin{minipage}[t]{13cm}{\footnotesize
      Photometric measurements of sources in the presence and absence of
ghosts.

      \vspace{\dp0}
      } \end{minipage} \\
      \\ \cdashline{2-3}

      & \begin{minipage}[t]{2cm}{Actual\\ Result}\end{minipage}   & 
      \begin{minipage}[t]{13cm}{\footnotesize
      
      \vspace{\dp0}
      } \end{minipage} \\
      \\ \cdashline{2-3}


      & Status          & Not Executed \\ \hline

      7 & Description &

      \begin{minipage}[t]{13cm}{\footnotesize
      Verify that the repeatability of photometric measurements on theses
sources has not degraded by more than 10\% due to the presence of
ghosts.

      \vspace{\dp0}
      } \end{minipage} \\
      \\ \cdashline{2-3}

      & Expected Result & 

      \begin{minipage}[t]{13cm}{\footnotesize
      
      \vspace{\dp0}
      } \end{minipage} \\
      \\ \cdashline{2-3}

      & \begin{minipage}[t]{2cm}{Actual\\ Result}\end{minipage}   & 
      \begin{minipage}[t]{13cm}{\footnotesize
      
      \vspace{\dp0}
      } \end{minipage} \\
      \\ \cdashline{2-3}


      & Status          & Not Executed \\ \hline

    \end{longtable}


    \paragraph{Test Case LVV-T595 - PSF ellipticity
 }\mbox{}\\

Open  \href{https://jira.lsstcorp.org/secure/Tests.jspa#/testCase/LVV-T595}{\textit{ LVV-T595 } }
test case in Jira.

    

    \textbf{ Preconditions}:\\
    

    Execution status: {\bf Not Executed }

    Final comment:\\


    Detailed step results:

    \begin{longtable}{p{1cm}p{2cm}p{13cm}}
    \hline
    {Step} & \multicolumn{2}{c}{Description, Results and Status}\\ \hline
      1 & Description &

      \begin{minipage}[t]{13cm}{\footnotesize
      Take a collection of images in all six filters at a diverse range of
airmasses and atmospheric seeing condtions in uncrowded fields.

      \vspace{\dp0}
      } \end{minipage} \\
      \\ \cdashline{2-3}

      & Expected Result & 

      \begin{minipage}[t]{13cm}{\footnotesize
      Collection of images

      \vspace{\dp0}
      } \end{minipage} \\
      \\ \cdashline{2-3}

      & \begin{minipage}[t]{2cm}{Actual\\ Result}\end{minipage}   & 
      \begin{minipage}[t]{13cm}{\footnotesize
      
      \vspace{\dp0}
      } \end{minipage} \\
      \\ \cdashline{2-3}


      & Status          & Not Executed \\ \hline

      2 & Description &

      \begin{minipage}[t]{13cm}{\footnotesize
      Perform single image processing on the images from step 1.

      \vspace{\dp0}
      } \end{minipage} \\
      \\ \cdashline{2-3}

      & Expected Result & 

      \begin{minipage}[t]{13cm}{\footnotesize
      Catalog of detected sources.

      \vspace{\dp0}
      } \end{minipage} \\
      \\ \cdashline{2-3}

      & \begin{minipage}[t]{2cm}{Actual\\ Result}\end{minipage}   & 
      \begin{minipage}[t]{13cm}{\footnotesize
      
      \vspace{\dp0}
      } \end{minipage} \\
      \\ \cdashline{2-3}


      & Status          & Not Executed \\ \hline

      3 & Description &

      \begin{minipage}[t]{13cm}{\footnotesize
      For each full-focal plane exposure, select all of the measured,
unresolved point sources brighter than some threshold (17th magnitude?).
~Calculate the ellipticity of the PSF measured at each of these sources.

      \vspace{\dp0}
      } \end{minipage} \\
      \\ \cdashline{2-3}

      & Expected Result & 

      \begin{minipage}[t]{13cm}{\footnotesize
      Distribution of PSF ellipticities in the images.

      \vspace{\dp0}
      } \end{minipage} \\
      \\ \cdashline{2-3}

      & \begin{minipage}[t]{2cm}{Actual\\ Result}\end{minipage}   & 
      \begin{minipage}[t]{13cm}{\footnotesize
      
      \vspace{\dp0}
      } \end{minipage} \\
      \\ \cdashline{2-3}


      & Status          & Not Executed \\ \hline

      4 & Description &

      \begin{minipage}[t]{13cm}{\footnotesize
      For each full-focal plane exposure, verify that the median PSF
ellipticity of the sources from step 3 is less than or equal to
0.04.\\[2\baselineskip]Verify that no more than 5\% of the sources from
step 3 have PSF ellipticity greater than 0.07.

      \vspace{\dp0}
      } \end{minipage} \\
      \\ \cdashline{2-3}

      & Expected Result & 

      \begin{minipage}[t]{13cm}{\footnotesize
      
      \vspace{\dp0}
      } \end{minipage} \\
      \\ \cdashline{2-3}

      & \begin{minipage}[t]{2cm}{Actual\\ Result}\end{minipage}   & 
      \begin{minipage}[t]{13cm}{\footnotesize
      
      \vspace{\dp0}
      } \end{minipage} \\
      \\ \cdashline{2-3}


      & Status          & Not Executed \\ \hline

    \end{longtable}


    \paragraph{Test Case LVV-T594 - Image quality - degradation from zenith
 }\mbox{}\\

Open  \href{https://jira.lsstcorp.org/secure/Tests.jspa#/testCase/LVV-T594}{\textit{ LVV-T594 } }
test case in Jira.

    

    \textbf{ Preconditions}:\\
    

    Execution status: {\bf Not Executed }

    Final comment:\\


    Detailed step results:

    \begin{longtable}{p{1cm}p{2cm}p{13cm}}
    \hline
    {Step} & \multicolumn{2}{c}{Description, Results and Status}\\ \hline
      1 & Description &

      \begin{minipage}[t]{13cm}{\footnotesize
      Take images at zenith, airmass=1.4, and airmass=2.0. ~Be sure to get
complete sets of images in all six filters at the three specified
airmasses, changing the airmass rapidly enough that observing conditions
do not change.

      \vspace{\dp0}
      } \end{minipage} \\
      \\ \cdashline{2-3}

      & Expected Result & 

      \begin{minipage}[t]{13cm}{\footnotesize
      In all six filters, collections of images at airmass=1, 1.4, 2 under
identical (or very similar) observing conditions.

      \vspace{\dp0}
      } \end{minipage} \\
      \\ \cdashline{2-3}

      & \begin{minipage}[t]{2cm}{Actual\\ Result}\end{minipage}   & 
      \begin{minipage}[t]{13cm}{\footnotesize
      
      \vspace{\dp0}
      } \end{minipage} \\
      \\ \cdashline{2-3}


      & Status          & Not Executed \\ \hline

      2 & Description &

      \begin{minipage}[t]{13cm}{\footnotesize
      Use the DIMM to measure observing conditions under which the images in
step 1 were taken.

      \vspace{\dp0}
      } \end{minipage} \\
      \\ \cdashline{2-3}

      & Expected Result & 

      \begin{minipage}[t]{13cm}{\footnotesize
      Observing conditions for images in step 1

      \vspace{\dp0}
      } \end{minipage} \\
      \\ \cdashline{2-3}

      & \begin{minipage}[t]{2cm}{Actual\\ Result}\end{minipage}   & 
      \begin{minipage}[t]{13cm}{\footnotesize
      
      \vspace{\dp0}
      } \end{minipage} \\
      \\ \cdashline{2-3}


      & Status          & Not Executed \\ \hline

      3 & Description &

      \begin{minipage}[t]{13cm}{\footnotesize
      Calculate the theoretical PSF size for each of the exposures in step 1
given the observing conditions measured in step 2.

      \vspace{\dp0}
      } \end{minipage} \\
      \\ \cdashline{2-3}

      & Expected Result & 

      \begin{minipage}[t]{13cm}{\footnotesize
      Theoretical model of PSF sizes for the images in step 1.

      \vspace{\dp0}
      } \end{minipage} \\
      \\ \cdashline{2-3}

      & \begin{minipage}[t]{2cm}{Actual\\ Result}\end{minipage}   & 
      \begin{minipage}[t]{13cm}{\footnotesize
      
      \vspace{\dp0}
      } \end{minipage} \\
      \\ \cdashline{2-3}


      & Status          & Not Executed \\ \hline

      4 & Description &

      \begin{minipage}[t]{13cm}{\footnotesize
      Perform single image processing on the images in step 1.

      \vspace{\dp0}
      } \end{minipage} \\
      \\ \cdashline{2-3}

      & Expected Result & 

      \begin{minipage}[t]{13cm}{\footnotesize
      Catalog of detected sources in step 1.

      \vspace{\dp0}
      } \end{minipage} \\
      \\ \cdashline{2-3}

      & \begin{minipage}[t]{2cm}{Actual\\ Result}\end{minipage}   & 
      \begin{minipage}[t]{13cm}{\footnotesize
      
      \vspace{\dp0}
      } \end{minipage} \\
      \\ \cdashline{2-3}


      & Status          & Not Executed \\ \hline

      5 & Description &

      \begin{minipage}[t]{13cm}{\footnotesize
      Subtract (in quadrature) the theoretical PSF sizes from the PSF sizes
measured in step 4.

      \vspace{\dp0}
      } \end{minipage} \\
      \\ \cdashline{2-3}

      & Expected Result & 

      \begin{minipage}[t]{13cm}{\footnotesize
      Residual PSF size

      \vspace{\dp0}
      } \end{minipage} \\
      \\ \cdashline{2-3}

      & \begin{minipage}[t]{2cm}{Actual\\ Result}\end{minipage}   & 
      \begin{minipage}[t]{13cm}{\footnotesize
      
      \vspace{\dp0}
      } \end{minipage} \\
      \\ \cdashline{2-3}


      & Status          & Not Executed \\ \hline

      6 & Description &

      \begin{minipage}[t]{13cm}{\footnotesize
      Verify that the residual PSF size at each airmass does not exceed the
following limits:\\[2\baselineskip]0.4 arcseconds at airmass=1\\
0.49 arcseconds at airmass=1.4\\
0.6 arcseconds at airmass=2\\[2\baselineskip]so that the system
contribution to PSF width degrades no more rapidly than airmass\^{}0.6

      \vspace{\dp0}
      } \end{minipage} \\
      \\ \cdashline{2-3}

      & Expected Result & 

      \begin{minipage}[t]{13cm}{\footnotesize
      
      \vspace{\dp0}
      } \end{minipage} \\
      \\ \cdashline{2-3}

      & \begin{minipage}[t]{2cm}{Actual\\ Result}\end{minipage}   & 
      \begin{minipage}[t]{13cm}{\footnotesize
      
      \vspace{\dp0}
      } \end{minipage} \\
      \\ \cdashline{2-3}


      & Status          & Not Executed \\ \hline

    \end{longtable}


    \paragraph{Test Case LVV-T593 - Image quality at zenith
 }\mbox{}\\

Open  \href{https://jira.lsstcorp.org/secure/Tests.jspa#/testCase/LVV-T593}{\textit{ LVV-T593 } }
test case in Jira.

    

    \textbf{ Preconditions}:\\
    

    Execution status: {\bf Not Executed }

    Final comment:\\


    Detailed step results:

    \begin{longtable}{p{1cm}p{2cm}p{13cm}}
    \hline
    {Step} & \multicolumn{2}{c}{Description, Results and Status}\\ \hline
      1 & Description &

      \begin{minipage}[t]{13cm}{\footnotesize
      Take a series of images at zenith in all six filters.

      \vspace{\dp0}
      } \end{minipage} \\
      \\ \cdashline{2-3}

      & Expected Result & 

      \begin{minipage}[t]{13cm}{\footnotesize
      Set of images

      \vspace{\dp0}
      } \end{minipage} \\
      \\ \cdashline{2-3}

      & \begin{minipage}[t]{2cm}{Actual\\ Result}\end{minipage}   & 
      \begin{minipage}[t]{13cm}{\footnotesize
      
      \vspace{\dp0}
      } \end{minipage} \\
      \\ \cdashline{2-3}


      & Status          & Not Executed \\ \hline

      2 & Description &

      \begin{minipage}[t]{13cm}{\footnotesize
      Calculate theoretical PSF size for the images in step 1 based on
observing conditions.

      \vspace{\dp0}
      } \end{minipage} \\
      \\ \cdashline{2-3}

      & Expected Result & 

      \begin{minipage}[t]{13cm}{\footnotesize
      Theoretical model of PSF size

      \vspace{\dp0}
      } \end{minipage} \\
      \\ \cdashline{2-3}

      & \begin{minipage}[t]{2cm}{Actual\\ Result}\end{minipage}   & 
      \begin{minipage}[t]{13cm}{\footnotesize
      
      \vspace{\dp0}
      } \end{minipage} \\
      \\ \cdashline{2-3}


      & Status          & Not Executed \\ \hline

      3 & Description &

      \begin{minipage}[t]{13cm}{\footnotesize
      Perform single image processing on images from step 1

      \vspace{\dp0}
      } \end{minipage} \\
      \\ \cdashline{2-3}

      & Expected Result & 

      \begin{minipage}[t]{13cm}{\footnotesize
      Catalog of detected sources

      \vspace{\dp0}
      } \end{minipage} \\
      \\ \cdashline{2-3}

      & \begin{minipage}[t]{2cm}{Actual\\ Result}\end{minipage}   & 
      \begin{minipage}[t]{13cm}{\footnotesize
      
      \vspace{\dp0}
      } \end{minipage} \\
      \\ \cdashline{2-3}


      & Status          & Not Executed \\ \hline

      4 & Description &

      \begin{minipage}[t]{13cm}{\footnotesize
      Subtract (in quadrature) theoretical PSF model from step 2 from measured
PSF sizes in catalog from step 3

      \vspace{\dp0}
      } \end{minipage} \\
      \\ \cdashline{2-3}

      & Expected Result & 

      \begin{minipage}[t]{13cm}{\footnotesize
      Residual PSF size

      \vspace{\dp0}
      } \end{minipage} \\
      \\ \cdashline{2-3}

      & \begin{minipage}[t]{2cm}{Actual\\ Result}\end{minipage}   & 
      \begin{minipage}[t]{13cm}{\footnotesize
      
      \vspace{\dp0}
      } \end{minipage} \\
      \\ \cdashline{2-3}


      & Status          & Not Executed \\ \hline

      5 & Description &

      \begin{minipage}[t]{13cm}{\footnotesize
      Verify that residual PSF size does not exceed 0.4 arcseconds

      \vspace{\dp0}
      } \end{minipage} \\
      \\ \cdashline{2-3}

      & Expected Result & 

      \begin{minipage}[t]{13cm}{\footnotesize
      
      \vspace{\dp0}
      } \end{minipage} \\
      \\ \cdashline{2-3}

      & \begin{minipage}[t]{2cm}{Actual\\ Result}\end{minipage}   & 
      \begin{minipage}[t]{13cm}{\footnotesize
      
      \vspace{\dp0}
      } \end{minipage} \\
      \\ \cdashline{2-3}


      & Status          & Not Executed \\ \hline

    \end{longtable}


    \paragraph{Test Case LVV-T592 - Image quality - maximum system contribution
 }\mbox{}\\

Open  \href{https://jira.lsstcorp.org/secure/Tests.jspa#/testCase/LVV-T592}{\textit{ LVV-T592 } }
test case in Jira.

    

    \textbf{ Preconditions}:\\
    

    Execution status: {\bf Not Executed }

    Final comment:\\


    Detailed step results:

    \begin{longtable}{p{1cm}p{2cm}p{13cm}}
    \hline
    {Step} & \multicolumn{2}{c}{Description, Results and Status}\\ \hline
      1 & Description &

      \begin{minipage}[t]{13cm}{\footnotesize
      Use DIMM to select observations taken at 0.6 arcsecond seeing.

      \vspace{\dp0}
      } \end{minipage} \\
      \\ \cdashline{2-3}

      & Expected Result & 

      \begin{minipage}[t]{13cm}{\footnotesize
      Set of images

      \vspace{\dp0}
      } \end{minipage} \\
      \\ \cdashline{2-3}

      & \begin{minipage}[t]{2cm}{Actual\\ Result}\end{minipage}   & 
      \begin{minipage}[t]{13cm}{\footnotesize
      
      \vspace{\dp0}
      } \end{minipage} \\
      \\ \cdashline{2-3}


      & Status          & Not Executed \\ \hline

      2 & Description &

      \begin{minipage}[t]{13cm}{\footnotesize
      Calculate theoretical PSF size for images in step 1 given observing
conditions.

      \vspace{\dp0}
      } \end{minipage} \\
      \\ \cdashline{2-3}

      & Expected Result & 

      \begin{minipage}[t]{13cm}{\footnotesize
      Theoretical model of PSF sizes

      \vspace{\dp0}
      } \end{minipage} \\
      \\ \cdashline{2-3}

      & \begin{minipage}[t]{2cm}{Actual\\ Result}\end{minipage}   & 
      \begin{minipage}[t]{13cm}{\footnotesize
      
      \vspace{\dp0}
      } \end{minipage} \\
      \\ \cdashline{2-3}


      & Status          & Not Executed \\ \hline

      3 & Description &

      \begin{minipage}[t]{13cm}{\footnotesize
      Perform single image processing on images from step 1.

      \vspace{\dp0}
      } \end{minipage} \\
      \\ \cdashline{2-3}

      & Expected Result & 

      \begin{minipage}[t]{13cm}{\footnotesize
      Catalog of detected sources

      \vspace{\dp0}
      } \end{minipage} \\
      \\ \cdashline{2-3}

      & \begin{minipage}[t]{2cm}{Actual\\ Result}\end{minipage}   & 
      \begin{minipage}[t]{13cm}{\footnotesize
      
      \vspace{\dp0}
      } \end{minipage} \\
      \\ \cdashline{2-3}


      & Status          & Not Executed \\ \hline

      4 & Description &

      \begin{minipage}[t]{13cm}{\footnotesize
      Subtract (in quadrature) theoretical model from step 2 from PSF sizes of
sources in catalogs from step 3

      \vspace{\dp0}
      } \end{minipage} \\
      \\ \cdashline{2-3}

      & Expected Result & 

      \begin{minipage}[t]{13cm}{\footnotesize
      Residual PSF sizes

      \vspace{\dp0}
      } \end{minipage} \\
      \\ \cdashline{2-3}

      & \begin{minipage}[t]{2cm}{Actual\\ Result}\end{minipage}   & 
      \begin{minipage}[t]{13cm}{\footnotesize
      
      \vspace{\dp0}
      } \end{minipage} \\
      \\ \cdashline{2-3}


      & Status          & Not Executed \\ \hline

      5 & Description &

      \begin{minipage}[t]{13cm}{\footnotesize
      Verify that residual PSF size, which will have been contributed by the
LSST system, is no more than 15\% of total PSF size.

      \vspace{\dp0}
      } \end{minipage} \\
      \\ \cdashline{2-3}

      & Expected Result & 

      \begin{minipage}[t]{13cm}{\footnotesize
      
      \vspace{\dp0}
      } \end{minipage} \\
      \\ \cdashline{2-3}

      & \begin{minipage}[t]{2cm}{Actual\\ Result}\end{minipage}   & 
      \begin{minipage}[t]{13cm}{\footnotesize
      
      \vspace{\dp0}
      } \end{minipage} \\
      \\ \cdashline{2-3}


      & Status          & Not Executed \\ \hline

    \end{longtable}


    \paragraph{Test Case LVV-T591 - Flux-enclosing radius
 }\mbox{}\\

Open  \href{https://jira.lsstcorp.org/secure/Tests.jspa#/testCase/LVV-T591}{\textit{ LVV-T591 } }
test case in Jira.

    

    \textbf{ Preconditions}:\\
    

    Execution status: {\bf Not Executed }

    Final comment:\\


    Detailed step results:

    \begin{longtable}{p{1cm}p{2cm}p{13cm}}
    \hline
    {Step} & \multicolumn{2}{c}{Description, Results and Status}\\ \hline
      1 & Description &

      \begin{minipage}[t]{13cm}{\footnotesize
      Select pointings in sparse fields (so that we don't have to worry about
more than one object getting encircled by the test apertures below)

      \vspace{\dp0}
      } \end{minipage} \\
      \\ \cdashline{2-3}

      & Expected Result & 

      \begin{minipage}[t]{13cm}{\footnotesize
      
      \vspace{\dp0}
      } \end{minipage} \\
      \\ \cdashline{2-3}

      & \begin{minipage}[t]{2cm}{Actual\\ Result}\end{minipage}   & 
      \begin{minipage}[t]{13cm}{\footnotesize
      
      \vspace{\dp0}
      } \end{minipage} \\
      \\ \cdashline{2-3}


      & Status          & Not Executed \\ \hline

      2 & Description &

      \begin{minipage}[t]{13cm}{\footnotesize
      Run single image processing on the exposures from step 1 to identify all
of the sources.

      \vspace{\dp0}
      } \end{minipage} \\
      \\ \cdashline{2-3}

      & Expected Result & 

      \begin{minipage}[t]{13cm}{\footnotesize
      Source catalog

      \vspace{\dp0}
      } \end{minipage} \\
      \\ \cdashline{2-3}

      & \begin{minipage}[t]{2cm}{Actual\\ Result}\end{minipage}   & 
      \begin{minipage}[t]{13cm}{\footnotesize
      
      \vspace{\dp0}
      } \end{minipage} \\
      \\ \cdashline{2-3}


      & Status          & Not Executed \\ \hline

      3 & Description &

      \begin{minipage}[t]{13cm}{\footnotesize
      Run forced photometry on all detected unresolved point sources,
measuring fluxes inside of a 2 arcescond (or some other unseemly large)
aperture.

      \vspace{\dp0}
      } \end{minipage} \\
      \\ \cdashline{2-3}

      & Expected Result & 

      \begin{minipage}[t]{13cm}{\footnotesize
      Catalog of source fluxes in a large aperture

      \vspace{\dp0}
      } \end{minipage} \\
      \\ \cdashline{2-3}

      & \begin{minipage}[t]{2cm}{Actual\\ Result}\end{minipage}   & 
      \begin{minipage}[t]{13cm}{\footnotesize
      
      \vspace{\dp0}
      } \end{minipage} \\
      \\ \cdashline{2-3}


      & Status          & Not Executed \\ \hline

      4 & Description &

      \begin{minipage}[t]{13cm}{\footnotesize
      Re-run forced photometry on sources from step 3, reducing the aperture
to 1.81 arcsecond, 1.31 arcsecond, and 0.8 arcsecond. ~Verify that three
measurements contain at least 95\%, 90\%, and 80\% of the flux for all
sources.

      \vspace{\dp0}
      } \end{minipage} \\
      \\ \cdashline{2-3}

      & Expected Result & 

      \begin{minipage}[t]{13cm}{\footnotesize
      
      \vspace{\dp0}
      } \end{minipage} \\
      \\ \cdashline{2-3}

      & \begin{minipage}[t]{2cm}{Actual\\ Result}\end{minipage}   & 
      \begin{minipage}[t]{13cm}{\footnotesize
      
      \vspace{\dp0}
      } \end{minipage} \\
      \\ \cdashline{2-3}


      & Status          & Not Executed \\ \hline

    \end{longtable}


    \paragraph{Test Case LVV-T590 - Median image quality at 0.8 arcsecond seeing
 }\mbox{}\\

Open  \href{https://jira.lsstcorp.org/secure/Tests.jspa#/testCase/LVV-T590}{\textit{ LVV-T590 } }
test case in Jira.

    

    \textbf{ Preconditions}:\\
    

    Execution status: {\bf Not Executed }

    Final comment:\\


    Detailed step results:

    \begin{longtable}{p{1cm}p{2cm}p{13cm}}
    \hline
    {Step} & \multicolumn{2}{c}{Description, Results and Status}\\ \hline
      1 & Description &

      \begin{minipage}[t]{13cm}{\footnotesize
      Use DIMM to select exposures taken at 0.8 arcsecond seeing in the r and
i bands

      \vspace{\dp0}
      } \end{minipage} \\
      \\ \cdashline{2-3}

      & Expected Result & 

      \begin{minipage}[t]{13cm}{\footnotesize
      
      \vspace{\dp0}
      } \end{minipage} \\
      \\ \cdashline{2-3}

      & \begin{minipage}[t]{2cm}{Actual\\ Result}\end{minipage}   & 
      \begin{minipage}[t]{13cm}{\footnotesize
      
      \vspace{\dp0}
      } \end{minipage} \\
      \\ \cdashline{2-3}


      & Status          & Not Executed \\ \hline

      2 & Description &

      \begin{minipage}[t]{13cm}{\footnotesize
      Run single image processing on the exposures from step 1

      \vspace{\dp0}
      } \end{minipage} \\
      \\ \cdashline{2-3}

      & Expected Result & 

      \begin{minipage}[t]{13cm}{\footnotesize
      Source catalog

      \vspace{\dp0}
      } \end{minipage} \\
      \\ \cdashline{2-3}

      & \begin{minipage}[t]{2cm}{Actual\\ Result}\end{minipage}   & 
      \begin{minipage}[t]{13cm}{\footnotesize
      
      \vspace{\dp0}
      } \end{minipage} \\
      \\ \cdashline{2-3}


      & Status          & Not Executed \\ \hline

      3 & Description &

      \begin{minipage}[t]{13cm}{\footnotesize
      Verify that median of PSF FWHM is approximately 0.89 arcsecond

      \vspace{\dp0}
      } \end{minipage} \\
      \\ \cdashline{2-3}

      & Expected Result & 

      \begin{minipage}[t]{13cm}{\footnotesize
      
      \vspace{\dp0}
      } \end{minipage} \\
      \\ \cdashline{2-3}

      & \begin{minipage}[t]{2cm}{Actual\\ Result}\end{minipage}   & 
      \begin{minipage}[t]{13cm}{\footnotesize
      
      \vspace{\dp0}
      } \end{minipage} \\
      \\ \cdashline{2-3}


      & Status          & Not Executed \\ \hline

    \end{longtable}


    \paragraph{Test Case LVV-T589 - Median image quality at 0.6 arcsecond seeing
 }\mbox{}\\

Open  \href{https://jira.lsstcorp.org/secure/Tests.jspa#/testCase/LVV-T589}{\textit{ LVV-T589 } }
test case in Jira.

    

    \textbf{ Preconditions}:\\
    

    Execution status: {\bf Not Executed }

    Final comment:\\


    Detailed step results:

    \begin{longtable}{p{1cm}p{2cm}p{13cm}}
    \hline
    {Step} & \multicolumn{2}{c}{Description, Results and Status}\\ \hline
      1 & Description &

      \begin{minipage}[t]{13cm}{\footnotesize
      Use DIMM to select pointings taken at 0.6 arcsecond seeing in the r and
i bands

      \vspace{\dp0}
      } \end{minipage} \\
      \\ \cdashline{2-3}

      & Expected Result & 

      \begin{minipage}[t]{13cm}{\footnotesize
      
      \vspace{\dp0}
      } \end{minipage} \\
      \\ \cdashline{2-3}

      & \begin{minipage}[t]{2cm}{Actual\\ Result}\end{minipage}   & 
      \begin{minipage}[t]{13cm}{\footnotesize
      
      \vspace{\dp0}
      } \end{minipage} \\
      \\ \cdashline{2-3}


      & Status          & Not Executed \\ \hline

      2 & Description &

      \begin{minipage}[t]{13cm}{\footnotesize
      Run single image processing on the exposures from step 1.

      \vspace{\dp0}
      } \end{minipage} \\
      \\ \cdashline{2-3}

      & Expected Result & 

      \begin{minipage}[t]{13cm}{\footnotesize
      Source catalog

      \vspace{\dp0}
      } \end{minipage} \\
      \\ \cdashline{2-3}

      & \begin{minipage}[t]{2cm}{Actual\\ Result}\end{minipage}   & 
      \begin{minipage}[t]{13cm}{\footnotesize
      
      \vspace{\dp0}
      } \end{minipage} \\
      \\ \cdashline{2-3}


      & Status          & Not Executed \\ \hline

      3 & Description &

      \begin{minipage}[t]{13cm}{\footnotesize
      Verify that the median PSF FWHM of unresolved point sources is 0.72
arcseconds.

      \vspace{\dp0}
      } \end{minipage} \\
      \\ \cdashline{2-3}

      & Expected Result & 

      \begin{minipage}[t]{13cm}{\footnotesize
      
      \vspace{\dp0}
      } \end{minipage} \\
      \\ \cdashline{2-3}

      & \begin{minipage}[t]{2cm}{Actual\\ Result}\end{minipage}   & 
      \begin{minipage}[t]{13cm}{\footnotesize
      
      \vspace{\dp0}
      } \end{minipage} \\
      \\ \cdashline{2-3}


      & Status          & Not Executed \\ \hline

    \end{longtable}


    \paragraph{Test Case LVV-T588 - Median image quality at 0.44 arcsecond seeing
 }\mbox{}\\

Open  \href{https://jira.lsstcorp.org/secure/Tests.jspa#/testCase/LVV-T588}{\textit{ LVV-T588 } }
test case in Jira.

    


    \textbf{ Preconditions}:\\
    

    Execution status: {\bf Not Executed }

    Final comment:\\


    Detailed step results:

    \begin{longtable}{p{1cm}p{2cm}p{13cm}}
    \hline
    {Step} & \multicolumn{2}{c}{Description, Results and Status}\\ \hline
      1 & Description &

      \begin{minipage}[t]{13cm}{\footnotesize
      Use DIMM to select pointings taken at 0.44 arcsecond seeing in the r and
i bands

      \vspace{\dp0}
      } \end{minipage} \\
      \\ \cdashline{2-3}

      & Expected Result & 

      \begin{minipage}[t]{13cm}{\footnotesize
      
      \vspace{\dp0}
      } \end{minipage} \\
      \\ \cdashline{2-3}

      & \begin{minipage}[t]{2cm}{Actual\\ Result}\end{minipage}   & 
      \begin{minipage}[t]{13cm}{\footnotesize
      
      \vspace{\dp0}
      } \end{minipage} \\
      \\ \cdashline{2-3}


      & Status          & Not Executed \\ \hline

      2 & Description &

      \begin{minipage}[t]{13cm}{\footnotesize
      Run single image processing on exposures selected in step 1

      \vspace{\dp0}
      } \end{minipage} \\
      \\ \cdashline{2-3}

      & Expected Result & 

      \begin{minipage}[t]{13cm}{\footnotesize
      Source catalog

      \vspace{\dp0}
      } \end{minipage} \\
      \\ \cdashline{2-3}

      & \begin{minipage}[t]{2cm}{Actual\\ Result}\end{minipage}   & 
      \begin{minipage}[t]{13cm}{\footnotesize
      
      \vspace{\dp0}
      } \end{minipage} \\
      \\ \cdashline{2-3}


      & Status          & Not Executed \\ \hline

      3 & Description &

      \begin{minipage}[t]{13cm}{\footnotesize
      Subtract (in quadrature) seeing from PSF FWHM of unresolved point
sources. ~Verify median of distribution of residual PSF FWHM is
approximately 0.59 arcseconds

      \vspace{\dp0}
      } \end{minipage} \\
      \\ \cdashline{2-3}

      & Expected Result & 

      \begin{minipage}[t]{13cm}{\footnotesize
      
      \vspace{\dp0}
      } \end{minipage} \\
      \\ \cdashline{2-3}

      & \begin{minipage}[t]{2cm}{Actual\\ Result}\end{minipage}   & 
      \begin{minipage}[t]{13cm}{\footnotesize
      
      \vspace{\dp0}
      } \end{minipage} \\
      \\ \cdashline{2-3}


      & Status          & Not Executed \\ \hline

    \end{longtable}


    \paragraph{Test Case LVV-T587 - PSF size in pixels
 }\mbox{}\\

Open  \href{https://jira.lsstcorp.org/secure/Tests.jspa#/testCase/LVV-T587}{\textit{ LVV-T587 } }
test case in Jira.

    

    \textbf{ Preconditions}:\\
    

    Execution status: {\bf Not Executed }

    Final comment:\\


    Detailed step results:

    \begin{longtable}{p{1cm}p{2cm}p{13cm}}
    \hline
    {Step} & \multicolumn{2}{c}{Description, Results and Status}\\ \hline
      1 & Description &

      \begin{minipage}[t]{13cm}{\footnotesize
      Use DIMM data to select a set of pointings observed at 0.6 arcsecond
seeing.

      \vspace{\dp0}
      } \end{minipage} \\
      \\ \cdashline{2-3}

      & Expected Result & 

      \begin{minipage}[t]{13cm}{\footnotesize
      
      \vspace{\dp0}
      } \end{minipage} \\
      \\ \cdashline{2-3}

      & \begin{minipage}[t]{2cm}{Actual\\ Result}\end{minipage}   & 
      \begin{minipage}[t]{13cm}{\footnotesize
      
      \vspace{\dp0}
      } \end{minipage} \\
      \\ \cdashline{2-3}


      & Status          & Not Executed \\ \hline

      2 & Description &

      \begin{minipage}[t]{13cm}{\footnotesize
      Run single image processing on pointings from step 1

      \vspace{\dp0}
      } \end{minipage} \\
      \\ \cdashline{2-3}

      & Expected Result & 

      \begin{minipage}[t]{13cm}{\footnotesize
      Catalog of detected sources

      \vspace{\dp0}
      } \end{minipage} \\
      \\ \cdashline{2-3}

      & \begin{minipage}[t]{2cm}{Actual\\ Result}\end{minipage}   & 
      \begin{minipage}[t]{13cm}{\footnotesize
      
      \vspace{\dp0}
      } \end{minipage} \\
      \\ \cdashline{2-3}


      & Status          & Not Executed \\ \hline

      3 & Description &

      \begin{minipage}[t]{13cm}{\footnotesize
      Verify that the minimum FWHM of the PSFs of unresolved point sources is
3 pixels or greater

      \vspace{\dp0}
      } \end{minipage} \\
      \\ \cdashline{2-3}

      & Expected Result & 

      \begin{minipage}[t]{13cm}{\footnotesize
      
      \vspace{\dp0}
      } \end{minipage} \\
      \\ \cdashline{2-3}

      & \begin{minipage}[t]{2cm}{Actual\\ Result}\end{minipage}   & 
      \begin{minipage}[t]{13cm}{\footnotesize
      
      \vspace{\dp0}
      } \end{minipage} \\
      \\ \cdashline{2-3}


      & Status          & Not Executed \\ \hline

    \end{longtable}


    \paragraph{Test Case LVV-T554 - Single exposure dynamic range
 }\mbox{}\\

Open  \href{https://jira.lsstcorp.org/secure/Tests.jspa#/testCase/LVV-T554}{\textit{ LVV-T554 } }
test case in Jira.

    Verify that objects within the specified dynamic range of a single image
are not saturated


    \textbf{ Preconditions}:\\
    

    Execution status: {\bf Not Executed }

    Final comment:\\


    Detailed step results:

    \begin{longtable}{p{1cm}p{2cm}p{13cm}}
    \hline
    {Step} & \multicolumn{2}{c}{Description, Results and Status}\\ \hline
      1 & Description &

      \begin{minipage}[t]{13cm}{\footnotesize
      Find images observed at:\\
airmass = 1.0\\
r-band skybrightness = 21 magnitude/arcsec\^{}2\\
r-band seeing = 0.7 arcsec

      \vspace{\dp0}
      } \end{minipage} \\
      \\ \cdashline{2-3}

      & Expected Result & 

      \begin{minipage}[t]{13cm}{\footnotesize
      Set of images to test

      \vspace{\dp0}
      } \end{minipage} \\
      \\ \cdashline{2-3}

      & \begin{minipage}[t]{2cm}{Actual\\ Result}\end{minipage}   & 
      \begin{minipage}[t]{13cm}{\footnotesize
      
      \vspace{\dp0}
      } \end{minipage} \\
      \\ \cdashline{2-3}


      & Status          & Not Executed \\ \hline

      2 & Description &

      \begin{minipage}[t]{13cm}{\footnotesize
      Run single-exposure processing on the images from step 1.

      \vspace{\dp0}
      } \end{minipage} \\
      \\ \cdashline{2-3}

      & Expected Result & 

      \begin{minipage}[t]{13cm}{\footnotesize
      5-sigma limiting magnitude and list of detected sources for each image.

      \vspace{\dp0}
      } \end{minipage} \\
      \\ \cdashline{2-3}

      & \begin{minipage}[t]{2cm}{Actual\\ Result}\end{minipage}   & 
      \begin{minipage}[t]{13cm}{\footnotesize
      
      \vspace{\dp0}
      } \end{minipage} \\
      \\ \cdashline{2-3}


      & Status          & Not Executed \\ \hline

      3 & Description &

      \begin{minipage}[t]{13cm}{\footnotesize
      Check that sources within specified dynamic range are not saturated.

      \vspace{\dp0}
      } \end{minipage} \\
      \\ \cdashline{2-3}

      & Expected Result & 

      \begin{minipage}[t]{13cm}{\footnotesize
      
      \vspace{\dp0}
      } \end{minipage} \\
      \\ \cdashline{2-3}

      & \begin{minipage}[t]{2cm}{Actual\\ Result}\end{minipage}   & 
      \begin{minipage}[t]{13cm}{\footnotesize
      
      \vspace{\dp0}
      } \end{minipage} \\
      \\ \cdashline{2-3}


      & Status          & Not Executed \\ \hline

    \end{longtable}


    \paragraph{Test Case LVV-T548 - Photometric errors -- level 1 processing -- reference catalog
 }\mbox{}\\

Open  \href{https://jira.lsstcorp.org/secure/Tests.jspa#/testCase/LVV-T548}{\textit{ LVV-T548 } }
test case in Jira.

    Test DM contribution to photometric errors by comparing LSST
measurements to external catalog


    \textbf{ Preconditions}:\\
    

    Execution status: {\bf Not Executed }

    Final comment:\\


    Detailed step results:

    \begin{longtable}{p{1cm}p{2cm}p{13cm}}
    \hline
    {Step} & \multicolumn{2}{c}{Description, Results and Status}\\ \hline
      1 & Description &

      \begin{minipage}[t]{13cm}{\footnotesize
      Identify catalog of static sources with well measured fluxes. ~We will
also need a sense for the historical RMS variation of the sources' flux.

      \vspace{\dp0}
      } \end{minipage} \\
      \\ \cdashline{2-3}

      & Expected Result & 

      \begin{minipage}[t]{13cm}{\footnotesize
      Catalog of static sources with flux values and uncertainties

      \vspace{\dp0}
      } \end{minipage} \\
      \\ \cdashline{2-3}

      & \begin{minipage}[t]{2cm}{Actual\\ Result}\end{minipage}   & 
      \begin{minipage}[t]{13cm}{\footnotesize
      
      \vspace{\dp0}
      } \end{minipage} \\
      \\ \cdashline{2-3}


      & Status          & Not Executed \\ \hline

      2 & Description &

      \begin{minipage}[t]{13cm}{\footnotesize
      Image region of sky overlapping the catalog in step 1 at varying
observing conditions (airmass, seeing, etc.)

      \vspace{\dp0}
      } \end{minipage} \\
      \\ \cdashline{2-3}

      & Expected Result & 

      \begin{minipage}[t]{13cm}{\footnotesize
      Images overlapping catalog from step 1

      \vspace{\dp0}
      } \end{minipage} \\
      \\ \cdashline{2-3}

      & \begin{minipage}[t]{2cm}{Actual\\ Result}\end{minipage}   & 
      \begin{minipage}[t]{13cm}{\footnotesize
      
      \vspace{\dp0}
      } \end{minipage} \\
      \\ \cdashline{2-3}


      & Status          & Not Executed \\ \hline

      3 & Description &

      \begin{minipage}[t]{13cm}{\footnotesize
      Perform level 1 processing on images from step 2. ~Keep difference
images.

      \vspace{\dp0}
      } \end{minipage} \\
      \\ \cdashline{2-3}

      & Expected Result & 

      \begin{minipage}[t]{13cm}{\footnotesize
      Catalog of DIASources\\
Difference images

      \vspace{\dp0}
      } \end{minipage} \\
      \\ \cdashline{2-3}

      & \begin{minipage}[t]{2cm}{Actual\\ Result}\end{minipage}   & 
      \begin{minipage}[t]{13cm}{\footnotesize
      
      \vspace{\dp0}
      } \end{minipage} \\
      \\ \cdashline{2-3}


      & Status          & Not Executed \\ \hline

      4 & Description &

      \begin{minipage}[t]{13cm}{\footnotesize
      Perform forced photometry on difference images from step 3 at location
of sources identified in the catalog from step 1.

      \vspace{\dp0}
      } \end{minipage} \\
      \\ \cdashline{2-3}

      & Expected Result & 

      \begin{minipage}[t]{13cm}{\footnotesize
      Catalog of forced difference image photometry measurements.

      \vspace{\dp0}
      } \end{minipage} \\
      \\ \cdashline{2-3}

      & \begin{minipage}[t]{2cm}{Actual\\ Result}\end{minipage}   & 
      \begin{minipage}[t]{13cm}{\footnotesize
      
      \vspace{\dp0}
      } \end{minipage} \\
      \\ \cdashline{2-3}


      & Status          & Not Executed \\ \hline

      5 & Description &

      \begin{minipage}[t]{13cm}{\footnotesize
      Construct model of photometric uncertainty based only on observing
conditions of images in step 1.

      \vspace{\dp0}
      } \end{minipage} \\
      \\ \cdashline{2-3}

      & Expected Result & 

      \begin{minipage}[t]{13cm}{\footnotesize
      Model of photometric uncertainty expected from observing conditions

      \vspace{\dp0}
      } \end{minipage} \\
      \\ \cdashline{2-3}

      & \begin{minipage}[t]{2cm}{Actual\\ Result}\end{minipage}   & 
      \begin{minipage}[t]{13cm}{\footnotesize
      
      \vspace{\dp0}
      } \end{minipage} \\
      \\ \cdashline{2-3}


      & Status          & Not Executed \\ \hline

      6 & Description &

      \begin{minipage}[t]{13cm}{\footnotesize
      Compare force difference image photometry to intrinsic width of source
photometry measurements in step 1 and model of uncertainty from step 5.
~Compare RMS residual to specified tolerance.

      \vspace{\dp0}
      } \end{minipage} \\
      \\ \cdashline{2-3}

      & Expected Result & 

      \begin{minipage}[t]{13cm}{\footnotesize
      
      \vspace{\dp0}
      } \end{minipage} \\
      \\ \cdashline{2-3}

      & \begin{minipage}[t]{2cm}{Actual\\ Result}\end{minipage}   & 
      \begin{minipage}[t]{13cm}{\footnotesize
      
      \vspace{\dp0}
      } \end{minipage} \\
      \\ \cdashline{2-3}


      & Status          & Not Executed \\ \hline

    \end{longtable}


    \paragraph{Test Case LVV-T545 - Astrometric error -- level 1 processing -- reference catalog
 }\mbox{}\\

Open  \href{https://jira.lsstcorp.org/secure/Tests.jspa#/testCase/LVV-T545}{\textit{ LVV-T545 } }
test case in Jira.

    Measure the astrometric performance requirements by comparing actual
data to a reference catalog (e.g. Gaia)


    \textbf{ Preconditions}:\\
    

    Execution status: {\bf Not Executed }

    Final comment:\\


    Detailed step results:

    \begin{longtable}{p{1cm}p{2cm}p{13cm}}
    \hline
    {Step} & \multicolumn{2}{c}{Description, Results and Status}\\ \hline
      1 & Description &

      \begin{minipage}[t]{13cm}{\footnotesize
      Find catalog of sources with well-measured astrometry.

      \vspace{\dp0}
      } \end{minipage} \\
      \\ \cdashline{2-3}

      & Expected Result & 

      \begin{minipage}[t]{13cm}{\footnotesize
      Catalog of sources to be used as ground truth

      \vspace{\dp0}
      } \end{minipage} \\
      \\ \cdashline{2-3}

      & \begin{minipage}[t]{2cm}{Actual\\ Result}\end{minipage}   & 
      \begin{minipage}[t]{13cm}{\footnotesize
      
      \vspace{\dp0}
      } \end{minipage} \\
      \\ \cdashline{2-3}


      & Status          & Not Executed \\ \hline

      2 & Description &

      \begin{minipage}[t]{13cm}{\footnotesize
      Image the area of sky overlapping ground truth catalog from step 1 under
diverse observing conditions (airmass, seeing, etc.).

      \vspace{\dp0}
      } \end{minipage} \\
      \\ \cdashline{2-3}

      & Expected Result & 

      \begin{minipage}[t]{13cm}{\footnotesize
      Images overlapping catalog from step 1

      \vspace{\dp0}
      } \end{minipage} \\
      \\ \cdashline{2-3}

      & \begin{minipage}[t]{2cm}{Actual\\ Result}\end{minipage}   & 
      \begin{minipage}[t]{13cm}{\footnotesize
      
      \vspace{\dp0}
      } \end{minipage} \\
      \\ \cdashline{2-3}


      & Status          & Not Executed \\ \hline

      3 & Description &

      \begin{minipage}[t]{13cm}{\footnotesize
      Perform Level 1 processing on images taken in step 2.

      \vspace{\dp0}
      } \end{minipage} \\
      \\ \cdashline{2-3}

      & Expected Result & 

      \begin{minipage}[t]{13cm}{\footnotesize
      Catalog of DIASources

      \vspace{\dp0}
      } \end{minipage} \\
      \\ \cdashline{2-3}

      & \begin{minipage}[t]{2cm}{Actual\\ Result}\end{minipage}   & 
      \begin{minipage}[t]{13cm}{\footnotesize
      
      \vspace{\dp0}
      } \end{minipage} \\
      \\ \cdashline{2-3}


      & Status          & Not Executed \\ \hline

      4 & Description &

      \begin{minipage}[t]{13cm}{\footnotesize
      Construct a model of astrometric errors due only to observing conditions
of images in step 2.

      \vspace{\dp0}
      } \end{minipage} \\
      \\ \cdashline{2-3}

      & Expected Result & 

      \begin{minipage}[t]{13cm}{\footnotesize
      Model of astrometric errors expected from observing conditions

      \vspace{\dp0}
      } \end{minipage} \\
      \\ \cdashline{2-3}

      & \begin{minipage}[t]{2cm}{Actual\\ Result}\end{minipage}   & 
      \begin{minipage}[t]{13cm}{\footnotesize
      
      \vspace{\dp0}
      } \end{minipage} \\
      \\ \cdashline{2-3}


      & Status          & Not Executed \\ \hline

      5 & Description &

      \begin{minipage}[t]{13cm}{\footnotesize
      Compare DIASources in step 3 to catalog from step 1 to find measured
astrometric errors.

      \vspace{\dp0}
      } \end{minipage} \\
      \\ \cdashline{2-3}

      & Expected Result & 

      \begin{minipage}[t]{13cm}{\footnotesize
      Catalog of measured astrometric errors

      \vspace{\dp0}
      } \end{minipage} \\
      \\ \cdashline{2-3}

      & \begin{minipage}[t]{2cm}{Actual\\ Result}\end{minipage}   & 
      \begin{minipage}[t]{13cm}{\footnotesize
      
      \vspace{\dp0}
      } \end{minipage} \\
      \\ \cdashline{2-3}


      & Status          & Not Executed \\ \hline

      6 & Description &

      \begin{minipage}[t]{13cm}{\footnotesize
      Calculate RMS residual between measured astrometric errors (step 5) and
astrometric errors expected due only to observing conditions (step 4).
~Verify that residual is less than specified limit.

      \vspace{\dp0}
      } \end{minipage} \\
      \\ \cdashline{2-3}

      & Expected Result & 

      \begin{minipage}[t]{13cm}{\footnotesize
      
      \vspace{\dp0}
      } \end{minipage} \\
      \\ \cdashline{2-3}

      & \begin{minipage}[t]{2cm}{Actual\\ Result}\end{minipage}   & 
      \begin{minipage}[t]{13cm}{\footnotesize
      
      \vspace{\dp0}
      } \end{minipage} \\
      \\ \cdashline{2-3}


      & Status          & Not Executed \\ \hline

    \end{longtable}


    \paragraph{Test Case LVV-T389 - Single Visit Photometric Repeatability
 }\mbox{}\\

Open  \href{https://jira.lsstcorp.org/secure/Tests.jspa#/testCase/LVV-T389}{\textit{ LVV-T389 } }
test case in Jira.

    Verify that the RMS of magnitudes in all filters and outlier rate of
magnitudes is within specification


    \textbf{ Preconditions}:\\
    Multi epoch observations of bright, isolated, unresolved, un-saturated
stars, observed at varying photometric conditions, air mass, and water
vapor.


    Execution status: {\bf Not Executed }

    Final comment:\\


    Detailed step results:

    \begin{longtable}{p{1cm}p{2cm}p{13cm}}
    \hline
    {Step} & \multicolumn{2}{c}{Description, Results and Status}\\ \hline
      1 & Description &

      \begin{minipage}[t]{13cm}{\footnotesize
      Define sample data of stars to be used in subsequent tests. ~Columns
needed are camera rotation angle, magnitude in all bands, RA, Dec,
detector position.

      \vspace{\dp0}
      } \end{minipage} \\
      \\ \cdashline{2-3}

      & Expected Result & 

      \begin{minipage}[t]{13cm}{\footnotesize
      
      \vspace{\dp0}
      } \end{minipage} \\
      \\ \cdashline{2-3}

      & \begin{minipage}[t]{2cm}{Actual\\ Result}\end{minipage}   & 
      \begin{minipage}[t]{13cm}{\footnotesize
      
      \vspace{\dp0}
      } \end{minipage} \\
      \\ \cdashline{2-3}


      & Status          & Not Executed \\ \hline

      2 & Description &

      \begin{minipage}[t]{13cm}{\footnotesize
      For each star, measure the RMS magnitude in each filter. This yields a
distribution of RMS values in each filter. Calculate the median of the
distributions for each filter.

      \vspace{\dp0}
      } \end{minipage} \\
      \\ \cdashline{2-3}

      & Expected Result & 

      \begin{minipage}[t]{13cm}{\footnotesize
      
      \vspace{\dp0}
      } \end{minipage} \\
      \\ \cdashline{2-3}

      & \begin{minipage}[t]{2cm}{Actual\\ Result}\end{minipage}   & 
      \begin{minipage}[t]{13cm}{\footnotesize
      
      \vspace{\dp0}
      } \end{minipage} \\
      \\ \cdashline{2-3}


      & Status          & Not Executed \\ \hline

      3 & Description &

      \begin{minipage}[t]{13cm}{\footnotesize
      For each star, calculate the mean magnitude. Calculate the number of
observations which deviate by more than PA2gri (15) millimags from their
means for magnitudes in the g, r, and i bands and PA2uzy (22.5)
millimags ~for magnitudes in the u, z, and y bands.

      \vspace{\dp0}
      } \end{minipage} \\
      \\ \cdashline{2-3}

      & Expected Result & 

      \begin{minipage}[t]{13cm}{\footnotesize
      
      \vspace{\dp0}
      } \end{minipage} \\
      \\ \cdashline{2-3}

      & \begin{minipage}[t]{2cm}{Actual\\ Result}\end{minipage}   & 
      \begin{minipage}[t]{13cm}{\footnotesize
      
      \vspace{\dp0}
      } \end{minipage} \\
      \\ \cdashline{2-3}


      & Status          & Not Executed \\ \hline

      4 & Description &

      \begin{minipage}[t]{13cm}{\footnotesize
      Check the median RMS values for u, z, ~are below PA1uzy (7.5)
millimagniutes

      \vspace{\dp0}
      } \end{minipage} \\
      \\ \cdashline{2-3}

      & Expected Result & 

      \begin{minipage}[t]{13cm}{\footnotesize
      
      \vspace{\dp0}
      } \end{minipage} \\
      \\ \cdashline{2-3}

      & \begin{minipage}[t]{2cm}{Actual\\ Result}\end{minipage}   & 
      \begin{minipage}[t]{13cm}{\footnotesize
      
      \vspace{\dp0}
      } \end{minipage} \\
      \\ \cdashline{2-3}


      & Status          & Not Executed \\ \hline

      5 & Description &

      \begin{minipage}[t]{13cm}{\footnotesize
      Check the median RMS values for g, r, i are below PA1gri (5)
millimagnitudes~

      \vspace{\dp0}
      } \end{minipage} \\
      \\ \cdashline{2-3}

      & Expected Result & 

      \begin{minipage}[t]{13cm}{\footnotesize
      
      \vspace{\dp0}
      } \end{minipage} \\
      \\ \cdashline{2-3}

      & \begin{minipage}[t]{2cm}{Actual\\ Result}\end{minipage}   & 
      \begin{minipage}[t]{13cm}{\footnotesize
      
      \vspace{\dp0}
      } \end{minipage} \\
      \\ \cdashline{2-3}


      & Status          & Not Executed \\ \hline

      6 & Description &

      \begin{minipage}[t]{13cm}{\footnotesize
      Check that less than PF1 (10\%) of measurements deviate from their means
by more than PA2 from step 3

      \vspace{\dp0}
      } \end{minipage} \\
      \\ \cdashline{2-3}

      & Expected Result & 

      \begin{minipage}[t]{13cm}{\footnotesize
      
      \vspace{\dp0}
      } \end{minipage} \\
      \\ \cdashline{2-3}

      & \begin{minipage}[t]{2cm}{Actual\\ Result}\end{minipage}   & 
      \begin{minipage}[t]{13cm}{\footnotesize
      
      \vspace{\dp0}
      } \end{minipage} \\
      \\ \cdashline{2-3}


      & Status          & Not Executed \\ \hline

    \end{longtable}


    \paragraph{Test Case LVV-T390 - The spatial uniformity of photometric zeropoints
 }\mbox{}\\

Open  \href{https://jira.lsstcorp.org/secure/Tests.jspa#/testCase/LVV-T390}{\textit{ LVV-T390 } }
test case in Jira.

    The distribution width (rms) of the internal photometric zero-point
error (the system stability across the sky) will not exceed PA3/PA3u
millimag, and no more than PF2 \% of the distribution will exceed PF4
millimag. ~Applies to both bright and faint ends to constrain
non-linearity of the flux scale


    \textbf{ Preconditions}:\\
    

    Execution status: {\bf Not Executed }

    Final comment:\\


    Detailed step results:

    \begin{longtable}{p{1cm}p{2cm}p{13cm}}
    \hline
    {Step} & \multicolumn{2}{c}{Description, Results and Status}\\ \hline
      1 & Description &

      \begin{minipage}[t]{13cm}{\footnotesize
      Identify fields spread across the sky with available standard stars so
that zero points can be determined. ~Alternatively, identify specific
marker points in color-color diagrams that can be used to define
relative zero point offsets. ~Requires both bright and faint samples.

      \vspace{\dp0}
      } \end{minipage} \\
      \\ \cdashline{2-3}

      & Expected Result & 

      \begin{minipage}[t]{13cm}{\footnotesize
      
      \vspace{\dp0}
      } \end{minipage} \\
      \\ \cdashline{2-3}

      & \begin{minipage}[t]{2cm}{Actual\\ Result}\end{minipage}   & 
      \begin{minipage}[t]{13cm}{\footnotesize
      
      \vspace{\dp0}
      } \end{minipage} \\
      \\ \cdashline{2-3}


      & Status          & Not Executed \\ \hline

      2 & Description &

      \begin{minipage}[t]{13cm}{\footnotesize
      Measure photometric zero points for stars for each patch of the sky

      \vspace{\dp0}
      } \end{minipage} \\
      \\ \cdashline{2-3}

      & Expected Result & 

      \begin{minipage}[t]{13cm}{\footnotesize
      
      \vspace{\dp0}
      } \end{minipage} \\
      \\ \cdashline{2-3}

      & \begin{minipage}[t]{2cm}{Actual\\ Result}\end{minipage}   & 
      \begin{minipage}[t]{13cm}{\footnotesize
      
      \vspace{\dp0}
      } \end{minipage} \\
      \\ \cdashline{2-3}


      & Status          & Not Executed \\ \hline

      3 & Description &

      \begin{minipage}[t]{13cm}{\footnotesize
      Calculate RMS of zero points, check that RMS is less than PA3u (20 mmag)
in the u-band, and less than PA3 (10 mmag) for grizy bands.

      \vspace{\dp0}
      } \end{minipage} \\
      \\ \cdashline{2-3}

      & Expected Result & 

      \begin{minipage}[t]{13cm}{\footnotesize
      
      \vspace{\dp0}
      } \end{minipage} \\
      \\ \cdashline{2-3}

      & \begin{minipage}[t]{2cm}{Actual\\ Result}\end{minipage}   & 
      \begin{minipage}[t]{13cm}{\footnotesize
      
      \vspace{\dp0}
      } \end{minipage} \\
      \\ \cdashline{2-3}


      & Status          & Not Executed \\ \hline

      4 & Description &

      \begin{minipage}[t]{13cm}{\footnotesize
      Check that no more than PF2 percent (10\%) of the zero points exceed PF4
(15 mmag)

      \vspace{\dp0}
      } \end{minipage} \\
      \\ \cdashline{2-3}

      & Expected Result & 

      \begin{minipage}[t]{13cm}{\footnotesize
      
      \vspace{\dp0}
      } \end{minipage} \\
      \\ \cdashline{2-3}

      & \begin{minipage}[t]{2cm}{Actual\\ Result}\end{minipage}   & 
      \begin{minipage}[t]{13cm}{\footnotesize
      
      \vspace{\dp0}
      } \end{minipage} \\
      \\ \cdashline{2-3}


      & Status          & Not Executed \\ \hline

    \end{longtable}


    \paragraph{Test Case LVV-T297 - Absolute Astrometric Performance
 }\mbox{}\\

Open  \href{https://jira.lsstcorp.org/secure/Tests.jspa#/testCase/LVV-T297}{\textit{ LVV-T297 } }
test case in Jira.

    Measure astrometric performance using Gaia as an external reference.


    \textbf{ Preconditions}:\\
    

    Execution status: {\bf Not Executed }

    Final comment:\\


    Detailed step results:

    \begin{longtable}{p{1cm}p{2cm}p{13cm}}
    \hline
    {Step} & \multicolumn{2}{c}{Description, Results and Status}\\ \hline
      1 & Description &

      \begin{minipage}[t]{13cm}{\footnotesize
      
      \vspace{\dp0}
      } \end{minipage} \\
      \\ \cdashline{2-3}

      & Expected Result & 

      \begin{minipage}[t]{13cm}{\footnotesize
      
      \vspace{\dp0}
      } \end{minipage} \\
      \\ \cdashline{2-3}

      & \begin{minipage}[t]{2cm}{Actual\\ Result}\end{minipage}   & 
      \begin{minipage}[t]{13cm}{\footnotesize
      
      \vspace{\dp0}
      } \end{minipage} \\
      \\ \cdashline{2-3}


      & Status          & Not Executed \\ \hline

    \end{longtable}


    \paragraph{Test Case LVV-T298 - Cross-band Astrometric Performance
 }\mbox{}\\

Open  \href{https://jira.lsstcorp.org/secure/Tests.jspa#/testCase/LVV-T298}{\textit{ LVV-T298 } }
test case in Jira.

    

    \textbf{ Preconditions}:\\
    

    Execution status: {\bf Not Executed }

    Final comment:\\


    Detailed step results:

    \begin{longtable}{p{1cm}p{2cm}p{13cm}}
    \hline
    {Step} & \multicolumn{2}{c}{Description, Results and Status}\\ \hline
      1 & Description &

      \begin{minipage}[t]{13cm}{\footnotesize
      
      \vspace{\dp0}
      } \end{minipage} \\
      \\ \cdashline{2-3}

      & Expected Result & 

      \begin{minipage}[t]{13cm}{\footnotesize
      
      \vspace{\dp0}
      } \end{minipage} \\
      \\ \cdashline{2-3}

      & \begin{minipage}[t]{2cm}{Actual\\ Result}\end{minipage}   & 
      \begin{minipage}[t]{13cm}{\footnotesize
      
      \vspace{\dp0}
      } \end{minipage} \\
      \\ \cdashline{2-3}


      & Status          & Not Executed \\ \hline

    \end{longtable}


    \paragraph{Test Case LVV-T299 - Relative Astrometric Performance
 }\mbox{}\\

Open  \href{https://jira.lsstcorp.org/secure/Tests.jspa#/testCase/LVV-T299}{\textit{ LVV-T299 } }
test case in Jira.

    

    \textbf{ Preconditions}:\\
    

    Execution status: {\bf Not Executed }

    Final comment:\\


    Detailed step results:

    \begin{longtable}{p{1cm}p{2cm}p{13cm}}
    \hline
    {Step} & \multicolumn{2}{c}{Description, Results and Status}\\ \hline
      1 & Description &

      \begin{minipage}[t]{13cm}{\footnotesize
      
      \vspace{\dp0}
      } \end{minipage} \\
      \\ \cdashline{2-3}

      & Expected Result & 

      \begin{minipage}[t]{13cm}{\footnotesize
      
      \vspace{\dp0}
      } \end{minipage} \\
      \\ \cdashline{2-3}

      & \begin{minipage}[t]{2cm}{Actual\\ Result}\end{minipage}   & 
      \begin{minipage}[t]{13cm}{\footnotesize
      
      \vspace{\dp0}
      } \end{minipage} \\
      \\ \cdashline{2-3}


      & Status          & Not Executed \\ \hline

    \end{longtable}


    \paragraph{Test Case LVV-T360 - Off Zenith Image Quality Degradation
 }\mbox{}\\

Open  \href{https://jira.lsstcorp.org/secure/Tests.jspa#/testCase/LVV-T360}{\textit{ LVV-T360 } }
test case in Jira.

    

    \textbf{ Preconditions}:\\
    

    Execution status: {\bf Not Executed }

    Final comment:\\


    Detailed step results:

    \begin{longtable}{p{1cm}p{2cm}p{13cm}}
    \hline
    {Step} & \multicolumn{2}{c}{Description, Results and Status}\\ \hline
      1 & Description &

      \begin{minipage}[t]{13cm}{\footnotesize
      
      \vspace{\dp0}
      } \end{minipage} \\
      \\ \cdashline{2-3}

      & Expected Result & 

      \begin{minipage}[t]{13cm}{\footnotesize
      
      \vspace{\dp0}
      } \end{minipage} \\
      \\ \cdashline{2-3}

      & \begin{minipage}[t]{2cm}{Actual\\ Result}\end{minipage}   & 
      \begin{minipage}[t]{13cm}{\footnotesize
      
      \vspace{\dp0}
      } \end{minipage} \\
      \\ \cdashline{2-3}


      & Status          & Not Executed \\ \hline

    \end{longtable}


  \subsection{Test Cycle LVV-C5 }

Open test cycle {\it \href{https://jira.lsstcorp.org/secure/Tests.jspa#/testrun/LVV-C5}{Commissioning SV: Full-survey Performance w/ ComCam
}} in Jira.

  Commissioning SV: Full-survey Performance w/ ComCam
\\
  Status: Not Executed

  

  \subsubsection{Software Version/Baseline}
    Not provided.

  \subsubsection{Configuration}
    Not provided.

  \subsubsection{Test Cases in LVV-C5 Test Cycle}


    \paragraph{Test Case LVV-T986 - MOPS -- orbit association at catalog level
 }\mbox{}\\

Open  \href{https://jira.lsstcorp.org/secure/Tests.jspa#/testCase/LVV-T986}{\textit{ LVV-T986 } }
test case in Jira.

    Use catalog-level tests to verify that MOPS is correctly associating
orbits at the correct rate


    \textbf{ Preconditions}:\\
    

    Execution status: {\bf Not Executed }

    Final comment:\\


    Detailed step results:

    \begin{longtable}{p{1cm}p{2cm}p{13cm}}
    \hline
    {Step} & \multicolumn{2}{c}{Description, Results and Status}\\ \hline
      1 & Description &

      \begin{minipage}[t]{13cm}{\footnotesize
      Extract catalogs with sequence of observations meeting the requirements
orbitNightlyObservationInterval = 90{[}minute{]} Interval over which a
reference test case Solar System object must be observed within a night,
orbitObservations = 2{[}integer{]} Number of detections within a single
night required to define the reference test case for Solar System
objects, orbitObservationInterval = 3{[}day{]}

      \vspace{\dp0}
      } \end{minipage} \\
      \\ \cdashline{2-3}

      & Expected Result & 

      \begin{minipage}[t]{13cm}{\footnotesize
      
      \vspace{\dp0}
      } \end{minipage} \\
      \\ \cdashline{2-3}

      & \begin{minipage}[t]{2cm}{Actual\\ Result}\end{minipage}   & 
      \begin{minipage}[t]{13cm}{\footnotesize
      
      \vspace{\dp0}
      } \end{minipage} \\
      \\ \cdashline{2-3}


      & Status          & Not Executed \\ \hline

      2 & Description &

      \begin{minipage}[t]{13cm}{\footnotesize
      Simulate sequence of asteroid positions given these observations
(pointing and time) using a current Solar System model

      \vspace{\dp0}
      } \end{minipage} \\
      \\ \cdashline{2-3}

      & Expected Result & 

      \begin{minipage}[t]{13cm}{\footnotesize
      
      \vspace{\dp0}
      } \end{minipage} \\
      \\ \cdashline{2-3}

      & \begin{minipage}[t]{2cm}{Actual\\ Result}\end{minipage}   & 
      \begin{minipage}[t]{13cm}{\footnotesize
      
      \vspace{\dp0}
      } \end{minipage} \\
      \\ \cdashline{2-3}


      & Status          & Not Executed \\ \hline

      3 & Description &

      \begin{minipage}[t]{13cm}{\footnotesize
      Inject simulated asteroids into LSST catalog

      \vspace{\dp0}
      } \end{minipage} \\
      \\ \cdashline{2-3}

      & Expected Result & 

      \begin{minipage}[t]{13cm}{\footnotesize
      
      \vspace{\dp0}
      } \end{minipage} \\
      \\ \cdashline{2-3}

      & \begin{minipage}[t]{2cm}{Actual\\ Result}\end{minipage}   & 
      \begin{minipage}[t]{13cm}{\footnotesize
      
      \vspace{\dp0}
      } \end{minipage} \\
      \\ \cdashline{2-3}


      & Status          & Not Executed \\ \hline

      4 & Description &

      \begin{minipage}[t]{13cm}{\footnotesize
      Run MOPS

      \vspace{\dp0}
      } \end{minipage} \\
      \\ \cdashline{2-3}

      & Expected Result & 

      \begin{minipage}[t]{13cm}{\footnotesize
      
      \vspace{\dp0}
      } \end{minipage} \\
      \\ \cdashline{2-3}

      & \begin{minipage}[t]{2cm}{Actual\\ Result}\end{minipage}   & 
      \begin{minipage}[t]{13cm}{\footnotesize
      
      \vspace{\dp0}
      } \end{minipage} \\
      \\ \cdashline{2-3}


      & Status          & Not Executed \\ \hline

      5 & Description &

      \begin{minipage}[t]{13cm}{\footnotesize
      Match simulated ~asteroids

      \vspace{\dp0}
      } \end{minipage} \\
      \\ \cdashline{2-3}

      & Expected Result & 

      \begin{minipage}[t]{13cm}{\footnotesize
      
      \vspace{\dp0}
      } \end{minipage} \\
      \\ \cdashline{2-3}

      & \begin{minipage}[t]{2cm}{Actual\\ Result}\end{minipage}   & 
      \begin{minipage}[t]{13cm}{\footnotesize
      
      \vspace{\dp0}
      } \end{minipage} \\
      \\ \cdashline{2-3}


      & Status          & Not Executed \\ \hline

      6 & Description &

      \begin{minipage}[t]{13cm}{\footnotesize
      Plot fraction of recovered asteroids as a function of limiting magnitude
(possibly for different populations of asteroids)

      \vspace{\dp0}
      } \end{minipage} \\
      \\ \cdashline{2-3}

      & Expected Result & 

      \begin{minipage}[t]{13cm}{\footnotesize
      
      \vspace{\dp0}
      } \end{minipage} \\
      \\ \cdashline{2-3}

      & \begin{minipage}[t]{2cm}{Actual\\ Result}\end{minipage}   & 
      \begin{minipage}[t]{13cm}{\footnotesize
      
      \vspace{\dp0}
      } \end{minipage} \\
      \\ \cdashline{2-3}


      & Status          & Not Executed \\ \hline

    \end{longtable}


    \paragraph{Test Case LVV-T969 - Alert completeness
 }\mbox{}\\

Open  \href{https://jira.lsstcorp.org/secure/Tests.jspa#/testCase/LVV-T969}{\textit{ LVV-T969 } }
test case in Jira.

    Verify that all difference image sources above SNR=5 get broadcast as
alerts


    \textbf{ Preconditions}:\\
    

    Execution status: {\bf Not Executed }

    Final comment:\\


    Detailed step results:

    \begin{longtable}{p{1cm}p{2cm}p{13cm}}
    \hline
    {Step} & \multicolumn{2}{c}{Description, Results and Status}\\ \hline
      1 & Description &

      \begin{minipage}[t]{13cm}{\footnotesize
      Run mini-survey over a densely populated field

      \vspace{\dp0}
      } \end{minipage} \\
      \\ \cdashline{2-3}

      & Expected Result & 

      \begin{minipage}[t]{13cm}{\footnotesize
      
      \vspace{\dp0}
      } \end{minipage} \\
      \\ \cdashline{2-3}

      & \begin{minipage}[t]{2cm}{Actual\\ Result}\end{minipage}   & 
      \begin{minipage}[t]{13cm}{\footnotesize
      
      \vspace{\dp0}
      } \end{minipage} \\
      \\ \cdashline{2-3}


      & Status          & Not Executed \\ \hline

      2 & Description &

      \begin{minipage}[t]{13cm}{\footnotesize
      Perform difference image analysis and alert production on that
mini-survey

      \vspace{\dp0}
      } \end{minipage} \\
      \\ \cdashline{2-3}

      & Expected Result & 

      \begin{minipage}[t]{13cm}{\footnotesize
      
      \vspace{\dp0}
      } \end{minipage} \\
      \\ \cdashline{2-3}

      & \begin{minipage}[t]{2cm}{Actual\\ Result}\end{minipage}   & 
      \begin{minipage}[t]{13cm}{\footnotesize
      
      \vspace{\dp0}
      } \end{minipage} \\
      \\ \cdashline{2-3}


      & Status          & Not Executed \\ \hline

      3 & Description &

      \begin{minipage}[t]{13cm}{\footnotesize
      Revisit difference images used to produce alerts in step 2. ~Verify by
hand that all 5-sigma detections in those difference images resulted in
alerts.

      \vspace{\dp0}
      } \end{minipage} \\
      \\ \cdashline{2-3}

      & Expected Result & 

      \begin{minipage}[t]{13cm}{\footnotesize
      
      \vspace{\dp0}
      } \end{minipage} \\
      \\ \cdashline{2-3}

      & \begin{minipage}[t]{2cm}{Actual\\ Result}\end{minipage}   & 
      \begin{minipage}[t]{13cm}{\footnotesize
      
      \vspace{\dp0}
      } \end{minipage} \\
      \\ \cdashline{2-3}


      & Status          & Not Executed \\ \hline

      4 & Description &

      \begin{minipage}[t]{13cm}{\footnotesize
      Verify that at least one image resulted in at least 10,000 alerts,
verifying that the Data Management system was able to handle a volume of
up to 10,000 alerts per image.

      \vspace{\dp0}
      } \end{minipage} \\
      \\ \cdashline{2-3}

      & Expected Result & 

      \begin{minipage}[t]{13cm}{\footnotesize
      
      \vspace{\dp0}
      } \end{minipage} \\
      \\ \cdashline{2-3}

      & \begin{minipage}[t]{2cm}{Actual\\ Result}\end{minipage}   & 
      \begin{minipage}[t]{13cm}{\footnotesize
      
      \vspace{\dp0}
      } \end{minipage} \\
      \\ \cdashline{2-3}


      & Status          & Not Executed \\ \hline

    \end{longtable}


    \paragraph{Test Case LVV-T968 - WCS measurement and reporting
 }\mbox{}\\

Open  \href{https://jira.lsstcorp.org/secure/Tests.jspa#/testCase/LVV-T968}{\textit{ LVV-T968 } }
test case in Jira.

    Verify that the Data Management system can produce and report a WCS
solution for an image with the required accuracy in the required amount
of time


    \textbf{ Preconditions}:\\
    

    Execution status: {\bf Not Executed }

    Final comment:\\


    Detailed step results:

    \begin{longtable}{p{1cm}p{2cm}p{13cm}}
    \hline
    {Step} & \multicolumn{2}{c}{Description, Results and Status}\\ \hline
      1 & Description &

      \begin{minipage}[t]{13cm}{\footnotesize
      While telescope is running in a survey-like setting, observe that WCS
data is being reported within 60 seconds of images being taken.

      \vspace{\dp0}
      } \end{minipage} \\
      \\ \cdashline{2-3}

      & Expected Result & 

      \begin{minipage}[t]{13cm}{\footnotesize
      
      \vspace{\dp0}
      } \end{minipage} \\
      \\ \cdashline{2-3}

      & \begin{minipage}[t]{2cm}{Actual\\ Result}\end{minipage}   & 
      \begin{minipage}[t]{13cm}{\footnotesize
      
      \vspace{\dp0}
      } \end{minipage} \\
      \\ \cdashline{2-3}


      & Status          & Not Executed \\ \hline

      2 & Description &

      \begin{minipage}[t]{13cm}{\footnotesize
      Identify the reference stars used for astrometric solutions in the
images from step 1. ~Compare their positions as reported in the
reference catalog to their positions according to the reported WCS.
~Verify that the residuals between catalog and WCS position have an RMS
less than 0.2 arcseconds.

      \vspace{\dp0}
      } \end{minipage} \\
      \\ \cdashline{2-3}

      & Expected Result & 

      \begin{minipage}[t]{13cm}{\footnotesize
      
      \vspace{\dp0}
      } \end{minipage} \\
      \\ \cdashline{2-3}

      & \begin{minipage}[t]{2cm}{Actual\\ Result}\end{minipage}   & 
      \begin{minipage}[t]{13cm}{\footnotesize
      
      \vspace{\dp0}
      } \end{minipage} \\
      \\ \cdashline{2-3}


      & Status          & Not Executed \\ \hline

    \end{longtable}


    \paragraph{Test Case LVV-T176 - Verify implementation of Infrastructure Sizing for ``catching up''
 }\mbox{}\\

Open  \href{https://jira.lsstcorp.org/secure/Tests.jspa#/testCase/LVV-T176}{\textit{ LVV-T176 } }
test case in Jira.

    Demonstrate Data Management System has sufficient excess capacity
(compute infrastructure) to process one night's data (2800 exposures)
within 24 hours while also maintaining nightly Alert Production (note
this is very similar to
\href{https://jira.lsstcorp.org/secure/Tests.jspa\#/testCase/LVV-T173}{LVV-T173}).~


    \textbf{ Preconditions}:\\
    

    Execution status: {\bf Not Executed }

    Final comment:\\


    Detailed step results:

    \begin{longtable}{p{1cm}p{2cm}p{13cm}}
    \hline
    {Step} & \multicolumn{2}{c}{Description, Results and Status}\\ \hline
      1 & Description &

      \begin{minipage}[t]{13cm}{\footnotesize
      Execute single-day operations rehearsal including catch-up after
failure, observe data products generated in time

      \vspace{\dp0}
      } \end{minipage} \\
      \\ \cdashline{2-3}

      & Expected Result & 

      \begin{minipage}[t]{13cm}{\footnotesize
      
      \vspace{\dp0}
      } \end{minipage} \\
      \\ \cdashline{2-3}

      & \begin{minipage}[t]{2cm}{Actual\\ Result}\end{minipage}   & 
      \begin{minipage}[t]{13cm}{\footnotesize
      
      \vspace{\dp0}
      } \end{minipage} \\
      \\ \cdashline{2-3}


      & Status          & Not Executed \\ \hline

    \end{longtable}


    \paragraph{Test Case LVV-T966 - Image storing during outage
 }\mbox{}\\

Open  \href{https://jira.lsstcorp.org/secure/Tests.jspa#/testCase/LVV-T966}{\textit{ LVV-T966 } }
test case in Jira.

    Verify that the Data Management system can hold up to 48 hours of images
in the event that base-to-summit communications are lost


    \textbf{ Preconditions}:\\
    

    Execution status: {\bf Not Executed }

    Final comment:\\


    Detailed step results:

    \begin{longtable}{p{1cm}p{2cm}p{13cm}}
    \hline
    {Step} & \multicolumn{2}{c}{Description, Results and Status}\\ \hline
      1 & Description &

      \begin{minipage}[t]{13cm}{\footnotesize
      While telescope is running, instruct the Data Management pipeline to
cache data as if it cannot communicate with the base facility. ~Verify
that the system can run for 48 hours in this mode without filling up the
cache of images.

      \vspace{\dp0}
      } \end{minipage} \\
      \\ \cdashline{2-3}

      & Expected Result & 

      \begin{minipage}[t]{13cm}{\footnotesize
      
      \vspace{\dp0}
      } \end{minipage} \\
      \\ \cdashline{2-3}

      & \begin{minipage}[t]{2cm}{Actual\\ Result}\end{minipage}   & 
      \begin{minipage}[t]{13cm}{\footnotesize
      
      \vspace{\dp0}
      } \end{minipage} \\
      \\ \cdashline{2-3}


      & Status          & Not Executed \\ \hline

    \end{longtable}


    \paragraph{Test Case LVV-T963 - Alert generation reliability
 }\mbox{}\\

Open  \href{https://jira.lsstcorp.org/secure/Tests.jspa#/testCase/LVV-T963}{\textit{ LVV-T963 } }
test case in Jira.

    Verify that the alert production pipeline does not skip or delay alert
production on too many images


    \textbf{ Preconditions}:\\
    

    Execution status: {\bf Not Executed }

    Final comment:\\


    Detailed step results:

    \begin{longtable}{p{1cm}p{2cm}p{13cm}}
    \hline
    {Step} & \multicolumn{2}{c}{Description, Results and Status}\\ \hline
      1 & Description &

      \begin{minipage}[t]{13cm}{\footnotesize
      Perform a mini-survey using a realistic scheduling algorithm. ~Run the
images through the full alert-production pipeline.\\[2\baselineskip]Log
all images taken by the camera and all alerts generated.

      \vspace{\dp0}
      } \end{minipage} \\
      \\ \cdashline{2-3}

      & Expected Result & 

      \begin{minipage}[t]{13cm}{\footnotesize
      A list of images taken and the resulting alerts.

      \vspace{\dp0}
      } \end{minipage} \\
      \\ \cdashline{2-3}

      & \begin{minipage}[t]{2cm}{Actual\\ Result}\end{minipage}   & 
      \begin{minipage}[t]{13cm}{\footnotesize
      
      \vspace{\dp0}
      } \end{minipage} \\
      \\ \cdashline{2-3}


      & Status          & Not Executed \\ \hline

      2 & Description &

      \begin{minipage}[t]{13cm}{\footnotesize
      Verify that 99\% of images resulted in alerts being broadcast within 1
minute of the image being taken.

      \vspace{\dp0}
      } \end{minipage} \\
      \\ \cdashline{2-3}

      & Expected Result & 

      \begin{minipage}[t]{13cm}{\footnotesize
      
      \vspace{\dp0}
      } \end{minipage} \\
      \\ \cdashline{2-3}

      & \begin{minipage}[t]{2cm}{Actual\\ Result}\end{minipage}   & 
      \begin{minipage}[t]{13cm}{\footnotesize
      
      \vspace{\dp0}
      } \end{minipage} \\
      \\ \cdashline{2-3}


      & Status          & Not Executed \\ \hline

      3 & Description &

      \begin{minipage}[t]{13cm}{\footnotesize
      Verify that 99.9\% of images resulted in alerts being broadcast at all.

      \vspace{\dp0}
      } \end{minipage} \\
      \\ \cdashline{2-3}

      & Expected Result & 

      \begin{minipage}[t]{13cm}{\footnotesize
      
      \vspace{\dp0}
      } \end{minipage} \\
      \\ \cdashline{2-3}

      & \begin{minipage}[t]{2cm}{Actual\\ Result}\end{minipage}   & 
      \begin{minipage}[t]{13cm}{\footnotesize
      
      \vspace{\dp0}
      } \end{minipage} \\
      \\ \cdashline{2-3}


      & Status          & Not Executed \\ \hline

    \end{longtable}


    \paragraph{Test Case LVV-T962 - PSF ellipticity correlations
 }\mbox{}\\

Open  \href{https://jira.lsstcorp.org/secure/Tests.jspa#/testCase/LVV-T962}{\textit{ LVV-T962 } }
test case in Jira.

    Verify that correlations in the ellipticity of the PSF measured on a
single image are not excessive


    \textbf{ Preconditions}:\\
    

    Execution status: {\bf Not Executed }

    Final comment:\\


    Detailed step results:

    \begin{longtable}{p{1cm}p{2cm}p{13cm}}
    \hline
    {Step} & \multicolumn{2}{c}{Description, Results and Status}\\ \hline
      1 & Description &

      \begin{minipage}[t]{13cm}{\footnotesize
      Image patches of sky at a wide variation of galactic latitude (to vary
the number of sources used for PSF measurement) and airmasses (to vary
the effects of the atmosphere on the PSF).

      \vspace{\dp0}
      } \end{minipage} \\
      \\ \cdashline{2-3}

      & Expected Result & 

      \begin{minipage}[t]{13cm}{\footnotesize
      A set of images.

      \vspace{\dp0}
      } \end{minipage} \\
      \\ \cdashline{2-3}

      & \begin{minipage}[t]{2cm}{Actual\\ Result}\end{minipage}   & 
      \begin{minipage}[t]{13cm}{\footnotesize
      
      \vspace{\dp0}
      } \end{minipage} \\
      \\ \cdashline{2-3}


      & Status          & Not Executed \\ \hline

      2 & Description &

      \begin{minipage}[t]{13cm}{\footnotesize
      For each image in step 1, measure the PSF and decompose it into the
functional form we are going to use to interpolate the PSF onto the
positions of galaxies.

      \vspace{\dp0}
      } \end{minipage} \\
      \\ \cdashline{2-3}

      & Expected Result & 

      \begin{minipage}[t]{13cm}{\footnotesize
      PSF measurements in the images from step 1

      \vspace{\dp0}
      } \end{minipage} \\
      \\ \cdashline{2-3}

      & \begin{minipage}[t]{2cm}{Actual\\ Result}\end{minipage}   & 
      \begin{minipage}[t]{13cm}{\footnotesize
      
      \vspace{\dp0}
      } \end{minipage} \\
      \\ \cdashline{2-3}


      & Status          & Not Executed \\ \hline

      3 & Description &

      \begin{minipage}[t]{13cm}{\footnotesize
      Use the PSF interpolation function to calculate the theoretically
expected PSF at the position of stars used to measure the PSF so that we
can accurately measure the residual between the measured PSF ellipticity
and the interpolated PSF ellipticity.

      \vspace{\dp0}
      } \end{minipage} \\
      \\ \cdashline{2-3}

      & Expected Result & 

      \begin{minipage}[t]{13cm}{\footnotesize
      Calculated PSF at positions of stars used to measure the PSF

      \vspace{\dp0}
      } \end{minipage} \\
      \\ \cdashline{2-3}

      & \begin{minipage}[t]{2cm}{Actual\\ Result}\end{minipage}   & 
      \begin{minipage}[t]{13cm}{\footnotesize
      
      \vspace{\dp0}
      } \end{minipage} \\
      \\ \cdashline{2-3}


      & Status          & Not Executed \\ \hline

      4 & Description &

      \begin{minipage}[t]{13cm}{\footnotesize
      For each of the stars used to calculate the PSF, calculate the e1 and e2
parameters specified in equations (10) and (11) of the Science
Requirements Document.\\[2\baselineskip]Calculate e1 and e2 for the PSF
as calculated from the interpolation function.\\[2\baselineskip]Subtract
the measured e1 and e2 from the interpolated e1 and e2 to get the
residuals delta\_e1, delta\_e2.

      \vspace{\dp0}
      } \end{minipage} \\
      \\ \cdashline{2-3}

      & Expected Result & 

      \begin{minipage}[t]{13cm}{\footnotesize
      distribution of PSF ellipticity residuals across the focal plane

      \vspace{\dp0}
      } \end{minipage} \\
      \\ \cdashline{2-3}

      & \begin{minipage}[t]{2cm}{Actual\\ Result}\end{minipage}   & 
      \begin{minipage}[t]{13cm}{\footnotesize
      
      \vspace{\dp0}
      } \end{minipage} \\
      \\ \cdashline{2-3}


      & Status          & Not Executed \\ \hline

      5 & Description &

      \begin{minipage}[t]{13cm}{\footnotesize
      Calculate the correlation between pairs of sources of the PSF
ellipticity residuals from step 4 as a function of separation angle
between sources.\\[2\baselineskip]Verify that:\\[2\baselineskip]1) The
median PSF ellipticity correlations on 1 arcminute scales is less than
2.0e-5 (on both E1 and E2)\\[2\baselineskip]2) The median PSF
ellipticity correlations on 5 arcminutes scales is less than
1.0e-7\\[2\baselineskip]3) No more than 15\% of source pairs at 1
arcminute scales have PSF ellipticity correlations that exceed 4.0e-5
(on both E1 and E2)\\[2\baselineskip]4) No more than 15\% of source
pairs at 5 arcminute scales have PSF ellipticity correlations greater
than 2.0e-7

      \vspace{\dp0}
      } \end{minipage} \\
      \\ \cdashline{2-3}

      & Expected Result & 

      \begin{minipage}[t]{13cm}{\footnotesize
      
      \vspace{\dp0}
      } \end{minipage} \\
      \\ \cdashline{2-3}

      & \begin{minipage}[t]{2cm}{Actual\\ Result}\end{minipage}   & 
      \begin{minipage}[t]{13cm}{\footnotesize
      
      \vspace{\dp0}
      } \end{minipage} \\
      \\ \cdashline{2-3}


      & Status          & Not Executed \\ \hline

    \end{longtable}


    \paragraph{Test Case LVV-T950 - DIASource misassociation rate
 }\mbox{}\\

Open  \href{https://jira.lsstcorp.org/secure/Tests.jspa#/testCase/LVV-T950}{\textit{ LVV-T950 } }
test case in Jira.

    Verify that DIASources are not misassociated with DIAObjects at too
large a rate


    \textbf{ Preconditions}:\\
    

    Execution status: {\bf Not Executed }

    Final comment:\\


    Detailed step results:

    \begin{longtable}{p{1cm}p{2cm}p{13cm}}
    \hline
    {Step} & \multicolumn{2}{c}{Description, Results and Status}\\ \hline
      1 & Description &

      \begin{minipage}[t]{13cm}{\footnotesize
      Collect images from a minisurvey

      \vspace{\dp0}
      } \end{minipage} \\
      \\ \cdashline{2-3}

      & Expected Result & 

      \begin{minipage}[t]{13cm}{\footnotesize
      Images and calibration products from a minisurvey

      \vspace{\dp0}
      } \end{minipage} \\
      \\ \cdashline{2-3}

      & \begin{minipage}[t]{2cm}{Actual\\ Result}\end{minipage}   & 
      \begin{minipage}[t]{13cm}{\footnotesize
      
      \vspace{\dp0}
      } \end{minipage} \\
      \\ \cdashline{2-3}


      & Status          & Not Executed \\ \hline

      2 & Description &

      \begin{minipage}[t]{13cm}{\footnotesize
      Use synpipe (or some other tool) to inject variable sources with
different degrees of variability and different signal-to-noise ratios
into the images from step 1

      \vspace{\dp0}
      } \end{minipage} \\
      \\ \cdashline{2-3}

      & Expected Result & 

      \begin{minipage}[t]{13cm}{\footnotesize
      Images with an artificial population of variable sources

      \vspace{\dp0}
      } \end{minipage} \\
      \\ \cdashline{2-3}

      & \begin{minipage}[t]{2cm}{Actual\\ Result}\end{minipage}   & 
      \begin{minipage}[t]{13cm}{\footnotesize
      
      \vspace{\dp0}
      } \end{minipage} \\
      \\ \cdashline{2-3}


      & Status          & Not Executed \\ \hline

      3 & Description &

      \begin{minipage}[t]{13cm}{\footnotesize
      Perform level 1 processing on the images with artificial variable
sources in them

      \vspace{\dp0}
      } \end{minipage} \\
      \\ \cdashline{2-3}

      & Expected Result & 

      \begin{minipage}[t]{13cm}{\footnotesize
      Catalog of DIASources and DIAObjects

      \vspace{\dp0}
      } \end{minipage} \\
      \\ \cdashline{2-3}

      & \begin{minipage}[t]{2cm}{Actual\\ Result}\end{minipage}   & 
      \begin{minipage}[t]{13cm}{\footnotesize
      
      \vspace{\dp0}
      } \end{minipage} \\
      \\ \cdashline{2-3}


      & Status          & Not Executed \\ \hline

      4 & Description &

      \begin{minipage}[t]{13cm}{\footnotesize
      Consider only the artificial variables injected in step 2 (for which we
know the ground truth). ~Verify that they are not misassociated with
DIAObjects at a larger than acceptable rate.

      \vspace{\dp0}
      } \end{minipage} \\
      \\ \cdashline{2-3}

      & Expected Result & 

      \begin{minipage}[t]{13cm}{\footnotesize
      
      \vspace{\dp0}
      } \end{minipage} \\
      \\ \cdashline{2-3}

      & \begin{minipage}[t]{2cm}{Actual\\ Result}\end{minipage}   & 
      \begin{minipage}[t]{13cm}{\footnotesize
      
      \vspace{\dp0}
      } \end{minipage} \\
      \\ \cdashline{2-3}


      & Status          & Not Executed \\ \hline

    \end{longtable}


    \paragraph{Test Case LVV-T943 - Provenance in Level 1 catalogs
 }\mbox{}\\

Open  \href{https://jira.lsstcorp.org/secure/Tests.jspa#/testCase/LVV-T943}{\textit{ LVV-T943 } }
test case in Jira.

    Verify that provenance information is correctly stored in Level 1
catalog products


    \textbf{ Preconditions}:\\
    

    Execution status: {\bf Not Executed }

    Final comment:\\


    Detailed step results:

    \begin{longtable}{p{1cm}p{2cm}p{13cm}}
    \hline
    {Step} & \multicolumn{2}{c}{Description, Results and Status}\\ \hline
      1 & Description &

      \begin{minipage}[t]{13cm}{\footnotesize
      {Ingest raw data from L1 Test Stand DAQ, simulating each observing
mode\\
}

      \vspace{\dp0}
      } \end{minipage} \\
      \\ \cdashline{2-3}

      & Expected Result & 

      \begin{minipage}[t]{13cm}{\footnotesize
      
      \vspace{\dp0}
      } \end{minipage} \\
      \\ \cdashline{2-3}

      & \begin{minipage}[t]{2cm}{Actual\\ Result}\end{minipage}   & 
      \begin{minipage}[t]{13cm}{\footnotesize
      
      \vspace{\dp0}
      } \end{minipage} \\
      \\ \cdashline{2-3}


      & Status          & Not Executed \\ \hline

      2 & Description &

      \begin{minipage}[t]{13cm}{\footnotesize
      O{bserve image metadata is present and queryable}

      \vspace{\dp0}
      } \end{minipage} \\
      \\ \cdashline{2-3}

      & Expected Result & 

      \begin{minipage}[t]{13cm}{\footnotesize
      
      \vspace{\dp0}
      } \end{minipage} \\
      \\ \cdashline{2-3}

      & \begin{minipage}[t]{2cm}{Actual\\ Result}\end{minipage}   & 
      \begin{minipage}[t]{13cm}{\footnotesize
      
      \vspace{\dp0}
      } \end{minipage} \\
      \\ \cdashline{2-3}


      & Status          & Not Executed \\ \hline

      3 & Description &

      \begin{minipage}[t]{13cm}{\footnotesize
      Ingest data from L1 Camera Test Stand DAQ

      \vspace{\dp0}
      } \end{minipage} \\
      \\ \cdashline{2-3}

      & Expected Result & 

      \begin{minipage}[t]{13cm}{\footnotesize
      
      \vspace{\dp0}
      } \end{minipage} \\
      \\ \cdashline{2-3}

      & \begin{minipage}[t]{2cm}{Actual\\ Result}\end{minipage}   & 
      \begin{minipage}[t]{13cm}{\footnotesize
      
      \vspace{\dp0}
      } \end{minipage} \\
      \\ \cdashline{2-3}


      & Status          & Not Executed \\ \hline

      4 & Description &

      \begin{minipage}[t]{13cm}{\footnotesize
      Simulate all different modes

      \vspace{\dp0}
      } \end{minipage} \\
      \\ \cdashline{2-3}

      & Expected Result & 

      \begin{minipage}[t]{13cm}{\footnotesize
      
      \vspace{\dp0}
      } \end{minipage} \\
      \\ \cdashline{2-3}

      & \begin{minipage}[t]{2cm}{Actual\\ Result}\end{minipage}   & 
      \begin{minipage}[t]{13cm}{\footnotesize
      
      \vspace{\dp0}
      } \end{minipage} \\
      \\ \cdashline{2-3}


      & Status          & Not Executed \\ \hline

      5 & Description &

      \begin{minipage}[t]{13cm}{\footnotesize
      Verify that a raw image is constructed in correct format

      \vspace{\dp0}
      } \end{minipage} \\
      \\ \cdashline{2-3}

      & Expected Result & 

      \begin{minipage}[t]{13cm}{\footnotesize
      
      \vspace{\dp0}
      } \end{minipage} \\
      \\ \cdashline{2-3}

      & \begin{minipage}[t]{2cm}{Actual\\ Result}\end{minipage}   & 
      \begin{minipage}[t]{13cm}{\footnotesize
      
      \vspace{\dp0}
      } \end{minipage} \\
      \\ \cdashline{2-3}


      & Status          & Not Executed \\ \hline

      6 & Description &

      \begin{minipage}[t]{13cm}{\footnotesize
      Verify that a raw image is constructed with correct metadata

      \vspace{\dp0}
      } \end{minipage} \\
      \\ \cdashline{2-3}

      & Expected Result & 

      \begin{minipage}[t]{13cm}{\footnotesize
      
      \vspace{\dp0}
      } \end{minipage} \\
      \\ \cdashline{2-3}

      & \begin{minipage}[t]{2cm}{Actual\\ Result}\end{minipage}   & 
      \begin{minipage}[t]{13cm}{\footnotesize
      
      \vspace{\dp0}
      } \end{minipage} \\
      \\ \cdashline{2-3}


      & Status          & Not Executed \\ \hline

      7 & Description &

      \begin{minipage}[t]{13cm}{\footnotesize
      Verify that time of exposure start/end, site metadata, telescope
metadata, and camera metadata are stored in DMS
system.\\[2\baselineskip]

      \vspace{\dp0}
      } \end{minipage} \\
      \\ \cdashline{2-3}

      & Expected Result & 

      \begin{minipage}[t]{13cm}{\footnotesize
      
      \vspace{\dp0}
      } \end{minipage} \\
      \\ \cdashline{2-3}

      & \begin{minipage}[t]{2cm}{Actual\\ Result}\end{minipage}   & 
      \begin{minipage}[t]{13cm}{\footnotesize
      
      \vspace{\dp0}
      } \end{minipage} \\
      \\ \cdashline{2-3}


      & Status          & Not Executed \\ \hline

      8 & Description &

      \begin{minipage}[t]{13cm}{\footnotesize
      Detect sources on difference images from step 1

      \vspace{\dp0}
      } \end{minipage} \\
      \\ \cdashline{2-3}

      & Expected Result & 

      \begin{minipage}[t]{13cm}{\footnotesize
      Catalog of DIASources

      \vspace{\dp0}
      } \end{minipage} \\
      \\ \cdashline{2-3}

      & \begin{minipage}[t]{2cm}{Actual\\ Result}\end{minipage}   & 
      \begin{minipage}[t]{13cm}{\footnotesize
      
      \vspace{\dp0}
      } \end{minipage} \\
      \\ \cdashline{2-3}


      & Status          & Not Executed \\ \hline

      9 & Description &

      \begin{minipage}[t]{13cm}{\footnotesize
      Verify that provenance information is correctly stored in DIASource
catalogs from step 2

      \vspace{\dp0}
      } \end{minipage} \\
      \\ \cdashline{2-3}

      & Expected Result & 

      \begin{minipage}[t]{13cm}{\footnotesize
      
      \vspace{\dp0}
      } \end{minipage} \\
      \\ \cdashline{2-3}

      & \begin{minipage}[t]{2cm}{Actual\\ Result}\end{minipage}   & 
      \begin{minipage}[t]{13cm}{\footnotesize
      
      \vspace{\dp0}
      } \end{minipage} \\
      \\ \cdashline{2-3}


      & Status          & Not Executed \\ \hline

    \end{longtable}


    \paragraph{Test Case LVV-T942 - Provenance on Level 2 catalogs
 }\mbox{}\\

Open  \href{https://jira.lsstcorp.org/secure/Tests.jspa#/testCase/LVV-T942}{\textit{ LVV-T942 } }
test case in Jira.

    Verify that provenance information is stored in level 2 catalog products


    \textbf{ Preconditions}:\\
    

    Execution status: {\bf Not Executed }

    Final comment:\\


    Detailed step results:

    \begin{longtable}{p{1cm}p{2cm}p{13cm}}
    \hline
    {Step} & \multicolumn{2}{c}{Description, Results and Status}\\ \hline
      1 & Description &

      \begin{minipage}[t]{13cm}{\footnotesize
      The DM Stack shall be initialized using the loadLSST script (as
described in DRP-00-00).

      \vspace{\dp0}
      } \end{minipage} \\
      \\ \cdashline{2-3}

      & Expected Result & 

      \begin{minipage}[t]{13cm}{\footnotesize
      
      \vspace{\dp0}
      } \end{minipage} \\
      \\ \cdashline{2-3}

      & \begin{minipage}[t]{2cm}{Actual\\ Result}\end{minipage}   & 
      \begin{minipage}[t]{13cm}{\footnotesize
      
      \vspace{\dp0}
      } \end{minipage} \\
      \\ \cdashline{2-3}


      & Status          & Not Executed \\ \hline

      2 & Description &

      \begin{minipage}[t]{13cm}{\footnotesize
      A ``Data Butler'' will be initialized to access the repository.

      \vspace{\dp0}
      } \end{minipage} \\
      \\ \cdashline{2-3}

      & Expected Result & 

      \begin{minipage}[t]{13cm}{\footnotesize
      
      \vspace{\dp0}
      } \end{minipage} \\
      \\ \cdashline{2-3}

      & \begin{minipage}[t]{2cm}{Actual\\ Result}\end{minipage}   & 
      \begin{minipage}[t]{13cm}{\footnotesize
      
      \vspace{\dp0}
      } \end{minipage} \\
      \\ \cdashline{2-3}


      & Status          & Not Executed \\ \hline

      3 & Description &

      \begin{minipage}[t]{13cm}{\footnotesize
      For each of the expected data products types (listed in Test Items
section §4.3.2) and each of the expected units (PVIs, coadds, etc), the
data product will be retrieved from the Butler and verified to be
non-empty.

      \vspace{\dp0}
      } \end{minipage} \\
      \\ \cdashline{2-3}

      & Expected Result & 

      \begin{minipage}[t]{13cm}{\footnotesize
      
      \vspace{\dp0}
      } \end{minipage} \\
      \\ \cdashline{2-3}

      & \begin{minipage}[t]{2cm}{Actual\\ Result}\end{minipage}   & 
      \begin{minipage}[t]{13cm}{\footnotesize
      
      \vspace{\dp0}
      } \end{minipage} \\
      \\ \cdashline{2-3}


      & Status          & Not Executed \\ \hline

      4 & Description &

      \begin{minipage}[t]{13cm}{\footnotesize
      Query and verify provenance of input images, and software versions that
went into producing stack.

      \vspace{\dp0}
      } \end{minipage} \\
      \\ \cdashline{2-3}

      & Expected Result & 

      \begin{minipage}[t]{13cm}{\footnotesize
      
      \vspace{\dp0}
      } \end{minipage} \\
      \\ \cdashline{2-3}

      & \begin{minipage}[t]{2cm}{Actual\\ Result}\end{minipage}   & 
      \begin{minipage}[t]{13cm}{\footnotesize
      
      \vspace{\dp0}
      } \end{minipage} \\
      \\ \cdashline{2-3}


      & Status          & Not Executed \\ \hline

      5 & Description &

      \begin{minipage}[t]{13cm}{\footnotesize
      Test re-generating 10 different coadds tract+patches based on the
provenance image given

      \vspace{\dp0}
      } \end{minipage} \\
      \\ \cdashline{2-3}

      & Expected Result & 

      \begin{minipage}[t]{13cm}{\footnotesize
      
      \vspace{\dp0}
      } \end{minipage} \\
      \\ \cdashline{2-3}

      & \begin{minipage}[t]{2cm}{Actual\\ Result}\end{minipage}   & 
      \begin{minipage}[t]{13cm}{\footnotesize
      
      \vspace{\dp0}
      } \end{minipage} \\
      \\ \cdashline{2-3}


      & Status          & Not Executed \\ \hline

      6 & Description &

      \begin{minipage}[t]{13cm}{\footnotesize
      Run source detection on coadded images from step 1

      \vspace{\dp0}
      } \end{minipage} \\
      \\ \cdashline{2-3}

      & Expected Result & 

      \begin{minipage}[t]{13cm}{\footnotesize
      Catalog of sources detected on coadds

      \vspace{\dp0}
      } \end{minipage} \\
      \\ \cdashline{2-3}

      & \begin{minipage}[t]{2cm}{Actual\\ Result}\end{minipage}   & 
      \begin{minipage}[t]{13cm}{\footnotesize
      
      \vspace{\dp0}
      } \end{minipage} \\
      \\ \cdashline{2-3}


      & Status          & Not Executed \\ \hline

      7 & Description &

      \begin{minipage}[t]{13cm}{\footnotesize
      Verify that correct provenance information is stored in catalogs from
step 2

      \vspace{\dp0}
      } \end{minipage} \\
      \\ \cdashline{2-3}

      & Expected Result & 

      \begin{minipage}[t]{13cm}{\footnotesize
      
      \vspace{\dp0}
      } \end{minipage} \\
      \\ \cdashline{2-3}

      & \begin{minipage}[t]{2cm}{Actual\\ Result}\end{minipage}   & 
      \begin{minipage}[t]{13cm}{\footnotesize
      
      \vspace{\dp0}
      } \end{minipage} \\
      \\ \cdashline{2-3}


      & Status          & Not Executed \\ \hline

    \end{longtable}


    \paragraph{Test Case LVV-T941 - Level 1 reproducibility (different computer hardware)
 }\mbox{}\\

Open  \href{https://jira.lsstcorp.org/secure/Tests.jspa#/testCase/LVV-T941}{\textit{ LVV-T941 } }
test case in Jira.

    

    \textbf{ Preconditions}:\\
    LVV-T940 has been run


    Execution status: {\bf Not Executed }

    Final comment:\\


    Detailed step results:

    \begin{longtable}{p{1cm}p{2cm}p{13cm}}
    \hline
    {Step} & \multicolumn{2}{c}{Description, Results and Status}\\ \hline
      1 & Description &

      \begin{minipage}[t]{13cm}{\footnotesize
      Run Level 1 analysis on data from step 1 of LVV-T940, using coadded
images from first Level 2 run as templates

      \vspace{\dp0}
      } \end{minipage} \\
      \\ \cdashline{2-3}

      & Expected Result & 

      \begin{minipage}[t]{13cm}{\footnotesize
      Catalog of DIASources

      \vspace{\dp0}
      } \end{minipage} \\
      \\ \cdashline{2-3}

      & \begin{minipage}[t]{2cm}{Actual\\ Result}\end{minipage}   & 
      \begin{minipage}[t]{13cm}{\footnotesize
      
      \vspace{\dp0}
      } \end{minipage} \\
      \\ \cdashline{2-3}


      & Status          & Not Executed \\ \hline

      2 & Description &

      \begin{minipage}[t]{13cm}{\footnotesize
      Re-run Level 1 analysis on data from step 1 of LVV-T940 using coadded
images from second Level 2 run and using a different computer system
than step 1 of this requirement

      \vspace{\dp0}
      } \end{minipage} \\
      \\ \cdashline{2-3}

      & Expected Result & 

      \begin{minipage}[t]{13cm}{\footnotesize
      Catalog of DIASources

      \vspace{\dp0}
      } \end{minipage} \\
      \\ \cdashline{2-3}

      & \begin{minipage}[t]{2cm}{Actual\\ Result}\end{minipage}   & 
      \begin{minipage}[t]{13cm}{\footnotesize
      
      \vspace{\dp0}
      } \end{minipage} \\
      \\ \cdashline{2-3}


      & Status          & Not Executed \\ \hline

      3 & Description &

      \begin{minipage}[t]{13cm}{\footnotesize
      Verify that DIASource catalogs from steps 1 and 2 agree to within some
small tolerance

      \vspace{\dp0}
      } \end{minipage} \\
      \\ \cdashline{2-3}

      & Expected Result & 

      \begin{minipage}[t]{13cm}{\footnotesize
      DIASource catalogs should agree to within some small tolerance

      \vspace{\dp0}
      } \end{minipage} \\
      \\ \cdashline{2-3}

      & \begin{minipage}[t]{2cm}{Actual\\ Result}\end{minipage}   & 
      \begin{minipage}[t]{13cm}{\footnotesize
      
      \vspace{\dp0}
      } \end{minipage} \\
      \\ \cdashline{2-3}


      & Status          & Not Executed \\ \hline

    \end{longtable}


    \paragraph{Test Case LVV-T939 - Level 1 reproducibility (same computer hardware)
 }\mbox{}\\

Open  \href{https://jira.lsstcorp.org/secure/Tests.jspa#/testCase/LVV-T939}{\textit{ LVV-T939 } }
test case in Jira.

    

    \textbf{ Preconditions}:\\
    LVV-T938 has been run


    Execution status: {\bf Not Executed }

    Final comment:\\


    Detailed step results:

    \begin{longtable}{p{1cm}p{2cm}p{13cm}}
    \hline
    {Step} & \multicolumn{2}{c}{Description, Results and Status}\\ \hline
      1 & Description &

      \begin{minipage}[t]{13cm}{\footnotesize
      Run Level 1 analysis on images taken in LVV-T938 using coadds from the
first run as templates.

      \vspace{\dp0}
      } \end{minipage} \\
      \\ \cdashline{2-3}

      & Expected Result & 

      \begin{minipage}[t]{13cm}{\footnotesize
      Set of DIASources

      \vspace{\dp0}
      } \end{minipage} \\
      \\ \cdashline{2-3}

      & \begin{minipage}[t]{2cm}{Actual\\ Result}\end{minipage}   & 
      \begin{minipage}[t]{13cm}{\footnotesize
      
      \vspace{\dp0}
      } \end{minipage} \\
      \\ \cdashline{2-3}


      & Status          & Not Executed \\ \hline

      2 & Description &

      \begin{minipage}[t]{13cm}{\footnotesize
      Run Level 1 analysis on images taken in LVV-T938 using coadds from the
first run as templates. ~Run on the same system as step 1.

      \vspace{\dp0}
      } \end{minipage} \\
      \\ \cdashline{2-3}

      & Expected Result & 

      \begin{minipage}[t]{13cm}{\footnotesize
      Set of DIASources

      \vspace{\dp0}
      } \end{minipage} \\
      \\ \cdashline{2-3}

      & \begin{minipage}[t]{2cm}{Actual\\ Result}\end{minipage}   & 
      \begin{minipage}[t]{13cm}{\footnotesize
      
      \vspace{\dp0}
      } \end{minipage} \\
      \\ \cdashline{2-3}


      & Status          & Not Executed \\ \hline

      3 & Description &

      \begin{minipage}[t]{13cm}{\footnotesize
      Verify that the sets of DIASources produced in steps 1 and 2 are
identical, since they were run on the same system.

      \vspace{\dp0}
      } \end{minipage} \\
      \\ \cdashline{2-3}

      & Expected Result & 

      \begin{minipage}[t]{13cm}{\footnotesize
      Sets of DIASources should be identical

      \vspace{\dp0}
      } \end{minipage} \\
      \\ \cdashline{2-3}

      & \begin{minipage}[t]{2cm}{Actual\\ Result}\end{minipage}   & 
      \begin{minipage}[t]{13cm}{\footnotesize
      
      \vspace{\dp0}
      } \end{minipage} \\
      \\ \cdashline{2-3}


      & Status          & Not Executed \\ \hline

    \end{longtable}


    \paragraph{Test Case LVV-T940 - Level 2 reproducibility (different computer hardware)
 }\mbox{}\\

Open  \href{https://jira.lsstcorp.org/secure/Tests.jspa#/testCase/LVV-T940}{\textit{ LVV-T940 } }
test case in Jira.

    

    \textbf{ Preconditions}:\\
    

    Execution status: {\bf Not Executed }

    Final comment:\\


    Detailed step results:

    \begin{longtable}{p{1cm}p{2cm}p{13cm}}
    \hline
    {Step} & \multicolumn{2}{c}{Description, Results and Status}\\ \hline
      1 & Description &

      \begin{minipage}[t]{13cm}{\footnotesize
      Take precursor data and calibration products from mini survey

      \vspace{\dp0}
      } \end{minipage} \\
      \\ \cdashline{2-3}

      & Expected Result & 

      \begin{minipage}[t]{13cm}{\footnotesize
      Calibration products\\
Images

      \vspace{\dp0}
      } \end{minipage} \\
      \\ \cdashline{2-3}

      & \begin{minipage}[t]{2cm}{Actual\\ Result}\end{minipage}   & 
      \begin{minipage}[t]{13cm}{\footnotesize
      
      \vspace{\dp0}
      } \end{minipage} \\
      \\ \cdashline{2-3}


      & Status          & Not Executed \\ \hline

      2 & Description &

      \begin{minipage}[t]{13cm}{\footnotesize
      Run Level 2 analysis on data from step 1

      \vspace{\dp0}
      } \end{minipage} \\
      \\ \cdashline{2-3}

      & Expected Result & 

      \begin{minipage}[t]{13cm}{\footnotesize
      Coadded images\\
Catalogs of detected sources

      \vspace{\dp0}
      } \end{minipage} \\
      \\ \cdashline{2-3}

      & \begin{minipage}[t]{2cm}{Actual\\ Result}\end{minipage}   & 
      \begin{minipage}[t]{13cm}{\footnotesize
      
      \vspace{\dp0}
      } \end{minipage} \\
      \\ \cdashline{2-3}


      & Status          & Not Executed \\ \hline

      3 & Description &

      \begin{minipage}[t]{13cm}{\footnotesize
      Re-run Level 2 analysis on data from step 1 using a different hardware
system

      \vspace{\dp0}
      } \end{minipage} \\
      \\ \cdashline{2-3}

      & Expected Result & 

      \begin{minipage}[t]{13cm}{\footnotesize
      Coadded images\\
Catalogs of detected sources

      \vspace{\dp0}
      } \end{minipage} \\
      \\ \cdashline{2-3}

      & \begin{minipage}[t]{2cm}{Actual\\ Result}\end{minipage}   & 
      \begin{minipage}[t]{13cm}{\footnotesize
      
      \vspace{\dp0}
      } \end{minipage} \\
      \\ \cdashline{2-3}


      & Status          & Not Executed \\ \hline

      4 & Description &

      \begin{minipage}[t]{13cm}{\footnotesize
      Verify that the catalogs from steps 2 and 3 agree to within some small
tolerance

      \vspace{\dp0}
      } \end{minipage} \\
      \\ \cdashline{2-3}

      & Expected Result & 

      \begin{minipage}[t]{13cm}{\footnotesize
      Catalogs will agree to within some tolerance

      \vspace{\dp0}
      } \end{minipage} \\
      \\ \cdashline{2-3}

      & \begin{minipage}[t]{2cm}{Actual\\ Result}\end{minipage}   & 
      \begin{minipage}[t]{13cm}{\footnotesize
      
      \vspace{\dp0}
      } \end{minipage} \\
      \\ \cdashline{2-3}


      & Status          & Not Executed \\ \hline

    \end{longtable}


    \paragraph{Test Case LVV-T938 - Level 2 reproducibility (same computer hardware)
 }\mbox{}\\

Open  \href{https://jira.lsstcorp.org/secure/Tests.jspa#/testCase/LVV-T938}{\textit{ LVV-T938 } }
test case in Jira.

    

    \textbf{ Preconditions}:\\
    

    Execution status: {\bf Not Executed }

    Final comment:\\


    Detailed step results:

    \begin{longtable}{p{1cm}p{2cm}p{13cm}}
    \hline
    {Step} & \multicolumn{2}{c}{Description, Results and Status}\\ \hline
      1 & Description &

      \begin{minipage}[t]{13cm}{\footnotesize
      Take precursor mini-survey data

      \vspace{\dp0}
      } \end{minipage} \\
      \\ \cdashline{2-3}

      & Expected Result & 

      \begin{minipage}[t]{13cm}{\footnotesize
      Images and calibration products

      \vspace{\dp0}
      } \end{minipage} \\
      \\ \cdashline{2-3}

      & \begin{minipage}[t]{2cm}{Actual\\ Result}\end{minipage}   & 
      \begin{minipage}[t]{13cm}{\footnotesize
      
      \vspace{\dp0}
      } \end{minipage} \\
      \\ \cdashline{2-3}


      & Status          & Not Executed \\ \hline

      2 & Description &

      \begin{minipage}[t]{13cm}{\footnotesize
      Run Level 2 processing on precursor data from step 1

      \vspace{\dp0}
      } \end{minipage} \\
      \\ \cdashline{2-3}

      & Expected Result & 

      \begin{minipage}[t]{13cm}{\footnotesize
      Coadded images\\
Catalogs detected on coadded images

      \vspace{\dp0}
      } \end{minipage} \\
      \\ \cdashline{2-3}

      & \begin{minipage}[t]{2cm}{Actual\\ Result}\end{minipage}   & 
      \begin{minipage}[t]{13cm}{\footnotesize
      
      \vspace{\dp0}
      } \end{minipage} \\
      \\ \cdashline{2-3}


      & Status          & Not Executed \\ \hline

      3 & Description &

      \begin{minipage}[t]{13cm}{\footnotesize
      Re-run Level 2 processing on data from step 1, using the same system as
in step 2

      \vspace{\dp0}
      } \end{minipage} \\
      \\ \cdashline{2-3}

      & Expected Result & 

      \begin{minipage}[t]{13cm}{\footnotesize
      Coadded images\\
Catalogs detected on coadded images

      \vspace{\dp0}
      } \end{minipage} \\
      \\ \cdashline{2-3}

      & \begin{minipage}[t]{2cm}{Actual\\ Result}\end{minipage}   & 
      \begin{minipage}[t]{13cm}{\footnotesize
      
      \vspace{\dp0}
      } \end{minipage} \\
      \\ \cdashline{2-3}


      & Status          & Not Executed \\ \hline

      4 & Description &

      \begin{minipage}[t]{13cm}{\footnotesize
      Verify that catalogs from step 2 and step 3 identify all of the same
sources and give identical measurements (since the two analyses were run
on the same system).

      \vspace{\dp0}
      } \end{minipage} \\
      \\ \cdashline{2-3}

      & Expected Result & 

      \begin{minipage}[t]{13cm}{\footnotesize
      Catalogs are identical

      \vspace{\dp0}
      } \end{minipage} \\
      \\ \cdashline{2-3}

      & \begin{minipage}[t]{2cm}{Actual\\ Result}\end{minipage}   & 
      \begin{minipage}[t]{13cm}{\footnotesize
      
      \vspace{\dp0}
      } \end{minipage} \\
      \\ \cdashline{2-3}


      & Status          & Not Executed \\ \hline

    \end{longtable}


    \paragraph{Test Case LVV-T597 - Depth variation over field of view
 }\mbox{}\\

Open  \href{https://jira.lsstcorp.org/secure/Tests.jspa#/testCase/LVV-T597}{\textit{ LVV-T597 } }
test case in Jira.

    

    \textbf{ Preconditions}:\\
    

    Execution status: {\bf Not Executed }

    Final comment:\\


    Detailed step results:

    \begin{longtable}{p{1cm}p{2cm}p{13cm}}
    \hline
    {Step} & \multicolumn{2}{c}{Description, Results and Status}\\ \hline
      1 & Description &

      \begin{minipage}[t]{13cm}{\footnotesize
      After conclusion of a mini-survey, select all pointings that meet the
depth requirement specified in LSR-REQ-0090 (LVV-263)

      \vspace{\dp0}
      } \end{minipage} \\
      \\ \cdashline{2-3}

      & Expected Result & 

      \begin{minipage}[t]{13cm}{\footnotesize
      Set of images that meet the fiducial depth requirement.

      \vspace{\dp0}
      } \end{minipage} \\
      \\ \cdashline{2-3}

      & \begin{minipage}[t]{2cm}{Actual\\ Result}\end{minipage}   & 
      \begin{minipage}[t]{13cm}{\footnotesize
      
      \vspace{\dp0}
      } \end{minipage} \\
      \\ \cdashline{2-3}


      & Status          & Not Executed \\ \hline

      2 & Description &

      \begin{minipage}[t]{13cm}{\footnotesize
      Perform single image processing on the images from step 1 (if not done
already)

      \vspace{\dp0}
      } \end{minipage} \\
      \\ \cdashline{2-3}

      & Expected Result & 

      \begin{minipage}[t]{13cm}{\footnotesize
      Catalogs of measured sources from the images in step 1

      \vspace{\dp0}
      } \end{minipage} \\
      \\ \cdashline{2-3}

      & \begin{minipage}[t]{2cm}{Actual\\ Result}\end{minipage}   & 
      \begin{minipage}[t]{13cm}{\footnotesize
      
      \vspace{\dp0}
      } \end{minipage} \\
      \\ \cdashline{2-3}


      & Status          & Not Executed \\ \hline

      3 & Description &

      \begin{minipage}[t]{13cm}{\footnotesize
      Subdivide each image into small regions (\textasciitilde{}1 CCD should
be enough, since 1/189 = 5*10\^{}-3). ~In each region, examine the SNR
distribution of measured sources to determine the 5-sigma limiting
magnitude for that region of the focal plane.

      \vspace{\dp0}
      } \end{minipage} \\
      \\ \cdashline{2-3}

      & Expected Result & 

      \begin{minipage}[t]{13cm}{\footnotesize
      Distribution of depths for subregions of the focal plane

      \vspace{\dp0}
      } \end{minipage} \\
      \\ \cdashline{2-3}

      & \begin{minipage}[t]{2cm}{Actual\\ Result}\end{minipage}   & 
      \begin{minipage}[t]{13cm}{\footnotesize
      
      \vspace{\dp0}
      } \end{minipage} \\
      \\ \cdashline{2-3}


      & Status          & Not Executed \\ \hline

      4 & Description &

      \begin{minipage}[t]{13cm}{\footnotesize
      Verify that, for each exposure, no more than 15\% of the focal plane
area has a 5-sigma limiting magnitude 0.2 AB magnitude brighter than the
median 5-sigma limiting magnitude for the entire exposure.

      \vspace{\dp0}
      } \end{minipage} \\
      \\ \cdashline{2-3}

      & Expected Result & 

      \begin{minipage}[t]{13cm}{\footnotesize
      
      \vspace{\dp0}
      } \end{minipage} \\
      \\ \cdashline{2-3}

      & \begin{minipage}[t]{2cm}{Actual\\ Result}\end{minipage}   & 
      \begin{minipage}[t]{13cm}{\footnotesize
      
      \vspace{\dp0}
      } \end{minipage} \\
      \\ \cdashline{2-3}


      & Status          & Not Executed \\ \hline

    \end{longtable}


    \paragraph{Test Case LVV-T596 - Depth: r-band
 }\mbox{}\\

Open  \href{https://jira.lsstcorp.org/secure/Tests.jspa#/testCase/LVV-T596}{\textit{ LVV-T596 } }
test case in Jira.

    

    \textbf{ Preconditions}:\\
    

    Execution status: {\bf Not Executed }

    Final comment:\\


    Detailed step results:

    \begin{longtable}{p{1cm}p{2cm}p{13cm}}
    \hline
    {Step} & \multicolumn{2}{c}{Description, Results and Status}\\ \hline
      1 & Description &

      \begin{minipage}[t]{13cm}{\footnotesize
      Upon completion of mini-survey, select all exposures taken at or
near\\[2\baselineskip]band = r\\
airmass = 1\\
seeing = 0.7 arcsecond\\
sky brightness = 21 magnitudes per square arcsecond

      \vspace{\dp0}
      } \end{minipage} \\
      \\ \cdashline{2-3}

      & Expected Result & 

      \begin{minipage}[t]{13cm}{\footnotesize
      Set of exposures taken at reference conditions

      \vspace{\dp0}
      } \end{minipage} \\
      \\ \cdashline{2-3}

      & \begin{minipage}[t]{2cm}{Actual\\ Result}\end{minipage}   & 
      \begin{minipage}[t]{13cm}{\footnotesize
      
      \vspace{\dp0}
      } \end{minipage} \\
      \\ \cdashline{2-3}


      & Status          & Not Executed \\ \hline

      2 & Description &

      \begin{minipage}[t]{13cm}{\footnotesize
      Run single visit processing on images from step 1

      \vspace{\dp0}
      } \end{minipage} \\
      \\ \cdashline{2-3}

      & Expected Result & 

      \begin{minipage}[t]{13cm}{\footnotesize
      catalog of measured sources

      \vspace{\dp0}
      } \end{minipage} \\
      \\ \cdashline{2-3}

      & \begin{minipage}[t]{2cm}{Actual\\ Result}\end{minipage}   & 
      \begin{minipage}[t]{13cm}{\footnotesize
      
      \vspace{\dp0}
      } \end{minipage} \\
      \\ \cdashline{2-3}


      & Status          & Not Executed \\ \hline

      3 & Description &

      \begin{minipage}[t]{13cm}{\footnotesize
      For each visit, find the 5-sigma limiting magnitude by examining the
distribution of sources detected at SNR=5

      \vspace{\dp0}
      } \end{minipage} \\
      \\ \cdashline{2-3}

      & Expected Result & 

      \begin{minipage}[t]{13cm}{\footnotesize
      distribution of 5-sigma limiting magnitudes

      \vspace{\dp0}
      } \end{minipage} \\
      \\ \cdashline{2-3}

      & \begin{minipage}[t]{2cm}{Actual\\ Result}\end{minipage}   & 
      \begin{minipage}[t]{13cm}{\footnotesize
      
      \vspace{\dp0}
      } \end{minipage} \\
      \\ \cdashline{2-3}


      & Status          & Not Executed \\ \hline

      4 & Description &

      \begin{minipage}[t]{13cm}{\footnotesize
      Verify that the median of the distribution of 5-sigma limiting
magnitudes is no brighter than 24.7 AB magnitudes

      \vspace{\dp0}
      } \end{minipage} \\
      \\ \cdashline{2-3}

      & Expected Result & 

      \begin{minipage}[t]{13cm}{\footnotesize
      
      \vspace{\dp0}
      } \end{minipage} \\
      \\ \cdashline{2-3}

      & \begin{minipage}[t]{2cm}{Actual\\ Result}\end{minipage}   & 
      \begin{minipage}[t]{13cm}{\footnotesize
      
      \vspace{\dp0}
      } \end{minipage} \\
      \\ \cdashline{2-3}


      & Status          & Not Executed \\ \hline

      5 & Description &

      \begin{minipage}[t]{13cm}{\footnotesize
      Verify that no more than 10\% of the images have a 5-sigma limiting
magnitudes brighter than 24.4 AB magnitudes

      \vspace{\dp0}
      } \end{minipage} \\
      \\ \cdashline{2-3}

      & Expected Result & 

      \begin{minipage}[t]{13cm}{\footnotesize
      
      \vspace{\dp0}
      } \end{minipage} \\
      \\ \cdashline{2-3}

      & \begin{minipage}[t]{2cm}{Actual\\ Result}\end{minipage}   & 
      \begin{minipage}[t]{13cm}{\footnotesize
      
      \vspace{\dp0}
      } \end{minipage} \\
      \\ \cdashline{2-3}


      & Status          & Not Executed \\ \hline

    \end{longtable}


    \paragraph{Test Case LVV-T549 - Zeropoint consistency
 }\mbox{}\\

Open  \href{https://jira.lsstcorp.org/secure/Tests.jspa#/testCase/LVV-T549}{\textit{ LVV-T549 } }
test case in Jira.

    Verify that the CCD AP pipeline zero point is consistent within
\textbf{photoZeroPointOffset (50)} millimags of the level 2 pipeline


    \textbf{ Preconditions}:\\
    

    Execution status: {\bf Not Executed }

    Final comment:\\


    Detailed step results:

    \begin{longtable}{p{1cm}p{2cm}p{13cm}}
    \hline
    {Step} & \multicolumn{2}{c}{Description, Results and Status}\\ \hline
      1 & Description &

      \begin{minipage}[t]{13cm}{\footnotesize
      Image a patch of sky to a specified depth (1yr? 3yr? full LSST?).

      \vspace{\dp0}
      } \end{minipage} \\
      \\ \cdashline{2-3}

      & Expected Result & 

      \begin{minipage}[t]{13cm}{\footnotesize
      Images of the same patch of sky

      \vspace{\dp0}
      } \end{minipage} \\
      \\ \cdashline{2-3}

      & \begin{minipage}[t]{2cm}{Actual\\ Result}\end{minipage}   & 
      \begin{minipage}[t]{13cm}{\footnotesize
      
      \vspace{\dp0}
      } \end{minipage} \\
      \\ \cdashline{2-3}


      & Status          & Not Executed \\ \hline

      2 & Description &

      \begin{minipage}[t]{13cm}{\footnotesize
      Perform Level 2 processing on images from step 1. ~Store final zeropoint
determination.

      \vspace{\dp0}
      } \end{minipage} \\
      \\ \cdashline{2-3}

      & Expected Result & 

      \begin{minipage}[t]{13cm}{\footnotesize
      Determination of photometric zeropoint from final photometric
calibration algorithm

      \vspace{\dp0}
      } \end{minipage} \\
      \\ \cdashline{2-3}

      & \begin{minipage}[t]{2cm}{Actual\\ Result}\end{minipage}   & 
      \begin{minipage}[t]{13cm}{\footnotesize
      
      \vspace{\dp0}
      } \end{minipage} \\
      \\ \cdashline{2-3}


      & Status          & Not Executed \\ \hline

      3 & Description &

      \begin{minipage}[t]{13cm}{\footnotesize
      Calibrate images from step 1 using Level 1 pipeline

      \vspace{\dp0}
      } \end{minipage} \\
      \\ \cdashline{2-3}

      & Expected Result & 

      \begin{minipage}[t]{13cm}{\footnotesize
      Calibrated images based on raw images from step 1

      \vspace{\dp0}
      } \end{minipage} \\
      \\ \cdashline{2-3}

      & \begin{minipage}[t]{2cm}{Actual\\ Result}\end{minipage}   & 
      \begin{minipage}[t]{13cm}{\footnotesize
      
      \vspace{\dp0}
      } \end{minipage} \\
      \\ \cdashline{2-3}


      & Status          & Not Executed \\ \hline

      4 & Description &

      \begin{minipage}[t]{13cm}{\footnotesize
      For each calibrated image produced in step 3, verify that the
photometric zeropoint agrees with the final zeropoint determine din step
2 to the specified tolerance.

      \vspace{\dp0}
      } \end{minipage} \\
      \\ \cdashline{2-3}

      & Expected Result & 

      \begin{minipage}[t]{13cm}{\footnotesize
      
      \vspace{\dp0}
      } \end{minipage} \\
      \\ \cdashline{2-3}

      & \begin{minipage}[t]{2cm}{Actual\\ Result}\end{minipage}   & 
      \begin{minipage}[t]{13cm}{\footnotesize
      
      \vspace{\dp0}
      } \end{minipage} \\
      \\ \cdashline{2-3}


      & Status          & Not Executed \\ \hline

    \end{longtable}


    \paragraph{Test Case LVV-T547 - Photometric errors -- level 1 processing -- on-sky data
 }\mbox{}\\

Open  \href{https://jira.lsstcorp.org/secure/Tests.jspa#/testCase/LVV-T547}{\textit{ LVV-T547 } }
test case in Jira.

    Test DM contribution to photometric erros with LSST images


    \textbf{ Preconditions}:\\
    

    Execution status: {\bf Not Executed }

    Final comment:\\


    Detailed step results:

    \begin{longtable}{p{1cm}p{2cm}p{13cm}}
    \hline
    {Step} & \multicolumn{2}{c}{Description, Results and Status}\\ \hline
      1 & Description &

      \begin{minipage}[t]{13cm}{\footnotesize
      Image a patch of sky at various observing conditions (airmass, seeing
,etc.).

      \vspace{\dp0}
      } \end{minipage} \\
      \\ \cdashline{2-3}

      & Expected Result & 

      \begin{minipage}[t]{13cm}{\footnotesize
      Images of sky at various observing conditions

      \vspace{\dp0}
      } \end{minipage} \\
      \\ \cdashline{2-3}

      & \begin{minipage}[t]{2cm}{Actual\\ Result}\end{minipage}   & 
      \begin{minipage}[t]{13cm}{\footnotesize
      
      \vspace{\dp0}
      } \end{minipage} \\
      \\ \cdashline{2-3}


      & Status          & Not Executed \\ \hline

      2 & Description &

      \begin{minipage}[t]{13cm}{\footnotesize
      Perform Level 2 processing on images from step 1. ~Save catalog of
fluxes as well as standard deviation of flux measurements from
individual images.

      \vspace{\dp0}
      } \end{minipage} \\
      \\ \cdashline{2-3}

      & Expected Result & 

      \begin{minipage}[t]{13cm}{\footnotesize
      Catalog of all sources in images from step 1\\
Characterization of width of flux measurements for each source

      \vspace{\dp0}
      } \end{minipage} \\
      \\ \cdashline{2-3}

      & \begin{minipage}[t]{2cm}{Actual\\ Result}\end{minipage}   & 
      \begin{minipage}[t]{13cm}{\footnotesize
      
      \vspace{\dp0}
      } \end{minipage} \\
      \\ \cdashline{2-3}


      & Status          & Not Executed \\ \hline

      3 & Description &

      \begin{minipage}[t]{13cm}{\footnotesize
      Perform Level 1 processing on images from step 1. ~Keep difference
images for ``forced DIA photometry'' in later steps.

      \vspace{\dp0}
      } \end{minipage} \\
      \\ \cdashline{2-3}

      & Expected Result & 

      \begin{minipage}[t]{13cm}{\footnotesize
      Catalog of variable sources in images from step 1\\
Difference images corresponding to images in step 1

      \vspace{\dp0}
      } \end{minipage} \\
      \\ \cdashline{2-3}

      & \begin{minipage}[t]{2cm}{Actual\\ Result}\end{minipage}   & 
      \begin{minipage}[t]{13cm}{\footnotesize
      
      \vspace{\dp0}
      } \end{minipage} \\
      \\ \cdashline{2-3}


      & Status          & Not Executed \\ \hline

      4 & Description &

      \begin{minipage}[t]{13cm}{\footnotesize
      Identify sources in step 2 that did not appear as DIASources. ~These
will be taken as totally static objects.

      \vspace{\dp0}
      } \end{minipage} \\
      \\ \cdashline{2-3}

      & Expected Result & 

      \begin{minipage}[t]{13cm}{\footnotesize
      Catalog of static sources

      \vspace{\dp0}
      } \end{minipage} \\
      \\ \cdashline{2-3}

      & \begin{minipage}[t]{2cm}{Actual\\ Result}\end{minipage}   & 
      \begin{minipage}[t]{13cm}{\footnotesize
      
      \vspace{\dp0}
      } \end{minipage} \\
      \\ \cdashline{2-3}


      & Status          & Not Executed \\ \hline

      5 & Description &

      \begin{minipage}[t]{13cm}{\footnotesize
      Perform forced photometry on difference images from step 3 at locations
of static objects identified in step 4.~

      \vspace{\dp0}
      } \end{minipage} \\
      \\ \cdashline{2-3}

      & Expected Result & 

      \begin{minipage}[t]{13cm}{\footnotesize
      Catalog of difference image photometry for static sources

      \vspace{\dp0}
      } \end{minipage} \\
      \\ \cdashline{2-3}

      & \begin{minipage}[t]{2cm}{Actual\\ Result}\end{minipage}   & 
      \begin{minipage}[t]{13cm}{\footnotesize
      
      \vspace{\dp0}
      } \end{minipage} \\
      \\ \cdashline{2-3}


      & Status          & Not Executed \\ \hline

      6 & Description &

      \begin{minipage}[t]{13cm}{\footnotesize
      Construct model of photometric uncertainty based only observing
conditions of images in step 1.

      \vspace{\dp0}
      } \end{minipage} \\
      \\ \cdashline{2-3}

      & Expected Result & 

      \begin{minipage}[t]{13cm}{\footnotesize
      Model of photometric uncertainty expected solely due to observing
conditions

      \vspace{\dp0}
      } \end{minipage} \\
      \\ \cdashline{2-3}

      & \begin{minipage}[t]{2cm}{Actual\\ Result}\end{minipage}   & 
      \begin{minipage}[t]{13cm}{\footnotesize
      
      \vspace{\dp0}
      } \end{minipage} \\
      \\ \cdashline{2-3}


      & Status          & Not Executed \\ \hline

      7 & Description &

      \begin{minipage}[t]{13cm}{\footnotesize
      Compare distribution of force difference image photometry measurements
from step 5 with intrinsic width of flux measurements in step 2 and
model of uncertainty due to observing conditions in step 6. ~Verify that
RMS residual is within specified tolerance.

      \vspace{\dp0}
      } \end{minipage} \\
      \\ \cdashline{2-3}

      & Expected Result & 

      \begin{minipage}[t]{13cm}{\footnotesize
      
      \vspace{\dp0}
      } \end{minipage} \\
      \\ \cdashline{2-3}

      & \begin{minipage}[t]{2cm}{Actual\\ Result}\end{minipage}   & 
      \begin{minipage}[t]{13cm}{\footnotesize
      
      \vspace{\dp0}
      } \end{minipage} \\
      \\ \cdashline{2-3}


      & Status          & Not Executed \\ \hline

    \end{longtable}


    \paragraph{Test Case LVV-T544 - Astrometric error -- level 1 processing -- on-sky data
 }\mbox{}\\

Open  \href{https://jira.lsstcorp.org/secure/Tests.jspa#/testCase/LVV-T544}{\textit{ LVV-T544 } }
test case in Jira.

    Measure the astrometric performance requirements using actual data


    \textbf{ Preconditions}:\\
    

    Execution status: {\bf Not Executed }

    Final comment:\\


    Detailed step results:

    \begin{longtable}{p{1cm}p{2cm}p{13cm}}
    \hline
    {Step} & \multicolumn{2}{c}{Description, Results and Status}\\ \hline
      1 & Description &

      \begin{minipage}[t]{13cm}{\footnotesize
      Perform full-depth mini-survey on a patch of sky.

      \vspace{\dp0}
      } \end{minipage} \\
      \\ \cdashline{2-3}

      & Expected Result & 

      \begin{minipage}[t]{13cm}{\footnotesize
      Images going down to full LSST depth

      \vspace{\dp0}
      } \end{minipage} \\
      \\ \cdashline{2-3}

      & \begin{minipage}[t]{2cm}{Actual\\ Result}\end{minipage}   & 
      \begin{minipage}[t]{13cm}{\footnotesize
      
      \vspace{\dp0}
      } \end{minipage} \\
      \\ \cdashline{2-3}


      & Status          & Not Executed \\ \hline

      2 & Description &

      \begin{minipage}[t]{13cm}{\footnotesize
      Perform Level 2 processing to get ground truth position of sources.

      \vspace{\dp0}
      } \end{minipage} \\
      \\ \cdashline{2-3}

      & Expected Result & 

      \begin{minipage}[t]{13cm}{\footnotesize
      Catalog of sources to be used as truth for analysis

      \vspace{\dp0}
      } \end{minipage} \\
      \\ \cdashline{2-3}

      & \begin{minipage}[t]{2cm}{Actual\\ Result}\end{minipage}   & 
      \begin{minipage}[t]{13cm}{\footnotesize
      
      \vspace{\dp0}
      } \end{minipage} \\
      \\ \cdashline{2-3}


      & Status          & Not Executed \\ \hline

      3 & Description &

      \begin{minipage}[t]{13cm}{\footnotesize
      Perform Level 1 analysis on images from step 1.

      \vspace{\dp0}
      } \end{minipage} \\
      \\ \cdashline{2-3}

      & Expected Result & 

      \begin{minipage}[t]{13cm}{\footnotesize
      Catalog of DIASources

      \vspace{\dp0}
      } \end{minipage} \\
      \\ \cdashline{2-3}

      & \begin{minipage}[t]{2cm}{Actual\\ Result}\end{minipage}   & 
      \begin{minipage}[t]{13cm}{\footnotesize
      
      \vspace{\dp0}
      } \end{minipage} \\
      \\ \cdashline{2-3}


      & Status          & Not Executed \\ \hline

      4 & Description &

      \begin{minipage}[t]{13cm}{\footnotesize
      Model astrometric errors as a function of observing conditions (airmass,
seeing, etc.) in images from step 1.

      \vspace{\dp0}
      } \end{minipage} \\
      \\ \cdashline{2-3}

      & Expected Result & 

      \begin{minipage}[t]{13cm}{\footnotesize
      Model of astrometric errors due only to observing conditions

      \vspace{\dp0}
      } \end{minipage} \\
      \\ \cdashline{2-3}

      & \begin{minipage}[t]{2cm}{Actual\\ Result}\end{minipage}   & 
      \begin{minipage}[t]{13cm}{\footnotesize
      
      \vspace{\dp0}
      } \end{minipage} \\
      \\ \cdashline{2-3}


      & Status          & Not Executed \\ \hline

      5 & Description &

      \begin{minipage}[t]{13cm}{\footnotesize
      Compare measured positions of DIASources to ground truth catalog from
step 2 to get distribution of astrometric errors.

      \vspace{\dp0}
      } \end{minipage} \\
      \\ \cdashline{2-3}

      & Expected Result & 

      \begin{minipage}[t]{13cm}{\footnotesize
      Distribution of measured astrometric errors

      \vspace{\dp0}
      } \end{minipage} \\
      \\ \cdashline{2-3}

      & \begin{minipage}[t]{2cm}{Actual\\ Result}\end{minipage}   & 
      \begin{minipage}[t]{13cm}{\footnotesize
      
      \vspace{\dp0}
      } \end{minipage} \\
      \\ \cdashline{2-3}


      & Status          & Not Executed \\ \hline

      6 & Description &

      \begin{minipage}[t]{13cm}{\footnotesize
      Compute the RMS residual between the measured astrometric errors in step
5 and the model of errors due just to observing conditions in step 4.
~Verify that residual is within specified tolerance.

      \vspace{\dp0}
      } \end{minipage} \\
      \\ \cdashline{2-3}

      & Expected Result & 

      \begin{minipage}[t]{13cm}{\footnotesize
      
      \vspace{\dp0}
      } \end{minipage} \\
      \\ \cdashline{2-3}

      & \begin{minipage}[t]{2cm}{Actual\\ Result}\end{minipage}   & 
      \begin{minipage}[t]{13cm}{\footnotesize
      
      \vspace{\dp0}
      } \end{minipage} \\
      \\ \cdashline{2-3}


      & Status          & Not Executed \\ \hline

    \end{longtable}


    \paragraph{Test Case LVV-T532 - MOPS completeness threshold
 }\mbox{}\\

Open  \href{https://jira.lsstcorp.org/secure/Tests.jspa#/testCase/LVV-T532}{\textit{ LVV-T532 } }
test case in Jira.

    Verify that spuriousness metric has a threshold value at which
completeness and purity requirements for MOPS are met


    \textbf{ Preconditions}:\\
    

    Execution status: {\bf Not Executed }

    Final comment:\\


    Detailed step results:

    \begin{longtable}{p{1cm}p{2cm}p{13cm}}
    \hline
    {Step} & \multicolumn{2}{c}{Description, Results and Status}\\ \hline
      1 & Description &

      \begin{minipage}[t]{13cm}{\footnotesize
      Generate catalog of simulated variable/transient sources

      \vspace{\dp0}
      } \end{minipage} \\
      \\ \cdashline{2-3}

      & Expected Result & 

      \begin{minipage}[t]{13cm}{\footnotesize
      
      \vspace{\dp0}
      } \end{minipage} \\
      \\ \cdashline{2-3}

      & \begin{minipage}[t]{2cm}{Actual\\ Result}\end{minipage}   & 
      \begin{minipage}[t]{13cm}{\footnotesize
      
      \vspace{\dp0}
      } \end{minipage} \\
      \\ \cdashline{2-3}


      & Status          & Not Executed \\ \hline

      2 & Description &

      \begin{minipage}[t]{13cm}{\footnotesize
      Inject simulated variables/transients into actual images

      \vspace{\dp0}
      } \end{minipage} \\
      \\ \cdashline{2-3}

      & Expected Result & 

      \begin{minipage}[t]{13cm}{\footnotesize
      Set of images with simulated variables/transients

      \vspace{\dp0}
      } \end{minipage} \\
      \\ \cdashline{2-3}

      & \begin{minipage}[t]{2cm}{Actual\\ Result}\end{minipage}   & 
      \begin{minipage}[t]{13cm}{\footnotesize
      
      \vspace{\dp0}
      } \end{minipage} \\
      \\ \cdashline{2-3}


      & Status          & Not Executed \\ \hline

      3 & Description &

      \begin{minipage}[t]{13cm}{\footnotesize
      Run difference imaging on images with injected variables/transients

      \vspace{\dp0}
      } \end{minipage} \\
      \\ \cdashline{2-3}

      & Expected Result & 

      \begin{minipage}[t]{13cm}{\footnotesize
      Catalog of detected DIASources

      \vspace{\dp0}
      } \end{minipage} \\
      \\ \cdashline{2-3}

      & \begin{minipage}[t]{2cm}{Actual\\ Result}\end{minipage}   & 
      \begin{minipage}[t]{13cm}{\footnotesize
      
      \vspace{\dp0}
      } \end{minipage} \\
      \\ \cdashline{2-3}


      & Status          & Not Executed \\ \hline

      4 & Description &

      \begin{minipage}[t]{13cm}{\footnotesize
      Rate DIASources according to spuriousness metric

      \vspace{\dp0}
      } \end{minipage} \\
      \\ \cdashline{2-3}

      & Expected Result & 

      \begin{minipage}[t]{13cm}{\footnotesize
      Catalog of DIASources with assigned spuriousness values

      \vspace{\dp0}
      } \end{minipage} \\
      \\ \cdashline{2-3}

      & \begin{minipage}[t]{2cm}{Actual\\ Result}\end{minipage}   & 
      \begin{minipage}[t]{13cm}{\footnotesize
      
      \vspace{\dp0}
      } \end{minipage} \\
      \\ \cdashline{2-3}


      & Status          & Not Executed \\ \hline

      5 & Description &

      \begin{minipage}[t]{13cm}{\footnotesize
      Find value of spuriousness threshold that preserves injected sources at
completeness mopsCompletenessMin

      \vspace{\dp0}
      } \end{minipage} \\
      \\ \cdashline{2-3}

      & Expected Result & 

      \begin{minipage}[t]{13cm}{\footnotesize
      Threshold in spuriousness metric

      \vspace{\dp0}
      } \end{minipage} \\
      \\ \cdashline{2-3}

      & \begin{minipage}[t]{2cm}{Actual\\ Result}\end{minipage}   & 
      \begin{minipage}[t]{13cm}{\footnotesize
      
      \vspace{\dp0}
      } \end{minipage} \\
      \\ \cdashline{2-3}


      & Status          & Not Executed \\ \hline

      6 & Description &

      \begin{minipage}[t]{13cm}{\footnotesize
      Compare to analysis of purity as a function of spuriousness metric

      \vspace{\dp0}
      } \end{minipage} \\
      \\ \cdashline{2-3}

      & Expected Result & 

      \begin{minipage}[t]{13cm}{\footnotesize
      
      \vspace{\dp0}
      } \end{minipage} \\
      \\ \cdashline{2-3}

      & \begin{minipage}[t]{2cm}{Actual\\ Result}\end{minipage}   & 
      \begin{minipage}[t]{13cm}{\footnotesize
      
      \vspace{\dp0}
      } \end{minipage} \\
      \\ \cdashline{2-3}


      & Status          & Not Executed \\ \hline

    \end{longtable}


    \paragraph{Test Case LVV-T533 - MOPS purity threshold
 }\mbox{}\\

Open  \href{https://jira.lsstcorp.org/secure/Tests.jspa#/testCase/LVV-T533}{\textit{ LVV-T533 } }
test case in Jira.

    

    \textbf{ Preconditions}:\\
    

    Execution status: {\bf Not Executed }

    Final comment:\\


    Detailed step results:

    \begin{longtable}{p{1cm}p{2cm}p{13cm}}
    \hline
    {Step} & \multicolumn{2}{c}{Description, Results and Status}\\ \hline
      1 & Description &

      \begin{minipage}[t]{13cm}{\footnotesize
      Identify all truly variable sources in a mini-survey area. ~This will
either be done by waiting for the mini-survey to complete and running a
full historical light curve analysis on the region, or through human
inspection of difference images (or some combination of both).

      \vspace{\dp0}
      } \end{minipage} \\
      \\ \cdashline{2-3}

      & Expected Result & 

      \begin{minipage}[t]{13cm}{\footnotesize
      Catalog of true variables in the region.

      \vspace{\dp0}
      } \end{minipage} \\
      \\ \cdashline{2-3}

      & \begin{minipage}[t]{2cm}{Actual\\ Result}\end{minipage}   & 
      \begin{minipage}[t]{13cm}{\footnotesize
      
      \vspace{\dp0}
      } \end{minipage} \\
      \\ \cdashline{2-3}


      & Status          & Not Executed \\ \hline

      2 & Description &

      \begin{minipage}[t]{13cm}{\footnotesize
      Go back to the individual images in the mini-survey and perform
difference image analysis.

      \vspace{\dp0}
      } \end{minipage} \\
      \\ \cdashline{2-3}

      & Expected Result & 

      \begin{minipage}[t]{13cm}{\footnotesize
      Catalogs of DIASources, some of them bogus.

      \vspace{\dp0}
      } \end{minipage} \\
      \\ \cdashline{2-3}

      & \begin{minipage}[t]{2cm}{Actual\\ Result}\end{minipage}   & 
      \begin{minipage}[t]{13cm}{\footnotesize
      
      \vspace{\dp0}
      } \end{minipage} \\
      \\ \cdashline{2-3}


      & Status          & Not Executed \\ \hline

      3 & Description &

      \begin{minipage}[t]{13cm}{\footnotesize
      Use catalog from step 1 to identify which of the DIASources in step 2
are real and which are artifacts.

      \vspace{\dp0}
      } \end{minipage} \\
      \\ \cdashline{2-3}

      & Expected Result & 

      \begin{minipage}[t]{13cm}{\footnotesize
      Catalog of DIASources labeled as either `real' or `bogus'.

      \vspace{\dp0}
      } \end{minipage} \\
      \\ \cdashline{2-3}

      & \begin{minipage}[t]{2cm}{Actual\\ Result}\end{minipage}   & 
      \begin{minipage}[t]{13cm}{\footnotesize
      
      \vspace{\dp0}
      } \end{minipage} \\
      \\ \cdashline{2-3}


      & Status          & Not Executed \\ \hline

      4 & Description &

      \begin{minipage}[t]{13cm}{\footnotesize
      Rate DIASource detections with spuriousness metric.

      \vspace{\dp0}
      } \end{minipage} \\
      \\ \cdashline{2-3}

      & Expected Result & 

      \begin{minipage}[t]{13cm}{\footnotesize
      Catalog of DIASources with spuriousness metric assigned.

      \vspace{\dp0}
      } \end{minipage} \\
      \\ \cdashline{2-3}

      & \begin{minipage}[t]{2cm}{Actual\\ Result}\end{minipage}   & 
      \begin{minipage}[t]{13cm}{\footnotesize
      
      \vspace{\dp0}
      } \end{minipage} \\
      \\ \cdashline{2-3}


      & Status          & Not Executed \\ \hline

      5 & Description &

      \begin{minipage}[t]{13cm}{\footnotesize
      Find value of spuriousness metric which gives desired purity
mopsPurityMin

      \vspace{\dp0}
      } \end{minipage} \\
      \\ \cdashline{2-3}

      & Expected Result & 

      \begin{minipage}[t]{13cm}{\footnotesize
      Threshold in spuriousness metric.

      \vspace{\dp0}
      } \end{minipage} \\
      \\ \cdashline{2-3}

      & \begin{minipage}[t]{2cm}{Actual\\ Result}\end{minipage}   & 
      \begin{minipage}[t]{13cm}{\footnotesize
      
      \vspace{\dp0}
      } \end{minipage} \\
      \\ \cdashline{2-3}


      & Status          & Not Executed \\ \hline

      6 & Description &

      \begin{minipage}[t]{13cm}{\footnotesize
      Compare to completeness threshold in spuriousness metric.

      \vspace{\dp0}
      } \end{minipage} \\
      \\ \cdashline{2-3}

      & Expected Result & 

      \begin{minipage}[t]{13cm}{\footnotesize
      
      \vspace{\dp0}
      } \end{minipage} \\
      \\ \cdashline{2-3}

      & \begin{minipage}[t]{2cm}{Actual\\ Result}\end{minipage}   & 
      \begin{minipage}[t]{13cm}{\footnotesize
      
      \vspace{\dp0}
      } \end{minipage} \\
      \\ \cdashline{2-3}


      & Status          & Not Executed \\ \hline

    \end{longtable}


    \paragraph{Test Case LVV-T294 - On-sky Observations: Full-survey Key Performance Metrics
 }\mbox{}\\

Open  \href{https://jira.lsstcorp.org/secure/Tests.jspa#/testCase/LVV-T294}{\textit{ LVV-T294 } }
test case in Jira.

    Repeated observations of a smaller number of fields reaching cumulative
exposures equivalent to the 10-year stack in the wide-fast-deep survey,
specifically, 200 visits in both the r and i band. These ~observations
are designed to measure residual PSF ellipticities, ~and to test
transient, variable, and moving object detection over a range of
timescales. Three fields should be chosen along the ecliptic that
together span a range of source densities. Each field should be observed
in multiple epochsdistributed over at least 3 consecutive nights and
cover a range of airmasses. Dithered pointings will be used to
approximate the coverage pattern expected in the wide-fast-deep
survey.\\[2\baselineskip]Estimated ~observing time = 34 seconds * 200
visits * 2 filters * 3 (dither pattern) * 3 fields =
\textasciitilde{}36hrs.


    \textbf{ Preconditions}:\\
    

    Execution status: {\bf Not Executed }

    Final comment:\\


    Detailed step results:

    \begin{longtable}{p{1cm}p{2cm}p{13cm}}
    \hline
    {Step} & \multicolumn{2}{c}{Description, Results and Status}\\ \hline
      1 & Description &

      \begin{minipage}[t]{13cm}{\footnotesize
      
      \vspace{\dp0}
      } \end{minipage} \\
      \\ \cdashline{2-3}

      & Expected Result & 

      \begin{minipage}[t]{13cm}{\footnotesize
      
      \vspace{\dp0}
      } \end{minipage} \\
      \\ \cdashline{2-3}

      & \begin{minipage}[t]{2cm}{Actual\\ Result}\end{minipage}   & 
      \begin{minipage}[t]{13cm}{\footnotesize
      
      \vspace{\dp0}
      } \end{minipage} \\
      \\ \cdashline{2-3}


      & Status          & Not Executed \\ \hline

    \end{longtable}


    \paragraph{Test Case LVV-T296 - Data Processing Campaign: Full-survey Key Performance Metrics
 }\mbox{}\\

Open  \href{https://jira.lsstcorp.org/secure/Tests.jspa#/testCase/LVV-T296}{\textit{ LVV-T296 } }
test case in Jira.

    

    \textbf{ Preconditions}:\\
    

    Execution status: {\bf Not Executed }

    Final comment:\\


    Detailed step results:

    \begin{longtable}{p{1cm}p{2cm}p{13cm}}
    \hline
    {Step} & \multicolumn{2}{c}{Description, Results and Status}\\ \hline
      1 & Description &

      \begin{minipage}[t]{13cm}{\footnotesize
      
      \vspace{\dp0}
      } \end{minipage} \\
      \\ \cdashline{2-3}

      & Expected Result & 

      \begin{minipage}[t]{13cm}{\footnotesize
      
      \vspace{\dp0}
      } \end{minipage} \\
      \\ \cdashline{2-3}

      & \begin{minipage}[t]{2cm}{Actual\\ Result}\end{minipage}   & 
      \begin{minipage}[t]{13cm}{\footnotesize
      
      \vspace{\dp0}
      } \end{minipage} \\
      \\ \cdashline{2-3}


      & Status          & Not Executed \\ \hline

    \end{longtable}


  \subsection{Test Cycle LVV-C36 }

Open test cycle {\it \href{https://jira.lsstcorp.org/secure/Tests.jspa#/testrun/LVV-C36}{Commissioning SV: 20-year Depth w/ ComCam
}} in Jira.

  Commissioning SV: 20-year Depth w/ ComCam
\\
  Status: Not Executed

  

  \subsubsection{Software Version/Baseline}
    Not provided.

  \subsubsection{Configuration}
    Not provided.

  \subsubsection{Test Cases in LVV-C36 Test Cycle}


    \paragraph{Test Case LVV-T1071 - On-sky Observations: 20-year Depth Test
 }\mbox{}\\

Open  \href{https://jira.lsstcorp.org/secure/Tests.jspa#/testCase/LVV-T1071}{\textit{ LVV-T1071 } }
test case in Jira.

    

    \textbf{ Preconditions}:\\
    

    Execution status: {\bf Not Executed }

    Final comment:\\


    Detailed step results:

    \begin{longtable}{p{1cm}p{2cm}p{13cm}}
    \hline
    {Step} & \multicolumn{2}{c}{Description, Results and Status}\\ \hline
      1 & Description &

      \begin{minipage}[t]{13cm}{\footnotesize
      
      \vspace{\dp0}
      } \end{minipage} \\
      \\ \cdashline{2-3}

      & Expected Result & 

      \begin{minipage}[t]{13cm}{\footnotesize
      
      \vspace{\dp0}
      } \end{minipage} \\
      \\ \cdashline{2-3}

      & \begin{minipage}[t]{2cm}{Actual\\ Result}\end{minipage}   & 
      \begin{minipage}[t]{13cm}{\footnotesize
      
      \vspace{\dp0}
      } \end{minipage} \\
      \\ \cdashline{2-3}


      & Status          & Not Executed \\ \hline

    \end{longtable}


  \subsection{Test Cycle LVV-C37 }

Open test cycle {\it \href{https://jira.lsstcorp.org/secure/Tests.jspa#/testrun/LVV-C37}{Commissioning SV: Scheduler Testing w/ ComCam
}} in Jira.

  Commissioning SV: Scheduler Testing w/ ComCam
\\
  Status: Not Executed

  

  \subsubsection{Software Version/Baseline}
    Not provided.

  \subsubsection{Configuration}
    Not provided.

  \subsubsection{Test Cases in LVV-C37 Test Cycle}


    \paragraph{Test Case LVV-T965 - Sub-system health during communication outage
 }\mbox{}\\

Open  \href{https://jira.lsstcorp.org/secure/Tests.jspa#/testCase/LVV-T965}{\textit{ LVV-T965 } }
test case in Jira.

    Verify that sub-system health reporting infrastructure continues to
function, even in the absence of base-to-summit communications


    \textbf{ Preconditions}:\\
    

    Execution status: {\bf Not Executed }

    Final comment:\\


    Detailed step results:

    \begin{longtable}{p{1cm}p{2cm}p{13cm}}
    \hline
    {Step} & \multicolumn{2}{c}{Description, Results and Status}\\ \hline
      1 & Description &

      \begin{minipage}[t]{13cm}{\footnotesize
      While the telescope is observing, deactivate base-to-summit
communication channels for some reasonable amount of time
(\textasciitilde{} a few hours?)

      \vspace{\dp0}
      } \end{minipage} \\
      \\ \cdashline{2-3}

      & Expected Result & 

      \begin{minipage}[t]{13cm}{\footnotesize
      
      \vspace{\dp0}
      } \end{minipage} \\
      \\ \cdashline{2-3}

      & \begin{minipage}[t]{2cm}{Actual\\ Result}\end{minipage}   & 
      \begin{minipage}[t]{13cm}{\footnotesize
      
      \vspace{\dp0}
      } \end{minipage} \\
      \\ \cdashline{2-3}


      & Status          & Not Executed \\ \hline

      2 & Description &

      \begin{minipage}[t]{13cm}{\footnotesize
      Verify that the sub-system health reporting infrastructure specified in
OSS-REQ-0065 continues to operate while communication is down.

      \vspace{\dp0}
      } \end{minipage} \\
      \\ \cdashline{2-3}

      & Expected Result & 

      \begin{minipage}[t]{13cm}{\footnotesize
      
      \vspace{\dp0}
      } \end{minipage} \\
      \\ \cdashline{2-3}

      & \begin{minipage}[t]{2cm}{Actual\\ Result}\end{minipage}   & 
      \begin{minipage}[t]{13cm}{\footnotesize
      
      \vspace{\dp0}
      } \end{minipage} \\
      \\ \cdashline{2-3}


      & Status          & Not Executed \\ \hline

    \end{longtable}


    \paragraph{Test Case LVV-T964 - Quick look during outage
 }\mbox{}\\

Open  \href{https://jira.lsstcorp.org/secure/Tests.jspa#/testCase/LVV-T964}{\textit{ LVV-T964 } }
test case in Jira.

    Verify that the quick look system still functions, even when
communication between the base and the summit has been lost


    \textbf{ Preconditions}:\\
    

    Execution status: {\bf Not Executed }

    Final comment:\\


    Detailed step results:

    \begin{longtable}{p{1cm}p{2cm}p{13cm}}
    \hline
    {Step} & \multicolumn{2}{c}{Description, Results and Status}\\ \hline
      1 & Description &

      \begin{minipage}[t]{13cm}{\footnotesize
      While the telescope is observing, deactivate all communication channels
between the base and summit for some reasonable amount of time
(\textasciitilde{} a few hours?)

      \vspace{\dp0}
      } \end{minipage} \\
      \\ \cdashline{2-3}

      & Expected Result & 

      \begin{minipage}[t]{13cm}{\footnotesize
      
      \vspace{\dp0}
      } \end{minipage} \\
      \\ \cdashline{2-3}

      & \begin{minipage}[t]{2cm}{Actual\\ Result}\end{minipage}   & 
      \begin{minipage}[t]{13cm}{\footnotesize
      
      \vspace{\dp0}
      } \end{minipage} \\
      \\ \cdashline{2-3}


      & Status          & Not Executed \\ \hline

      2 & Description &

      \begin{minipage}[t]{13cm}{\footnotesize
      Verify that the quick look system specified in OSS-REQ-0057 (LVV-1024)
continues to operate, even in the absence of communications.

      \vspace{\dp0}
      } \end{minipage} \\
      \\ \cdashline{2-3}

      & Expected Result & 

      \begin{minipage}[t]{13cm}{\footnotesize
      
      \vspace{\dp0}
      } \end{minipage} \\
      \\ \cdashline{2-3}

      & \begin{minipage}[t]{2cm}{Actual\\ Result}\end{minipage}   & 
      \begin{minipage}[t]{13cm}{\footnotesize
      
      \vspace{\dp0}
      } \end{minipage} \\
      \\ \cdashline{2-3}


      & Status          & Not Executed \\ \hline

    \end{longtable}


    \paragraph{Test Case LVV-T967 - Scheduler functioning during communication outage
 }\mbox{}\\

Open  \href{https://jira.lsstcorp.org/secure/Tests.jspa#/testCase/LVV-T967}{\textit{ LVV-T967 } }
test case in Jira.

    Verify that the scheduler can continue to operate the telescope, even
when communication between the base and the summit has been lost


    \textbf{ Preconditions}:\\
    

    Execution status: {\bf Not Executed }

    Final comment:\\


    Detailed step results:

    \begin{longtable}{p{1cm}p{2cm}p{13cm}}
    \hline
    {Step} & \multicolumn{2}{c}{Description, Results and Status}\\ \hline
      1 & Description &

      \begin{minipage}[t]{13cm}{\footnotesize
      Run OpSim with the scheduler in a mode that simulates loss of
communication between the base and the summit. ~Verify that the
scheduler can continue to operate for 48 simulated hours of observing in
this state.

      \vspace{\dp0}
      } \end{minipage} \\
      \\ \cdashline{2-3}

      & Expected Result & 

      \begin{minipage}[t]{13cm}{\footnotesize
      
      \vspace{\dp0}
      } \end{minipage} \\
      \\ \cdashline{2-3}

      & \begin{minipage}[t]{2cm}{Actual\\ Result}\end{minipage}   & 
      \begin{minipage}[t]{13cm}{\footnotesize
      
      \vspace{\dp0}
      } \end{minipage} \\
      \\ \cdashline{2-3}


      & Status          & Not Executed \\ \hline

    \end{longtable}


\input{appendix.tex}
\end{document}
